%
% File naaclhlt2012.tex
%

\documentclass[a4paper,11pt]{article}
\usepackage{naaclhlt2012}
\usepackage{apacite}
\usepackage{amsmath}
\usepackage{pgf}

\title{Modeling Referential Coordination as a Particle Swarm Optimization Task}

\author{ Anonymous PDF:  author names withheld\\}
%\author{H. Chase Stevens \\
%  University of Edinburgh \\
%  3 Charles Street \\
%  Edinburgh, UK \\
%  {\tt chase@chasestevens.com} \\\And
%  Hannah Rohde \\
%  University of Edinburgh \\
%  3 Charles Street \\
%  Edinburgh, UK \\
%  {\tt hannah.rohde@ed.ac.uk} \\}

\date{}

\begin{document}
\maketitle
\begin{abstract}
The question of how referential choice and interpretation are influenced by production cost remains unresolved in the literature. Recent research \cite{rohde2012,degen2012,frank2012} investigates the conditions under which speakers coordinate the use of ambiguous expressions. This paper takes a novel approach to modeling referential coordination by simulating the results of one such study on referring expression costs \cite{rohde2012}.  The simulation uses a general-purpose optimization method, particle swarm optimization, to capture the previously reported referential choices and to extrapolate to new communicative conditions.  The results replicate observed overinformativity behaviors \cite{brennan1996} and demonstrate that dyadic referential coordination can be framed as a constrained optimization problem in which agents may need only to maintain a simplified representation of the common ground and of each other.
\end{abstract}

\section{Introduction}
An open question in the field of linguistics is how participants in a conversation coordinate their use of referring expressions. When producing referring expressions, speakers must weigh the cost of an expression, in terms of both construction and articulation, against the ease with which their conversational partners will be able to infer the intended referent. When speakers employ referring expressions that do not uniquely select a referent within the context of the discourse, they risk miscommunication.

For example, consider a context in which there are three available referents, a chocolate Labrador, a black Poodle, and a brown American Water Spaniel/German Longhaired Pointer mix. Given the high cost of producing an unambiguous referring expression for the third animal, a speaker might attempt to refer to it as ``that brown dog". However, this expression could also indicate the chocolate Labrador, and rectifying this misinterpretation could be costly. Interlocutors therefore jointly seek to minimize their expended effort while not violating the constraint imposed by their partner's ability to disambiguate the referring expressions used \cite{benz2005}. Thus, speakers' referential strategies must be sensitive to the relative costs of producing each referring expression as well as to the evolving state of mappings between referential forms and intended referents, as coordinated with their interlocutors. 

Recent work has used language games to investigate how speakers make selections from a given set of available ambiguous and unambiguous referring expressions \cite{rohde2012,degen2012,frank2012}. Unlike some studies which target one-time production choices (e.g. \citeauthor{degen2012}, \citeyearNP{degen2012}), \citeauthor{rohde2012}\ analyze pairs of interlocutors coordinating referring expressions in a ``conversational" (multiple turn) scenario. ``Conversations" consisted of iterated language games in which participant dyads were rewarded for coordinating their use of ambiguous and unambiguous referring expressions. The results showed that participants' ability to successfully coordinate their use of less costly ambiguous forms was sensitive to the relative costs of the available competing unambiguous forms. 

This paper seeks to better understand the role of form costs in referential coordination by building a computational model of \citeauthor{rohde2012}'s findings. A suitable model must (i) represent the internal state of an agent with respect to the game and allow the agent's actions to be derived from this state, (ii) use the agent's communicative success with their partner to update their internal state within the model, even as their partner's internal state itself changes, and (iii) extend easily to other potential language games. 

Particle swarm optimization (PSO) is suitable in all three regards. PSO can serve not only as a general optimization method, but also as a means of modeling human social behavior, especially collaborative problem solving \cite{kennedy1997}. The optimized PSO model presented here, which utilizes a mixed strategy search space to represent form production and comprehension, is shown to outperform a baseline model in replicating human responses to form cost changes. The optimized model is used to extrapolate beyond previously tested conditions, providing a more comprehensive picture of the circumstances under which ambiguous form coordination is possible and likely. The parameters of the model and its predictions can be interpreted both in terms of the PSO algorithm itself and in terms of a broader understanding of the process of referential coordination.


\section{Background}
\subsection{Referential coordination}
\subsubsection{Game theoretic approaches}
Game theory provides a framework for modeling agents' actions as strategies within a game. To decide which action to choose, agents attempt to maximize their expected utility by leveraging their knowledge of the game's state \cite{benz2005}. Game theoretic concepts capture many linguistic phenomena that depend on the establishment of conventions between agents whose interests are aligned \cite{lewis1969}.

Recent work applies game theory to problems of referential coordination. For example, the use of more general but costly expressions is conventionally taken to implicitly exclude the senses of easily accessed and more specific forms (e.g. ``caused to die" excluding ``killed"). The establishment of this convention has been explained using an evolutionary game-theoretic approach: Both speaker and hearer benefit from an interpretation of the former which carries a greater degree of information (``caused to die but didn't kill") \cite{benz2005}. Models have also been used to predict participant behavior in referential inference tasks \cite{degen2012}; however, the need for more nuanced, comprehensive models (an example of which this paper attempts to provide) has become clear. 

\subsubsection{Psycholinguistic approaches}
The problem of referential coordination within the psycholinguistics literature has primarily been framed in terms of audience design \cite{clark1982}, whereby speakers carefully tailor utterances to best target their interlocutors; likewise, listeners interpret the meaning of utterances with respect not only to the speaker but also with respect to the speaker's presumed beliefs about the listener. Within such an account, a speaker's audience design enables the listener to disambiguate the referent of an otherwise ambiguous referring expression, provided the listener's internal model of the speaker is sufficient to capture why the speaker would have chosen a particular expression. 

As an example (adapted from \citeauthor{clark1982}), suppose Alice uses the referring expression ``your friend" when talking to Bob. Assuming Bob has more than one friend, the referent of this expression is ambiguous. However, if Alice has met only one of Bob's friends, she may rely on him to recognize this fact and interpret the expression accordingly. From Bob's perspective, to disambiguate Alice's referring expression, he must both recall that Alice has met only one of his friends and correctly reason that she will expect him to leverage this information. As interlocutors update their models of each other and the common information shared between them, they become increasingly entrained on specific lexical forms which are reflective of this shared knowledge and experience.

This understanding of referential coordination is borne out in work on natural language generation. For example, \citeA{golland2010} utilize a game-theoretic approach to realize a language game in which an artificial agent must successfully communicate a specified referent to a human listener. Their results show that when this artificial speaker is endowed with an internally embedded model of the listener, it substantially outperforms simpler models which do not take the listener into account.

Such research implicitly assumes that the common ground between speaker and hearer informs successful utterance planning, whereas more egocentric models of communication offer a competing view. In a human behavioral study \cite{horton1996}, participants who produced referring expressions for a partner were more likely to fail to take into account the shared common ground when under time pressure than when speed was not encouraged. This effect of time pressure was taken as evidence that common ground constraints do not inform initial utterance planning, but instead acts as a filtering mechanism during later-stage utterance production. Further, \citeA{horton1996} posit that common ground may play a minimal role in the production of the majority of utterances, given the cost of routinely taking this information into consideration.

\subsubsection{Coordination of ambiguous forms}
\citeA{rohde2012} present an iterated language game in which players attempt to indicate an object to their partner via one of several possible referring expressions. Players gain points upon successful communication but must spend points to communicate. Each referent has an unambiguous form that speakers may use; alternatively, players may use an ambiguous form that could potentially indicate a number of referents. Players were incentivized to find the least costly forms with which they could achieve a high rate of communicative success. Two studies were conducted using the same set of six referents (various flowers and trees). A constrained lexicon consisted of a set of unambiguous forms (e.g., ``daisy", ``pine") with alternative ambiguous forms (``flower" and ``tree").

On one hand, players' strategies matched game theoretic predictions: Ambiguous forms were reserved for the referents with the most costly unambiguous form (Rohde et al.\ study 1).  On the other hand, coordination was also partially contingent on the relative costs of unambiguous and ambiguous forms: Pairs were more likely to successfully coordinate when unambiguous form costs were lower and more similar to each other (study 2), contra the prediction that the greater the cost disparity, the greater the expected gains in utility for the systematic exchange of a high cost unambiguous form for a low cost ambiguous one. \citeauthor{rohde2012}\ suggested that lower stakes encouraged players to explore a wider range of referential strategies. The question that this paper asks is whether this behavior can be better understood as an emergent property of coordination if game constraints are simulated in a PSO-based model.


\subsection{Particle swarm optimization}
\subsubsection{Formulation and parameters}
\label{sec:2.2.3}
PSO, an iterative global optimization algorithm, represents potential solutions to some problem as particles existing within an $n$-dimensional search space \cite{kennedy1995}. Particles have both a position and velocity within this space. Each particle keeps track of the best position it has been in as evaluated by the given objective function, as well as the best solution found by any member of the particle's group \cite{chong2013}. 

In this paper's formulation of PSO, a particle $i$ with position $x_i$ has velocity $v_i$ at state $t$:
\begin{multline} \label{eq:pso_vel}
\begin{split}
v_i^t = \theta(t) \cdot v_i^{t-1} & + \alpha \cdot \epsilon_1 \cdot (x_i^* - x_i^{t-1}) \\
                                  & + \beta \cdot \epsilon_2 \cdot (x_{N(i)}^* - x_i^{t-1})
\end{split}
\end{multline}
where $\theta$ is the inertial scheduling function, $\alpha$ is the cognitive component governing $i$'s attraction to its personal best known position $x_i^*$, $\beta$ is the social component governing $i$'s attraction to the best position $x_{N(i)}^*$ known for its group $N(i)$, and $\epsilon_1$ and $\epsilon_2$ are randomly chosen values within $\left(0, 1\right]$. $\theta$ is defined with respect to a base inertia $\tau$ and inertial dampening factor $\sigma$ such that
\begin{equation}
\theta(t) = \frac{\tau}{\sigma^t} 
\end{equation}
The position $x$ of $i$ at state $t$ is defined as
\begin{equation}
x_i^t = x_i^{t-1} + \phi \cdot v_i^t 
\end{equation}
where $\phi$ is a constant velocity dampening factor.

\subsubsection{Previous work}
\label{sec:pso_prev_work}

While PSO has been applied to a number of problems within the fields of linguistics and psychology, its primary use has been as a means of optimizing parameters for other models, as opposed to direct application as a model in and of itself (e.g. \citeauthor{mehdad2009}, \citeyearNP{mehdad2009}).\footnote{A notable exception is the use of PSO to perform unsupervised phoneme clustering \cite{ahmadi2007}.}

PSO has likewise been applied to game learning, often using a coevolutionary paradigm. Traditionally this method has involved PSO over a search space of neural network weights, or for classical constraint optimization ``games", such as the $n$-queens problem \cite{engelbrecht2005}. By contrast, in this paper, the positions of particles themselves comprise agents' internal states, which directly define a mixed strategy; further, the aggregate locations of the particles within the search space after running the optimization task are of primary interest, as opposed to the best solutions found by the swarm.


\section{Methods}
\subsection{Search space}
\label{sec:search_space}
To represent a solution to a referential coordination language game, the game-theoretic notion of a mixed strategy was adopted, whereby each possible action $a$ within a game is performed by a player $i$ with some probability $P_i(a)$ \cite{benz2005}. For a referent $r$, the player can be said to maintain a probability of producing the associated ambiguous form $A$, $P_i(A|r)$. Conversely, the probability of a player producting the unambiguous form to refer to $r$ is $1 - P_i(A|r)$. Therefore, in a game with $n$ possible referents for which $A$ could be used, a player's strategy was represented with $n$ independent probabilities, yielding an $n$-dimensional search space. 

It is important to note that this does not constitute a traditional mixed strategy, in that, for a given player, the probabilities over all referents of using the ambiguous form may not sum to $1$.  For example, a player is capable of opting to not use the ambiguous form at all. In this sense, it is more accurate to state that a player $i$ maintains a separate mixed strategy for each referent $r$, where, for the ambiguous form $A$ and unambiguous form $U$, $P_i(A|r) + P_i(U|r) = 1$.

Players were assumed to optimize their strategies for groups of referents sharing the same ambiguous form independently of other groups. As such, \citeauthor{rohde2012}'s two studies were each treated as two independent language games being run concurrently, yielding search spaces with three dimensions, not six. Note that some player pairs in \citeauthor{rohde2012}'s second study coordinated the use of the ambiguous form for one group, but not the other, suggesting that behavior with one form need not mirror the other.

Finally, because each dimension in the search space defined above reflects a probability, values outside the interval $[0, 1]$ are invalid. To adapt the PSO algorithm to this constraint, particles in our optimized model which moved outside the desired search space immediately had a repair method applied to them, whereby they were relocated to the nearest point which did not violate the given constraints; as a baseline, a standard ``rejection" constraint-handling technique was employed (see \citeNP{engelbrecht2005}).
 
\subsection{Objective function}
\label{sec:objective_func}
In applying PSO to the \citeauthor{rohde2012}\ language game, an appropriate representation of the game's goals is formulated as an objective function. The expected number of points awarded to player $i$ communicating referent $r$ to their partner $j$ can be calculated as follows:

\begin{multline}
\begin{split}
EP_{i}(r|j) = & P_i(A|r)(S \cdot P_j(r|A) - cost_A) \\
              & + (1 - P_i(A|r))(S - cost_r) 
\end{split}
\end{multline}
where $cost_A$ is the cost of producing the ambiguous form, $S$ is the number of points awarded on successful communication (80 and 85 for \citeauthor{rohde2012}'s study 1 and 2, respectively), and $cost_r$ is the cost of producing $r$'s unambiguous form.

In each round, the actual number of points awarded to players is dependent on samples from $P_i$ and $P_j$, as well as the randomly-chosen $r$. As such, without the strategies of $i$ or $j$ changing, there are, for any given round, a number of possible scores $i$ might attain. To avoid optimizing a stochastic objective function, the objective function $f$ was ergo chosen as:
\begin{equation}
f(i) = \sum_{r \in R} EP_{i}(r|j) + EP_{j}(r|i)
\end{equation}


\subsection{PSO parameter optimization}
\label{sec:param_opt}
The PSO model parameters were optimized to best fit the experimental data. The values of the cognitive and social components $\alpha$ and $\beta$ were restricted to the interval $[0,4]$, the inertial dampening factor $\sigma$ to $[1,1.1]$, base inertia $\tau$ to $[0,4]$, and the velocity dampening constant $\phi$ to $[0,2]$. The number of iterations over which the model was run, $[100,1000]$, was also optimized.

Optimization of these parameters was itself performed via PSO, over $1255$ iterations. The parameter values used for this meta-optimization followed published recommendations \cite{shi1998,solnon2010}. A baseline model, against which the optimized model was compared, also used these recommended parameter values. Initial particle positions were assigned using a randomized nonuniform method \cite{mitchell1991}. 

The parameter optimization task sought to minimize the discrepancy in rates of ambiguous form coordination, unambiguous form coordination, and failure to coordinate between the model and the experimental data, across all language game variants. To evaluate this and compare model behavior with the behavior of human players in the task, pairs were considered to have coordinated if, when the PSO task had completed, the pair could successfully communicate about referents $\geq 95\%$ of the time.


\section{Results and discussion}
\subsection{Meta-optimization task}
\label{sec:parameter_optimization}

\begin{table}[]
\begin{center}
    \begin{tabular}{ l l l r }
    Parameter  & Optimized & Baseline & $\% \Delta$ \\ \hline
    $\alpha$   & $0.689$   & $2.0$    & $-65.55\%$ \\ \hline
    $\beta$    & $2.897$   & $2.0$    & $+44.86\%$ \\ \hline
    $\sigma$   & $1.027$   & $1.001$  & $+2.58\%$\\ \hline
    $\tau$     & $0.658$   & $1.2$    & $-45.17\%$ \\ \hline
    $\phi$     & $1.202$   & $1.0$    & $+20.20\%$\\ \hline
    Iterations & $305$     & N/A      & N/A\\ 
    \end{tabular}
   	\caption{Comparison of optimized PSO parameters against baseline published settings}
	\label{table:1}
\end{center}
\end{table}

The results of the model parameter optimization task are presented in Table \ref{table:1}. Of particular note are the substantial discrepancies between the baseline and optimized values for the cognitive component ($\alpha$), social component ($\beta$), and base inertia ($\tau$). One possible explanation for these discrepancies concerns the dynamic nature of the language game. Recall from equation \ref{eq:pso_vel} that cognitive component ($\alpha$) determines how attracted a particle is to its own previously best-found position $x_i^*$, and that the social component ($\beta$) determines how attracted a particle is to the best-found solution within its group (here, dyad) $x_{N(i)}^*$. One possible technique for handling dynamic objective functions such as that presented by the \citeauthor{rohde2012} language game is, on objective function change, to reinitialize or discard $x_i^*$ and to recalculate $f(x_{N(i)}^*)$ \cite{engelbrecht2005}. While the latter is only achievable through a modification of the core PSO algorithm, in a scenario where the objective function changes every iteration, the former is tantamount to lowering $\alpha$ to zero. As the game modeled here is evaluated for each particle against its partner's position, which is updated on every iteration, the game's objective function matches this description; the lowering of $\alpha$ in the game's optimized parameters can therefore be interpreted as the emergence of this dynamic function adaptation technique.

Another possible explanation is that the divergence reflects the chosen objective function. Language is an inherently cooperative endeavour; referential coordination games may therefore encourage favoring the best strategies across all communicative partners over those that maximize individual fitness. Indeed, the parameter optimization results may reflect the fundamentally social underpinnings of the human condition. It is important, however, to note that any conclusions of this nature drawn from the parameter optimization task must be taken with a grain of salt; the underabundance of data from \citeauthor{rohde2012} and resultant infeasibility of dividing the data into test and training sets makes it likely that the models have been grossly overfit. Confirmation of these results would require not only more extensive human trials under a number of conditions, but also a thorough examination of how adjusting each of the PSO parameters in isolation impacts the model's results.


\subsection{Comparison against baseline}
\label{sec:model_comparison}
Results from the optimized model were compared against a baseline model, which made use of the standard parameters used in the meta-optimization task. For both models, 250 simulations of 10 pairs were performed for each of \citeauthor{rohde2012}'s sets of costs (see Table \ref{table:2}).
\begin{table}[]
\begin{center}
    \begin{tabular}{l c c c c | c}
    Set & $cost_{r_1}$ & $cost_{r_2}$ & $cost_{r_3}$ & $cost_A$ & \\ \hline
    1/Flowers & $60$ & $120$ & $280$ & $80$ & 1\\ \hline
    1/Trees & $60$ & $120$ & $250$ & $80$ & 2\\ \hline
    2/Flowers & $80$ & $140$ & $165$ & $80$ & 3\\ \hline
    2/Trees & $80$ & $135$ & $170$ & $80$ & 4\\ 
    \end{tabular}
    \caption{\citeauthor{rohde2012}'s referring expression costs by set (Study 1 or 2 / Flowers or Trees). For the simulation in this paper, each study has  been split into two experiments. The last column shows the experiment number used for reference in this paper.}
    \label{table:2}
\end{center}
\end{table}

\begin{figure}[t]
\centering
\scalebox{.55}{%% Creator: Matplotlib, PGF backend
%%
%% To include the figure in your LaTeX document, write
%%   \input{<filename>.pgf}
%%
%% Make sure the required packages are loaded in your preamble
%%   \usepackage{pgf}
%%
%% Figures using additional raster images can only be included by \input if
%% they are in the same directory as the main LaTeX file. For loading figures
%% from other directories you can use the `import` package
%%   \usepackage{import}
%% and then include the figures with
%%   \import{<path to file>}{<filename>.pgf}
%%
%% Matplotlib used the following preamble
%%   \usepackage{fontspec}
%%   \setsansfont{DejaVu Sans}
%%   \setmonofont{DejaVu Sans Mono}
%%
\begingroup%
\makeatletter%
\begin{pgfpicture}%
\pgfpathrectangle{\pgfpointorigin}{\pgfqpoint{8.150000in}{6.150000in}}%
\pgfusepath{use as bounding box, clip}%
\begin{pgfscope}%
\pgfsetbuttcap%
\pgfsetmiterjoin%
\definecolor{currentfill}{rgb}{1.000000,1.000000,1.000000}%
\pgfsetfillcolor{currentfill}%
\pgfsetlinewidth{0.000000pt}%
\definecolor{currentstroke}{rgb}{1.000000,1.000000,1.000000}%
\pgfsetstrokecolor{currentstroke}%
\pgfsetdash{}{0pt}%
\pgfpathmoveto{\pgfqpoint{0.000000in}{0.000000in}}%
\pgfpathlineto{\pgfqpoint{8.150000in}{0.000000in}}%
\pgfpathlineto{\pgfqpoint{8.150000in}{6.150000in}}%
\pgfpathlineto{\pgfqpoint{0.000000in}{6.150000in}}%
\pgfpathclose%
\pgfusepath{fill}%
\end{pgfscope}%
\begin{pgfscope}%
\pgfsetbuttcap%
\pgfsetmiterjoin%
\definecolor{currentfill}{rgb}{1.000000,1.000000,1.000000}%
\pgfsetfillcolor{currentfill}%
\pgfsetlinewidth{0.000000pt}%
\definecolor{currentstroke}{rgb}{0.000000,0.000000,0.000000}%
\pgfsetstrokecolor{currentstroke}%
\pgfsetstrokeopacity{0.000000}%
\pgfsetdash{}{0pt}%
\pgfpathmoveto{\pgfqpoint{1.018750in}{0.615000in}}%
\pgfpathlineto{\pgfqpoint{7.335000in}{0.615000in}}%
\pgfpathlineto{\pgfqpoint{7.335000in}{5.535000in}}%
\pgfpathlineto{\pgfqpoint{1.018750in}{5.535000in}}%
\pgfpathclose%
\pgfusepath{fill}%
\end{pgfscope}%
\begin{pgfscope}%
\pgfpathrectangle{\pgfqpoint{1.018750in}{0.615000in}}{\pgfqpoint{6.316250in}{4.920000in}} %
\pgfusepath{clip}%
\pgfsetbuttcap%
\pgfsetmiterjoin%
\definecolor{currentfill}{rgb}{0.200000,0.200000,0.200000}%
\pgfsetfillcolor{currentfill}%
\pgfsetlinewidth{1.003750pt}%
\definecolor{currentstroke}{rgb}{0.000000,0.000000,0.000000}%
\pgfsetstrokecolor{currentstroke}%
\pgfsetdash{}{0pt}%
\pgfpathmoveto{\pgfqpoint{1.176656in}{0.615000in}}%
\pgfpathlineto{\pgfqpoint{1.492469in}{0.615000in}}%
\pgfpathlineto{\pgfqpoint{1.492469in}{4.630904in}}%
\pgfpathlineto{\pgfqpoint{1.176656in}{4.630904in}}%
\pgfpathclose%
\pgfusepath{stroke,fill}%
\end{pgfscope}%
\begin{pgfscope}%
\pgfpathrectangle{\pgfqpoint{1.018750in}{0.615000in}}{\pgfqpoint{6.316250in}{4.920000in}} %
\pgfusepath{clip}%
\pgfsetbuttcap%
\pgfsetmiterjoin%
\definecolor{currentfill}{rgb}{0.200000,0.200000,0.200000}%
\pgfsetfillcolor{currentfill}%
\pgfsetlinewidth{1.003750pt}%
\definecolor{currentstroke}{rgb}{0.000000,0.000000,0.000000}%
\pgfsetstrokecolor{currentstroke}%
\pgfsetdash{}{0pt}%
\pgfpathmoveto{\pgfqpoint{2.755719in}{0.615000in}}%
\pgfpathlineto{\pgfqpoint{3.071531in}{0.615000in}}%
\pgfpathlineto{\pgfqpoint{3.071531in}{3.475972in}}%
\pgfpathlineto{\pgfqpoint{2.755719in}{3.475972in}}%
\pgfpathclose%
\pgfusepath{stroke,fill}%
\end{pgfscope}%
\begin{pgfscope}%
\pgfpathrectangle{\pgfqpoint{1.018750in}{0.615000in}}{\pgfqpoint{6.316250in}{4.920000in}} %
\pgfusepath{clip}%
\pgfsetbuttcap%
\pgfsetmiterjoin%
\definecolor{currentfill}{rgb}{0.200000,0.200000,0.200000}%
\pgfsetfillcolor{currentfill}%
\pgfsetlinewidth{1.003750pt}%
\definecolor{currentstroke}{rgb}{0.000000,0.000000,0.000000}%
\pgfsetstrokecolor{currentstroke}%
\pgfsetdash{}{0pt}%
\pgfpathmoveto{\pgfqpoint{4.334781in}{0.615000in}}%
\pgfpathlineto{\pgfqpoint{4.650594in}{0.615000in}}%
\pgfpathlineto{\pgfqpoint{4.650594in}{4.429629in}}%
\pgfpathlineto{\pgfqpoint{4.334781in}{4.429629in}}%
\pgfpathclose%
\pgfusepath{stroke,fill}%
\end{pgfscope}%
\begin{pgfscope}%
\pgfpathrectangle{\pgfqpoint{1.018750in}{0.615000in}}{\pgfqpoint{6.316250in}{4.920000in}} %
\pgfusepath{clip}%
\pgfsetbuttcap%
\pgfsetmiterjoin%
\definecolor{currentfill}{rgb}{0.200000,0.200000,0.200000}%
\pgfsetfillcolor{currentfill}%
\pgfsetlinewidth{1.003750pt}%
\definecolor{currentstroke}{rgb}{0.000000,0.000000,0.000000}%
\pgfsetstrokecolor{currentstroke}%
\pgfsetdash{}{0pt}%
\pgfpathmoveto{\pgfqpoint{5.913844in}{0.615000in}}%
\pgfpathlineto{\pgfqpoint{6.229656in}{0.615000in}}%
\pgfpathlineto{\pgfqpoint{6.229656in}{3.231567in}}%
\pgfpathlineto{\pgfqpoint{5.913844in}{3.231567in}}%
\pgfpathclose%
\pgfusepath{stroke,fill}%
\end{pgfscope}%
\begin{pgfscope}%
\pgfpathrectangle{\pgfqpoint{1.018750in}{0.615000in}}{\pgfqpoint{6.316250in}{4.920000in}} %
\pgfusepath{clip}%
\pgfsetbuttcap%
\pgfsetmiterjoin%
\definecolor{currentfill}{rgb}{0.400000,0.400000,0.400000}%
\pgfsetfillcolor{currentfill}%
\pgfsetlinewidth{1.003750pt}%
\definecolor{currentstroke}{rgb}{0.000000,0.000000,0.000000}%
\pgfsetstrokecolor{currentstroke}%
\pgfsetdash{}{0pt}%
\pgfpathmoveto{\pgfqpoint{1.492469in}{0.615000in}}%
\pgfpathlineto{\pgfqpoint{1.808281in}{0.615000in}}%
\pgfpathlineto{\pgfqpoint{1.808281in}{4.317159in}}%
\pgfpathlineto{\pgfqpoint{1.492469in}{4.317159in}}%
\pgfpathclose%
\pgfusepath{stroke,fill}%
\end{pgfscope}%
\begin{pgfscope}%
\pgfpathrectangle{\pgfqpoint{1.018750in}{0.615000in}}{\pgfqpoint{6.316250in}{4.920000in}} %
\pgfusepath{clip}%
\pgfsetbuttcap%
\pgfsetmiterjoin%
\definecolor{currentfill}{rgb}{0.400000,0.400000,0.400000}%
\pgfsetfillcolor{currentfill}%
\pgfsetlinewidth{1.003750pt}%
\definecolor{currentstroke}{rgb}{0.000000,0.000000,0.000000}%
\pgfsetstrokecolor{currentstroke}%
\pgfsetdash{}{0pt}%
\pgfpathmoveto{\pgfqpoint{3.071531in}{0.615000in}}%
\pgfpathlineto{\pgfqpoint{3.387344in}{0.615000in}}%
\pgfpathlineto{\pgfqpoint{3.387344in}{3.715558in}}%
\pgfpathlineto{\pgfqpoint{3.071531in}{3.715558in}}%
\pgfpathclose%
\pgfusepath{stroke,fill}%
\end{pgfscope}%
\begin{pgfscope}%
\pgfpathrectangle{\pgfqpoint{1.018750in}{0.615000in}}{\pgfqpoint{6.316250in}{4.920000in}} %
\pgfusepath{clip}%
\pgfsetbuttcap%
\pgfsetmiterjoin%
\definecolor{currentfill}{rgb}{0.400000,0.400000,0.400000}%
\pgfsetfillcolor{currentfill}%
\pgfsetlinewidth{1.003750pt}%
\definecolor{currentstroke}{rgb}{0.000000,0.000000,0.000000}%
\pgfsetstrokecolor{currentstroke}%
\pgfsetdash{}{0pt}%
\pgfpathmoveto{\pgfqpoint{4.650594in}{0.615000in}}%
\pgfpathlineto{\pgfqpoint{4.966406in}{0.615000in}}%
\pgfpathlineto{\pgfqpoint{4.966406in}{4.270882in}}%
\pgfpathlineto{\pgfqpoint{4.650594in}{4.270882in}}%
\pgfpathclose%
\pgfusepath{stroke,fill}%
\end{pgfscope}%
\begin{pgfscope}%
\pgfpathrectangle{\pgfqpoint{1.018750in}{0.615000in}}{\pgfqpoint{6.316250in}{4.920000in}} %
\pgfusepath{clip}%
\pgfsetbuttcap%
\pgfsetmiterjoin%
\definecolor{currentfill}{rgb}{0.400000,0.400000,0.400000}%
\pgfsetfillcolor{currentfill}%
\pgfsetlinewidth{1.003750pt}%
\definecolor{currentstroke}{rgb}{0.000000,0.000000,0.000000}%
\pgfsetstrokecolor{currentstroke}%
\pgfsetdash{}{0pt}%
\pgfpathmoveto{\pgfqpoint{6.229656in}{0.615000in}}%
\pgfpathlineto{\pgfqpoint{6.545469in}{0.615000in}}%
\pgfpathlineto{\pgfqpoint{6.545469in}{3.623004in}}%
\pgfpathlineto{\pgfqpoint{6.229656in}{3.623004in}}%
\pgfpathclose%
\pgfusepath{stroke,fill}%
\end{pgfscope}%
\begin{pgfscope}%
\pgfpathrectangle{\pgfqpoint{1.018750in}{0.615000in}}{\pgfqpoint{6.316250in}{4.920000in}} %
\pgfusepath{clip}%
\pgfsetbuttcap%
\pgfsetmiterjoin%
\definecolor{currentfill}{rgb}{0.600000,0.600000,0.600000}%
\pgfsetfillcolor{currentfill}%
\pgfsetlinewidth{1.003750pt}%
\definecolor{currentstroke}{rgb}{0.000000,0.000000,0.000000}%
\pgfsetstrokecolor{currentstroke}%
\pgfsetdash{}{0pt}%
\pgfpathmoveto{\pgfqpoint{1.808281in}{0.615000in}}%
\pgfpathlineto{\pgfqpoint{2.124094in}{0.615000in}}%
\pgfpathlineto{\pgfqpoint{2.124094in}{4.109128in}}%
\pgfpathlineto{\pgfqpoint{1.808281in}{4.109128in}}%
\pgfpathclose%
\pgfusepath{stroke,fill}%
\end{pgfscope}%
\begin{pgfscope}%
\pgfpathrectangle{\pgfqpoint{1.018750in}{0.615000in}}{\pgfqpoint{6.316250in}{4.920000in}} %
\pgfusepath{clip}%
\pgfsetbuttcap%
\pgfsetmiterjoin%
\definecolor{currentfill}{rgb}{0.600000,0.600000,0.600000}%
\pgfsetfillcolor{currentfill}%
\pgfsetlinewidth{1.003750pt}%
\definecolor{currentstroke}{rgb}{0.000000,0.000000,0.000000}%
\pgfsetstrokecolor{currentstroke}%
\pgfsetdash{}{0pt}%
\pgfpathmoveto{\pgfqpoint{3.387344in}{0.615000in}}%
\pgfpathlineto{\pgfqpoint{3.703156in}{0.615000in}}%
\pgfpathlineto{\pgfqpoint{3.703156in}{3.390478in}}%
\pgfpathlineto{\pgfqpoint{3.387344in}{3.390478in}}%
\pgfpathclose%
\pgfusepath{stroke,fill}%
\end{pgfscope}%
\begin{pgfscope}%
\pgfpathrectangle{\pgfqpoint{1.018750in}{0.615000in}}{\pgfqpoint{6.316250in}{4.920000in}} %
\pgfusepath{clip}%
\pgfsetbuttcap%
\pgfsetmiterjoin%
\definecolor{currentfill}{rgb}{0.600000,0.600000,0.600000}%
\pgfsetfillcolor{currentfill}%
\pgfsetlinewidth{1.003750pt}%
\definecolor{currentstroke}{rgb}{0.000000,0.000000,0.000000}%
\pgfsetstrokecolor{currentstroke}%
\pgfsetdash{}{0pt}%
\pgfpathmoveto{\pgfqpoint{4.966406in}{0.615000in}}%
\pgfpathlineto{\pgfqpoint{5.282219in}{0.615000in}}%
\pgfpathlineto{\pgfqpoint{5.282219in}{4.480844in}}%
\pgfpathlineto{\pgfqpoint{4.966406in}{4.480844in}}%
\pgfpathclose%
\pgfusepath{stroke,fill}%
\end{pgfscope}%
\begin{pgfscope}%
\pgfpathrectangle{\pgfqpoint{1.018750in}{0.615000in}}{\pgfqpoint{6.316250in}{4.920000in}} %
\pgfusepath{clip}%
\pgfsetbuttcap%
\pgfsetmiterjoin%
\definecolor{currentfill}{rgb}{0.600000,0.600000,0.600000}%
\pgfsetfillcolor{currentfill}%
\pgfsetlinewidth{1.003750pt}%
\definecolor{currentstroke}{rgb}{0.000000,0.000000,0.000000}%
\pgfsetstrokecolor{currentstroke}%
\pgfsetdash{}{0pt}%
\pgfpathmoveto{\pgfqpoint{6.545469in}{0.615000in}}%
\pgfpathlineto{\pgfqpoint{6.861281in}{0.615000in}}%
\pgfpathlineto{\pgfqpoint{6.861281in}{3.861318in}}%
\pgfpathlineto{\pgfqpoint{6.545469in}{3.861318in}}%
\pgfpathclose%
\pgfusepath{stroke,fill}%
\end{pgfscope}%
\begin{pgfscope}%
\pgfpathrectangle{\pgfqpoint{1.018750in}{0.615000in}}{\pgfqpoint{6.316250in}{4.920000in}} %
\pgfusepath{clip}%
\pgfsetbuttcap%
\pgfsetmiterjoin%
\definecolor{currentfill}{rgb}{0.800000,0.800000,0.800000}%
\pgfsetfillcolor{currentfill}%
\pgfsetlinewidth{1.003750pt}%
\definecolor{currentstroke}{rgb}{0.000000,0.000000,0.000000}%
\pgfsetstrokecolor{currentstroke}%
\pgfsetdash{}{0pt}%
\pgfpathmoveto{\pgfqpoint{2.124094in}{0.615000in}}%
\pgfpathlineto{\pgfqpoint{2.439906in}{0.615000in}}%
\pgfpathlineto{\pgfqpoint{2.439906in}{3.895000in}}%
\pgfpathlineto{\pgfqpoint{2.124094in}{3.895000in}}%
\pgfpathclose%
\pgfusepath{stroke,fill}%
\end{pgfscope}%
\begin{pgfscope}%
\pgfpathrectangle{\pgfqpoint{1.018750in}{0.615000in}}{\pgfqpoint{6.316250in}{4.920000in}} %
\pgfusepath{clip}%
\pgfsetbuttcap%
\pgfsetmiterjoin%
\definecolor{currentfill}{rgb}{0.800000,0.800000,0.800000}%
\pgfsetfillcolor{currentfill}%
\pgfsetlinewidth{1.003750pt}%
\definecolor{currentstroke}{rgb}{0.000000,0.000000,0.000000}%
\pgfsetstrokecolor{currentstroke}%
\pgfsetdash{}{0pt}%
\pgfpathmoveto{\pgfqpoint{3.703156in}{0.615000in}}%
\pgfpathlineto{\pgfqpoint{4.018969in}{0.615000in}}%
\pgfpathlineto{\pgfqpoint{4.018969in}{3.348333in}}%
\pgfpathlineto{\pgfqpoint{3.703156in}{3.348333in}}%
\pgfpathclose%
\pgfusepath{stroke,fill}%
\end{pgfscope}%
\begin{pgfscope}%
\pgfpathrectangle{\pgfqpoint{1.018750in}{0.615000in}}{\pgfqpoint{6.316250in}{4.920000in}} %
\pgfusepath{clip}%
\pgfsetbuttcap%
\pgfsetmiterjoin%
\definecolor{currentfill}{rgb}{0.800000,0.800000,0.800000}%
\pgfsetfillcolor{currentfill}%
\pgfsetlinewidth{1.003750pt}%
\definecolor{currentstroke}{rgb}{0.000000,0.000000,0.000000}%
\pgfsetstrokecolor{currentstroke}%
\pgfsetdash{}{0pt}%
\pgfpathmoveto{\pgfqpoint{5.282219in}{0.615000in}}%
\pgfpathlineto{\pgfqpoint{5.598031in}{0.615000in}}%
\pgfpathlineto{\pgfqpoint{5.598031in}{4.988333in}}%
\pgfpathlineto{\pgfqpoint{5.282219in}{4.988333in}}%
\pgfpathclose%
\pgfusepath{stroke,fill}%
\end{pgfscope}%
\begin{pgfscope}%
\pgfpathrectangle{\pgfqpoint{1.018750in}{0.615000in}}{\pgfqpoint{6.316250in}{4.920000in}} %
\pgfusepath{clip}%
\pgfsetbuttcap%
\pgfsetmiterjoin%
\definecolor{currentfill}{rgb}{0.800000,0.800000,0.800000}%
\pgfsetfillcolor{currentfill}%
\pgfsetlinewidth{1.003750pt}%
\definecolor{currentstroke}{rgb}{0.000000,0.000000,0.000000}%
\pgfsetstrokecolor{currentstroke}%
\pgfsetdash{}{0pt}%
\pgfpathmoveto{\pgfqpoint{6.861281in}{0.615000in}}%
\pgfpathlineto{\pgfqpoint{7.177094in}{0.615000in}}%
\pgfpathlineto{\pgfqpoint{7.177094in}{3.348333in}}%
\pgfpathlineto{\pgfqpoint{6.861281in}{3.348333in}}%
\pgfpathclose%
\pgfusepath{stroke,fill}%
\end{pgfscope}%
\begin{pgfscope}%
\pgfsetrectcap%
\pgfsetmiterjoin%
\pgfsetlinewidth{1.003750pt}%
\definecolor{currentstroke}{rgb}{0.000000,0.000000,0.000000}%
\pgfsetstrokecolor{currentstroke}%
\pgfsetdash{}{0pt}%
\pgfpathmoveto{\pgfqpoint{1.018750in}{5.535000in}}%
\pgfpathlineto{\pgfqpoint{7.335000in}{5.535000in}}%
\pgfusepath{stroke}%
\end{pgfscope}%
\begin{pgfscope}%
\pgfsetrectcap%
\pgfsetmiterjoin%
\pgfsetlinewidth{1.003750pt}%
\definecolor{currentstroke}{rgb}{0.000000,0.000000,0.000000}%
\pgfsetstrokecolor{currentstroke}%
\pgfsetdash{}{0pt}%
\pgfpathmoveto{\pgfqpoint{7.335000in}{0.615000in}}%
\pgfpathlineto{\pgfqpoint{7.335000in}{5.535000in}}%
\pgfusepath{stroke}%
\end{pgfscope}%
\begin{pgfscope}%
\pgfsetrectcap%
\pgfsetmiterjoin%
\pgfsetlinewidth{1.003750pt}%
\definecolor{currentstroke}{rgb}{0.000000,0.000000,0.000000}%
\pgfsetstrokecolor{currentstroke}%
\pgfsetdash{}{0pt}%
\pgfpathmoveto{\pgfqpoint{1.018750in}{0.615000in}}%
\pgfpathlineto{\pgfqpoint{7.335000in}{0.615000in}}%
\pgfusepath{stroke}%
\end{pgfscope}%
\begin{pgfscope}%
\pgfsetrectcap%
\pgfsetmiterjoin%
\pgfsetlinewidth{1.003750pt}%
\definecolor{currentstroke}{rgb}{0.000000,0.000000,0.000000}%
\pgfsetstrokecolor{currentstroke}%
\pgfsetdash{}{0pt}%
\pgfpathmoveto{\pgfqpoint{1.018750in}{0.615000in}}%
\pgfpathlineto{\pgfqpoint{1.018750in}{5.535000in}}%
\pgfusepath{stroke}%
\end{pgfscope}%
\begin{pgfscope}%
\pgfsetbuttcap%
\pgfsetroundjoin%
\definecolor{currentfill}{rgb}{0.000000,0.000000,0.000000}%
\pgfsetfillcolor{currentfill}%
\pgfsetlinewidth{0.501875pt}%
\definecolor{currentstroke}{rgb}{0.000000,0.000000,0.000000}%
\pgfsetstrokecolor{currentstroke}%
\pgfsetdash{}{0pt}%
\pgfsys@defobject{currentmarker}{\pgfqpoint{0.000000in}{0.000000in}}{\pgfqpoint{0.000000in}{0.055556in}}{%
\pgfpathmoveto{\pgfqpoint{0.000000in}{0.000000in}}%
\pgfpathlineto{\pgfqpoint{0.000000in}{0.055556in}}%
\pgfusepath{stroke,fill}%
}%
\begin{pgfscope}%
\pgfsys@transformshift{1.926711in}{0.615000in}%
\pgfsys@useobject{currentmarker}{}%
\end{pgfscope}%
\end{pgfscope}%
\begin{pgfscope}%
\pgfsetbuttcap%
\pgfsetroundjoin%
\definecolor{currentfill}{rgb}{0.000000,0.000000,0.000000}%
\pgfsetfillcolor{currentfill}%
\pgfsetlinewidth{0.501875pt}%
\definecolor{currentstroke}{rgb}{0.000000,0.000000,0.000000}%
\pgfsetstrokecolor{currentstroke}%
\pgfsetdash{}{0pt}%
\pgfsys@defobject{currentmarker}{\pgfqpoint{0.000000in}{-0.055556in}}{\pgfqpoint{0.000000in}{0.000000in}}{%
\pgfpathmoveto{\pgfqpoint{0.000000in}{0.000000in}}%
\pgfpathlineto{\pgfqpoint{0.000000in}{-0.055556in}}%
\pgfusepath{stroke,fill}%
}%
\begin{pgfscope}%
\pgfsys@transformshift{1.926711in}{5.535000in}%
\pgfsys@useobject{currentmarker}{}%
\end{pgfscope}%
\end{pgfscope}%
\begin{pgfscope}%
\pgftext[x=1.926711in,y=0.559444in,,top]{{\sffamily\fontsize{14.000000}{16.800000}\selectfont \textrm{Experiment 1}}}%
\end{pgfscope}%
\begin{pgfscope}%
\pgfsetbuttcap%
\pgfsetroundjoin%
\definecolor{currentfill}{rgb}{0.000000,0.000000,0.000000}%
\pgfsetfillcolor{currentfill}%
\pgfsetlinewidth{0.501875pt}%
\definecolor{currentstroke}{rgb}{0.000000,0.000000,0.000000}%
\pgfsetstrokecolor{currentstroke}%
\pgfsetdash{}{0pt}%
\pgfsys@defobject{currentmarker}{\pgfqpoint{0.000000in}{0.000000in}}{\pgfqpoint{0.000000in}{0.055556in}}{%
\pgfpathmoveto{\pgfqpoint{0.000000in}{0.000000in}}%
\pgfpathlineto{\pgfqpoint{0.000000in}{0.055556in}}%
\pgfusepath{stroke,fill}%
}%
\begin{pgfscope}%
\pgfsys@transformshift{3.505773in}{0.615000in}%
\pgfsys@useobject{currentmarker}{}%
\end{pgfscope}%
\end{pgfscope}%
\begin{pgfscope}%
\pgfsetbuttcap%
\pgfsetroundjoin%
\definecolor{currentfill}{rgb}{0.000000,0.000000,0.000000}%
\pgfsetfillcolor{currentfill}%
\pgfsetlinewidth{0.501875pt}%
\definecolor{currentstroke}{rgb}{0.000000,0.000000,0.000000}%
\pgfsetstrokecolor{currentstroke}%
\pgfsetdash{}{0pt}%
\pgfsys@defobject{currentmarker}{\pgfqpoint{0.000000in}{-0.055556in}}{\pgfqpoint{0.000000in}{0.000000in}}{%
\pgfpathmoveto{\pgfqpoint{0.000000in}{0.000000in}}%
\pgfpathlineto{\pgfqpoint{0.000000in}{-0.055556in}}%
\pgfusepath{stroke,fill}%
}%
\begin{pgfscope}%
\pgfsys@transformshift{3.505773in}{5.535000in}%
\pgfsys@useobject{currentmarker}{}%
\end{pgfscope}%
\end{pgfscope}%
\begin{pgfscope}%
\pgftext[x=3.505773in,y=0.559444in,,top]{{\sffamily\fontsize{14.000000}{16.800000}\selectfont \textrm{Experiment 2}}}%
\end{pgfscope}%
\begin{pgfscope}%
\pgfsetbuttcap%
\pgfsetroundjoin%
\definecolor{currentfill}{rgb}{0.000000,0.000000,0.000000}%
\pgfsetfillcolor{currentfill}%
\pgfsetlinewidth{0.501875pt}%
\definecolor{currentstroke}{rgb}{0.000000,0.000000,0.000000}%
\pgfsetstrokecolor{currentstroke}%
\pgfsetdash{}{0pt}%
\pgfsys@defobject{currentmarker}{\pgfqpoint{0.000000in}{0.000000in}}{\pgfqpoint{0.000000in}{0.055556in}}{%
\pgfpathmoveto{\pgfqpoint{0.000000in}{0.000000in}}%
\pgfpathlineto{\pgfqpoint{0.000000in}{0.055556in}}%
\pgfusepath{stroke,fill}%
}%
\begin{pgfscope}%
\pgfsys@transformshift{5.084836in}{0.615000in}%
\pgfsys@useobject{currentmarker}{}%
\end{pgfscope}%
\end{pgfscope}%
\begin{pgfscope}%
\pgfsetbuttcap%
\pgfsetroundjoin%
\definecolor{currentfill}{rgb}{0.000000,0.000000,0.000000}%
\pgfsetfillcolor{currentfill}%
\pgfsetlinewidth{0.501875pt}%
\definecolor{currentstroke}{rgb}{0.000000,0.000000,0.000000}%
\pgfsetstrokecolor{currentstroke}%
\pgfsetdash{}{0pt}%
\pgfsys@defobject{currentmarker}{\pgfqpoint{0.000000in}{-0.055556in}}{\pgfqpoint{0.000000in}{0.000000in}}{%
\pgfpathmoveto{\pgfqpoint{0.000000in}{0.000000in}}%
\pgfpathlineto{\pgfqpoint{0.000000in}{-0.055556in}}%
\pgfusepath{stroke,fill}%
}%
\begin{pgfscope}%
\pgfsys@transformshift{5.084836in}{5.535000in}%
\pgfsys@useobject{currentmarker}{}%
\end{pgfscope}%
\end{pgfscope}%
\begin{pgfscope}%
\pgftext[x=5.084836in,y=0.559444in,,top]{{\sffamily\fontsize{14.000000}{16.800000}\selectfont \textrm{Experiment 3}}}%
\end{pgfscope}%
\begin{pgfscope}%
\pgfsetbuttcap%
\pgfsetroundjoin%
\definecolor{currentfill}{rgb}{0.000000,0.000000,0.000000}%
\pgfsetfillcolor{currentfill}%
\pgfsetlinewidth{0.501875pt}%
\definecolor{currentstroke}{rgb}{0.000000,0.000000,0.000000}%
\pgfsetstrokecolor{currentstroke}%
\pgfsetdash{}{0pt}%
\pgfsys@defobject{currentmarker}{\pgfqpoint{0.000000in}{0.000000in}}{\pgfqpoint{0.000000in}{0.055556in}}{%
\pgfpathmoveto{\pgfqpoint{0.000000in}{0.000000in}}%
\pgfpathlineto{\pgfqpoint{0.000000in}{0.055556in}}%
\pgfusepath{stroke,fill}%
}%
\begin{pgfscope}%
\pgfsys@transformshift{6.663898in}{0.615000in}%
\pgfsys@useobject{currentmarker}{}%
\end{pgfscope}%
\end{pgfscope}%
\begin{pgfscope}%
\pgfsetbuttcap%
\pgfsetroundjoin%
\definecolor{currentfill}{rgb}{0.000000,0.000000,0.000000}%
\pgfsetfillcolor{currentfill}%
\pgfsetlinewidth{0.501875pt}%
\definecolor{currentstroke}{rgb}{0.000000,0.000000,0.000000}%
\pgfsetstrokecolor{currentstroke}%
\pgfsetdash{}{0pt}%
\pgfsys@defobject{currentmarker}{\pgfqpoint{0.000000in}{-0.055556in}}{\pgfqpoint{0.000000in}{0.000000in}}{%
\pgfpathmoveto{\pgfqpoint{0.000000in}{0.000000in}}%
\pgfpathlineto{\pgfqpoint{0.000000in}{-0.055556in}}%
\pgfusepath{stroke,fill}%
}%
\begin{pgfscope}%
\pgfsys@transformshift{6.663898in}{5.535000in}%
\pgfsys@useobject{currentmarker}{}%
\end{pgfscope}%
\end{pgfscope}%
\begin{pgfscope}%
\pgftext[x=6.663898in,y=0.559444in,,top]{{\sffamily\fontsize{14.000000}{16.800000}\selectfont \textrm{Experiment 4}}}%
\end{pgfscope}%
\begin{pgfscope}%
\pgfsetbuttcap%
\pgfsetroundjoin%
\definecolor{currentfill}{rgb}{0.000000,0.000000,0.000000}%
\pgfsetfillcolor{currentfill}%
\pgfsetlinewidth{0.501875pt}%
\definecolor{currentstroke}{rgb}{0.000000,0.000000,0.000000}%
\pgfsetstrokecolor{currentstroke}%
\pgfsetdash{}{0pt}%
\pgfsys@defobject{currentmarker}{\pgfqpoint{0.000000in}{0.000000in}}{\pgfqpoint{0.055556in}{0.000000in}}{%
\pgfpathmoveto{\pgfqpoint{0.000000in}{0.000000in}}%
\pgfpathlineto{\pgfqpoint{0.055556in}{0.000000in}}%
\pgfusepath{stroke,fill}%
}%
\begin{pgfscope}%
\pgfsys@transformshift{1.018750in}{0.615000in}%
\pgfsys@useobject{currentmarker}{}%
\end{pgfscope}%
\end{pgfscope}%
\begin{pgfscope}%
\pgfsetbuttcap%
\pgfsetroundjoin%
\definecolor{currentfill}{rgb}{0.000000,0.000000,0.000000}%
\pgfsetfillcolor{currentfill}%
\pgfsetlinewidth{0.501875pt}%
\definecolor{currentstroke}{rgb}{0.000000,0.000000,0.000000}%
\pgfsetstrokecolor{currentstroke}%
\pgfsetdash{}{0pt}%
\pgfsys@defobject{currentmarker}{\pgfqpoint{-0.055556in}{0.000000in}}{\pgfqpoint{0.000000in}{0.000000in}}{%
\pgfpathmoveto{\pgfqpoint{0.000000in}{0.000000in}}%
\pgfpathlineto{\pgfqpoint{-0.055556in}{0.000000in}}%
\pgfusepath{stroke,fill}%
}%
\begin{pgfscope}%
\pgfsys@transformshift{7.335000in}{0.615000in}%
\pgfsys@useobject{currentmarker}{}%
\end{pgfscope}%
\end{pgfscope}%
\begin{pgfscope}%
\pgftext[x=0.963194in,y=0.615000in,right,]{{\sffamily\fontsize{14.000000}{16.800000}\selectfont \(\displaystyle 0\%\)}}%
\end{pgfscope}%
\begin{pgfscope}%
\pgfsetbuttcap%
\pgfsetroundjoin%
\definecolor{currentfill}{rgb}{0.000000,0.000000,0.000000}%
\pgfsetfillcolor{currentfill}%
\pgfsetlinewidth{0.501875pt}%
\definecolor{currentstroke}{rgb}{0.000000,0.000000,0.000000}%
\pgfsetstrokecolor{currentstroke}%
\pgfsetdash{}{0pt}%
\pgfsys@defobject{currentmarker}{\pgfqpoint{0.000000in}{0.000000in}}{\pgfqpoint{0.055556in}{0.000000in}}{%
\pgfpathmoveto{\pgfqpoint{0.000000in}{0.000000in}}%
\pgfpathlineto{\pgfqpoint{0.055556in}{0.000000in}}%
\pgfusepath{stroke,fill}%
}%
\begin{pgfscope}%
\pgfsys@transformshift{1.018750in}{1.161667in}%
\pgfsys@useobject{currentmarker}{}%
\end{pgfscope}%
\end{pgfscope}%
\begin{pgfscope}%
\pgfsetbuttcap%
\pgfsetroundjoin%
\definecolor{currentfill}{rgb}{0.000000,0.000000,0.000000}%
\pgfsetfillcolor{currentfill}%
\pgfsetlinewidth{0.501875pt}%
\definecolor{currentstroke}{rgb}{0.000000,0.000000,0.000000}%
\pgfsetstrokecolor{currentstroke}%
\pgfsetdash{}{0pt}%
\pgfsys@defobject{currentmarker}{\pgfqpoint{-0.055556in}{0.000000in}}{\pgfqpoint{0.000000in}{0.000000in}}{%
\pgfpathmoveto{\pgfqpoint{0.000000in}{0.000000in}}%
\pgfpathlineto{\pgfqpoint{-0.055556in}{0.000000in}}%
\pgfusepath{stroke,fill}%
}%
\begin{pgfscope}%
\pgfsys@transformshift{7.335000in}{1.161667in}%
\pgfsys@useobject{currentmarker}{}%
\end{pgfscope}%
\end{pgfscope}%
\begin{pgfscope}%
\pgftext[x=0.963194in,y=1.161667in,right,]{{\sffamily\fontsize{14.000000}{16.800000}\selectfont \(\displaystyle 10\%\)}}%
\end{pgfscope}%
\begin{pgfscope}%
\pgfsetbuttcap%
\pgfsetroundjoin%
\definecolor{currentfill}{rgb}{0.000000,0.000000,0.000000}%
\pgfsetfillcolor{currentfill}%
\pgfsetlinewidth{0.501875pt}%
\definecolor{currentstroke}{rgb}{0.000000,0.000000,0.000000}%
\pgfsetstrokecolor{currentstroke}%
\pgfsetdash{}{0pt}%
\pgfsys@defobject{currentmarker}{\pgfqpoint{0.000000in}{0.000000in}}{\pgfqpoint{0.055556in}{0.000000in}}{%
\pgfpathmoveto{\pgfqpoint{0.000000in}{0.000000in}}%
\pgfpathlineto{\pgfqpoint{0.055556in}{0.000000in}}%
\pgfusepath{stroke,fill}%
}%
\begin{pgfscope}%
\pgfsys@transformshift{1.018750in}{1.708333in}%
\pgfsys@useobject{currentmarker}{}%
\end{pgfscope}%
\end{pgfscope}%
\begin{pgfscope}%
\pgfsetbuttcap%
\pgfsetroundjoin%
\definecolor{currentfill}{rgb}{0.000000,0.000000,0.000000}%
\pgfsetfillcolor{currentfill}%
\pgfsetlinewidth{0.501875pt}%
\definecolor{currentstroke}{rgb}{0.000000,0.000000,0.000000}%
\pgfsetstrokecolor{currentstroke}%
\pgfsetdash{}{0pt}%
\pgfsys@defobject{currentmarker}{\pgfqpoint{-0.055556in}{0.000000in}}{\pgfqpoint{0.000000in}{0.000000in}}{%
\pgfpathmoveto{\pgfqpoint{0.000000in}{0.000000in}}%
\pgfpathlineto{\pgfqpoint{-0.055556in}{0.000000in}}%
\pgfusepath{stroke,fill}%
}%
\begin{pgfscope}%
\pgfsys@transformshift{7.335000in}{1.708333in}%
\pgfsys@useobject{currentmarker}{}%
\end{pgfscope}%
\end{pgfscope}%
\begin{pgfscope}%
\pgftext[x=0.963194in,y=1.708333in,right,]{{\sffamily\fontsize{14.000000}{16.800000}\selectfont \(\displaystyle 20\%\)}}%
\end{pgfscope}%
\begin{pgfscope}%
\pgfsetbuttcap%
\pgfsetroundjoin%
\definecolor{currentfill}{rgb}{0.000000,0.000000,0.000000}%
\pgfsetfillcolor{currentfill}%
\pgfsetlinewidth{0.501875pt}%
\definecolor{currentstroke}{rgb}{0.000000,0.000000,0.000000}%
\pgfsetstrokecolor{currentstroke}%
\pgfsetdash{}{0pt}%
\pgfsys@defobject{currentmarker}{\pgfqpoint{0.000000in}{0.000000in}}{\pgfqpoint{0.055556in}{0.000000in}}{%
\pgfpathmoveto{\pgfqpoint{0.000000in}{0.000000in}}%
\pgfpathlineto{\pgfqpoint{0.055556in}{0.000000in}}%
\pgfusepath{stroke,fill}%
}%
\begin{pgfscope}%
\pgfsys@transformshift{1.018750in}{2.255000in}%
\pgfsys@useobject{currentmarker}{}%
\end{pgfscope}%
\end{pgfscope}%
\begin{pgfscope}%
\pgfsetbuttcap%
\pgfsetroundjoin%
\definecolor{currentfill}{rgb}{0.000000,0.000000,0.000000}%
\pgfsetfillcolor{currentfill}%
\pgfsetlinewidth{0.501875pt}%
\definecolor{currentstroke}{rgb}{0.000000,0.000000,0.000000}%
\pgfsetstrokecolor{currentstroke}%
\pgfsetdash{}{0pt}%
\pgfsys@defobject{currentmarker}{\pgfqpoint{-0.055556in}{0.000000in}}{\pgfqpoint{0.000000in}{0.000000in}}{%
\pgfpathmoveto{\pgfqpoint{0.000000in}{0.000000in}}%
\pgfpathlineto{\pgfqpoint{-0.055556in}{0.000000in}}%
\pgfusepath{stroke,fill}%
}%
\begin{pgfscope}%
\pgfsys@transformshift{7.335000in}{2.255000in}%
\pgfsys@useobject{currentmarker}{}%
\end{pgfscope}%
\end{pgfscope}%
\begin{pgfscope}%
\pgftext[x=0.963194in,y=2.255000in,right,]{{\sffamily\fontsize{14.000000}{16.800000}\selectfont \(\displaystyle 30\%\)}}%
\end{pgfscope}%
\begin{pgfscope}%
\pgfsetbuttcap%
\pgfsetroundjoin%
\definecolor{currentfill}{rgb}{0.000000,0.000000,0.000000}%
\pgfsetfillcolor{currentfill}%
\pgfsetlinewidth{0.501875pt}%
\definecolor{currentstroke}{rgb}{0.000000,0.000000,0.000000}%
\pgfsetstrokecolor{currentstroke}%
\pgfsetdash{}{0pt}%
\pgfsys@defobject{currentmarker}{\pgfqpoint{0.000000in}{0.000000in}}{\pgfqpoint{0.055556in}{0.000000in}}{%
\pgfpathmoveto{\pgfqpoint{0.000000in}{0.000000in}}%
\pgfpathlineto{\pgfqpoint{0.055556in}{0.000000in}}%
\pgfusepath{stroke,fill}%
}%
\begin{pgfscope}%
\pgfsys@transformshift{1.018750in}{2.801667in}%
\pgfsys@useobject{currentmarker}{}%
\end{pgfscope}%
\end{pgfscope}%
\begin{pgfscope}%
\pgfsetbuttcap%
\pgfsetroundjoin%
\definecolor{currentfill}{rgb}{0.000000,0.000000,0.000000}%
\pgfsetfillcolor{currentfill}%
\pgfsetlinewidth{0.501875pt}%
\definecolor{currentstroke}{rgb}{0.000000,0.000000,0.000000}%
\pgfsetstrokecolor{currentstroke}%
\pgfsetdash{}{0pt}%
\pgfsys@defobject{currentmarker}{\pgfqpoint{-0.055556in}{0.000000in}}{\pgfqpoint{0.000000in}{0.000000in}}{%
\pgfpathmoveto{\pgfqpoint{0.000000in}{0.000000in}}%
\pgfpathlineto{\pgfqpoint{-0.055556in}{0.000000in}}%
\pgfusepath{stroke,fill}%
}%
\begin{pgfscope}%
\pgfsys@transformshift{7.335000in}{2.801667in}%
\pgfsys@useobject{currentmarker}{}%
\end{pgfscope}%
\end{pgfscope}%
\begin{pgfscope}%
\pgftext[x=0.963194in,y=2.801667in,right,]{{\sffamily\fontsize{14.000000}{16.800000}\selectfont \(\displaystyle 40\%\)}}%
\end{pgfscope}%
\begin{pgfscope}%
\pgfsetbuttcap%
\pgfsetroundjoin%
\definecolor{currentfill}{rgb}{0.000000,0.000000,0.000000}%
\pgfsetfillcolor{currentfill}%
\pgfsetlinewidth{0.501875pt}%
\definecolor{currentstroke}{rgb}{0.000000,0.000000,0.000000}%
\pgfsetstrokecolor{currentstroke}%
\pgfsetdash{}{0pt}%
\pgfsys@defobject{currentmarker}{\pgfqpoint{0.000000in}{0.000000in}}{\pgfqpoint{0.055556in}{0.000000in}}{%
\pgfpathmoveto{\pgfqpoint{0.000000in}{0.000000in}}%
\pgfpathlineto{\pgfqpoint{0.055556in}{0.000000in}}%
\pgfusepath{stroke,fill}%
}%
\begin{pgfscope}%
\pgfsys@transformshift{1.018750in}{3.348333in}%
\pgfsys@useobject{currentmarker}{}%
\end{pgfscope}%
\end{pgfscope}%
\begin{pgfscope}%
\pgfsetbuttcap%
\pgfsetroundjoin%
\definecolor{currentfill}{rgb}{0.000000,0.000000,0.000000}%
\pgfsetfillcolor{currentfill}%
\pgfsetlinewidth{0.501875pt}%
\definecolor{currentstroke}{rgb}{0.000000,0.000000,0.000000}%
\pgfsetstrokecolor{currentstroke}%
\pgfsetdash{}{0pt}%
\pgfsys@defobject{currentmarker}{\pgfqpoint{-0.055556in}{0.000000in}}{\pgfqpoint{0.000000in}{0.000000in}}{%
\pgfpathmoveto{\pgfqpoint{0.000000in}{0.000000in}}%
\pgfpathlineto{\pgfqpoint{-0.055556in}{0.000000in}}%
\pgfusepath{stroke,fill}%
}%
\begin{pgfscope}%
\pgfsys@transformshift{7.335000in}{3.348333in}%
\pgfsys@useobject{currentmarker}{}%
\end{pgfscope}%
\end{pgfscope}%
\begin{pgfscope}%
\pgftext[x=0.963194in,y=3.348333in,right,]{{\sffamily\fontsize{14.000000}{16.800000}\selectfont \(\displaystyle 50\%\)}}%
\end{pgfscope}%
\begin{pgfscope}%
\pgfsetbuttcap%
\pgfsetroundjoin%
\definecolor{currentfill}{rgb}{0.000000,0.000000,0.000000}%
\pgfsetfillcolor{currentfill}%
\pgfsetlinewidth{0.501875pt}%
\definecolor{currentstroke}{rgb}{0.000000,0.000000,0.000000}%
\pgfsetstrokecolor{currentstroke}%
\pgfsetdash{}{0pt}%
\pgfsys@defobject{currentmarker}{\pgfqpoint{0.000000in}{0.000000in}}{\pgfqpoint{0.055556in}{0.000000in}}{%
\pgfpathmoveto{\pgfqpoint{0.000000in}{0.000000in}}%
\pgfpathlineto{\pgfqpoint{0.055556in}{0.000000in}}%
\pgfusepath{stroke,fill}%
}%
\begin{pgfscope}%
\pgfsys@transformshift{1.018750in}{3.895000in}%
\pgfsys@useobject{currentmarker}{}%
\end{pgfscope}%
\end{pgfscope}%
\begin{pgfscope}%
\pgfsetbuttcap%
\pgfsetroundjoin%
\definecolor{currentfill}{rgb}{0.000000,0.000000,0.000000}%
\pgfsetfillcolor{currentfill}%
\pgfsetlinewidth{0.501875pt}%
\definecolor{currentstroke}{rgb}{0.000000,0.000000,0.000000}%
\pgfsetstrokecolor{currentstroke}%
\pgfsetdash{}{0pt}%
\pgfsys@defobject{currentmarker}{\pgfqpoint{-0.055556in}{0.000000in}}{\pgfqpoint{0.000000in}{0.000000in}}{%
\pgfpathmoveto{\pgfqpoint{0.000000in}{0.000000in}}%
\pgfpathlineto{\pgfqpoint{-0.055556in}{0.000000in}}%
\pgfusepath{stroke,fill}%
}%
\begin{pgfscope}%
\pgfsys@transformshift{7.335000in}{3.895000in}%
\pgfsys@useobject{currentmarker}{}%
\end{pgfscope}%
\end{pgfscope}%
\begin{pgfscope}%
\pgftext[x=0.963194in,y=3.895000in,right,]{{\sffamily\fontsize{14.000000}{16.800000}\selectfont \(\displaystyle 60\%\)}}%
\end{pgfscope}%
\begin{pgfscope}%
\pgfsetbuttcap%
\pgfsetroundjoin%
\definecolor{currentfill}{rgb}{0.000000,0.000000,0.000000}%
\pgfsetfillcolor{currentfill}%
\pgfsetlinewidth{0.501875pt}%
\definecolor{currentstroke}{rgb}{0.000000,0.000000,0.000000}%
\pgfsetstrokecolor{currentstroke}%
\pgfsetdash{}{0pt}%
\pgfsys@defobject{currentmarker}{\pgfqpoint{0.000000in}{0.000000in}}{\pgfqpoint{0.055556in}{0.000000in}}{%
\pgfpathmoveto{\pgfqpoint{0.000000in}{0.000000in}}%
\pgfpathlineto{\pgfqpoint{0.055556in}{0.000000in}}%
\pgfusepath{stroke,fill}%
}%
\begin{pgfscope}%
\pgfsys@transformshift{1.018750in}{4.441667in}%
\pgfsys@useobject{currentmarker}{}%
\end{pgfscope}%
\end{pgfscope}%
\begin{pgfscope}%
\pgfsetbuttcap%
\pgfsetroundjoin%
\definecolor{currentfill}{rgb}{0.000000,0.000000,0.000000}%
\pgfsetfillcolor{currentfill}%
\pgfsetlinewidth{0.501875pt}%
\definecolor{currentstroke}{rgb}{0.000000,0.000000,0.000000}%
\pgfsetstrokecolor{currentstroke}%
\pgfsetdash{}{0pt}%
\pgfsys@defobject{currentmarker}{\pgfqpoint{-0.055556in}{0.000000in}}{\pgfqpoint{0.000000in}{0.000000in}}{%
\pgfpathmoveto{\pgfqpoint{0.000000in}{0.000000in}}%
\pgfpathlineto{\pgfqpoint{-0.055556in}{0.000000in}}%
\pgfusepath{stroke,fill}%
}%
\begin{pgfscope}%
\pgfsys@transformshift{7.335000in}{4.441667in}%
\pgfsys@useobject{currentmarker}{}%
\end{pgfscope}%
\end{pgfscope}%
\begin{pgfscope}%
\pgftext[x=0.963194in,y=4.441667in,right,]{{\sffamily\fontsize{14.000000}{16.800000}\selectfont \(\displaystyle 70\%\)}}%
\end{pgfscope}%
\begin{pgfscope}%
\pgfsetbuttcap%
\pgfsetroundjoin%
\definecolor{currentfill}{rgb}{0.000000,0.000000,0.000000}%
\pgfsetfillcolor{currentfill}%
\pgfsetlinewidth{0.501875pt}%
\definecolor{currentstroke}{rgb}{0.000000,0.000000,0.000000}%
\pgfsetstrokecolor{currentstroke}%
\pgfsetdash{}{0pt}%
\pgfsys@defobject{currentmarker}{\pgfqpoint{0.000000in}{0.000000in}}{\pgfqpoint{0.055556in}{0.000000in}}{%
\pgfpathmoveto{\pgfqpoint{0.000000in}{0.000000in}}%
\pgfpathlineto{\pgfqpoint{0.055556in}{0.000000in}}%
\pgfusepath{stroke,fill}%
}%
\begin{pgfscope}%
\pgfsys@transformshift{1.018750in}{4.988333in}%
\pgfsys@useobject{currentmarker}{}%
\end{pgfscope}%
\end{pgfscope}%
\begin{pgfscope}%
\pgfsetbuttcap%
\pgfsetroundjoin%
\definecolor{currentfill}{rgb}{0.000000,0.000000,0.000000}%
\pgfsetfillcolor{currentfill}%
\pgfsetlinewidth{0.501875pt}%
\definecolor{currentstroke}{rgb}{0.000000,0.000000,0.000000}%
\pgfsetstrokecolor{currentstroke}%
\pgfsetdash{}{0pt}%
\pgfsys@defobject{currentmarker}{\pgfqpoint{-0.055556in}{0.000000in}}{\pgfqpoint{0.000000in}{0.000000in}}{%
\pgfpathmoveto{\pgfqpoint{0.000000in}{0.000000in}}%
\pgfpathlineto{\pgfqpoint{-0.055556in}{0.000000in}}%
\pgfusepath{stroke,fill}%
}%
\begin{pgfscope}%
\pgfsys@transformshift{7.335000in}{4.988333in}%
\pgfsys@useobject{currentmarker}{}%
\end{pgfscope}%
\end{pgfscope}%
\begin{pgfscope}%
\pgftext[x=0.963194in,y=4.988333in,right,]{{\sffamily\fontsize{14.000000}{16.800000}\selectfont \(\displaystyle 80\%\)}}%
\end{pgfscope}%
\begin{pgfscope}%
\pgfsetbuttcap%
\pgfsetroundjoin%
\definecolor{currentfill}{rgb}{0.000000,0.000000,0.000000}%
\pgfsetfillcolor{currentfill}%
\pgfsetlinewidth{0.501875pt}%
\definecolor{currentstroke}{rgb}{0.000000,0.000000,0.000000}%
\pgfsetstrokecolor{currentstroke}%
\pgfsetdash{}{0pt}%
\pgfsys@defobject{currentmarker}{\pgfqpoint{0.000000in}{0.000000in}}{\pgfqpoint{0.055556in}{0.000000in}}{%
\pgfpathmoveto{\pgfqpoint{0.000000in}{0.000000in}}%
\pgfpathlineto{\pgfqpoint{0.055556in}{0.000000in}}%
\pgfusepath{stroke,fill}%
}%
\begin{pgfscope}%
\pgfsys@transformshift{1.018750in}{5.535000in}%
\pgfsys@useobject{currentmarker}{}%
\end{pgfscope}%
\end{pgfscope}%
\begin{pgfscope}%
\pgfsetbuttcap%
\pgfsetroundjoin%
\definecolor{currentfill}{rgb}{0.000000,0.000000,0.000000}%
\pgfsetfillcolor{currentfill}%
\pgfsetlinewidth{0.501875pt}%
\definecolor{currentstroke}{rgb}{0.000000,0.000000,0.000000}%
\pgfsetstrokecolor{currentstroke}%
\pgfsetdash{}{0pt}%
\pgfsys@defobject{currentmarker}{\pgfqpoint{-0.055556in}{0.000000in}}{\pgfqpoint{0.000000in}{0.000000in}}{%
\pgfpathmoveto{\pgfqpoint{0.000000in}{0.000000in}}%
\pgfpathlineto{\pgfqpoint{-0.055556in}{0.000000in}}%
\pgfusepath{stroke,fill}%
}%
\begin{pgfscope}%
\pgfsys@transformshift{7.335000in}{5.535000in}%
\pgfsys@useobject{currentmarker}{}%
\end{pgfscope}%
\end{pgfscope}%
\begin{pgfscope}%
\pgftext[x=0.963194in,y=5.535000in,right,]{{\sffamily\fontsize{14.000000}{16.800000}\selectfont \(\displaystyle 90\%\)}}%
\end{pgfscope}%
\begin{pgfscope}%
\pgftext[x=0.539253in,y=3.075000in,,bottom,rotate=90.000000]{{\sffamily\fontsize{14.000000}{16.800000}\selectfont \textrm{Pairs coordinating using ambiguous form (scaled)}}}%
\end{pgfscope}%
\begin{pgfscope}%
\pgfsetbuttcap%
\pgfsetmiterjoin%
\definecolor{currentfill}{rgb}{1.000000,1.000000,1.000000}%
\pgfsetfillcolor{currentfill}%
\pgfsetlinewidth{1.003750pt}%
\definecolor{currentstroke}{rgb}{0.000000,0.000000,0.000000}%
\pgfsetstrokecolor{currentstroke}%
\pgfsetdash{}{0pt}%
\pgfpathmoveto{\pgfqpoint{1.911014in}{5.081389in}}%
\pgfpathlineto{\pgfqpoint{6.442736in}{5.081389in}}%
\pgfpathlineto{\pgfqpoint{6.442736in}{5.683778in}}%
\pgfpathlineto{\pgfqpoint{1.911014in}{5.683778in}}%
\pgfpathclose%
\pgfusepath{stroke,fill}%
\end{pgfscope}%
\begin{pgfscope}%
\pgfsetbuttcap%
\pgfsetmiterjoin%
\definecolor{currentfill}{rgb}{0.200000,0.200000,0.200000}%
\pgfsetfillcolor{currentfill}%
\pgfsetlinewidth{1.003750pt}%
\definecolor{currentstroke}{rgb}{0.000000,0.000000,0.000000}%
\pgfsetstrokecolor{currentstroke}%
\pgfsetdash{}{0pt}%
\pgfpathmoveto{\pgfqpoint{1.988792in}{5.469889in}}%
\pgfpathlineto{\pgfqpoint{2.377681in}{5.469889in}}%
\pgfpathlineto{\pgfqpoint{2.377681in}{5.606000in}}%
\pgfpathlineto{\pgfqpoint{1.988792in}{5.606000in}}%
\pgfpathclose%
\pgfusepath{stroke,fill}%
\end{pgfscope}%
\begin{pgfscope}%
\pgftext[x=2.533236in,y=5.469889in,left,base]{{\rmfamily\fontsize{14.000000}{16.800000}\selectfont Repair model}}%
\end{pgfscope}%
\begin{pgfscope}%
\pgfsetbuttcap%
\pgfsetmiterjoin%
\definecolor{currentfill}{rgb}{0.400000,0.400000,0.400000}%
\pgfsetfillcolor{currentfill}%
\pgfsetlinewidth{1.003750pt}%
\definecolor{currentstroke}{rgb}{0.000000,0.000000,0.000000}%
\pgfsetstrokecolor{currentstroke}%
\pgfsetdash{}{0pt}%
\pgfpathmoveto{\pgfqpoint{1.988792in}{5.198833in}}%
\pgfpathlineto{\pgfqpoint{2.377681in}{5.198833in}}%
\pgfpathlineto{\pgfqpoint{2.377681in}{5.334945in}}%
\pgfpathlineto{\pgfqpoint{1.988792in}{5.334945in}}%
\pgfpathclose%
\pgfusepath{stroke,fill}%
\end{pgfscope}%
\begin{pgfscope}%
\pgftext[x=2.533236in,y=5.198833in,left,base]{{\rmfamily\fontsize{14.000000}{16.800000}\selectfont Rejection model}}%
\end{pgfscope}%
\begin{pgfscope}%
\pgfsetbuttcap%
\pgfsetmiterjoin%
\definecolor{currentfill}{rgb}{0.600000,0.600000,0.600000}%
\pgfsetfillcolor{currentfill}%
\pgfsetlinewidth{1.003750pt}%
\definecolor{currentstroke}{rgb}{0.000000,0.000000,0.000000}%
\pgfsetstrokecolor{currentstroke}%
\pgfsetdash{}{0pt}%
\pgfpathmoveto{\pgfqpoint{4.268069in}{5.469889in}}%
\pgfpathlineto{\pgfqpoint{4.656958in}{5.469889in}}%
\pgfpathlineto{\pgfqpoint{4.656958in}{5.606000in}}%
\pgfpathlineto{\pgfqpoint{4.268069in}{5.606000in}}%
\pgfpathclose%
\pgfusepath{stroke,fill}%
\end{pgfscope}%
\begin{pgfscope}%
\pgftext[x=4.812514in,y=5.469889in,left,base]{{\rmfamily\fontsize{14.000000}{16.800000}\selectfont Baseline model}}%
\end{pgfscope}%
\begin{pgfscope}%
\pgfsetbuttcap%
\pgfsetmiterjoin%
\definecolor{currentfill}{rgb}{0.800000,0.800000,0.800000}%
\pgfsetfillcolor{currentfill}%
\pgfsetlinewidth{1.003750pt}%
\definecolor{currentstroke}{rgb}{0.000000,0.000000,0.000000}%
\pgfsetstrokecolor{currentstroke}%
\pgfsetdash{}{0pt}%
\pgfpathmoveto{\pgfqpoint{4.268069in}{5.198833in}}%
\pgfpathlineto{\pgfqpoint{4.656958in}{5.198833in}}%
\pgfpathlineto{\pgfqpoint{4.656958in}{5.334945in}}%
\pgfpathlineto{\pgfqpoint{4.268069in}{5.334945in}}%
\pgfpathclose%
\pgfusepath{stroke,fill}%
\end{pgfscope}%
\begin{pgfscope}%
\pgftext[x=4.812514in,y=5.198833in,left,base]{{\rmfamily\fontsize{14.000000}{16.800000}\selectfont Experimental data}}%
\end{pgfscope}%
\end{pgfpicture}%
\makeatother%
\endgroup%
}
\caption{Comparison of pairs coordinating using the ambiguous form for both models against the experimental data}
\label{fig:model_comp}
\end{figure}

To assess how well the two models captured the human behavior on each of \citeauthor{rohde2012}'s cost sets, scaling factors for the optimized and baseline models were derived by minimizing the squared error in ambiguous form coordination rates. The scaling factors found for the optimized and baseline models were $2.19$ and $11.33$ respectively. A comparison of the scaled results from each model to the experimental data is given in Figure \ref{fig:model_comp}. An ANOVA was conducted predicting scaled error rate as a function of model type with each simulation treated as an independent observation. The optimized model showed a significantly lower error rate than the baseline model ($F(1, 1998) = 921.3$, $p<0.001$). This suggests that, while both models demonstrated sensitivity to variations in form costs and neither perfectly replicated the rates of ambiguous coordination observed experimentally, the model using optimized parameters best captured the proportional responses attested in the \citeauthor{rohde2012}\ studies. 

It remains to be seen whether parameters optimized to a PSO model using the same constraint handling technique as the baseline might yield more favorable results.\footnote{An alternative model using this technique in conjunction with the above-presented PSO parameters did not outperform the baseline.} A further question concerns the initial particle positions: In this paper an initial position for each particle was assigned randomly; humans, however, are likely to have prior biases that inform their initial strategies, and this may account for the increased rates of coordination on the ambiguous form observed in the human studies.

\subsection{Implications for audience design}
A notable feature of the PSO model is the simplicity with which agents are represented. Recall that a particle comprises four vectors, $x_i^t$, $v_i^t$, $x_i^*$, and $x_{N(i)}^*$. Apart from these attributes, agents have no form of memory whatsoever. Despite this, referential coordination and entrainment behaviors which mirror those of human interlocutors are possible. This presents an account of referential coordination much more in keeping with more egocentric models of communication \cite{horton1996} than the audience design view \cite{clark1982}, especially given that agents maintain no explicit model of their communicative partners or of the common ground. Instead, referential coordination occurs as a response to previous successes, failures, and incurred production costs, resulting in an implicit representation or reflection of the common ground via the agents' mixed strategies. 


\subsection{Predictions from optimized model}
\subsubsection{Cost and reward variation}

\begin{figure}[t]
\centering
\scalebox{.425}{%% Creator: Matplotlib, PGF backend
%%
%% To include the figure in your LaTeX document, write
%%   \input{<filename>.pgf}
%%
%% Make sure the required packages are loaded in your preamble
%%   \usepackage{pgf}
%%
%% Figures using additional raster images can only be included by \input if
%% they are in the same directory as the main LaTeX file. For loading figures
%% from other directories you can use the `import` package
%%   \usepackage{import}
%% and then include the figures with
%%   \import{<path to file>}{<filename>.pgf}
%%
%% Matplotlib used the following preamble
%%   \usepackage{fontspec}
%%   \setsansfont{DejaVu Sans}
%%   \setmonofont{DejaVu Sans Mono}
%%
\begingroup%
\makeatletter%
\begin{pgfpicture}%
\pgfpathrectangle{\pgfpointorigin}{\pgfqpoint{8.150000in}{6.150000in}}%
\pgfusepath{use as bounding box, clip}%
\begin{pgfscope}%
\pgfsetbuttcap%
\pgfsetmiterjoin%
\definecolor{currentfill}{rgb}{1.000000,1.000000,1.000000}%
\pgfsetfillcolor{currentfill}%
\pgfsetlinewidth{0.000000pt}%
\definecolor{currentstroke}{rgb}{1.000000,1.000000,1.000000}%
\pgfsetstrokecolor{currentstroke}%
\pgfsetdash{}{0pt}%
\pgfpathmoveto{\pgfqpoint{0.000000in}{0.000000in}}%
\pgfpathlineto{\pgfqpoint{8.150000in}{0.000000in}}%
\pgfpathlineto{\pgfqpoint{8.150000in}{6.150000in}}%
\pgfpathlineto{\pgfqpoint{0.000000in}{6.150000in}}%
\pgfpathclose%
\pgfusepath{fill}%
\end{pgfscope}%
\begin{pgfscope}%
\pgfsetbuttcap%
\pgfsetmiterjoin%
\definecolor{currentfill}{rgb}{1.000000,1.000000,1.000000}%
\pgfsetfillcolor{currentfill}%
\pgfsetlinewidth{0.000000pt}%
\definecolor{currentstroke}{rgb}{0.000000,0.000000,0.000000}%
\pgfsetstrokecolor{currentstroke}%
\pgfsetstrokeopacity{0.000000}%
\pgfsetdash{}{0pt}%
\pgfpathmoveto{\pgfqpoint{1.018750in}{0.615000in}}%
\pgfpathlineto{\pgfqpoint{7.335000in}{0.615000in}}%
\pgfpathlineto{\pgfqpoint{7.335000in}{5.535000in}}%
\pgfpathlineto{\pgfqpoint{1.018750in}{5.535000in}}%
\pgfpathclose%
\pgfusepath{fill}%
\end{pgfscope}%
\begin{pgfscope}%
\pgfpathrectangle{\pgfqpoint{1.018750in}{0.615000in}}{\pgfqpoint{6.316250in}{4.920000in}} %
\pgfusepath{clip}%
\pgfsetrectcap%
\pgfsetroundjoin%
\pgfsetlinewidth{1.003750pt}%
\definecolor{currentstroke}{rgb}{0.000000,0.000000,1.000000}%
\pgfsetstrokecolor{currentstroke}%
\pgfsetdash{}{0pt}%
\pgfpathmoveto{\pgfqpoint{1.018750in}{1.676209in}}%
\pgfpathlineto{\pgfqpoint{1.305852in}{2.029945in}}%
\pgfpathlineto{\pgfqpoint{1.592955in}{2.172302in}}%
\pgfpathlineto{\pgfqpoint{1.880057in}{2.357798in}}%
\pgfpathlineto{\pgfqpoint{2.167159in}{2.633885in}}%
\pgfpathlineto{\pgfqpoint{2.454261in}{3.030759in}}%
\pgfpathlineto{\pgfqpoint{2.741364in}{3.276649in}}%
\pgfpathlineto{\pgfqpoint{3.028466in}{3.557050in}}%
\pgfpathlineto{\pgfqpoint{3.315568in}{3.526853in}}%
\pgfpathlineto{\pgfqpoint{3.602670in}{3.975494in}}%
\pgfpathlineto{\pgfqpoint{3.889773in}{4.053143in}}%
\pgfpathlineto{\pgfqpoint{4.176875in}{4.178245in}}%
\pgfpathlineto{\pgfqpoint{4.463977in}{4.320602in}}%
\pgfpathlineto{\pgfqpoint{4.751080in}{4.389624in}}%
\pgfpathlineto{\pgfqpoint{5.038182in}{4.372369in}}%
\pgfpathlineto{\pgfqpoint{5.325284in}{4.337858in}}%
\pgfpathlineto{\pgfqpoint{5.612386in}{4.329230in}}%
\pgfpathlineto{\pgfqpoint{5.899489in}{4.406879in}}%
\pgfpathlineto{\pgfqpoint{6.186591in}{4.398252in}}%
\pgfpathlineto{\pgfqpoint{6.473693in}{4.506098in}}%
\pgfpathlineto{\pgfqpoint{6.760795in}{4.225697in}}%
\pgfpathlineto{\pgfqpoint{7.047898in}{4.329230in}}%
\pgfpathlineto{\pgfqpoint{7.335000in}{4.156676in}}%
\pgfusepath{stroke}%
\end{pgfscope}%
\begin{pgfscope}%
\pgfpathrectangle{\pgfqpoint{1.018750in}{0.615000in}}{\pgfqpoint{6.316250in}{4.920000in}} %
\pgfusepath{clip}%
\pgfsetbuttcap%
\pgfsetroundjoin%
\pgfsetlinewidth{1.003750pt}%
\definecolor{currentstroke}{rgb}{0.000000,0.500000,0.000000}%
\pgfsetstrokecolor{currentstroke}%
\pgfsetdash{{6.000000pt}{6.000000pt}}{0.000000pt}%
\pgfpathmoveto{\pgfqpoint{1.018750in}{0.636569in}}%
\pgfpathlineto{\pgfqpoint{1.305852in}{0.649511in}}%
\pgfpathlineto{\pgfqpoint{1.592955in}{0.645197in}}%
\pgfpathlineto{\pgfqpoint{1.880057in}{0.645197in}}%
\pgfpathlineto{\pgfqpoint{2.167159in}{0.748730in}}%
\pgfpathlineto{\pgfqpoint{2.454261in}{0.809124in}}%
\pgfpathlineto{\pgfqpoint{2.741364in}{0.951481in}}%
\pgfpathlineto{\pgfqpoint{3.028466in}{1.180115in}}%
\pgfpathlineto{\pgfqpoint{3.315568in}{1.568362in}}%
\pgfpathlineto{\pgfqpoint{3.602670in}{1.607187in}}%
\pgfpathlineto{\pgfqpoint{3.889773in}{1.697778in}}%
\pgfpathlineto{\pgfqpoint{4.176875in}{1.827194in}}%
\pgfpathlineto{\pgfqpoint{4.463977in}{1.861704in}}%
\pgfpathlineto{\pgfqpoint{4.751080in}{1.814252in}}%
\pgfpathlineto{\pgfqpoint{5.038182in}{2.038572in}}%
\pgfpathlineto{\pgfqpoint{5.325284in}{2.021317in}}%
\pgfpathlineto{\pgfqpoint{5.612386in}{2.047200in}}%
\pgfpathlineto{\pgfqpoint{5.899489in}{2.124850in}}%
\pgfpathlineto{\pgfqpoint{6.186591in}{2.124850in}}%
\pgfpathlineto{\pgfqpoint{6.473693in}{2.142105in}}%
\pgfpathlineto{\pgfqpoint{6.760795in}{2.323287in}}%
\pgfpathlineto{\pgfqpoint{7.047898in}{2.323287in}}%
\pgfpathlineto{\pgfqpoint{7.335000in}{2.159360in}}%
\pgfusepath{stroke}%
\end{pgfscope}%
\begin{pgfscope}%
\pgfpathrectangle{\pgfqpoint{1.018750in}{0.615000in}}{\pgfqpoint{6.316250in}{4.920000in}} %
\pgfusepath{clip}%
\pgfsetrectcap%
\pgfsetroundjoin%
\pgfsetlinewidth{1.003750pt}%
\definecolor{currentstroke}{rgb}{0.000000,0.000000,0.000000}%
\pgfsetstrokecolor{currentstroke}%
\pgfsetdash{}{0pt}%
\pgfpathmoveto{\pgfqpoint{2.167159in}{0.615000in}}%
\pgfpathlineto{\pgfqpoint{2.167159in}{5.535000in}}%
\pgfusepath{stroke}%
\end{pgfscope}%
\begin{pgfscope}%
\pgfpathrectangle{\pgfqpoint{1.018750in}{0.615000in}}{\pgfqpoint{6.316250in}{4.920000in}} %
\pgfusepath{clip}%
\pgfsetrectcap%
\pgfsetroundjoin%
\pgfsetlinewidth{1.003750pt}%
\definecolor{currentstroke}{rgb}{0.000000,0.000000,0.000000}%
\pgfsetstrokecolor{currentstroke}%
\pgfsetdash{}{0pt}%
\pgfpathmoveto{\pgfqpoint{3.028466in}{0.615000in}}%
\pgfpathlineto{\pgfqpoint{3.028466in}{5.535000in}}%
\pgfusepath{stroke}%
\end{pgfscope}%
\begin{pgfscope}%
\pgfpathrectangle{\pgfqpoint{1.018750in}{0.615000in}}{\pgfqpoint{6.316250in}{4.920000in}} %
\pgfusepath{clip}%
\pgfsetrectcap%
\pgfsetroundjoin%
\pgfsetlinewidth{1.003750pt}%
\definecolor{currentstroke}{rgb}{0.000000,0.000000,0.000000}%
\pgfsetstrokecolor{currentstroke}%
\pgfsetdash{}{0pt}%
\pgfpathmoveto{\pgfqpoint{5.325284in}{0.615000in}}%
\pgfpathlineto{\pgfqpoint{5.325284in}{5.535000in}}%
\pgfusepath{stroke}%
\end{pgfscope}%
\begin{pgfscope}%
\pgfsetrectcap%
\pgfsetmiterjoin%
\pgfsetlinewidth{1.003750pt}%
\definecolor{currentstroke}{rgb}{0.000000,0.000000,0.000000}%
\pgfsetstrokecolor{currentstroke}%
\pgfsetdash{}{0pt}%
\pgfpathmoveto{\pgfqpoint{1.018750in}{5.535000in}}%
\pgfpathlineto{\pgfqpoint{7.335000in}{5.535000in}}%
\pgfusepath{stroke}%
\end{pgfscope}%
\begin{pgfscope}%
\pgfsetrectcap%
\pgfsetmiterjoin%
\pgfsetlinewidth{1.003750pt}%
\definecolor{currentstroke}{rgb}{0.000000,0.000000,0.000000}%
\pgfsetstrokecolor{currentstroke}%
\pgfsetdash{}{0pt}%
\pgfpathmoveto{\pgfqpoint{7.335000in}{0.615000in}}%
\pgfpathlineto{\pgfqpoint{7.335000in}{5.535000in}}%
\pgfusepath{stroke}%
\end{pgfscope}%
\begin{pgfscope}%
\pgfsetrectcap%
\pgfsetmiterjoin%
\pgfsetlinewidth{1.003750pt}%
\definecolor{currentstroke}{rgb}{0.000000,0.000000,0.000000}%
\pgfsetstrokecolor{currentstroke}%
\pgfsetdash{}{0pt}%
\pgfpathmoveto{\pgfqpoint{1.018750in}{0.615000in}}%
\pgfpathlineto{\pgfqpoint{7.335000in}{0.615000in}}%
\pgfusepath{stroke}%
\end{pgfscope}%
\begin{pgfscope}%
\pgfsetrectcap%
\pgfsetmiterjoin%
\pgfsetlinewidth{1.003750pt}%
\definecolor{currentstroke}{rgb}{0.000000,0.000000,0.000000}%
\pgfsetstrokecolor{currentstroke}%
\pgfsetdash{}{0pt}%
\pgfpathmoveto{\pgfqpoint{1.018750in}{0.615000in}}%
\pgfpathlineto{\pgfqpoint{1.018750in}{5.535000in}}%
\pgfusepath{stroke}%
\end{pgfscope}%
\begin{pgfscope}%
\pgfsetbuttcap%
\pgfsetroundjoin%
\definecolor{currentfill}{rgb}{0.000000,0.000000,0.000000}%
\pgfsetfillcolor{currentfill}%
\pgfsetlinewidth{0.501875pt}%
\definecolor{currentstroke}{rgb}{0.000000,0.000000,0.000000}%
\pgfsetstrokecolor{currentstroke}%
\pgfsetdash{}{0pt}%
\pgfsys@defobject{currentmarker}{\pgfqpoint{0.000000in}{0.000000in}}{\pgfqpoint{0.000000in}{0.055556in}}{%
\pgfpathmoveto{\pgfqpoint{0.000000in}{0.000000in}}%
\pgfpathlineto{\pgfqpoint{0.000000in}{0.055556in}}%
\pgfusepath{stroke,fill}%
}%
\begin{pgfscope}%
\pgfsys@transformshift{1.018750in}{0.615000in}%
\pgfsys@useobject{currentmarker}{}%
\end{pgfscope}%
\end{pgfscope}%
\begin{pgfscope}%
\pgfsetbuttcap%
\pgfsetroundjoin%
\definecolor{currentfill}{rgb}{0.000000,0.000000,0.000000}%
\pgfsetfillcolor{currentfill}%
\pgfsetlinewidth{0.501875pt}%
\definecolor{currentstroke}{rgb}{0.000000,0.000000,0.000000}%
\pgfsetstrokecolor{currentstroke}%
\pgfsetdash{}{0pt}%
\pgfsys@defobject{currentmarker}{\pgfqpoint{0.000000in}{-0.055556in}}{\pgfqpoint{0.000000in}{0.000000in}}{%
\pgfpathmoveto{\pgfqpoint{0.000000in}{0.000000in}}%
\pgfpathlineto{\pgfqpoint{0.000000in}{-0.055556in}}%
\pgfusepath{stroke,fill}%
}%
\begin{pgfscope}%
\pgfsys@transformshift{1.018750in}{5.535000in}%
\pgfsys@useobject{currentmarker}{}%
\end{pgfscope}%
\end{pgfscope}%
\begin{pgfscope}%
\pgftext[x=1.018750in,y=0.559444in,,top]{{\sffamily\fontsize{14.000000}{16.800000}\selectfont \(\displaystyle -20\)}}%
\end{pgfscope}%
\begin{pgfscope}%
\pgfsetbuttcap%
\pgfsetroundjoin%
\definecolor{currentfill}{rgb}{0.000000,0.000000,0.000000}%
\pgfsetfillcolor{currentfill}%
\pgfsetlinewidth{0.501875pt}%
\definecolor{currentstroke}{rgb}{0.000000,0.000000,0.000000}%
\pgfsetstrokecolor{currentstroke}%
\pgfsetdash{}{0pt}%
\pgfsys@defobject{currentmarker}{\pgfqpoint{0.000000in}{0.000000in}}{\pgfqpoint{0.000000in}{0.055556in}}{%
\pgfpathmoveto{\pgfqpoint{0.000000in}{0.000000in}}%
\pgfpathlineto{\pgfqpoint{0.000000in}{0.055556in}}%
\pgfusepath{stroke,fill}%
}%
\begin{pgfscope}%
\pgfsys@transformshift{2.454261in}{0.615000in}%
\pgfsys@useobject{currentmarker}{}%
\end{pgfscope}%
\end{pgfscope}%
\begin{pgfscope}%
\pgfsetbuttcap%
\pgfsetroundjoin%
\definecolor{currentfill}{rgb}{0.000000,0.000000,0.000000}%
\pgfsetfillcolor{currentfill}%
\pgfsetlinewidth{0.501875pt}%
\definecolor{currentstroke}{rgb}{0.000000,0.000000,0.000000}%
\pgfsetstrokecolor{currentstroke}%
\pgfsetdash{}{0pt}%
\pgfsys@defobject{currentmarker}{\pgfqpoint{0.000000in}{-0.055556in}}{\pgfqpoint{0.000000in}{0.000000in}}{%
\pgfpathmoveto{\pgfqpoint{0.000000in}{0.000000in}}%
\pgfpathlineto{\pgfqpoint{0.000000in}{-0.055556in}}%
\pgfusepath{stroke,fill}%
}%
\begin{pgfscope}%
\pgfsys@transformshift{2.454261in}{5.535000in}%
\pgfsys@useobject{currentmarker}{}%
\end{pgfscope}%
\end{pgfscope}%
\begin{pgfscope}%
\pgftext[x=2.454261in,y=0.559444in,,top]{{\sffamily\fontsize{14.000000}{16.800000}\selectfont \(\displaystyle 80\)}}%
\end{pgfscope}%
\begin{pgfscope}%
\pgfsetbuttcap%
\pgfsetroundjoin%
\definecolor{currentfill}{rgb}{0.000000,0.000000,0.000000}%
\pgfsetfillcolor{currentfill}%
\pgfsetlinewidth{0.501875pt}%
\definecolor{currentstroke}{rgb}{0.000000,0.000000,0.000000}%
\pgfsetstrokecolor{currentstroke}%
\pgfsetdash{}{0pt}%
\pgfsys@defobject{currentmarker}{\pgfqpoint{0.000000in}{0.000000in}}{\pgfqpoint{0.000000in}{0.055556in}}{%
\pgfpathmoveto{\pgfqpoint{0.000000in}{0.000000in}}%
\pgfpathlineto{\pgfqpoint{0.000000in}{0.055556in}}%
\pgfusepath{stroke,fill}%
}%
\begin{pgfscope}%
\pgfsys@transformshift{3.889773in}{0.615000in}%
\pgfsys@useobject{currentmarker}{}%
\end{pgfscope}%
\end{pgfscope}%
\begin{pgfscope}%
\pgfsetbuttcap%
\pgfsetroundjoin%
\definecolor{currentfill}{rgb}{0.000000,0.000000,0.000000}%
\pgfsetfillcolor{currentfill}%
\pgfsetlinewidth{0.501875pt}%
\definecolor{currentstroke}{rgb}{0.000000,0.000000,0.000000}%
\pgfsetstrokecolor{currentstroke}%
\pgfsetdash{}{0pt}%
\pgfsys@defobject{currentmarker}{\pgfqpoint{0.000000in}{-0.055556in}}{\pgfqpoint{0.000000in}{0.000000in}}{%
\pgfpathmoveto{\pgfqpoint{0.000000in}{0.000000in}}%
\pgfpathlineto{\pgfqpoint{0.000000in}{-0.055556in}}%
\pgfusepath{stroke,fill}%
}%
\begin{pgfscope}%
\pgfsys@transformshift{3.889773in}{5.535000in}%
\pgfsys@useobject{currentmarker}{}%
\end{pgfscope}%
\end{pgfscope}%
\begin{pgfscope}%
\pgftext[x=3.889773in,y=0.559444in,,top]{{\sffamily\fontsize{14.000000}{16.800000}\selectfont \(\displaystyle 180\)}}%
\end{pgfscope}%
\begin{pgfscope}%
\pgfsetbuttcap%
\pgfsetroundjoin%
\definecolor{currentfill}{rgb}{0.000000,0.000000,0.000000}%
\pgfsetfillcolor{currentfill}%
\pgfsetlinewidth{0.501875pt}%
\definecolor{currentstroke}{rgb}{0.000000,0.000000,0.000000}%
\pgfsetstrokecolor{currentstroke}%
\pgfsetdash{}{0pt}%
\pgfsys@defobject{currentmarker}{\pgfqpoint{0.000000in}{0.000000in}}{\pgfqpoint{0.000000in}{0.055556in}}{%
\pgfpathmoveto{\pgfqpoint{0.000000in}{0.000000in}}%
\pgfpathlineto{\pgfqpoint{0.000000in}{0.055556in}}%
\pgfusepath{stroke,fill}%
}%
\begin{pgfscope}%
\pgfsys@transformshift{5.325284in}{0.615000in}%
\pgfsys@useobject{currentmarker}{}%
\end{pgfscope}%
\end{pgfscope}%
\begin{pgfscope}%
\pgfsetbuttcap%
\pgfsetroundjoin%
\definecolor{currentfill}{rgb}{0.000000,0.000000,0.000000}%
\pgfsetfillcolor{currentfill}%
\pgfsetlinewidth{0.501875pt}%
\definecolor{currentstroke}{rgb}{0.000000,0.000000,0.000000}%
\pgfsetstrokecolor{currentstroke}%
\pgfsetdash{}{0pt}%
\pgfsys@defobject{currentmarker}{\pgfqpoint{0.000000in}{-0.055556in}}{\pgfqpoint{0.000000in}{0.000000in}}{%
\pgfpathmoveto{\pgfqpoint{0.000000in}{0.000000in}}%
\pgfpathlineto{\pgfqpoint{0.000000in}{-0.055556in}}%
\pgfusepath{stroke,fill}%
}%
\begin{pgfscope}%
\pgfsys@transformshift{5.325284in}{5.535000in}%
\pgfsys@useobject{currentmarker}{}%
\end{pgfscope}%
\end{pgfscope}%
\begin{pgfscope}%
\pgftext[x=5.325284in,y=0.559444in,,top]{{\sffamily\fontsize{14.000000}{16.800000}\selectfont \(\displaystyle 280\)}}%
\end{pgfscope}%
\begin{pgfscope}%
\pgfsetbuttcap%
\pgfsetroundjoin%
\definecolor{currentfill}{rgb}{0.000000,0.000000,0.000000}%
\pgfsetfillcolor{currentfill}%
\pgfsetlinewidth{0.501875pt}%
\definecolor{currentstroke}{rgb}{0.000000,0.000000,0.000000}%
\pgfsetstrokecolor{currentstroke}%
\pgfsetdash{}{0pt}%
\pgfsys@defobject{currentmarker}{\pgfqpoint{0.000000in}{0.000000in}}{\pgfqpoint{0.000000in}{0.055556in}}{%
\pgfpathmoveto{\pgfqpoint{0.000000in}{0.000000in}}%
\pgfpathlineto{\pgfqpoint{0.000000in}{0.055556in}}%
\pgfusepath{stroke,fill}%
}%
\begin{pgfscope}%
\pgfsys@transformshift{6.760795in}{0.615000in}%
\pgfsys@useobject{currentmarker}{}%
\end{pgfscope}%
\end{pgfscope}%
\begin{pgfscope}%
\pgfsetbuttcap%
\pgfsetroundjoin%
\definecolor{currentfill}{rgb}{0.000000,0.000000,0.000000}%
\pgfsetfillcolor{currentfill}%
\pgfsetlinewidth{0.501875pt}%
\definecolor{currentstroke}{rgb}{0.000000,0.000000,0.000000}%
\pgfsetstrokecolor{currentstroke}%
\pgfsetdash{}{0pt}%
\pgfsys@defobject{currentmarker}{\pgfqpoint{0.000000in}{-0.055556in}}{\pgfqpoint{0.000000in}{0.000000in}}{%
\pgfpathmoveto{\pgfqpoint{0.000000in}{0.000000in}}%
\pgfpathlineto{\pgfqpoint{0.000000in}{-0.055556in}}%
\pgfusepath{stroke,fill}%
}%
\begin{pgfscope}%
\pgfsys@transformshift{6.760795in}{5.535000in}%
\pgfsys@useobject{currentmarker}{}%
\end{pgfscope}%
\end{pgfscope}%
\begin{pgfscope}%
\pgftext[x=6.760795in,y=0.559444in,,top]{{\sffamily\fontsize{14.000000}{16.800000}\selectfont \(\displaystyle 380\)}}%
\end{pgfscope}%
\begin{pgfscope}%
\pgftext[x=4.176875in,y=0.301822in,,top]{{\sffamily\fontsize{14.000000}{16.800000}\selectfont \textrm{Ambiguous form cost}}}%
\end{pgfscope}%
\begin{pgfscope}%
\pgfsetbuttcap%
\pgfsetroundjoin%
\definecolor{currentfill}{rgb}{0.000000,0.000000,0.000000}%
\pgfsetfillcolor{currentfill}%
\pgfsetlinewidth{0.501875pt}%
\definecolor{currentstroke}{rgb}{0.000000,0.000000,0.000000}%
\pgfsetstrokecolor{currentstroke}%
\pgfsetdash{}{0pt}%
\pgfsys@defobject{currentmarker}{\pgfqpoint{0.000000in}{0.000000in}}{\pgfqpoint{0.055556in}{0.000000in}}{%
\pgfpathmoveto{\pgfqpoint{0.000000in}{0.000000in}}%
\pgfpathlineto{\pgfqpoint{0.055556in}{0.000000in}}%
\pgfusepath{stroke,fill}%
}%
\begin{pgfscope}%
\pgfsys@transformshift{1.018750in}{0.615000in}%
\pgfsys@useobject{currentmarker}{}%
\end{pgfscope}%
\end{pgfscope}%
\begin{pgfscope}%
\pgfsetbuttcap%
\pgfsetroundjoin%
\definecolor{currentfill}{rgb}{0.000000,0.000000,0.000000}%
\pgfsetfillcolor{currentfill}%
\pgfsetlinewidth{0.501875pt}%
\definecolor{currentstroke}{rgb}{0.000000,0.000000,0.000000}%
\pgfsetstrokecolor{currentstroke}%
\pgfsetdash{}{0pt}%
\pgfsys@defobject{currentmarker}{\pgfqpoint{-0.055556in}{0.000000in}}{\pgfqpoint{0.000000in}{0.000000in}}{%
\pgfpathmoveto{\pgfqpoint{0.000000in}{0.000000in}}%
\pgfpathlineto{\pgfqpoint{-0.055556in}{0.000000in}}%
\pgfusepath{stroke,fill}%
}%
\begin{pgfscope}%
\pgfsys@transformshift{7.335000in}{0.615000in}%
\pgfsys@useobject{currentmarker}{}%
\end{pgfscope}%
\end{pgfscope}%
\begin{pgfscope}%
\pgftext[x=0.963194in,y=0.615000in,right,]{{\sffamily\fontsize{14.000000}{16.800000}\selectfont \(\displaystyle 0\%\)}}%
\end{pgfscope}%
\begin{pgfscope}%
\pgfsetbuttcap%
\pgfsetroundjoin%
\definecolor{currentfill}{rgb}{0.000000,0.000000,0.000000}%
\pgfsetfillcolor{currentfill}%
\pgfsetlinewidth{0.501875pt}%
\definecolor{currentstroke}{rgb}{0.000000,0.000000,0.000000}%
\pgfsetstrokecolor{currentstroke}%
\pgfsetdash{}{0pt}%
\pgfsys@defobject{currentmarker}{\pgfqpoint{0.000000in}{0.000000in}}{\pgfqpoint{0.055556in}{0.000000in}}{%
\pgfpathmoveto{\pgfqpoint{0.000000in}{0.000000in}}%
\pgfpathlineto{\pgfqpoint{0.055556in}{0.000000in}}%
\pgfusepath{stroke,fill}%
}%
\begin{pgfscope}%
\pgfsys@transformshift{1.018750in}{1.599000in}%
\pgfsys@useobject{currentmarker}{}%
\end{pgfscope}%
\end{pgfscope}%
\begin{pgfscope}%
\pgfsetbuttcap%
\pgfsetroundjoin%
\definecolor{currentfill}{rgb}{0.000000,0.000000,0.000000}%
\pgfsetfillcolor{currentfill}%
\pgfsetlinewidth{0.501875pt}%
\definecolor{currentstroke}{rgb}{0.000000,0.000000,0.000000}%
\pgfsetstrokecolor{currentstroke}%
\pgfsetdash{}{0pt}%
\pgfsys@defobject{currentmarker}{\pgfqpoint{-0.055556in}{0.000000in}}{\pgfqpoint{0.000000in}{0.000000in}}{%
\pgfpathmoveto{\pgfqpoint{0.000000in}{0.000000in}}%
\pgfpathlineto{\pgfqpoint{-0.055556in}{0.000000in}}%
\pgfusepath{stroke,fill}%
}%
\begin{pgfscope}%
\pgfsys@transformshift{7.335000in}{1.599000in}%
\pgfsys@useobject{currentmarker}{}%
\end{pgfscope}%
\end{pgfscope}%
\begin{pgfscope}%
\pgftext[x=0.963194in,y=1.599000in,right,]{{\sffamily\fontsize{14.000000}{16.800000}\selectfont \(\displaystyle 20\%\)}}%
\end{pgfscope}%
\begin{pgfscope}%
\pgfsetbuttcap%
\pgfsetroundjoin%
\definecolor{currentfill}{rgb}{0.000000,0.000000,0.000000}%
\pgfsetfillcolor{currentfill}%
\pgfsetlinewidth{0.501875pt}%
\definecolor{currentstroke}{rgb}{0.000000,0.000000,0.000000}%
\pgfsetstrokecolor{currentstroke}%
\pgfsetdash{}{0pt}%
\pgfsys@defobject{currentmarker}{\pgfqpoint{0.000000in}{0.000000in}}{\pgfqpoint{0.055556in}{0.000000in}}{%
\pgfpathmoveto{\pgfqpoint{0.000000in}{0.000000in}}%
\pgfpathlineto{\pgfqpoint{0.055556in}{0.000000in}}%
\pgfusepath{stroke,fill}%
}%
\begin{pgfscope}%
\pgfsys@transformshift{1.018750in}{2.583000in}%
\pgfsys@useobject{currentmarker}{}%
\end{pgfscope}%
\end{pgfscope}%
\begin{pgfscope}%
\pgfsetbuttcap%
\pgfsetroundjoin%
\definecolor{currentfill}{rgb}{0.000000,0.000000,0.000000}%
\pgfsetfillcolor{currentfill}%
\pgfsetlinewidth{0.501875pt}%
\definecolor{currentstroke}{rgb}{0.000000,0.000000,0.000000}%
\pgfsetstrokecolor{currentstroke}%
\pgfsetdash{}{0pt}%
\pgfsys@defobject{currentmarker}{\pgfqpoint{-0.055556in}{0.000000in}}{\pgfqpoint{0.000000in}{0.000000in}}{%
\pgfpathmoveto{\pgfqpoint{0.000000in}{0.000000in}}%
\pgfpathlineto{\pgfqpoint{-0.055556in}{0.000000in}}%
\pgfusepath{stroke,fill}%
}%
\begin{pgfscope}%
\pgfsys@transformshift{7.335000in}{2.583000in}%
\pgfsys@useobject{currentmarker}{}%
\end{pgfscope}%
\end{pgfscope}%
\begin{pgfscope}%
\pgftext[x=0.963194in,y=2.583000in,right,]{{\sffamily\fontsize{14.000000}{16.800000}\selectfont \(\displaystyle 40\%\)}}%
\end{pgfscope}%
\begin{pgfscope}%
\pgfsetbuttcap%
\pgfsetroundjoin%
\definecolor{currentfill}{rgb}{0.000000,0.000000,0.000000}%
\pgfsetfillcolor{currentfill}%
\pgfsetlinewidth{0.501875pt}%
\definecolor{currentstroke}{rgb}{0.000000,0.000000,0.000000}%
\pgfsetstrokecolor{currentstroke}%
\pgfsetdash{}{0pt}%
\pgfsys@defobject{currentmarker}{\pgfqpoint{0.000000in}{0.000000in}}{\pgfqpoint{0.055556in}{0.000000in}}{%
\pgfpathmoveto{\pgfqpoint{0.000000in}{0.000000in}}%
\pgfpathlineto{\pgfqpoint{0.055556in}{0.000000in}}%
\pgfusepath{stroke,fill}%
}%
\begin{pgfscope}%
\pgfsys@transformshift{1.018750in}{3.567000in}%
\pgfsys@useobject{currentmarker}{}%
\end{pgfscope}%
\end{pgfscope}%
\begin{pgfscope}%
\pgfsetbuttcap%
\pgfsetroundjoin%
\definecolor{currentfill}{rgb}{0.000000,0.000000,0.000000}%
\pgfsetfillcolor{currentfill}%
\pgfsetlinewidth{0.501875pt}%
\definecolor{currentstroke}{rgb}{0.000000,0.000000,0.000000}%
\pgfsetstrokecolor{currentstroke}%
\pgfsetdash{}{0pt}%
\pgfsys@defobject{currentmarker}{\pgfqpoint{-0.055556in}{0.000000in}}{\pgfqpoint{0.000000in}{0.000000in}}{%
\pgfpathmoveto{\pgfqpoint{0.000000in}{0.000000in}}%
\pgfpathlineto{\pgfqpoint{-0.055556in}{0.000000in}}%
\pgfusepath{stroke,fill}%
}%
\begin{pgfscope}%
\pgfsys@transformshift{7.335000in}{3.567000in}%
\pgfsys@useobject{currentmarker}{}%
\end{pgfscope}%
\end{pgfscope}%
\begin{pgfscope}%
\pgftext[x=0.963194in,y=3.567000in,right,]{{\sffamily\fontsize{14.000000}{16.800000}\selectfont \(\displaystyle 60\%\)}}%
\end{pgfscope}%
\begin{pgfscope}%
\pgfsetbuttcap%
\pgfsetroundjoin%
\definecolor{currentfill}{rgb}{0.000000,0.000000,0.000000}%
\pgfsetfillcolor{currentfill}%
\pgfsetlinewidth{0.501875pt}%
\definecolor{currentstroke}{rgb}{0.000000,0.000000,0.000000}%
\pgfsetstrokecolor{currentstroke}%
\pgfsetdash{}{0pt}%
\pgfsys@defobject{currentmarker}{\pgfqpoint{0.000000in}{0.000000in}}{\pgfqpoint{0.055556in}{0.000000in}}{%
\pgfpathmoveto{\pgfqpoint{0.000000in}{0.000000in}}%
\pgfpathlineto{\pgfqpoint{0.055556in}{0.000000in}}%
\pgfusepath{stroke,fill}%
}%
\begin{pgfscope}%
\pgfsys@transformshift{1.018750in}{4.551000in}%
\pgfsys@useobject{currentmarker}{}%
\end{pgfscope}%
\end{pgfscope}%
\begin{pgfscope}%
\pgfsetbuttcap%
\pgfsetroundjoin%
\definecolor{currentfill}{rgb}{0.000000,0.000000,0.000000}%
\pgfsetfillcolor{currentfill}%
\pgfsetlinewidth{0.501875pt}%
\definecolor{currentstroke}{rgb}{0.000000,0.000000,0.000000}%
\pgfsetstrokecolor{currentstroke}%
\pgfsetdash{}{0pt}%
\pgfsys@defobject{currentmarker}{\pgfqpoint{-0.055556in}{0.000000in}}{\pgfqpoint{0.000000in}{0.000000in}}{%
\pgfpathmoveto{\pgfqpoint{0.000000in}{0.000000in}}%
\pgfpathlineto{\pgfqpoint{-0.055556in}{0.000000in}}%
\pgfusepath{stroke,fill}%
}%
\begin{pgfscope}%
\pgfsys@transformshift{7.335000in}{4.551000in}%
\pgfsys@useobject{currentmarker}{}%
\end{pgfscope}%
\end{pgfscope}%
\begin{pgfscope}%
\pgftext[x=0.963194in,y=4.551000in,right,]{{\sffamily\fontsize{14.000000}{16.800000}\selectfont \(\displaystyle 80\%\)}}%
\end{pgfscope}%
\begin{pgfscope}%
\pgfsetbuttcap%
\pgfsetroundjoin%
\definecolor{currentfill}{rgb}{0.000000,0.000000,0.000000}%
\pgfsetfillcolor{currentfill}%
\pgfsetlinewidth{0.501875pt}%
\definecolor{currentstroke}{rgb}{0.000000,0.000000,0.000000}%
\pgfsetstrokecolor{currentstroke}%
\pgfsetdash{}{0pt}%
\pgfsys@defobject{currentmarker}{\pgfqpoint{0.000000in}{0.000000in}}{\pgfqpoint{0.055556in}{0.000000in}}{%
\pgfpathmoveto{\pgfqpoint{0.000000in}{0.000000in}}%
\pgfpathlineto{\pgfqpoint{0.055556in}{0.000000in}}%
\pgfusepath{stroke,fill}%
}%
\begin{pgfscope}%
\pgfsys@transformshift{1.018750in}{5.535000in}%
\pgfsys@useobject{currentmarker}{}%
\end{pgfscope}%
\end{pgfscope}%
\begin{pgfscope}%
\pgfsetbuttcap%
\pgfsetroundjoin%
\definecolor{currentfill}{rgb}{0.000000,0.000000,0.000000}%
\pgfsetfillcolor{currentfill}%
\pgfsetlinewidth{0.501875pt}%
\definecolor{currentstroke}{rgb}{0.000000,0.000000,0.000000}%
\pgfsetstrokecolor{currentstroke}%
\pgfsetdash{}{0pt}%
\pgfsys@defobject{currentmarker}{\pgfqpoint{-0.055556in}{0.000000in}}{\pgfqpoint{0.000000in}{0.000000in}}{%
\pgfpathmoveto{\pgfqpoint{0.000000in}{0.000000in}}%
\pgfpathlineto{\pgfqpoint{-0.055556in}{0.000000in}}%
\pgfusepath{stroke,fill}%
}%
\begin{pgfscope}%
\pgfsys@transformshift{7.335000in}{5.535000in}%
\pgfsys@useobject{currentmarker}{}%
\end{pgfscope}%
\end{pgfscope}%
\begin{pgfscope}%
\pgftext[x=0.963194in,y=5.535000in,right,]{{\sffamily\fontsize{14.000000}{16.800000}\selectfont \(\displaystyle 100\%\)}}%
\end{pgfscope}%
\begin{pgfscope}%
\pgftext[x=0.441337in,y=3.075000in,,bottom,rotate=90.000000]{{\sffamily\fontsize{14.000000}{16.800000}\selectfont \textrm{Coordination rate}}}%
\end{pgfscope}%
\begin{pgfscope}%
\pgfsetbuttcap%
\pgfsetmiterjoin%
\definecolor{currentfill}{rgb}{1.000000,1.000000,1.000000}%
\pgfsetfillcolor{currentfill}%
\pgfsetlinewidth{1.003750pt}%
\definecolor{currentstroke}{rgb}{0.000000,0.000000,0.000000}%
\pgfsetstrokecolor{currentstroke}%
\pgfsetdash{}{0pt}%
\pgfpathmoveto{\pgfqpoint{1.115972in}{4.829556in}}%
\pgfpathlineto{\pgfqpoint{3.398556in}{4.829556in}}%
\pgfpathlineto{\pgfqpoint{3.398556in}{5.437778in}}%
\pgfpathlineto{\pgfqpoint{1.115972in}{5.437778in}}%
\pgfpathclose%
\pgfusepath{stroke,fill}%
\end{pgfscope}%
\begin{pgfscope}%
\pgfsetrectcap%
\pgfsetroundjoin%
\pgfsetlinewidth{1.003750pt}%
\definecolor{currentstroke}{rgb}{0.000000,0.000000,1.000000}%
\pgfsetstrokecolor{currentstroke}%
\pgfsetdash{}{0pt}%
\pgfpathmoveto{\pgfqpoint{1.252083in}{5.289222in}}%
\pgfpathlineto{\pgfqpoint{1.524306in}{5.289222in}}%
\pgfusepath{stroke}%
\end{pgfscope}%
\begin{pgfscope}%
\pgftext[x=1.738194in,y=5.221167in,left,base]{{\rmfamily\fontsize{14.000000}{16.800000}\selectfont Ambiguous form}}%
\end{pgfscope}%
\begin{pgfscope}%
\pgfsetbuttcap%
\pgfsetroundjoin%
\pgfsetlinewidth{1.003750pt}%
\definecolor{currentstroke}{rgb}{0.000000,0.500000,0.000000}%
\pgfsetstrokecolor{currentstroke}%
\pgfsetdash{{6.000000pt}{6.000000pt}}{0.000000pt}%
\pgfpathmoveto{\pgfqpoint{1.252083in}{5.015250in}}%
\pgfpathlineto{\pgfqpoint{1.524306in}{5.015250in}}%
\pgfusepath{stroke}%
\end{pgfscope}%
\begin{pgfscope}%
\pgftext[x=1.738194in,y=4.947195in,left,base]{{\rmfamily\fontsize{14.000000}{16.800000}\selectfont Unambiguous form}}%
\end{pgfscope}%
\end{pgfpicture}%
\makeatother%
\endgroup%
}
\caption{Coordination rate predictions for Experiment 1 with ambiguous form cost varied (lines at 80, 120, 280 showing costs of unambiguous forms)}
\label{fig:3}
\end{figure}

\begin{figure}[t]
\centering
\scalebox{.425}{%% Creator: Matplotlib, PGF backend
%%
%% To include the figure in your LaTeX document, write
%%   \input{<filename>.pgf}
%%
%% Make sure the required packages are loaded in your preamble
%%   \usepackage{pgf}
%%
%% Figures using additional raster images can only be included by \input if
%% they are in the same directory as the main LaTeX file. For loading figures
%% from other directories you can use the `import` package
%%   \usepackage{import}
%% and then include the figures with
%%   \import{<path to file>}{<filename>.pgf}
%%
%% Matplotlib used the following preamble
%%   \usepackage{fontspec}
%%   \setsansfont{DejaVu Sans}
%%   \setmonofont{DejaVu Sans Mono}
%%
\begingroup%
\makeatletter%
\begin{pgfpicture}%
\pgfpathrectangle{\pgfpointorigin}{\pgfqpoint{8.150000in}{6.150000in}}%
\pgfusepath{use as bounding box, clip}%
\begin{pgfscope}%
\pgfsetbuttcap%
\pgfsetmiterjoin%
\definecolor{currentfill}{rgb}{1.000000,1.000000,1.000000}%
\pgfsetfillcolor{currentfill}%
\pgfsetlinewidth{0.000000pt}%
\definecolor{currentstroke}{rgb}{1.000000,1.000000,1.000000}%
\pgfsetstrokecolor{currentstroke}%
\pgfsetdash{}{0pt}%
\pgfpathmoveto{\pgfqpoint{0.000000in}{0.000000in}}%
\pgfpathlineto{\pgfqpoint{8.150000in}{0.000000in}}%
\pgfpathlineto{\pgfqpoint{8.150000in}{6.150000in}}%
\pgfpathlineto{\pgfqpoint{0.000000in}{6.150000in}}%
\pgfpathclose%
\pgfusepath{fill}%
\end{pgfscope}%
\begin{pgfscope}%
\pgfsetbuttcap%
\pgfsetmiterjoin%
\definecolor{currentfill}{rgb}{1.000000,1.000000,1.000000}%
\pgfsetfillcolor{currentfill}%
\pgfsetlinewidth{0.000000pt}%
\definecolor{currentstroke}{rgb}{0.000000,0.000000,0.000000}%
\pgfsetstrokecolor{currentstroke}%
\pgfsetstrokeopacity{0.000000}%
\pgfsetdash{}{0pt}%
\pgfpathmoveto{\pgfqpoint{1.018750in}{0.615000in}}%
\pgfpathlineto{\pgfqpoint{7.335000in}{0.615000in}}%
\pgfpathlineto{\pgfqpoint{7.335000in}{5.535000in}}%
\pgfpathlineto{\pgfqpoint{1.018750in}{5.535000in}}%
\pgfpathclose%
\pgfusepath{fill}%
\end{pgfscope}%
\begin{pgfscope}%
\pgfpathrectangle{\pgfqpoint{1.018750in}{0.615000in}}{\pgfqpoint{6.316250in}{4.920000in}} %
\pgfusepath{clip}%
\pgfsetrectcap%
\pgfsetroundjoin%
\pgfsetlinewidth{1.003750pt}%
\definecolor{currentstroke}{rgb}{0.000000,0.000000,1.000000}%
\pgfsetstrokecolor{currentstroke}%
\pgfsetdash{}{0pt}%
\pgfpathmoveto{\pgfqpoint{1.018750in}{1.387180in}}%
\pgfpathlineto{\pgfqpoint{1.305852in}{1.796997in}}%
\pgfpathlineto{\pgfqpoint{1.592955in}{1.995434in}}%
\pgfpathlineto{\pgfqpoint{1.880057in}{2.284462in}}%
\pgfpathlineto{\pgfqpoint{2.167159in}{2.646826in}}%
\pgfpathlineto{\pgfqpoint{2.454261in}{2.849577in}}%
\pgfpathlineto{\pgfqpoint{2.741364in}{3.186058in}}%
\pgfpathlineto{\pgfqpoint{3.028466in}{3.509597in}}%
\pgfpathlineto{\pgfqpoint{3.315568in}{3.785684in}}%
\pgfpathlineto{\pgfqpoint{3.602670in}{3.971180in}}%
\pgfpathlineto{\pgfqpoint{3.889773in}{4.100596in}}%
\pgfpathlineto{\pgfqpoint{4.176875in}{4.480215in}}%
\pgfpathlineto{\pgfqpoint{4.463977in}{4.497470in}}%
\pgfpathlineto{\pgfqpoint{4.751080in}{4.596689in}}%
\pgfpathlineto{\pgfqpoint{5.038182in}{4.743360in}}%
\pgfpathlineto{\pgfqpoint{5.325284in}{4.760616in}}%
\pgfpathlineto{\pgfqpoint{5.612386in}{4.838265in}}%
\pgfpathlineto{\pgfqpoint{5.899489in}{4.812382in}}%
\pgfpathlineto{\pgfqpoint{6.186591in}{4.734732in}}%
\pgfpathlineto{\pgfqpoint{6.473693in}{4.751988in}}%
\pgfpathlineto{\pgfqpoint{6.760795in}{4.838265in}}%
\pgfpathlineto{\pgfqpoint{7.047898in}{5.032389in}}%
\pgfpathlineto{\pgfqpoint{7.335000in}{4.980622in}}%
\pgfusepath{stroke}%
\end{pgfscope}%
\begin{pgfscope}%
\pgfpathrectangle{\pgfqpoint{1.018750in}{0.615000in}}{\pgfqpoint{6.316250in}{4.920000in}} %
\pgfusepath{clip}%
\pgfsetbuttcap%
\pgfsetroundjoin%
\pgfsetlinewidth{1.003750pt}%
\definecolor{currentstroke}{rgb}{0.000000,0.500000,0.000000}%
\pgfsetstrokecolor{currentstroke}%
\pgfsetdash{{6.000000pt}{6.000000pt}}{0.000000pt}%
\pgfpathmoveto{\pgfqpoint{1.018750in}{0.645197in}}%
\pgfpathlineto{\pgfqpoint{1.305852in}{0.662452in}}%
\pgfpathlineto{\pgfqpoint{1.592955in}{0.645197in}}%
\pgfpathlineto{\pgfqpoint{1.880057in}{0.653825in}}%
\pgfpathlineto{\pgfqpoint{2.167159in}{0.718533in}}%
\pgfpathlineto{\pgfqpoint{2.454261in}{0.740102in}}%
\pgfpathlineto{\pgfqpoint{2.741364in}{0.899714in}}%
\pgfpathlineto{\pgfqpoint{3.028466in}{1.050699in}}%
\pgfpathlineto{\pgfqpoint{3.315568in}{1.188743in}}%
\pgfpathlineto{\pgfqpoint{3.602670in}{1.253451in}}%
\pgfpathlineto{\pgfqpoint{3.889773in}{1.223254in}}%
\pgfpathlineto{\pgfqpoint{4.176875in}{1.318159in}}%
\pgfpathlineto{\pgfqpoint{4.463977in}{1.426005in}}%
\pgfpathlineto{\pgfqpoint{4.751080in}{1.495027in}}%
\pgfpathlineto{\pgfqpoint{5.038182in}{1.451888in}}%
\pgfpathlineto{\pgfqpoint{5.325284in}{1.395808in}}%
\pgfpathlineto{\pgfqpoint{5.612386in}{1.546793in}}%
\pgfpathlineto{\pgfqpoint{5.899489in}{1.456202in}}%
\pgfpathlineto{\pgfqpoint{6.186591in}{1.516596in}}%
\pgfpathlineto{\pgfqpoint{6.473693in}{1.529537in}}%
\pgfpathlineto{\pgfqpoint{6.760795in}{1.520910in}}%
\pgfpathlineto{\pgfqpoint{7.047898in}{1.581304in}}%
\pgfpathlineto{\pgfqpoint{7.335000in}{1.507968in}}%
\pgfusepath{stroke}%
\end{pgfscope}%
\begin{pgfscope}%
\pgfsetrectcap%
\pgfsetmiterjoin%
\pgfsetlinewidth{1.003750pt}%
\definecolor{currentstroke}{rgb}{0.000000,0.000000,0.000000}%
\pgfsetstrokecolor{currentstroke}%
\pgfsetdash{}{0pt}%
\pgfpathmoveto{\pgfqpoint{1.018750in}{5.535000in}}%
\pgfpathlineto{\pgfqpoint{7.335000in}{5.535000in}}%
\pgfusepath{stroke}%
\end{pgfscope}%
\begin{pgfscope}%
\pgfsetrectcap%
\pgfsetmiterjoin%
\pgfsetlinewidth{1.003750pt}%
\definecolor{currentstroke}{rgb}{0.000000,0.000000,0.000000}%
\pgfsetstrokecolor{currentstroke}%
\pgfsetdash{}{0pt}%
\pgfpathmoveto{\pgfqpoint{7.335000in}{0.615000in}}%
\pgfpathlineto{\pgfqpoint{7.335000in}{5.535000in}}%
\pgfusepath{stroke}%
\end{pgfscope}%
\begin{pgfscope}%
\pgfsetrectcap%
\pgfsetmiterjoin%
\pgfsetlinewidth{1.003750pt}%
\definecolor{currentstroke}{rgb}{0.000000,0.000000,0.000000}%
\pgfsetstrokecolor{currentstroke}%
\pgfsetdash{}{0pt}%
\pgfpathmoveto{\pgfqpoint{1.018750in}{0.615000in}}%
\pgfpathlineto{\pgfqpoint{7.335000in}{0.615000in}}%
\pgfusepath{stroke}%
\end{pgfscope}%
\begin{pgfscope}%
\pgfsetrectcap%
\pgfsetmiterjoin%
\pgfsetlinewidth{1.003750pt}%
\definecolor{currentstroke}{rgb}{0.000000,0.000000,0.000000}%
\pgfsetstrokecolor{currentstroke}%
\pgfsetdash{}{0pt}%
\pgfpathmoveto{\pgfqpoint{1.018750in}{0.615000in}}%
\pgfpathlineto{\pgfqpoint{1.018750in}{5.535000in}}%
\pgfusepath{stroke}%
\end{pgfscope}%
\begin{pgfscope}%
\pgfsetbuttcap%
\pgfsetroundjoin%
\definecolor{currentfill}{rgb}{0.000000,0.000000,0.000000}%
\pgfsetfillcolor{currentfill}%
\pgfsetlinewidth{0.501875pt}%
\definecolor{currentstroke}{rgb}{0.000000,0.000000,0.000000}%
\pgfsetstrokecolor{currentstroke}%
\pgfsetdash{}{0pt}%
\pgfsys@defobject{currentmarker}{\pgfqpoint{0.000000in}{0.000000in}}{\pgfqpoint{0.000000in}{0.055556in}}{%
\pgfpathmoveto{\pgfqpoint{0.000000in}{0.000000in}}%
\pgfpathlineto{\pgfqpoint{0.000000in}{0.055556in}}%
\pgfusepath{stroke,fill}%
}%
\begin{pgfscope}%
\pgfsys@transformshift{1.018750in}{0.615000in}%
\pgfsys@useobject{currentmarker}{}%
\end{pgfscope}%
\end{pgfscope}%
\begin{pgfscope}%
\pgfsetbuttcap%
\pgfsetroundjoin%
\definecolor{currentfill}{rgb}{0.000000,0.000000,0.000000}%
\pgfsetfillcolor{currentfill}%
\pgfsetlinewidth{0.501875pt}%
\definecolor{currentstroke}{rgb}{0.000000,0.000000,0.000000}%
\pgfsetstrokecolor{currentstroke}%
\pgfsetdash{}{0pt}%
\pgfsys@defobject{currentmarker}{\pgfqpoint{0.000000in}{-0.055556in}}{\pgfqpoint{0.000000in}{0.000000in}}{%
\pgfpathmoveto{\pgfqpoint{0.000000in}{0.000000in}}%
\pgfpathlineto{\pgfqpoint{0.000000in}{-0.055556in}}%
\pgfusepath{stroke,fill}%
}%
\begin{pgfscope}%
\pgfsys@transformshift{1.018750in}{5.535000in}%
\pgfsys@useobject{currentmarker}{}%
\end{pgfscope}%
\end{pgfscope}%
\begin{pgfscope}%
\pgftext[x=1.018750in,y=0.559444in,,top]{{\sffamily\fontsize{14.000000}{16.800000}\selectfont \(\displaystyle -20\)}}%
\end{pgfscope}%
\begin{pgfscope}%
\pgfsetbuttcap%
\pgfsetroundjoin%
\definecolor{currentfill}{rgb}{0.000000,0.000000,0.000000}%
\pgfsetfillcolor{currentfill}%
\pgfsetlinewidth{0.501875pt}%
\definecolor{currentstroke}{rgb}{0.000000,0.000000,0.000000}%
\pgfsetstrokecolor{currentstroke}%
\pgfsetdash{}{0pt}%
\pgfsys@defobject{currentmarker}{\pgfqpoint{0.000000in}{0.000000in}}{\pgfqpoint{0.000000in}{0.055556in}}{%
\pgfpathmoveto{\pgfqpoint{0.000000in}{0.000000in}}%
\pgfpathlineto{\pgfqpoint{0.000000in}{0.055556in}}%
\pgfusepath{stroke,fill}%
}%
\begin{pgfscope}%
\pgfsys@transformshift{2.454261in}{0.615000in}%
\pgfsys@useobject{currentmarker}{}%
\end{pgfscope}%
\end{pgfscope}%
\begin{pgfscope}%
\pgfsetbuttcap%
\pgfsetroundjoin%
\definecolor{currentfill}{rgb}{0.000000,0.000000,0.000000}%
\pgfsetfillcolor{currentfill}%
\pgfsetlinewidth{0.501875pt}%
\definecolor{currentstroke}{rgb}{0.000000,0.000000,0.000000}%
\pgfsetstrokecolor{currentstroke}%
\pgfsetdash{}{0pt}%
\pgfsys@defobject{currentmarker}{\pgfqpoint{0.000000in}{-0.055556in}}{\pgfqpoint{0.000000in}{0.000000in}}{%
\pgfpathmoveto{\pgfqpoint{0.000000in}{0.000000in}}%
\pgfpathlineto{\pgfqpoint{0.000000in}{-0.055556in}}%
\pgfusepath{stroke,fill}%
}%
\begin{pgfscope}%
\pgfsys@transformshift{2.454261in}{5.535000in}%
\pgfsys@useobject{currentmarker}{}%
\end{pgfscope}%
\end{pgfscope}%
\begin{pgfscope}%
\pgftext[x=2.454261in,y=0.559444in,,top]{{\sffamily\fontsize{14.000000}{16.800000}\selectfont \(\displaystyle 80\)}}%
\end{pgfscope}%
\begin{pgfscope}%
\pgfsetbuttcap%
\pgfsetroundjoin%
\definecolor{currentfill}{rgb}{0.000000,0.000000,0.000000}%
\pgfsetfillcolor{currentfill}%
\pgfsetlinewidth{0.501875pt}%
\definecolor{currentstroke}{rgb}{0.000000,0.000000,0.000000}%
\pgfsetstrokecolor{currentstroke}%
\pgfsetdash{}{0pt}%
\pgfsys@defobject{currentmarker}{\pgfqpoint{0.000000in}{0.000000in}}{\pgfqpoint{0.000000in}{0.055556in}}{%
\pgfpathmoveto{\pgfqpoint{0.000000in}{0.000000in}}%
\pgfpathlineto{\pgfqpoint{0.000000in}{0.055556in}}%
\pgfusepath{stroke,fill}%
}%
\begin{pgfscope}%
\pgfsys@transformshift{3.889773in}{0.615000in}%
\pgfsys@useobject{currentmarker}{}%
\end{pgfscope}%
\end{pgfscope}%
\begin{pgfscope}%
\pgfsetbuttcap%
\pgfsetroundjoin%
\definecolor{currentfill}{rgb}{0.000000,0.000000,0.000000}%
\pgfsetfillcolor{currentfill}%
\pgfsetlinewidth{0.501875pt}%
\definecolor{currentstroke}{rgb}{0.000000,0.000000,0.000000}%
\pgfsetstrokecolor{currentstroke}%
\pgfsetdash{}{0pt}%
\pgfsys@defobject{currentmarker}{\pgfqpoint{0.000000in}{-0.055556in}}{\pgfqpoint{0.000000in}{0.000000in}}{%
\pgfpathmoveto{\pgfqpoint{0.000000in}{0.000000in}}%
\pgfpathlineto{\pgfqpoint{0.000000in}{-0.055556in}}%
\pgfusepath{stroke,fill}%
}%
\begin{pgfscope}%
\pgfsys@transformshift{3.889773in}{5.535000in}%
\pgfsys@useobject{currentmarker}{}%
\end{pgfscope}%
\end{pgfscope}%
\begin{pgfscope}%
\pgftext[x=3.889773in,y=0.559444in,,top]{{\sffamily\fontsize{14.000000}{16.800000}\selectfont \(\displaystyle 180\)}}%
\end{pgfscope}%
\begin{pgfscope}%
\pgfsetbuttcap%
\pgfsetroundjoin%
\definecolor{currentfill}{rgb}{0.000000,0.000000,0.000000}%
\pgfsetfillcolor{currentfill}%
\pgfsetlinewidth{0.501875pt}%
\definecolor{currentstroke}{rgb}{0.000000,0.000000,0.000000}%
\pgfsetstrokecolor{currentstroke}%
\pgfsetdash{}{0pt}%
\pgfsys@defobject{currentmarker}{\pgfqpoint{0.000000in}{0.000000in}}{\pgfqpoint{0.000000in}{0.055556in}}{%
\pgfpathmoveto{\pgfqpoint{0.000000in}{0.000000in}}%
\pgfpathlineto{\pgfqpoint{0.000000in}{0.055556in}}%
\pgfusepath{stroke,fill}%
}%
\begin{pgfscope}%
\pgfsys@transformshift{5.325284in}{0.615000in}%
\pgfsys@useobject{currentmarker}{}%
\end{pgfscope}%
\end{pgfscope}%
\begin{pgfscope}%
\pgfsetbuttcap%
\pgfsetroundjoin%
\definecolor{currentfill}{rgb}{0.000000,0.000000,0.000000}%
\pgfsetfillcolor{currentfill}%
\pgfsetlinewidth{0.501875pt}%
\definecolor{currentstroke}{rgb}{0.000000,0.000000,0.000000}%
\pgfsetstrokecolor{currentstroke}%
\pgfsetdash{}{0pt}%
\pgfsys@defobject{currentmarker}{\pgfqpoint{0.000000in}{-0.055556in}}{\pgfqpoint{0.000000in}{0.000000in}}{%
\pgfpathmoveto{\pgfqpoint{0.000000in}{0.000000in}}%
\pgfpathlineto{\pgfqpoint{0.000000in}{-0.055556in}}%
\pgfusepath{stroke,fill}%
}%
\begin{pgfscope}%
\pgfsys@transformshift{5.325284in}{5.535000in}%
\pgfsys@useobject{currentmarker}{}%
\end{pgfscope}%
\end{pgfscope}%
\begin{pgfscope}%
\pgftext[x=5.325284in,y=0.559444in,,top]{{\sffamily\fontsize{14.000000}{16.800000}\selectfont \(\displaystyle 280\)}}%
\end{pgfscope}%
\begin{pgfscope}%
\pgfsetbuttcap%
\pgfsetroundjoin%
\definecolor{currentfill}{rgb}{0.000000,0.000000,0.000000}%
\pgfsetfillcolor{currentfill}%
\pgfsetlinewidth{0.501875pt}%
\definecolor{currentstroke}{rgb}{0.000000,0.000000,0.000000}%
\pgfsetstrokecolor{currentstroke}%
\pgfsetdash{}{0pt}%
\pgfsys@defobject{currentmarker}{\pgfqpoint{0.000000in}{0.000000in}}{\pgfqpoint{0.000000in}{0.055556in}}{%
\pgfpathmoveto{\pgfqpoint{0.000000in}{0.000000in}}%
\pgfpathlineto{\pgfqpoint{0.000000in}{0.055556in}}%
\pgfusepath{stroke,fill}%
}%
\begin{pgfscope}%
\pgfsys@transformshift{6.760795in}{0.615000in}%
\pgfsys@useobject{currentmarker}{}%
\end{pgfscope}%
\end{pgfscope}%
\begin{pgfscope}%
\pgfsetbuttcap%
\pgfsetroundjoin%
\definecolor{currentfill}{rgb}{0.000000,0.000000,0.000000}%
\pgfsetfillcolor{currentfill}%
\pgfsetlinewidth{0.501875pt}%
\definecolor{currentstroke}{rgb}{0.000000,0.000000,0.000000}%
\pgfsetstrokecolor{currentstroke}%
\pgfsetdash{}{0pt}%
\pgfsys@defobject{currentmarker}{\pgfqpoint{0.000000in}{-0.055556in}}{\pgfqpoint{0.000000in}{0.000000in}}{%
\pgfpathmoveto{\pgfqpoint{0.000000in}{0.000000in}}%
\pgfpathlineto{\pgfqpoint{0.000000in}{-0.055556in}}%
\pgfusepath{stroke,fill}%
}%
\begin{pgfscope}%
\pgfsys@transformshift{6.760795in}{5.535000in}%
\pgfsys@useobject{currentmarker}{}%
\end{pgfscope}%
\end{pgfscope}%
\begin{pgfscope}%
\pgftext[x=6.760795in,y=0.559444in,,top]{{\sffamily\fontsize{14.000000}{16.800000}\selectfont \(\displaystyle 380\)}}%
\end{pgfscope}%
\begin{pgfscope}%
\pgftext[x=4.176875in,y=0.301822in,,top]{{\sffamily\fontsize{22.000000}{22.800000}\selectfont \(\displaystyle S\)}}%
\end{pgfscope}%
\begin{pgfscope}%
\pgfsetbuttcap%
\pgfsetroundjoin%
\definecolor{currentfill}{rgb}{0.000000,0.000000,0.000000}%
\pgfsetfillcolor{currentfill}%
\pgfsetlinewidth{0.501875pt}%
\definecolor{currentstroke}{rgb}{0.000000,0.000000,0.000000}%
\pgfsetstrokecolor{currentstroke}%
\pgfsetdash{}{0pt}%
\pgfsys@defobject{currentmarker}{\pgfqpoint{0.000000in}{0.000000in}}{\pgfqpoint{0.055556in}{0.000000in}}{%
\pgfpathmoveto{\pgfqpoint{0.000000in}{0.000000in}}%
\pgfpathlineto{\pgfqpoint{0.055556in}{0.000000in}}%
\pgfusepath{stroke,fill}%
}%
\begin{pgfscope}%
\pgfsys@transformshift{1.018750in}{0.615000in}%
\pgfsys@useobject{currentmarker}{}%
\end{pgfscope}%
\end{pgfscope}%
\begin{pgfscope}%
\pgfsetbuttcap%
\pgfsetroundjoin%
\definecolor{currentfill}{rgb}{0.000000,0.000000,0.000000}%
\pgfsetfillcolor{currentfill}%
\pgfsetlinewidth{0.501875pt}%
\definecolor{currentstroke}{rgb}{0.000000,0.000000,0.000000}%
\pgfsetstrokecolor{currentstroke}%
\pgfsetdash{}{0pt}%
\pgfsys@defobject{currentmarker}{\pgfqpoint{-0.055556in}{0.000000in}}{\pgfqpoint{0.000000in}{0.000000in}}{%
\pgfpathmoveto{\pgfqpoint{0.000000in}{0.000000in}}%
\pgfpathlineto{\pgfqpoint{-0.055556in}{0.000000in}}%
\pgfusepath{stroke,fill}%
}%
\begin{pgfscope}%
\pgfsys@transformshift{7.335000in}{0.615000in}%
\pgfsys@useobject{currentmarker}{}%
\end{pgfscope}%
\end{pgfscope}%
\begin{pgfscope}%
\pgftext[x=0.963194in,y=0.615000in,right,]{{\sffamily\fontsize{14.000000}{16.800000}\selectfont \(\displaystyle 0\%\)}}%
\end{pgfscope}%
\begin{pgfscope}%
\pgfsetbuttcap%
\pgfsetroundjoin%
\definecolor{currentfill}{rgb}{0.000000,0.000000,0.000000}%
\pgfsetfillcolor{currentfill}%
\pgfsetlinewidth{0.501875pt}%
\definecolor{currentstroke}{rgb}{0.000000,0.000000,0.000000}%
\pgfsetstrokecolor{currentstroke}%
\pgfsetdash{}{0pt}%
\pgfsys@defobject{currentmarker}{\pgfqpoint{0.000000in}{0.000000in}}{\pgfqpoint{0.055556in}{0.000000in}}{%
\pgfpathmoveto{\pgfqpoint{0.000000in}{0.000000in}}%
\pgfpathlineto{\pgfqpoint{0.055556in}{0.000000in}}%
\pgfusepath{stroke,fill}%
}%
\begin{pgfscope}%
\pgfsys@transformshift{1.018750in}{1.599000in}%
\pgfsys@useobject{currentmarker}{}%
\end{pgfscope}%
\end{pgfscope}%
\begin{pgfscope}%
\pgfsetbuttcap%
\pgfsetroundjoin%
\definecolor{currentfill}{rgb}{0.000000,0.000000,0.000000}%
\pgfsetfillcolor{currentfill}%
\pgfsetlinewidth{0.501875pt}%
\definecolor{currentstroke}{rgb}{0.000000,0.000000,0.000000}%
\pgfsetstrokecolor{currentstroke}%
\pgfsetdash{}{0pt}%
\pgfsys@defobject{currentmarker}{\pgfqpoint{-0.055556in}{0.000000in}}{\pgfqpoint{0.000000in}{0.000000in}}{%
\pgfpathmoveto{\pgfqpoint{0.000000in}{0.000000in}}%
\pgfpathlineto{\pgfqpoint{-0.055556in}{0.000000in}}%
\pgfusepath{stroke,fill}%
}%
\begin{pgfscope}%
\pgfsys@transformshift{7.335000in}{1.599000in}%
\pgfsys@useobject{currentmarker}{}%
\end{pgfscope}%
\end{pgfscope}%
\begin{pgfscope}%
\pgftext[x=0.963194in,y=1.599000in,right,]{{\sffamily\fontsize{14.000000}{16.800000}\selectfont \(\displaystyle 20\%\)}}%
\end{pgfscope}%
\begin{pgfscope}%
\pgfsetbuttcap%
\pgfsetroundjoin%
\definecolor{currentfill}{rgb}{0.000000,0.000000,0.000000}%
\pgfsetfillcolor{currentfill}%
\pgfsetlinewidth{0.501875pt}%
\definecolor{currentstroke}{rgb}{0.000000,0.000000,0.000000}%
\pgfsetstrokecolor{currentstroke}%
\pgfsetdash{}{0pt}%
\pgfsys@defobject{currentmarker}{\pgfqpoint{0.000000in}{0.000000in}}{\pgfqpoint{0.055556in}{0.000000in}}{%
\pgfpathmoveto{\pgfqpoint{0.000000in}{0.000000in}}%
\pgfpathlineto{\pgfqpoint{0.055556in}{0.000000in}}%
\pgfusepath{stroke,fill}%
}%
\begin{pgfscope}%
\pgfsys@transformshift{1.018750in}{2.583000in}%
\pgfsys@useobject{currentmarker}{}%
\end{pgfscope}%
\end{pgfscope}%
\begin{pgfscope}%
\pgfsetbuttcap%
\pgfsetroundjoin%
\definecolor{currentfill}{rgb}{0.000000,0.000000,0.000000}%
\pgfsetfillcolor{currentfill}%
\pgfsetlinewidth{0.501875pt}%
\definecolor{currentstroke}{rgb}{0.000000,0.000000,0.000000}%
\pgfsetstrokecolor{currentstroke}%
\pgfsetdash{}{0pt}%
\pgfsys@defobject{currentmarker}{\pgfqpoint{-0.055556in}{0.000000in}}{\pgfqpoint{0.000000in}{0.000000in}}{%
\pgfpathmoveto{\pgfqpoint{0.000000in}{0.000000in}}%
\pgfpathlineto{\pgfqpoint{-0.055556in}{0.000000in}}%
\pgfusepath{stroke,fill}%
}%
\begin{pgfscope}%
\pgfsys@transformshift{7.335000in}{2.583000in}%
\pgfsys@useobject{currentmarker}{}%
\end{pgfscope}%
\end{pgfscope}%
\begin{pgfscope}%
\pgftext[x=0.963194in,y=2.583000in,right,]{{\sffamily\fontsize{14.000000}{16.800000}\selectfont \(\displaystyle 40\%\)}}%
\end{pgfscope}%
\begin{pgfscope}%
\pgfsetbuttcap%
\pgfsetroundjoin%
\definecolor{currentfill}{rgb}{0.000000,0.000000,0.000000}%
\pgfsetfillcolor{currentfill}%
\pgfsetlinewidth{0.501875pt}%
\definecolor{currentstroke}{rgb}{0.000000,0.000000,0.000000}%
\pgfsetstrokecolor{currentstroke}%
\pgfsetdash{}{0pt}%
\pgfsys@defobject{currentmarker}{\pgfqpoint{0.000000in}{0.000000in}}{\pgfqpoint{0.055556in}{0.000000in}}{%
\pgfpathmoveto{\pgfqpoint{0.000000in}{0.000000in}}%
\pgfpathlineto{\pgfqpoint{0.055556in}{0.000000in}}%
\pgfusepath{stroke,fill}%
}%
\begin{pgfscope}%
\pgfsys@transformshift{1.018750in}{3.567000in}%
\pgfsys@useobject{currentmarker}{}%
\end{pgfscope}%
\end{pgfscope}%
\begin{pgfscope}%
\pgfsetbuttcap%
\pgfsetroundjoin%
\definecolor{currentfill}{rgb}{0.000000,0.000000,0.000000}%
\pgfsetfillcolor{currentfill}%
\pgfsetlinewidth{0.501875pt}%
\definecolor{currentstroke}{rgb}{0.000000,0.000000,0.000000}%
\pgfsetstrokecolor{currentstroke}%
\pgfsetdash{}{0pt}%
\pgfsys@defobject{currentmarker}{\pgfqpoint{-0.055556in}{0.000000in}}{\pgfqpoint{0.000000in}{0.000000in}}{%
\pgfpathmoveto{\pgfqpoint{0.000000in}{0.000000in}}%
\pgfpathlineto{\pgfqpoint{-0.055556in}{0.000000in}}%
\pgfusepath{stroke,fill}%
}%
\begin{pgfscope}%
\pgfsys@transformshift{7.335000in}{3.567000in}%
\pgfsys@useobject{currentmarker}{}%
\end{pgfscope}%
\end{pgfscope}%
\begin{pgfscope}%
\pgftext[x=0.963194in,y=3.567000in,right,]{{\sffamily\fontsize{14.000000}{16.800000}\selectfont \(\displaystyle 60\%\)}}%
\end{pgfscope}%
\begin{pgfscope}%
\pgfsetbuttcap%
\pgfsetroundjoin%
\definecolor{currentfill}{rgb}{0.000000,0.000000,0.000000}%
\pgfsetfillcolor{currentfill}%
\pgfsetlinewidth{0.501875pt}%
\definecolor{currentstroke}{rgb}{0.000000,0.000000,0.000000}%
\pgfsetstrokecolor{currentstroke}%
\pgfsetdash{}{0pt}%
\pgfsys@defobject{currentmarker}{\pgfqpoint{0.000000in}{0.000000in}}{\pgfqpoint{0.055556in}{0.000000in}}{%
\pgfpathmoveto{\pgfqpoint{0.000000in}{0.000000in}}%
\pgfpathlineto{\pgfqpoint{0.055556in}{0.000000in}}%
\pgfusepath{stroke,fill}%
}%
\begin{pgfscope}%
\pgfsys@transformshift{1.018750in}{4.551000in}%
\pgfsys@useobject{currentmarker}{}%
\end{pgfscope}%
\end{pgfscope}%
\begin{pgfscope}%
\pgfsetbuttcap%
\pgfsetroundjoin%
\definecolor{currentfill}{rgb}{0.000000,0.000000,0.000000}%
\pgfsetfillcolor{currentfill}%
\pgfsetlinewidth{0.501875pt}%
\definecolor{currentstroke}{rgb}{0.000000,0.000000,0.000000}%
\pgfsetstrokecolor{currentstroke}%
\pgfsetdash{}{0pt}%
\pgfsys@defobject{currentmarker}{\pgfqpoint{-0.055556in}{0.000000in}}{\pgfqpoint{0.000000in}{0.000000in}}{%
\pgfpathmoveto{\pgfqpoint{0.000000in}{0.000000in}}%
\pgfpathlineto{\pgfqpoint{-0.055556in}{0.000000in}}%
\pgfusepath{stroke,fill}%
}%
\begin{pgfscope}%
\pgfsys@transformshift{7.335000in}{4.551000in}%
\pgfsys@useobject{currentmarker}{}%
\end{pgfscope}%
\end{pgfscope}%
\begin{pgfscope}%
\pgftext[x=0.963194in,y=4.551000in,right,]{{\sffamily\fontsize{14.000000}{16.800000}\selectfont \(\displaystyle 80\%\)}}%
\end{pgfscope}%
\begin{pgfscope}%
\pgfsetbuttcap%
\pgfsetroundjoin%
\definecolor{currentfill}{rgb}{0.000000,0.000000,0.000000}%
\pgfsetfillcolor{currentfill}%
\pgfsetlinewidth{0.501875pt}%
\definecolor{currentstroke}{rgb}{0.000000,0.000000,0.000000}%
\pgfsetstrokecolor{currentstroke}%
\pgfsetdash{}{0pt}%
\pgfsys@defobject{currentmarker}{\pgfqpoint{0.000000in}{0.000000in}}{\pgfqpoint{0.055556in}{0.000000in}}{%
\pgfpathmoveto{\pgfqpoint{0.000000in}{0.000000in}}%
\pgfpathlineto{\pgfqpoint{0.055556in}{0.000000in}}%
\pgfusepath{stroke,fill}%
}%
\begin{pgfscope}%
\pgfsys@transformshift{1.018750in}{5.535000in}%
\pgfsys@useobject{currentmarker}{}%
\end{pgfscope}%
\end{pgfscope}%
\begin{pgfscope}%
\pgfsetbuttcap%
\pgfsetroundjoin%
\definecolor{currentfill}{rgb}{0.000000,0.000000,0.000000}%
\pgfsetfillcolor{currentfill}%
\pgfsetlinewidth{0.501875pt}%
\definecolor{currentstroke}{rgb}{0.000000,0.000000,0.000000}%
\pgfsetstrokecolor{currentstroke}%
\pgfsetdash{}{0pt}%
\pgfsys@defobject{currentmarker}{\pgfqpoint{-0.055556in}{0.000000in}}{\pgfqpoint{0.000000in}{0.000000in}}{%
\pgfpathmoveto{\pgfqpoint{0.000000in}{0.000000in}}%
\pgfpathlineto{\pgfqpoint{-0.055556in}{0.000000in}}%
\pgfusepath{stroke,fill}%
}%
\begin{pgfscope}%
\pgfsys@transformshift{7.335000in}{5.535000in}%
\pgfsys@useobject{currentmarker}{}%
\end{pgfscope}%
\end{pgfscope}%
\begin{pgfscope}%
\pgftext[x=0.963194in,y=5.535000in,right,]{{\sffamily\fontsize{14.000000}{16.800000}\selectfont \(\displaystyle 100\%\)}}%
\end{pgfscope}%
\begin{pgfscope}%
\pgftext[x=0.441337in,y=3.075000in,,bottom,rotate=90.000000]{{\sffamily\fontsize{22.000000}{22.800000}\selectfont \textrm{Coordination rate}}}%
\end{pgfscope}%
\begin{pgfscope}%
\pgfsetbuttcap%
\pgfsetmiterjoin%
\definecolor{currentfill}{rgb}{1.000000,1.000000,1.000000}%
\pgfsetfillcolor{currentfill}%
\pgfsetlinewidth{1.003750pt}%
\definecolor{currentstroke}{rgb}{0.000000,0.000000,0.000000}%
\pgfsetstrokecolor{currentstroke}%
\pgfsetdash{}{0pt}%
\pgfpathmoveto{\pgfqpoint{1.115972in}{4.829556in}}%
\pgfpathlineto{\pgfqpoint{3.398556in}{4.829556in}}%
\pgfpathlineto{\pgfqpoint{3.398556in}{5.437778in}}%
\pgfpathlineto{\pgfqpoint{1.115972in}{5.437778in}}%
\pgfpathclose%
\pgfusepath{stroke,fill}%
\end{pgfscope}%
\begin{pgfscope}%
\pgfsetrectcap%
\pgfsetroundjoin%
\pgfsetlinewidth{1.003750pt}%
\definecolor{currentstroke}{rgb}{0.000000,0.000000,1.000000}%
\pgfsetstrokecolor{currentstroke}%
\pgfsetdash{}{0pt}%
\pgfpathmoveto{\pgfqpoint{1.252083in}{5.289222in}}%
\pgfpathlineto{\pgfqpoint{1.524306in}{5.289222in}}%
\pgfusepath{stroke}%
\end{pgfscope}%
\begin{pgfscope}%
\pgftext[x=1.738194in,y=5.221167in,left,base]{{\rmfamily\fontsize{14.000000}{16.800000}\selectfont Ambiguous form}}%
\end{pgfscope}%
\begin{pgfscope}%
\pgfsetbuttcap%
\pgfsetroundjoin%
\pgfsetlinewidth{1.003750pt}%
\definecolor{currentstroke}{rgb}{0.000000,0.500000,0.000000}%
\pgfsetstrokecolor{currentstroke}%
\pgfsetdash{{6.000000pt}{6.000000pt}}{0.000000pt}%
\pgfpathmoveto{\pgfqpoint{1.252083in}{5.015250in}}%
\pgfpathlineto{\pgfqpoint{1.524306in}{5.015250in}}%
\pgfusepath{stroke}%
\end{pgfscope}%
\begin{pgfscope}%
\pgftext[x=1.738194in,y=4.947195in,left,base]{{\rmfamily\fontsize{14.000000}{16.800000}\selectfont Unambiguous form}}%
\end{pgfscope}%
\end{pgfpicture}%
\makeatother%
\endgroup%
}
\caption{Coordination rate predictions for Experiment 1 with successful communication reward varied }
\label{fig:4}
\end{figure}

Having been shown to outperform the baseline model in mimicking experimental data when scaled, the optimized PSO model was used to predict referential coordination behaviors outside those conditions studied in \citeA{rohde2012}. The results of varying ambiguous form cost on coordination rates are presented in Figure \ref{fig:3}, while Figure \ref{fig:4} demonstrates the effects of adjusting the number of points awarded for successful communication. As can be seen, the optimized model predicts that overall coordination rates increase in response to both higher ambiguous form cost and higher successful communication reward.\footnote{It should be noted that because coordination rates have been scaled as per \ref{sec:model_comparison}, the combined rates in some instances exceed $100\%$. The rates could be normalized or the scaling factor adjusted to avoid this.}


The positive correlation between ambiguous form cost and ambiguous form coordination rate (Figure \ref{fig:3}) likely reflects the fact that increasing ambiguous form costs more readily indicate for which of the three referents the form should be produced. For example, in experiment 1, $cost_A$ is less than $cost_{r2}$ and $cost_{r3}$, and hence using the ambiguous form for either referent can increase the achieved score, or indeed agents may attempt to use the ambiguous form for both referents simultaneously, reducing overall communicative success. When $cost_A$ instead exceeds the cost of all but one unambiguous form, the referential strategy to replace the costliest unambiguous form with the ambiguous form becomes highly preferred, an effect which becomes more pronounced as ambiguous form cost increases. As one would predict, when $cost_A$ surpasses the cost of all unambiguous forms, coordination with the ambiguous form stops increasing. Whereas humans likely would avoid replacing any less costly unambiguous form with the ambiguous one, the benefits in the model of retaining coordination and maximizing successful communication rewards are shown to outweigh the attraction of not using the ambiguous form at all. That coordination on the costly ambiguous form arises at all in such cases reflects the random initialization of the agents' strategies.


\citeauthor{rohde2012} argue that the lower stakes in study 2 resulted in more frequent coordination on the ambiguous form by encouraging players to explore a variety of referential strategies. This aligns with the PSO predictions that higher rewards on each successful round (Figure \ref{fig:4}) favor coordination.  This is because higher rewards for success in effect create a cushion to absorb the costs of communication failure.

\subsubsection{Variation of discourse context after entrainment}
In apparent violation of the Gricean maxim of quantity \cite{grice1975}, interlocutors who entrain on the use of a high-cost but unambiguous referring expression in one discourse environment have been observed to only infrequently switch to using a less costly but more general referring expression when a new discourse context would allow them to do so unambiguously, i.e., they maintain a `conceptual pact' \cite{brennan1996}. To investigate whether this property holds for the PSO model, two experiments were conducted which varied in their dimensionality and form costs (see Table \ref{table:4}). To simulate previous entrainment on the unambiguous form, all particles $i$ were initialized such that $\forall r \in R,\, P_i(A|r) = 0$; further, the base inertia parameter ($\tau$) was modified from the optimized value presented in Table \ref{table:1} so as to reflect 305 iterations of the PSO algorithm having already been run. With these initial conditions set, 305 ``additional" iterations were run.

For both experiment A and B, 250 simulations were run. In no simulation did the number of agent pairs entrained on unambiguous forms initially differ from the final number after the simulation had concluded; this is to say, all agents continued using the more costly unambiguous forms. Although \citeauthor{brennan1996} do not report a pattern of uniform overinformativity, this is not taken to be problematic, considering no attempt was made to establish a correspondence between iterations of the PSO model in the \citeauthor{rohde2012} language games and duration of entrainment in the \citeauthor{brennan1996} studies. 

\begin{table}
\begin{center}
    \begin{tabular}{ l r r r r }
    Experiment & $cost_{r_1}$ & $cost_{r_2}$ & $cost_{r_3}$ & $cost_A$\\ \hline
    Experim. A & $80$  & $140$ & $165$ & $80$ \\ \hline
    Experim. B & $280$ &       &       & $80$ \\ \hline
    \end{tabular}
    \caption{Form costs for experiments A and B, which differ in their dimensionality. Note that experiment A's costs are identical to those of Table \ref{table:2}, experiment 3, for which \citeauthor{rohde2012}\ recorded the highest rate of ambiguous form coordination.}
    \label{table:4}
\end{center}
\end{table}



\section{Conclusion}
This paper seeks to model referential coordination computationally. The paper demonstrates that PSO offers a framework for replicating the human responses observed in \citeauthor{rohde2012}'s \citeyear{rohde2012} language game studies; in particular, this paper identifies a set of parameters for the PSO algorithm which yield a significant improvement in replication over baseline PSO parameters. The resultant model is used to predict the effects of varying costs and rewards, extrapolating beyond the conditions tested in \citeauthor{rohde2012}'s studies. Predictions were also made which follow effects noted in \citeauthor{brennan1996}. As a model of human behavior, PSO should not be taken as an unqualified snapshot of how human interlocutors make contextually appropriate reference decisions; rather, this modeling suggests a lower bound for how complex agents need to be to respond to form costs similarly to humans (i.e., a speaker may very well model a listener modeling a speaker modeling a listener, but that may not be necessary to achieve coordination). Overall, the findings demonstrate that it is possible to explain referential coordination in terms of a generalized optimization process (in which communicative success is maximized and communicative costs minimized), without needing to model complex or specialized linguistic processes or reasoning methods.





\bibliographystyle{apacite}
\bibliography{conference}

\end{document}

