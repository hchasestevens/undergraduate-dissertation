\documentclass[12pt]{article}
\usepackage{apacite}
\usepackage{graphicx}
\usepackage[top=2.5cm, bottom=2.5cm, left=2.5cm, right=2.5cm]{geometry}
\usepackage[toc,page]{appendix}
\usepackage{hyperref}
\usepackage{fancyref}
\usepackage[round]{natbib}

\newcommand{\citetwo}[4]{(\citeauthor{#1}, \citeyear{#1}, p.~#2; \citeauthor{#3}, \citeyear{#3}, p.~#4)}

\begin{document}

\title{``Dissertation in Language Sciences"}
\author{s1107496}

\maketitle

\begin{abstract}
Particle swarm optimization (PSO) has been proposed as a means of modeling changes in human behaviour in a social context. PSO has also been shown to be an effective optimization method for non-differentiable problems with continuous search spaces, and has seen widespread use as such. In this paper, the language game presented in \citeauthor{rohde2012}'s ``Communicating with Cost-based Implicature: a Game-Theoretic Approach to Ambiguity" is modeled as a particle swarm optimization task, with the aim of creating a system which captures the results seen in \citeauthor{rohde2012}'s experiments. Two such possible systems are presented, one of which is successful in modeling these results. Further to this, PSO's suitability for use in language games with highly dynamic objective functions, how we might frame PSO's behaviour from game-theoretic and psycholinguistic standpoints, and predictions made by the successful PSO-based model are also discussed.
\end{abstract}



\section{Introduction}
Computational modeling is a powerful tool which has been used extensively in the field of linguistics. By building computational models of linguistic phenomena, it is possible to create simulations which explore the experimentally infeasible or impossible. Computational models can also be used to extrapolate from data collected in the lab, the results of which can be used to drive further studies in potentially fruitful directions. Further to this, computational models uniquely allow for direct inspection of their parameters and processes, which can shed light on aspects of human cognition. Of the multitudinous uses of modeling in the recent linguistics literature, notable examples include the use of iterated learning to demonstrate the spontaneous emergence of syntax from holistic languages \citep{kirby2002}, the use of incremental probabilistic Earley parsing to model garden-path sentence comprehension \citep{hale2001}, and the use of Bayesian statistics to model referential inference \citep{frank2012}. This paper extends this growing body of work by considering how modeling may be applied to the problem of referential coordination in a discourse context.

When using references, interlocutors must weigh the costs they incur when producing the reference, both in terms of construction and articulation, against the ease with which their conversational partners will be able to infer the intended referent. When speakers employ references that do not uniquely select a referent within the context of the discourse, they further risk their conversational partner failing to infer the correct referent from those possible. While communicating, interlocutors therefore jointly seek to minimize their expended effort while not violating the constraint imposed by their partner's ability to disambiguate the references used \citep[p.~65-66]{benz2005}. Because of this, the referential strategy a speaker opts to adopt must be sensitive to the relative costs of producing each reference as well as the evolving state of mappings between referential form and intended referent, as coordinated with their interlocutors. 

In ``Communicating with Cost-based Implicature: a Game-Theoretic Approach to Ambiguity" \citeyearpar{rohde2012}, \citeauthor{rohde2012} present an iterated language game in which participants aim to indicate an object to their partner via use of one of several possible referring terms. Participants gain points upon successful communication, but must spend points in order to communicate. Each referent has a corresponding unambiguous form that players may choose to send to
their partners, alternatively, players may send an ambiguous form with a different cost that could potentially indicate a number of referents. The studies run by \citeauthor{rohde2012} demonstrated the likelihood of a pair of participants to successfully coordinate their use of an ambiguous reference to be partially contingent on the relative costs of the unambiguous and ambiguous references.

Particle swarm optimization, as originally introduced by \citet*{kennedy1995}, can serve not only as a general optimization method, but also as a means of modeling human social behaviour, especially in the context of collaborative problem solving \citep{kennedy1997}. Modeling the language games used in \citeauthor{rohde2012} as particle swarm optimization tasks, if possible, would allow a more exhaustive exploration of the effects of various form costs on reference coordination, offering a comprehensive picture of the circumstances under which ambiguous form entrainment is possible and likely. Further to this, an accurate particle-swarm-based model of human coordination in a discourse setting might shed light on more fundamental aspects of human cognition and social interaction as represented via the model's parameters. A successful model could also suggest particle swarm optimization's suitability for the modeling of other linguistic phenomena. 

Conversely, if particle swarm optimization is not a viable means of modeling the results observed in \citeauthor{rohde2012}, this could suggest a fundamental difference between human linguistic behaviours and other social behaviours as successfully modeled via PSO. A negative finding might also suggest that specialized models, such as \citeauthor{liu2014}'s ``human behaviour-based particle swarm optimization" \citeyearpar{liu2014}, are required for the modeling of more complex human social interactions.

To address these possibilities, this paper models \citeauthor{rohde2012}'s experiments as a particle swarm optimization task, utilizing a mixed strategy search space to represent form production and comprehension. Two model variants are presented. In the first, particles are restricted to a unit interval search space, resulting in a model which fails to capture the effects seen in \citeauthor{rohde2012} The second model, which utilizes a technique suggested in \citealt[p.~252]{engelbrecht2005} to allow for an unrestricted search space, exhibits responses to form costs in keeping proportionally with the results found by \citeauthor{rohde2012}.



\section{Background}
\subsection{Particle swarm optimization}
\subsubsection{Overview}
Particle swarm optimization was first formulated by \cite{kennedy1995} in an attempt to model human social behaviour. The initial model was based off of \cite{heppner1990}'s work in modeling bird flocking and roosting behaviour in two-dimensional space, refined to allow for an arbitrary number of dimensions and for particles to share the same or arbitrarily close positions within said multidimensional space. \citeauthor{kennedy1995} demonstrated that their new optimization method was suitable for use not only in modeling human social behaviour, but also for the general-purpose optimization of non-linear continuous problems. Specifically, particle swarm optimization was found to be successful both in optimizing the weights of an artificial neural network and in the Schaffer $f6$ problem, a standard benchmark for general-purpose optimization methods \citep{davis1991}.

The concept behind the particle swarm optimization algorithm is relatively simple. Potential solutions to some problem with $n$ dimensions are represented as particles existing within an $n$-dimensional search space. Each particle has both a position within this space and some velocity vector. Each particle also keeps track of the best position it has been in as evaluated by the given objective function, which is known as its ``personal" best position; a ``global" best position representing the best position found by all particles across a swarm or group is also maintained \citep{chong2013}. A particle swarm optimization task is run iteratively, until some stopping criterion is reached \citep[p.~80]{solnon2010}. Every iteration begins by updating the velocities of all particles. In doing so, a particle is acted upon by two forces: the attraction of the particle to its personal previous best known position, as governed by a ``cognitive" parameter, and its attraction to the best known position within its group, as governed by a ``social" parameter \citep{chong2013}. The particle also maintains some momentum from its previous velocity. Following this, particle positions are updated in accordance with their velocities, with new positions being evaluated via the given objective function and best-known personal and global positions updated where appropriate. Initial particle velocities are assigned randomly; in doing so, exploration of the search space is encouraged, reducing the likelihood of the swarm as a whole becoming caught in a local extremum or local extrema of the objective function \citetwo{yang2014}{32}{solnon2010}{78, 81}. 

As noted by \citet*[p.~99]{yang2014}, particle swarm optimization's applicability to a large domain of problems paired with its straightforward conceptual groundings and implementation have spurred on its widespread use in a number of fields, as well as the development of numerous refinements and derivatives of the original algorithm. This paper uses the well-known inertial variant of particle swarm optimization, which offers a ``noticeable improvement" in speedy convergence on good solutions as compared to standard PSO \citep[p.~101]{yang2014}. Further to this variant, for the purposes of this paper each particle's position is updated with respect to the best-found solution within a predefined neighborhood or group of particles, as opposed to the global best-found solution, as presented in \citet*[p.~79]{solnon2010}. When these groups do not intersect, this is simply equivalent to running a number of independent particle swarm optimization tasks equal to the number of groups.

% mention velocity clamping...? Engelbrecht 109 and in others


\subsubsection{Previous work}
\begin{itemize}


\item Noto et al. (2013) ``Agent-based Social Simulation Model for Analyzing Human Behaviors using Particle Swarm Optimization"

\item Cheng et al. (2008) ``A modified Particle Swarm Optimization-based human behavior modeling for emergency evacuation simulation system"

\item Franken \& Engelbrecht (2005) ``Particle swarm optimization approaches to coevolve strategies for the iterated prisoner's dilemma"

\item should be noted that the approach used here differs from traditional application of PSO for linguistics modelling, i.e. using PSO directly to model human optimization of language game (as objective function) vs. fine-tuning parameters
\end{itemize}
\subsubsection{Formulation and parameters}
In the formulation of PSO employed in this paper, a particle $i$ with position $x_i$ has a velocity $v_i$ at time $t$ such that
\[
v_i^t = \theta(t) v_i^{t-1} + \alpha \epsilon_1 (x_{N(i)}^* - x_i^{t-1}) + \beta \epsilon_2 (x_i^* - x_i^{t-1})
\],
where $\theta$ is the inertial scheduling function, $\alpha$ is the cognitive component, $\beta$ is the social component, $x_{N(i)}^*$ is the global best position known for $i$'s neighborhood or group $N(i)$, $x_i^*$ is $i$'s personal best known position, and $\epsilon_1$ and $\epsilon_2$ are randomly chosen values within $\left(0.0, 1.0\right]$. Although $\theta$ can take many forms including, most commonly, a constant function \citep[p.~101]{yang2014}, for the purposes of this paper $\theta$ is defined with respect to a base inertia $\tau$ and inertial dampening factor $\sigma$ such that
\[
\theta(t) = \frac{\tau}{\sigma^t} 
\]
Finally, the position $x$ of $i$ at time $t$ is defined as
\[
x_i^t = x_i^{t-1} + v_i^t 
\]

\subsection{Referential communication games}
\subsubsection{Overview}
\subsubsection{Game theoretic approaches}
\begin{itemize}
\item Benz, Jaeger, Van Rooij (2005) - general application of evolutionary game theory to reference

\item Degen (2012) - Iterated best response
\end{itemize}
\subsubsection{Psycholinguistic approaches}
\begin{itemize}
\item Golland, Liang, Klein (2010) - speaker modeling listener
\end{itemize}
\subsubsection{Rohde et al. 2012}
\begin{itemize}
\item When costs are similar, fewer pairs converge
\end{itemize}

\section{Methods}
\subsection{Modeling language game as PSO task}
\subsubsection{Mixed-strategy search space}
\subsubsection{Particles are agents}
\begin{itemize}
\item mention highly dynamic objective function, and why traditional solutions to this (Engelbrecht p320-326) are not appropriate in context

\item possibly, mention that this is coevolutionary (pair responding to changes each other make to objective function) - see if there is any research on this (e.g. franken and engelbrecht 2005)

\item Possibly measure severity? (Engelbrecht p50) - at least mention env. - probably type 3

\item niching? 

\item coevolutionary game learning (engelbrecht 349)
\end{itemize}
\subsubsection{Criteria for ``ambiguous coordination" and ``unambiguous coordination"}
\subsubsection{Model with particles restricted to search space vs. not}

\subsection{Optimizing parameters for PSO with PSO}
\subsubsection{Evaluating error}

\section{Results}
\subsection{Results of meta-optimization (PSO parameters)}
\subsection{Results for model with search space restriction}
\begin{itemize}
\item Rates at approximately correct levels, but reacts incorrectly to cost adjustments
\end{itemize}
\subsection{Results for model without}
\begin{itemize}
\item Ratios of rates track well onto data from experiments, but rates much lower
\end{itemize}

\section{Discussion}
\begin{itemize}
\item Perhaps mention coevolutionary aspect and difficulties applying PSO to radically dynamic objective functions here?

\item Model has response in keeping with experimental data without agents performing complex modeling (i.e. general optimization is sufficient to produce effects)

\item Generating predictions from PSO model
\end{itemize}
\section{Conclusion}



\bibliographystyle{apacite}
\bibliography{dissertation.bib}


\end{document}
