\documentclass[12pt]{article}
\usepackage{graphicx}
\usepackage[top=2.5cm, bottom=2.5cm, left=2.5cm, right=2.5cm]{geometry}
\usepackage[toc,page]{appendix}
\usepackage{hyperref}
\usepackage{fancyref}
\usepackage[round]{natbib}

\begin{document}

\title{Functional Reactive Programming in elm}
\author{s1107496}

\maketitle

\begin{abstract}
Particle swarm optimization (PSO) has been proposed as a means of modelling changes in human behaviour in a social context. PSO has also been shown to be an effective optimization method for non-differentiable problems with continuous search spaces, and has seen widespread use as such. In this report, the language game presented in Rohde et al.'s "Communicating with Cost-based Implicature: a Game-Theoretic Approach to Ambiguity" is modelled as a particle swarm optimization task, with the aim of creating a system which captures the results seen in Rodhe et al.'s experiments. Further to this, PSO's suitability for use in language games with dynamic fitness functions and how we might frame PSO's behaviour from game-theoretic and psycholinguistic standpoints are also discussed.
\end{abstract}


\section{Introduction}

\section{Background}

\section{Method}

\section{Results}

\section{Discussion}

\section{Conclusion}

\end{document}