\documentclass[12pt,a4paper]{article}
\usepackage{apacite}
\usepackage{graphicx}
%\usepackage[top=4cm, bottom=4cm, outer=5cm, inner=3cm, heightrounded, marginparwidth=3cm, marginparsep=1cm]{geometry} %TODO: comment out
%\usepackage[marginparwidth=2.5cm, marginparsep=2cm]{geometry}
\usepackage[toc,page]{appendix}
\usepackage{hyperref}
\usepackage{fancyref}
\usepackage[round]{natbib}
\usepackage{fancyhdr}
\usepackage{setspace}
\usepackage{pgf}
%\usepackage[justification=centering]{caption}
\pagestyle{fancy}

\newcommand{\citetwo}[4]{(\citeauthor{#1}, \citeyear{#1}, p.~#2; \citeauthor{#3}, \citeyear{#3}, p.~#4)}

\newcommand{\HRule}{\rule{\linewidth}{0.5mm}}



\begin{document}

\begin{titlepage}

\begin{center}

\textsc{\LARGE University of Edinburgh}\\[1cm]

\textsc{\Large Dissertation in Language Sciences}\\[1.5cm]

\HRule \\[0.4cm]
{ \huge \bfseries Understanding Referential Coordination as a Particle Swarm Optimization Task \\[0.4cm] }

\HRule \\[1.5cm]


\noindent
\begin{minipage}{0.4\textwidth}
\begin{flushleft} \large
\emph{Exam Number:}\\
---
\end{flushleft}
\end{minipage}%
\begin{minipage}{0.4\textwidth}
\begin{flushright} \large
\emph{Supervisor:} \\
Dr.~Hannah \textsc{Rohde}
\end{flushright}
\end{minipage}

\vfill

% Bottom of the page
{\large 1st April, 2015}
%\title{Understanding Referential Coordination as a Particle Swarm Optimization Task}
%\author{s1107496}
%\date{1st April, 2015}

%\maketitle

\end{center}
\end{titlepage}

\onehalfspace
%\doublespace

\begin{abstract}
In ``Communicating with Cost-based Implicature: a Game-Theoretic Approach to Ambiguity" \citeyearpar{rohde2012}, \citeauthor{rohde2012} sought to investigate the effects of ambiguous and unambiguous form costs on referential coordination. To do so, \citeauthor{rohde2012} ran a number of experiments in which participants played an iterated referential coordination language game. This paper takes a novel approach by modeling their results using particle swarm optimization (PSO), a general-purpose optimization method for non-differentiable problems with continuous search spaces. Two PSO-based models are presented and shown to perform well against a baseline model; predictions from the more favourable of the two models are also presented for several variants of the \citeauthor{rohde2012} language games.
% no foreshadowing of discussion here... but maybe that's okay
\end{abstract}

\pagebreak


\tableofcontents

\pagebreak


\section{Introduction}
An open question in the field of linguistics is how participants in a conversation coordinate their use of referring expressions. When producing referring expressions, interlocutors must weigh the costs they incur when producing the expression, both in terms of construction and articulation, against the ease with which their conversational partners will be able to infer the intended referent. When speakers employ referring expressions that do not uniquely select a referent within the context of the discourse, they further risk their conversational partner failing to infer the correct referent from those possible. For example, consider a discourse context in which there are three plausible referents, a chocolate Labrador, a black Poodle, and a brown American Water Spaniel/German Longhaired Pointer mix. Given the high cost of producing an unambiguous referring expression for the latter, a speaker might attempt to refer to the American Water Spaniel/German Longhaired Pointer mix as ``that brown dog". However, this expression could also plausibly be used to indicate the chocolate Labrador, and should the speaker's communicative partner interpret the referring expression as such, rectifying this misinterpretation is likely to be very costly. While communicating, interlocutors therefore jointly seek to minimize their expended effort while not violating the constraint imposed by their partner's ability to disambiguate the referring expressions used \citep[p.~65-66]{benz2005}. Consequentially, the referential strategy a speaker opts to adopt must be sensitive to the relative costs of producing each reference as well as the evolving state of mappings between referential form and intended referent, as coordinated with their interlocutors. 

In ``Communicating with Cost-based Implicature: a Game-Theoretic Approach to Ambiguity" \citeyearpar{rohde2012}, \citeauthor{rohde2012} present an iterated language game in which participants aim to indicate an object to their partner via use of one of several possible referring terms. Participants gain points upon successful communication, but must spend points in order to communicate. Each referent has a corresponding unambiguous form that players may choose to send to their partners; alternatively, players may send an ambiguous form with a different cost that could potentially indicate a number of referents. The studies run by \citeauthor{rohde2012} demonstrated the likelihood of a pair of participants to successfully coordinate their use of an ambiguous reference to be partially contingent on the relative costs of the unambiguous and ambiguous referring expressions.

This paper seeks to introduce a computational model of \citeauthor{rohde2012}'s findings, and in doing so consider how modeling may be applied to the problem of referential coordination. Such a model would allow simulation of the experimentally infeasible or impossible, as well as enabling extrapolation from the data collected in the lab, the results of which can be used to drive further studies in potentially fruitful directions. Further to this, computational models in general uniquely allow for direct inspection of their parameters and processes, which can shed light on aspects of human cognition. As such, computational modeling has seen extensive use in the field of linguistics, notable examples including the use of iterated learning to demonstrate the spontaneous emergence of syntax \citep{kirby2002}, the use of incremental probabilistic parsing to model garden-path sentence comprehension \citep{hale2001}, and the use of Bayesian statistics to model referential inference \citep{frank2012}.

When selecting a modeling approach for \cite{rohde2012}, three crucial considerations must be taken into account: first, the chosen approach needs to represent the internal state of a participant with respect to the game, and to allow the participant's actions to be derived from this state, second, changes in a participant's internal state within the model need to be reflective of the participant's communicative success with their partner, even as their partner's internal state itself changes, and third, the modeling approach should be easily extensible to many variants of the \citeauthor{rohde2012} language games; particle swarm optimization is suitable in all three regards. As originally introduced by \citet*{kennedy1995}, particle swarm optimization (PSO) can serve not only as a general optimization method, but also as a means of modeling human social behaviour, especially in the context of collaborative problem solving \citep{kennedy1997}. These factors make the technique an ideal candidate for modeling the \citeauthor{rohde2012} studies. 

If possible, a particle-swarm-based model would allow a more exhaustive exploration of the effects of various form costs on referential coordination, offering a comprehensive picture of the circumstances under which ambiguous form entrainment is possible and likely. Further to this, an accurate particle-swarm-based model of human coordination in a discourse setting might shed light on more fundamental aspects of human cognition and social interaction as represented via the model's parameters. A successful model could also suggest PSO's suitability for the modeling of other linguistic phenomena. 

Conversely, if PSO is not a viable means of modeling the results observed in \citeauthor{rohde2012}, this could suggest a fundamental difference between human linguistic behaviours and other social behaviours as successfully modeled via PSO. A negative finding might also suggest that specialized methods are required to model more complex human social interactions.

To address these possibilities, this paper models \citeauthor{rohde2012}'s experiments as a PSO task, utilizing a mixed strategy search space to represent form production and comprehension. Two model variants as optimized to the \citeauthor{rohde2012} language games are presented, which differ in their treatment of the search space; these models are shown to compare favorably to a baseline PSO model. Both models respond to changes in form costs in a promisingly similar fashion to the observed experimental data, although neither perfectly replicates human trials.


\section{Background}
\subsection{Referential coordination}
\subsubsection{Overview}
Language is an inherently cooperative endeavour. To successfully understand and be understood by their interlocutors, speakers must actively coordinate their use of referring expressions, and rely on their communicative partners to do the same. The mechanisms by which this coordination can occur with relative facility are an active topic of research, and of special interest is to what extent speakers maintain internal models of their interlocutors in order to inform their communicative strategies. Research has also been devoted to the processes by which speakers balance the costs they incur when communicating against those incurred by their interlocutors, and how the interplay of these combined factors directs the evolution of referential mappings during discourse. While \citeauthor{rohde2012} is emblematic of a game-theoretic approach to these issues, the psycholinguistics literature has also covered them in detail, and thus it is necessary to provide an overview of previous work done in both communities in order to sufficiently establish a point of departure for this work.

\subsubsection{Game theoretic approaches}
Game theory provides a methodology for understanding agents' actions by modeling them as strategies within games. When deciding on which actions to choose, agents are said to attempt to maximize their expected utility by leveraging their knowledge of the game's state \citep{benz2005}. As applied to linguistics, game theoretic concepts can be used to describe a number of phenomena; for instance, \cite{jaeger2008} demonstrates how an evolutionary game-theoretic framework (in which systematic stability is maximized) can be successfully applied to the division of the vowel space in order to predict the vowel systems of modern-day languages.

\cite{lewis1969} established much of the foundation necessary to frame language in game-theoretic terms, including providing an account of the establishment of conventions as a coordinative (this is to say, positive-sum) game ``in which the agents' interests coincide perfectly" \citep[p.~14]{lewis1969}.
His work has been built upon in order to apply game theory to the more specific problem of referential coordination. For example, by convention, the use of more general but costly referring expressions implicitly excludes the referents of easily accessed and more specific forms (e.g. ``cutter" versus ``knife", or ``some" versus ``all"). The establishment of this convention has been explained using an evolutionary (but ontogenetic) game-theoretic approach: both speaker and hearer benefit from an interpretation of the former which carries a greater degree of information (e.g. ``cutters which are not knives", ``some but not all") \citep{benz2005}. More recent work by \cite{degen2012} has used game theoretic models to predict participant behaviour in referential inference tasks with some success, although the researchers expressed the need for more nuanced, comprehensive models. 

\subsubsection{Psycholinguistic approaches}
\begin{itemize}
\item Golland, Liang, Klein (2010) - speaker modeling listener
\item Herb Clark - audience design
\item cf. Keysar and Horton - audience design without internal audience modeling
\end{itemize}
\subsubsection{Rohde et al. 2012}
\begin{itemize}
\item When costs are similar, fewer pairs converge
\end{itemize}


\subsection{Particle swarm optimization}
\subsubsection{Overview}
Particle swarm optimization was first formulated by \cite{kennedy1995} in an attempt to model human social behaviour. The initial model was based off of \cite{heppner1990}'s work in modeling bird flocking and roosting behaviour in two-dimensional space, refined to allow for an arbitrary number of dimensions and for particles to share the same or arbitrarily close positions within said multidimensional space. \citeauthor{kennedy1995} demonstrated that their new optimization method was suitable for use not only in modeling human social behaviour, but also for the general-purpose optimization of non-linear continuous problems. Specifically, PSO was found to be successful both in optimizing the weights of an artificial neural network and in the Schaffer $f6$ problem, a standard benchmark for general-purpose optimization methods \citep{davis1991}.

The concept behind the PSO algorithm is relatively simple. Potential solutions to some problem with $n$ dimensions are represented as particles existing within an $n$-dimensional search space. Each particle has both a position within this space and some velocity vector. Each particle also keeps track of the best position it has been in as evaluated by the given objective function, which is known as its ``personal" best position; a ``global" best position representing the best position found by all particles across a swarm or group is also maintained \citep{chong2013}. A PSO task is run iteratively, until some stopping criterion is reached \citep[p.~80]{solnon2010}. Every iteration begins by updating the velocities of all particles. In doing so, a particle is acted upon by two forces: the attraction of the particle to its personal previous best known position, as governed by a ``cognitive" parameter, and its attraction to the best known position within its group, as governed by a ``social" parameter \citep{chong2013}. The particle also maintains some momentum from its previous velocity. Following this, particle positions are updated in accordance with their velocities, with new positions being evaluated via the given objective function and best-known personal and global positions updated where appropriate. Initial particle velocities are assigned randomly; in doing so, exploration of the search space is encouraged, reducing the likelihood of the swarm as a whole becoming caught in a local extremum or local extrema of the objective function \citetwo{yang2014}{32}{solnon2010}{78, 81}. 

As noted by \citet*[p.~99]{yang2014}, PSO is applicable to a large domain of problems, has straightforward conceptual groundings, and is simple to implement; these factors have spurred on its widespread use in a number of fields, and resulted in the development of numerous refinements and derivatives of the original algorithm. This paper uses the well-known inertial variant of PSO, which offers a ``noticeable improvement" in speedy convergence on good solutions as compared to standard PSO \citep[p.~101]{yang2014}. Further to this variant, for the purposes of this paper each particle's position is updated with respect to the best-found solution within a predefined neighborhood or group of particles, as opposed to the global best-found solution, as presented in \citet*[p.~79]{solnon2010}. When these particle groups do not intersect, this is simply equivalent to running a number of independent PSO tasks equal to the number of groups.

% mention velocity clamping...? Engelbrecht 109 and in others


\subsubsection{Previous work}
\label{sec:pso_prev_work}

While PSO has been applied to a number of problems within the fields of linguistics and psychology, its primary use has been as a means of optimizing parameters for other models, as opposed to direct application as a model in and of itself (e.g. \citeauthor{chatterjee2005} \citeyear{chatterjee2005}; \citeauthor{mehdad2009} \citeyear{mehdad2009}). Notable exceptions to this trend include the use of PSO to perform unsupervised phoneme clustering \citep{ahmadi2007} and the modeling of human emergency evacuation behaviours via PSO \citep{cheng2008}.

PSO has likewise been applied to game learning, often using a coevolutionary paradigm in which agents play against one another in order to evaluate their fitness. However, traditionally this method has involved PSO over a search space of neural network weights, where the neural networks are used to choose actions given a game state, or in cases where the ``game" is a classical constraint optimization problem, such as the $n$-queens problem \citep[p.~349-351]{engelbrecht2005}. By contrast, in the new approach presented in this paper, the positions of particles themselves comprise agents' internal states, which directly define a mixed strategy (see \autoref{sec:search_space}).  

\subsubsection{Formulation and parameters}
In the formulation of PSO employed in this paper, a particle $i$ with position $x_i$ has a velocity $v_i$ at time $t$ such that
\begin{equation}
v_i^t = \theta(t) \cdot v_i^{t-1} + \alpha \cdot \epsilon_1 \cdot (x_{N(i)}^* - x_i^{t-1}) + \beta \cdot \epsilon_2 \cdot (x_i^* - x_i^{t-1})
\end{equation},
where $\theta$ is the inertial scheduling function, $\alpha$ is the cognitive component, $\beta$ is the social component, $x_{N(i)}^*$ is the global best position known for $i$'s neighborhood or group $N(i)$, $x_i^*$ is $i$'s personal best known position, and $\epsilon_1$ and $\epsilon_2$ are randomly chosen values within $\left(0.0, 1.0\right]$. Although $\theta$ can take many forms including, most commonly, a constant function \citep[p.~101]{yang2014}, for the purposes of this paper $\theta$ is defined with respect to a base inertia $\tau$ and inertial dampening factor $\sigma$ such that
\begin{equation}
\theta(t) = \frac{\tau}{\sigma^t} 
\end{equation}
Finally, the position $x$ of $i$ at time $t$ is defined as
\begin{equation}
x_i^t = x_i^{t-1} + v_i^t 
\end{equation}

\section{Methods}

\subsection{Overview}

To model referential coordination within the \citeauthor{rohde2012} language game as a PSO task, individual participants were modeled as particles in groups of size $2$. In exploring the search space of possible game strategies, each particle sought to maximize its score within the language game in relation to the strategy of its partner. The scoring function of the language game itself was parameterized on the relative costs of unambiguous forms, as in \citeauthor{rohde2012}. Further to this, the parameters of the PSO algorithm (cognitive component, social component, etc.) were optimized to best replicate \citeauthor{rohde2012}'s experimental findings. Two models were ultimately produced, which differed in their handling of the search space; both were evaluated against the experimental data.

\subsection{Models}
\subsubsection{Search space}
\label{sec:search_space}

In modeling the \citeauthor{rohde2012} language game as a PSO task, the form a solution to the game takes must be considered in order to establish a search space.  To do so, the game-theoretic notion of a mixed strategy was adopted, in which each possible action $a$ within a game is performed by a participant $i$ with some probability $P_i(a)$ \citep[p.~22]{benz2005}. For the given language game, for each referent $r$ the participant can be said to maintain a probability of using the associated ambiguous form $A$, $P_i(A|r)$. Conversely, the probability of a participant using the available unambiguous form for $r$ can be given as $1 - P_i(A|r)$. Therefore, in an instance of the \citeauthor{rohde2012} language game with $n$ possible referents, a participant's strategy was represented with $n$ independent probabilities, yielding an $n$-dimensional search space. 

It is important to note that this does not constitute a traditional mixed strategy, in that a participant's strategy is not a probability distribution. This is to say, the sum of a participant's probability of using the ambiguous referring expression over all referents may not equal $1$. As an example, given two referents $r_1$ and $r_2$, a participant is capable of opting to use the ambiguous form for neither. In this sense, it may be more accurate to state that a participant $i$ maintains a separate mixed strategy for each referent $r$, where, for the ambiguous form $A$ and unambiguous form $U$, $P_i(A|r) + P_i(U|r) = 1$.

It is also important to note that for the purposes of this paper, participants were assumed to seek to optimize their strategies for groups of referents sharing the same ambiguous form independently of other groups. As such, each of the studies presented in \citeauthor{rohde2012} were considered as two independent language games being run concurrently, and the PSO approach presented used 3-dimensional search spaces as opposed to 6-dimensional search spaces. This assumption was justified by the observation that participant pairs in \citeauthor{rohde2012}'s second study were able to coordinate their use of the ambiguous form for one group, but not the other. 

Finally, because each dimension in the search space defined above reflects a probability, values outside the interval $[0, 1]$ are invalid. A number of approaches for adapting PSO to constrained optimization problems have been suggested in the literature; two plausible alternatives as presented in \cite{engelbrecht2005} were considered, resulting in two competing models. 
In the first model, particles which moved outside the desired search space immediately had a repair method applied to them, whereby they were relocated to the nearest point which did not violate the given constraints. This model will be referred to as the ``repair" model. In the second model, particles were allowed to move freely within the search space, including to regions that violated constraints. However, particles not meeting the given constraints were allowed to update neither their personal best known position nor the global best position. Because all personal best positions and the global best position therefore remained in feasible space, particles were naturally attracted back to the region of the search space which respected the constraints. The model utilizing this technique will be referred to as the ``rejection" model. In both models, initial particle positions were assigned randomly within the valid search space.


 
\subsubsection{Objective function}
In applying PSO to the \citeauthor{rohde2012} language game, an appropriate representation of the game's goals must be formulated as an objective function. In the game as presented, the expected number of points awarded to participant $i$ given their partner $j$ when asked to communicate referent $r$ can be calculated as follows:
\begin{equation}
EP_{i}(r|j) = P_i(A|r)(S \cdot P_j(r|A) - cost_A) \,+\, (1 - P_i(A|r))(S - cost_r) 
\end{equation},
where $cost_A$ is the cost of production of the ambiguous form, $S$ is the number of points awarded on successful communication (set at 80 and 85 in \citeauthor{rohde2012}, respectively), and $cost_r$ is the cost of production of the unambiguous form for $r$.

Each round, the actual number of points awarded to participants is dependent on samples from $P_i$ and $P_j$, as well as the randomly-chosen $r$. As such, without the strategies of $i$ or $j$ changing, there are for any given round a number of possible scores $i$ might attain. Relying on the game unmodified as the objective function for a PSO task would therefore be imprudent, as an inconsistent objective function would be much harder to optimize. Instead, the objective function $f$ used for both models was chosen as 
\begin{equation}
f(i) = \sum_{r \in R} EP_{i}(r|j) + EP_{j}(r|i)
\end{equation}

\begin{figure}
\centering
%\includegraphics[width=\textwidth]{objective_function_cropped.png}
\scalebox{.725}{%% Creator: Matplotlib, PGF backend
%%
%% To include the figure in your LaTeX document, write
%%   \input{<filename>.pgf}
%%
%% Make sure the required packages are loaded in your preamble
%%   \usepackage{pgf}
%%
%% Figures using additional raster images can only be included by \input if
%% they are in the same directory as the main LaTeX file. For loading figures
%% from other directories you can use the `import` package
%%   \usepackage{import}
%% and then include the figures with
%%   \import{<path to file>}{<filename>.pgf}
%%
%% Matplotlib used the following preamble
%%   \usepackage{fontspec}
%%   \setmainfont{DejaVu Serif}
%%   \setsansfont{DejaVu Sans}
%%   \setmonofont{DejaVu Sans Mono}
%%
\begingroup%
\makeatletter%
\begin{pgfpicture}%
\pgfpathrectangle{\pgfpointorigin}{\pgfqpoint{7.300000in}{4.075000in}}%
\pgfusepath{use as bounding box}%
\begin{pgfscope}%
\pgfsetbuttcap%
\pgfsetroundjoin%
\definecolor{currentfill}{rgb}{1.000000,1.000000,1.000000}%
\pgfsetfillcolor{currentfill}%
\pgfsetlinewidth{0.000000pt}%
\definecolor{currentstroke}{rgb}{1.000000,1.000000,1.000000}%
\pgfsetstrokecolor{currentstroke}%
\pgfsetdash{}{0pt}%
\pgfpathmoveto{\pgfqpoint{0.000000in}{0.000000in}}%
\pgfpathlineto{\pgfqpoint{7.300000in}{0.000000in}}%
\pgfpathlineto{\pgfqpoint{7.300000in}{4.075000in}}%
\pgfpathlineto{\pgfqpoint{0.000000in}{4.075000in}}%
\pgfpathclose%
\pgfusepath{fill}%
\end{pgfscope}%
\begin{pgfscope}%
\pgfsetbuttcap%
\pgfsetroundjoin%
\definecolor{currentfill}{rgb}{1.000000,1.000000,1.000000}%
\pgfsetfillcolor{currentfill}%
\pgfsetlinewidth{0.000000pt}%
\definecolor{currentstroke}{rgb}{0.000000,0.000000,0.000000}%
\pgfsetstrokecolor{currentstroke}%
\pgfsetstrokeopacity{0.000000}%
\pgfsetdash{}{0pt}%
\pgfpathmoveto{\pgfqpoint{0.619395in}{0.804260in}}%
\pgfpathlineto{\pgfqpoint{3.085874in}{0.804260in}}%
\pgfpathlineto{\pgfqpoint{3.085874in}{3.270740in}}%
\pgfpathlineto{\pgfqpoint{0.619395in}{3.270740in}}%
\pgfpathclose%
\pgfusepath{fill}%
\end{pgfscope}%
\begin{pgfscope}%
\pgfpathrectangle{\pgfqpoint{0.619395in}{0.804260in}}{\pgfqpoint{2.466479in}{2.466479in}} %
\pgfusepath{clip}%
\pgftext[at=\pgfqpoint{0.619395in}{0.804260in},left,bottom]{\pgfimage[interpolate=true,width=2.480000in,height=2.480000in]{objective_function-img0.png}}%
\end{pgfscope}%
\begin{pgfscope}%
\pgfpathrectangle{\pgfqpoint{0.619395in}{0.804260in}}{\pgfqpoint{2.466479in}{2.466479in}} %
\pgfusepath{clip}%
\pgfsetrectcap%
\pgfsetroundjoin%
\pgfsetlinewidth{1.003750pt}%
\definecolor{currentstroke}{rgb}{0.000000,0.000000,0.000000}%
\pgfsetstrokecolor{currentstroke}%
\pgfsetdash{}{0pt}%
\pgfpathmoveto{\pgfqpoint{0.866043in}{3.024092in}}%
\pgfusepath{stroke}%
\end{pgfscope}%
\begin{pgfscope}%
\pgfpathrectangle{\pgfqpoint{0.619395in}{0.804260in}}{\pgfqpoint{2.466479in}{2.466479in}} %
\pgfusepath{clip}%
\pgfsetbuttcap%
\pgfsetroundjoin%
\definecolor{currentfill}{rgb}{0.000000,0.000000,0.000000}%
\pgfsetfillcolor{currentfill}%
\pgfsetlinewidth{2.007500pt}%
\definecolor{currentstroke}{rgb}{0.000000,0.000000,0.000000}%
\pgfsetstrokecolor{currentstroke}%
\pgfsetdash{}{0pt}%
\pgfsys@defobject{currentmarker}{\pgfqpoint{-0.055556in}{-0.055556in}}{\pgfqpoint{0.055556in}{0.055556in}}{%
\pgfpathmoveto{\pgfqpoint{-0.055556in}{-0.055556in}}%
\pgfpathlineto{\pgfqpoint{0.055556in}{0.055556in}}%
\pgfpathmoveto{\pgfqpoint{-0.055556in}{0.055556in}}%
\pgfpathlineto{\pgfqpoint{0.055556in}{-0.055556in}}%
\pgfusepath{stroke,fill}%
}%
\begin{pgfscope}%
\pgfsys@transformshift{0.866043in}{3.024092in}%
\pgfsys@useobject{currentmarker}{}%
\end{pgfscope}%
\end{pgfscope}%
\begin{pgfscope}%
\pgfsetbuttcap%
\pgfsetroundjoin%
\definecolor{currentfill}{rgb}{0.000000,0.000000,0.000000}%
\pgfsetfillcolor{currentfill}%
\pgfsetlinewidth{0.501875pt}%
\definecolor{currentstroke}{rgb}{0.000000,0.000000,0.000000}%
\pgfsetstrokecolor{currentstroke}%
\pgfsetdash{}{0pt}%
\pgfsys@defobject{currentmarker}{\pgfqpoint{0.000000in}{0.000000in}}{\pgfqpoint{0.000000in}{0.055556in}}{%
\pgfpathmoveto{\pgfqpoint{0.000000in}{0.000000in}}%
\pgfpathlineto{\pgfqpoint{0.000000in}{0.055556in}}%
\pgfusepath{stroke,fill}%
}%
\begin{pgfscope}%
\pgfsys@transformshift{0.619395in}{0.804260in}%
\pgfsys@useobject{currentmarker}{}%
\end{pgfscope}%
\end{pgfscope}%
\begin{pgfscope}%
\pgfsetbuttcap%
\pgfsetroundjoin%
\definecolor{currentfill}{rgb}{0.000000,0.000000,0.000000}%
\pgfsetfillcolor{currentfill}%
\pgfsetlinewidth{0.501875pt}%
\definecolor{currentstroke}{rgb}{0.000000,0.000000,0.000000}%
\pgfsetstrokecolor{currentstroke}%
\pgfsetdash{}{0pt}%
\pgfsys@defobject{currentmarker}{\pgfqpoint{0.000000in}{-0.055556in}}{\pgfqpoint{0.000000in}{0.000000in}}{%
\pgfpathmoveto{\pgfqpoint{0.000000in}{0.000000in}}%
\pgfpathlineto{\pgfqpoint{0.000000in}{-0.055556in}}%
\pgfusepath{stroke,fill}%
}%
\begin{pgfscope}%
\pgfsys@transformshift{0.619395in}{3.270740in}%
\pgfsys@useobject{currentmarker}{}%
\end{pgfscope}%
\end{pgfscope}%
\begin{pgfscope}%
\pgftext[x=0.619395in,y=0.748705in,,top]{{\sffamily\fontsize{12.000000}{14.400000}\selectfont \(\displaystyle 0.0\)}}%
\end{pgfscope}%
\begin{pgfscope}%
\pgfsetbuttcap%
\pgfsetroundjoin%
\definecolor{currentfill}{rgb}{0.000000,0.000000,0.000000}%
\pgfsetfillcolor{currentfill}%
\pgfsetlinewidth{0.501875pt}%
\definecolor{currentstroke}{rgb}{0.000000,0.000000,0.000000}%
\pgfsetstrokecolor{currentstroke}%
\pgfsetdash{}{0pt}%
\pgfsys@defobject{currentmarker}{\pgfqpoint{0.000000in}{0.000000in}}{\pgfqpoint{0.000000in}{0.055556in}}{%
\pgfpathmoveto{\pgfqpoint{0.000000in}{0.000000in}}%
\pgfpathlineto{\pgfqpoint{0.000000in}{0.055556in}}%
\pgfusepath{stroke,fill}%
}%
\begin{pgfscope}%
\pgfsys@transformshift{1.112691in}{0.804260in}%
\pgfsys@useobject{currentmarker}{}%
\end{pgfscope}%
\end{pgfscope}%
\begin{pgfscope}%
\pgfsetbuttcap%
\pgfsetroundjoin%
\definecolor{currentfill}{rgb}{0.000000,0.000000,0.000000}%
\pgfsetfillcolor{currentfill}%
\pgfsetlinewidth{0.501875pt}%
\definecolor{currentstroke}{rgb}{0.000000,0.000000,0.000000}%
\pgfsetstrokecolor{currentstroke}%
\pgfsetdash{}{0pt}%
\pgfsys@defobject{currentmarker}{\pgfqpoint{0.000000in}{-0.055556in}}{\pgfqpoint{0.000000in}{0.000000in}}{%
\pgfpathmoveto{\pgfqpoint{0.000000in}{0.000000in}}%
\pgfpathlineto{\pgfqpoint{0.000000in}{-0.055556in}}%
\pgfusepath{stroke,fill}%
}%
\begin{pgfscope}%
\pgfsys@transformshift{1.112691in}{3.270740in}%
\pgfsys@useobject{currentmarker}{}%
\end{pgfscope}%
\end{pgfscope}%
\begin{pgfscope}%
\pgftext[x=1.112691in,y=0.748705in,,top]{{\sffamily\fontsize{12.000000}{14.400000}\selectfont \(\displaystyle 0.2\)}}%
\end{pgfscope}%
\begin{pgfscope}%
\pgfsetbuttcap%
\pgfsetroundjoin%
\definecolor{currentfill}{rgb}{0.000000,0.000000,0.000000}%
\pgfsetfillcolor{currentfill}%
\pgfsetlinewidth{0.501875pt}%
\definecolor{currentstroke}{rgb}{0.000000,0.000000,0.000000}%
\pgfsetstrokecolor{currentstroke}%
\pgfsetdash{}{0pt}%
\pgfsys@defobject{currentmarker}{\pgfqpoint{0.000000in}{0.000000in}}{\pgfqpoint{0.000000in}{0.055556in}}{%
\pgfpathmoveto{\pgfqpoint{0.000000in}{0.000000in}}%
\pgfpathlineto{\pgfqpoint{0.000000in}{0.055556in}}%
\pgfusepath{stroke,fill}%
}%
\begin{pgfscope}%
\pgfsys@transformshift{1.605987in}{0.804260in}%
\pgfsys@useobject{currentmarker}{}%
\end{pgfscope}%
\end{pgfscope}%
\begin{pgfscope}%
\pgfsetbuttcap%
\pgfsetroundjoin%
\definecolor{currentfill}{rgb}{0.000000,0.000000,0.000000}%
\pgfsetfillcolor{currentfill}%
\pgfsetlinewidth{0.501875pt}%
\definecolor{currentstroke}{rgb}{0.000000,0.000000,0.000000}%
\pgfsetstrokecolor{currentstroke}%
\pgfsetdash{}{0pt}%
\pgfsys@defobject{currentmarker}{\pgfqpoint{0.000000in}{-0.055556in}}{\pgfqpoint{0.000000in}{0.000000in}}{%
\pgfpathmoveto{\pgfqpoint{0.000000in}{0.000000in}}%
\pgfpathlineto{\pgfqpoint{0.000000in}{-0.055556in}}%
\pgfusepath{stroke,fill}%
}%
\begin{pgfscope}%
\pgfsys@transformshift{1.605987in}{3.270740in}%
\pgfsys@useobject{currentmarker}{}%
\end{pgfscope}%
\end{pgfscope}%
\begin{pgfscope}%
\pgftext[x=1.605987in,y=0.748705in,,top]{{\sffamily\fontsize{12.000000}{14.400000}\selectfont \(\displaystyle 0.4\)}}%
\end{pgfscope}%
\begin{pgfscope}%
\pgfsetbuttcap%
\pgfsetroundjoin%
\definecolor{currentfill}{rgb}{0.000000,0.000000,0.000000}%
\pgfsetfillcolor{currentfill}%
\pgfsetlinewidth{0.501875pt}%
\definecolor{currentstroke}{rgb}{0.000000,0.000000,0.000000}%
\pgfsetstrokecolor{currentstroke}%
\pgfsetdash{}{0pt}%
\pgfsys@defobject{currentmarker}{\pgfqpoint{0.000000in}{0.000000in}}{\pgfqpoint{0.000000in}{0.055556in}}{%
\pgfpathmoveto{\pgfqpoint{0.000000in}{0.000000in}}%
\pgfpathlineto{\pgfqpoint{0.000000in}{0.055556in}}%
\pgfusepath{stroke,fill}%
}%
\begin{pgfscope}%
\pgfsys@transformshift{2.099283in}{0.804260in}%
\pgfsys@useobject{currentmarker}{}%
\end{pgfscope}%
\end{pgfscope}%
\begin{pgfscope}%
\pgfsetbuttcap%
\pgfsetroundjoin%
\definecolor{currentfill}{rgb}{0.000000,0.000000,0.000000}%
\pgfsetfillcolor{currentfill}%
\pgfsetlinewidth{0.501875pt}%
\definecolor{currentstroke}{rgb}{0.000000,0.000000,0.000000}%
\pgfsetstrokecolor{currentstroke}%
\pgfsetdash{}{0pt}%
\pgfsys@defobject{currentmarker}{\pgfqpoint{0.000000in}{-0.055556in}}{\pgfqpoint{0.000000in}{0.000000in}}{%
\pgfpathmoveto{\pgfqpoint{0.000000in}{0.000000in}}%
\pgfpathlineto{\pgfqpoint{0.000000in}{-0.055556in}}%
\pgfusepath{stroke,fill}%
}%
\begin{pgfscope}%
\pgfsys@transformshift{2.099283in}{3.270740in}%
\pgfsys@useobject{currentmarker}{}%
\end{pgfscope}%
\end{pgfscope}%
\begin{pgfscope}%
\pgftext[x=2.099283in,y=0.748705in,,top]{{\sffamily\fontsize{12.000000}{14.400000}\selectfont \(\displaystyle 0.6\)}}%
\end{pgfscope}%
\begin{pgfscope}%
\pgfsetbuttcap%
\pgfsetroundjoin%
\definecolor{currentfill}{rgb}{0.000000,0.000000,0.000000}%
\pgfsetfillcolor{currentfill}%
\pgfsetlinewidth{0.501875pt}%
\definecolor{currentstroke}{rgb}{0.000000,0.000000,0.000000}%
\pgfsetstrokecolor{currentstroke}%
\pgfsetdash{}{0pt}%
\pgfsys@defobject{currentmarker}{\pgfqpoint{0.000000in}{0.000000in}}{\pgfqpoint{0.000000in}{0.055556in}}{%
\pgfpathmoveto{\pgfqpoint{0.000000in}{0.000000in}}%
\pgfpathlineto{\pgfqpoint{0.000000in}{0.055556in}}%
\pgfusepath{stroke,fill}%
}%
\begin{pgfscope}%
\pgfsys@transformshift{2.592579in}{0.804260in}%
\pgfsys@useobject{currentmarker}{}%
\end{pgfscope}%
\end{pgfscope}%
\begin{pgfscope}%
\pgfsetbuttcap%
\pgfsetroundjoin%
\definecolor{currentfill}{rgb}{0.000000,0.000000,0.000000}%
\pgfsetfillcolor{currentfill}%
\pgfsetlinewidth{0.501875pt}%
\definecolor{currentstroke}{rgb}{0.000000,0.000000,0.000000}%
\pgfsetstrokecolor{currentstroke}%
\pgfsetdash{}{0pt}%
\pgfsys@defobject{currentmarker}{\pgfqpoint{0.000000in}{-0.055556in}}{\pgfqpoint{0.000000in}{0.000000in}}{%
\pgfpathmoveto{\pgfqpoint{0.000000in}{0.000000in}}%
\pgfpathlineto{\pgfqpoint{0.000000in}{-0.055556in}}%
\pgfusepath{stroke,fill}%
}%
\begin{pgfscope}%
\pgfsys@transformshift{2.592579in}{3.270740in}%
\pgfsys@useobject{currentmarker}{}%
\end{pgfscope}%
\end{pgfscope}%
\begin{pgfscope}%
\pgftext[x=2.592579in,y=0.748705in,,top]{{\sffamily\fontsize{12.000000}{14.400000}\selectfont \(\displaystyle 0.8\)}}%
\end{pgfscope}%
\begin{pgfscope}%
\pgfsetbuttcap%
\pgfsetroundjoin%
\definecolor{currentfill}{rgb}{0.000000,0.000000,0.000000}%
\pgfsetfillcolor{currentfill}%
\pgfsetlinewidth{0.501875pt}%
\definecolor{currentstroke}{rgb}{0.000000,0.000000,0.000000}%
\pgfsetstrokecolor{currentstroke}%
\pgfsetdash{}{0pt}%
\pgfsys@defobject{currentmarker}{\pgfqpoint{0.000000in}{0.000000in}}{\pgfqpoint{0.000000in}{0.055556in}}{%
\pgfpathmoveto{\pgfqpoint{0.000000in}{0.000000in}}%
\pgfpathlineto{\pgfqpoint{0.000000in}{0.055556in}}%
\pgfusepath{stroke,fill}%
}%
\begin{pgfscope}%
\pgfsys@transformshift{3.085874in}{0.804260in}%
\pgfsys@useobject{currentmarker}{}%
\end{pgfscope}%
\end{pgfscope}%
\begin{pgfscope}%
\pgfsetbuttcap%
\pgfsetroundjoin%
\definecolor{currentfill}{rgb}{0.000000,0.000000,0.000000}%
\pgfsetfillcolor{currentfill}%
\pgfsetlinewidth{0.501875pt}%
\definecolor{currentstroke}{rgb}{0.000000,0.000000,0.000000}%
\pgfsetstrokecolor{currentstroke}%
\pgfsetdash{}{0pt}%
\pgfsys@defobject{currentmarker}{\pgfqpoint{0.000000in}{-0.055556in}}{\pgfqpoint{0.000000in}{0.000000in}}{%
\pgfpathmoveto{\pgfqpoint{0.000000in}{0.000000in}}%
\pgfpathlineto{\pgfqpoint{0.000000in}{-0.055556in}}%
\pgfusepath{stroke,fill}%
}%
\begin{pgfscope}%
\pgfsys@transformshift{3.085874in}{3.270740in}%
\pgfsys@useobject{currentmarker}{}%
\end{pgfscope}%
\end{pgfscope}%
\begin{pgfscope}%
\pgftext[x=3.085874in,y=0.748705in,,top]{{\sffamily\fontsize{12.000000}{14.400000}\selectfont \(\displaystyle 1.0\)}}%
\end{pgfscope}%
\begin{pgfscope}%
\pgftext[x=1.852635in,y=0.517965in,,top]{{\sffamily\fontsize{12.000000}{14.400000}\selectfont \(\displaystyle P(A|r_2)\)}}%
\end{pgfscope}%
\begin{pgfscope}%
\pgfsetbuttcap%
\pgfsetroundjoin%
\definecolor{currentfill}{rgb}{0.000000,0.000000,0.000000}%
\pgfsetfillcolor{currentfill}%
\pgfsetlinewidth{0.501875pt}%
\definecolor{currentstroke}{rgb}{0.000000,0.000000,0.000000}%
\pgfsetstrokecolor{currentstroke}%
\pgfsetdash{}{0pt}%
\pgfsys@defobject{currentmarker}{\pgfqpoint{0.000000in}{0.000000in}}{\pgfqpoint{0.055556in}{0.000000in}}{%
\pgfpathmoveto{\pgfqpoint{0.000000in}{0.000000in}}%
\pgfpathlineto{\pgfqpoint{0.055556in}{0.000000in}}%
\pgfusepath{stroke,fill}%
}%
\begin{pgfscope}%
\pgfsys@transformshift{0.619395in}{0.804260in}%
\pgfsys@useobject{currentmarker}{}%
\end{pgfscope}%
\end{pgfscope}%
\begin{pgfscope}%
\pgfsetbuttcap%
\pgfsetroundjoin%
\definecolor{currentfill}{rgb}{0.000000,0.000000,0.000000}%
\pgfsetfillcolor{currentfill}%
\pgfsetlinewidth{0.501875pt}%
\definecolor{currentstroke}{rgb}{0.000000,0.000000,0.000000}%
\pgfsetstrokecolor{currentstroke}%
\pgfsetdash{}{0pt}%
\pgfsys@defobject{currentmarker}{\pgfqpoint{-0.055556in}{0.000000in}}{\pgfqpoint{0.000000in}{0.000000in}}{%
\pgfpathmoveto{\pgfqpoint{0.000000in}{0.000000in}}%
\pgfpathlineto{\pgfqpoint{-0.055556in}{0.000000in}}%
\pgfusepath{stroke,fill}%
}%
\begin{pgfscope}%
\pgfsys@transformshift{3.085874in}{0.804260in}%
\pgfsys@useobject{currentmarker}{}%
\end{pgfscope}%
\end{pgfscope}%
\begin{pgfscope}%
\pgftext[x=0.563839in,y=0.804260in,right,]{{\sffamily\fontsize{12.000000}{14.400000}\selectfont \(\displaystyle 0.0\)}}%
\end{pgfscope}%
\begin{pgfscope}%
\pgfsetbuttcap%
\pgfsetroundjoin%
\definecolor{currentfill}{rgb}{0.000000,0.000000,0.000000}%
\pgfsetfillcolor{currentfill}%
\pgfsetlinewidth{0.501875pt}%
\definecolor{currentstroke}{rgb}{0.000000,0.000000,0.000000}%
\pgfsetstrokecolor{currentstroke}%
\pgfsetdash{}{0pt}%
\pgfsys@defobject{currentmarker}{\pgfqpoint{0.000000in}{0.000000in}}{\pgfqpoint{0.055556in}{0.000000in}}{%
\pgfpathmoveto{\pgfqpoint{0.000000in}{0.000000in}}%
\pgfpathlineto{\pgfqpoint{0.055556in}{0.000000in}}%
\pgfusepath{stroke,fill}%
}%
\begin{pgfscope}%
\pgfsys@transformshift{0.619395in}{1.297556in}%
\pgfsys@useobject{currentmarker}{}%
\end{pgfscope}%
\end{pgfscope}%
\begin{pgfscope}%
\pgfsetbuttcap%
\pgfsetroundjoin%
\definecolor{currentfill}{rgb}{0.000000,0.000000,0.000000}%
\pgfsetfillcolor{currentfill}%
\pgfsetlinewidth{0.501875pt}%
\definecolor{currentstroke}{rgb}{0.000000,0.000000,0.000000}%
\pgfsetstrokecolor{currentstroke}%
\pgfsetdash{}{0pt}%
\pgfsys@defobject{currentmarker}{\pgfqpoint{-0.055556in}{0.000000in}}{\pgfqpoint{0.000000in}{0.000000in}}{%
\pgfpathmoveto{\pgfqpoint{0.000000in}{0.000000in}}%
\pgfpathlineto{\pgfqpoint{-0.055556in}{0.000000in}}%
\pgfusepath{stroke,fill}%
}%
\begin{pgfscope}%
\pgfsys@transformshift{3.085874in}{1.297556in}%
\pgfsys@useobject{currentmarker}{}%
\end{pgfscope}%
\end{pgfscope}%
\begin{pgfscope}%
\pgftext[x=0.563839in,y=1.297556in,right,]{{\sffamily\fontsize{12.000000}{14.400000}\selectfont \(\displaystyle 0.2\)}}%
\end{pgfscope}%
\begin{pgfscope}%
\pgfsetbuttcap%
\pgfsetroundjoin%
\definecolor{currentfill}{rgb}{0.000000,0.000000,0.000000}%
\pgfsetfillcolor{currentfill}%
\pgfsetlinewidth{0.501875pt}%
\definecolor{currentstroke}{rgb}{0.000000,0.000000,0.000000}%
\pgfsetstrokecolor{currentstroke}%
\pgfsetdash{}{0pt}%
\pgfsys@defobject{currentmarker}{\pgfqpoint{0.000000in}{0.000000in}}{\pgfqpoint{0.055556in}{0.000000in}}{%
\pgfpathmoveto{\pgfqpoint{0.000000in}{0.000000in}}%
\pgfpathlineto{\pgfqpoint{0.055556in}{0.000000in}}%
\pgfusepath{stroke,fill}%
}%
\begin{pgfscope}%
\pgfsys@transformshift{0.619395in}{1.790852in}%
\pgfsys@useobject{currentmarker}{}%
\end{pgfscope}%
\end{pgfscope}%
\begin{pgfscope}%
\pgfsetbuttcap%
\pgfsetroundjoin%
\definecolor{currentfill}{rgb}{0.000000,0.000000,0.000000}%
\pgfsetfillcolor{currentfill}%
\pgfsetlinewidth{0.501875pt}%
\definecolor{currentstroke}{rgb}{0.000000,0.000000,0.000000}%
\pgfsetstrokecolor{currentstroke}%
\pgfsetdash{}{0pt}%
\pgfsys@defobject{currentmarker}{\pgfqpoint{-0.055556in}{0.000000in}}{\pgfqpoint{0.000000in}{0.000000in}}{%
\pgfpathmoveto{\pgfqpoint{0.000000in}{0.000000in}}%
\pgfpathlineto{\pgfqpoint{-0.055556in}{0.000000in}}%
\pgfusepath{stroke,fill}%
}%
\begin{pgfscope}%
\pgfsys@transformshift{3.085874in}{1.790852in}%
\pgfsys@useobject{currentmarker}{}%
\end{pgfscope}%
\end{pgfscope}%
\begin{pgfscope}%
\pgftext[x=0.563839in,y=1.790852in,right,]{{\sffamily\fontsize{12.000000}{14.400000}\selectfont \(\displaystyle 0.4\)}}%
\end{pgfscope}%
\begin{pgfscope}%
\pgfsetbuttcap%
\pgfsetroundjoin%
\definecolor{currentfill}{rgb}{0.000000,0.000000,0.000000}%
\pgfsetfillcolor{currentfill}%
\pgfsetlinewidth{0.501875pt}%
\definecolor{currentstroke}{rgb}{0.000000,0.000000,0.000000}%
\pgfsetstrokecolor{currentstroke}%
\pgfsetdash{}{0pt}%
\pgfsys@defobject{currentmarker}{\pgfqpoint{0.000000in}{0.000000in}}{\pgfqpoint{0.055556in}{0.000000in}}{%
\pgfpathmoveto{\pgfqpoint{0.000000in}{0.000000in}}%
\pgfpathlineto{\pgfqpoint{0.055556in}{0.000000in}}%
\pgfusepath{stroke,fill}%
}%
\begin{pgfscope}%
\pgfsys@transformshift{0.619395in}{2.284148in}%
\pgfsys@useobject{currentmarker}{}%
\end{pgfscope}%
\end{pgfscope}%
\begin{pgfscope}%
\pgfsetbuttcap%
\pgfsetroundjoin%
\definecolor{currentfill}{rgb}{0.000000,0.000000,0.000000}%
\pgfsetfillcolor{currentfill}%
\pgfsetlinewidth{0.501875pt}%
\definecolor{currentstroke}{rgb}{0.000000,0.000000,0.000000}%
\pgfsetstrokecolor{currentstroke}%
\pgfsetdash{}{0pt}%
\pgfsys@defobject{currentmarker}{\pgfqpoint{-0.055556in}{0.000000in}}{\pgfqpoint{0.000000in}{0.000000in}}{%
\pgfpathmoveto{\pgfqpoint{0.000000in}{0.000000in}}%
\pgfpathlineto{\pgfqpoint{-0.055556in}{0.000000in}}%
\pgfusepath{stroke,fill}%
}%
\begin{pgfscope}%
\pgfsys@transformshift{3.085874in}{2.284148in}%
\pgfsys@useobject{currentmarker}{}%
\end{pgfscope}%
\end{pgfscope}%
\begin{pgfscope}%
\pgftext[x=0.563839in,y=2.284148in,right,]{{\sffamily\fontsize{12.000000}{14.400000}\selectfont \(\displaystyle 0.6\)}}%
\end{pgfscope}%
\begin{pgfscope}%
\pgfsetbuttcap%
\pgfsetroundjoin%
\definecolor{currentfill}{rgb}{0.000000,0.000000,0.000000}%
\pgfsetfillcolor{currentfill}%
\pgfsetlinewidth{0.501875pt}%
\definecolor{currentstroke}{rgb}{0.000000,0.000000,0.000000}%
\pgfsetstrokecolor{currentstroke}%
\pgfsetdash{}{0pt}%
\pgfsys@defobject{currentmarker}{\pgfqpoint{0.000000in}{0.000000in}}{\pgfqpoint{0.055556in}{0.000000in}}{%
\pgfpathmoveto{\pgfqpoint{0.000000in}{0.000000in}}%
\pgfpathlineto{\pgfqpoint{0.055556in}{0.000000in}}%
\pgfusepath{stroke,fill}%
}%
\begin{pgfscope}%
\pgfsys@transformshift{0.619395in}{2.777444in}%
\pgfsys@useobject{currentmarker}{}%
\end{pgfscope}%
\end{pgfscope}%
\begin{pgfscope}%
\pgfsetbuttcap%
\pgfsetroundjoin%
\definecolor{currentfill}{rgb}{0.000000,0.000000,0.000000}%
\pgfsetfillcolor{currentfill}%
\pgfsetlinewidth{0.501875pt}%
\definecolor{currentstroke}{rgb}{0.000000,0.000000,0.000000}%
\pgfsetstrokecolor{currentstroke}%
\pgfsetdash{}{0pt}%
\pgfsys@defobject{currentmarker}{\pgfqpoint{-0.055556in}{0.000000in}}{\pgfqpoint{0.000000in}{0.000000in}}{%
\pgfpathmoveto{\pgfqpoint{0.000000in}{0.000000in}}%
\pgfpathlineto{\pgfqpoint{-0.055556in}{0.000000in}}%
\pgfusepath{stroke,fill}%
}%
\begin{pgfscope}%
\pgfsys@transformshift{3.085874in}{2.777444in}%
\pgfsys@useobject{currentmarker}{}%
\end{pgfscope}%
\end{pgfscope}%
\begin{pgfscope}%
\pgftext[x=0.563839in,y=2.777444in,right,]{{\sffamily\fontsize{12.000000}{14.400000}\selectfont \(\displaystyle 0.8\)}}%
\end{pgfscope}%
\begin{pgfscope}%
\pgfsetbuttcap%
\pgfsetroundjoin%
\definecolor{currentfill}{rgb}{0.000000,0.000000,0.000000}%
\pgfsetfillcolor{currentfill}%
\pgfsetlinewidth{0.501875pt}%
\definecolor{currentstroke}{rgb}{0.000000,0.000000,0.000000}%
\pgfsetstrokecolor{currentstroke}%
\pgfsetdash{}{0pt}%
\pgfsys@defobject{currentmarker}{\pgfqpoint{0.000000in}{0.000000in}}{\pgfqpoint{0.055556in}{0.000000in}}{%
\pgfpathmoveto{\pgfqpoint{0.000000in}{0.000000in}}%
\pgfpathlineto{\pgfqpoint{0.055556in}{0.000000in}}%
\pgfusepath{stroke,fill}%
}%
\begin{pgfscope}%
\pgfsys@transformshift{0.619395in}{3.270740in}%
\pgfsys@useobject{currentmarker}{}%
\end{pgfscope}%
\end{pgfscope}%
\begin{pgfscope}%
\pgfsetbuttcap%
\pgfsetroundjoin%
\definecolor{currentfill}{rgb}{0.000000,0.000000,0.000000}%
\pgfsetfillcolor{currentfill}%
\pgfsetlinewidth{0.501875pt}%
\definecolor{currentstroke}{rgb}{0.000000,0.000000,0.000000}%
\pgfsetstrokecolor{currentstroke}%
\pgfsetdash{}{0pt}%
\pgfsys@defobject{currentmarker}{\pgfqpoint{-0.055556in}{0.000000in}}{\pgfqpoint{0.000000in}{0.000000in}}{%
\pgfpathmoveto{\pgfqpoint{0.000000in}{0.000000in}}%
\pgfpathlineto{\pgfqpoint{-0.055556in}{0.000000in}}%
\pgfusepath{stroke,fill}%
}%
\begin{pgfscope}%
\pgfsys@transformshift{3.085874in}{3.270740in}%
\pgfsys@useobject{currentmarker}{}%
\end{pgfscope}%
\end{pgfscope}%
\begin{pgfscope}%
\pgftext[x=0.563839in,y=3.270740in,right,]{{\sffamily\fontsize{12.000000}{14.400000}\selectfont \(\displaystyle 1.0\)}}%
\end{pgfscope}%
\begin{pgfscope}%
\pgftext[x=0.285871in,y=2.037500in,,bottom,rotate=90.000000]{{\sffamily\fontsize{12.000000}{14.400000}\selectfont \(\displaystyle P(A|r_1)\)}}%
\end{pgfscope}%
\begin{pgfscope}%
\pgfsetbuttcap%
\pgfsetroundjoin%
\pgfsetlinewidth{1.003750pt}%
\definecolor{currentstroke}{rgb}{0.000000,0.000000,0.000000}%
\pgfsetstrokecolor{currentstroke}%
\pgfsetdash{}{0pt}%
\pgfpathmoveto{\pgfqpoint{0.619395in}{3.270740in}}%
\pgfpathlineto{\pgfqpoint{3.085874in}{3.270740in}}%
\pgfusepath{stroke}%
\end{pgfscope}%
\begin{pgfscope}%
\pgfsetbuttcap%
\pgfsetroundjoin%
\pgfsetlinewidth{1.003750pt}%
\definecolor{currentstroke}{rgb}{0.000000,0.000000,0.000000}%
\pgfsetstrokecolor{currentstroke}%
\pgfsetdash{}{0pt}%
\pgfpathmoveto{\pgfqpoint{3.085874in}{0.804260in}}%
\pgfpathlineto{\pgfqpoint{3.085874in}{3.270740in}}%
\pgfusepath{stroke}%
\end{pgfscope}%
\begin{pgfscope}%
\pgfsetbuttcap%
\pgfsetroundjoin%
\pgfsetlinewidth{1.003750pt}%
\definecolor{currentstroke}{rgb}{0.000000,0.000000,0.000000}%
\pgfsetstrokecolor{currentstroke}%
\pgfsetdash{}{0pt}%
\pgfpathmoveto{\pgfqpoint{0.619395in}{0.804260in}}%
\pgfpathlineto{\pgfqpoint{3.085874in}{0.804260in}}%
\pgfusepath{stroke}%
\end{pgfscope}%
\begin{pgfscope}%
\pgfsetbuttcap%
\pgfsetroundjoin%
\pgfsetlinewidth{1.003750pt}%
\definecolor{currentstroke}{rgb}{0.000000,0.000000,0.000000}%
\pgfsetstrokecolor{currentstroke}%
\pgfsetdash{}{0pt}%
\pgfpathmoveto{\pgfqpoint{0.619395in}{0.804260in}}%
\pgfpathlineto{\pgfqpoint{0.619395in}{3.270740in}}%
\pgfusepath{stroke}%
\end{pgfscope}%
\begin{pgfscope}%
\pgfsetbuttcap%
\pgfsetroundjoin%
\definecolor{currentfill}{rgb}{1.000000,1.000000,1.000000}%
\pgfsetfillcolor{currentfill}%
\pgfsetlinewidth{0.000000pt}%
\definecolor{currentstroke}{rgb}{0.000000,0.000000,0.000000}%
\pgfsetstrokecolor{currentstroke}%
\pgfsetstrokeopacity{0.000000}%
\pgfsetdash{}{0pt}%
\pgfpathmoveto{\pgfqpoint{3.373521in}{0.804260in}}%
\pgfpathlineto{\pgfqpoint{5.840000in}{0.804260in}}%
\pgfpathlineto{\pgfqpoint{5.840000in}{3.270740in}}%
\pgfpathlineto{\pgfqpoint{3.373521in}{3.270740in}}%
\pgfpathclose%
\pgfusepath{fill}%
\end{pgfscope}%
\begin{pgfscope}%
\pgfpathrectangle{\pgfqpoint{3.373521in}{0.804260in}}{\pgfqpoint{2.466479in}{2.466479in}} %
\pgfusepath{clip}%
\pgftext[at=\pgfqpoint{3.373521in}{0.804260in},left,bottom]{\pgfimage[interpolate=true,width=2.480000in,height=2.480000in]{objective_function-img1.png}}%
\end{pgfscope}%
\begin{pgfscope}%
\pgfpathrectangle{\pgfqpoint{3.373521in}{0.804260in}}{\pgfqpoint{2.466479in}{2.466479in}} %
\pgfusepath{clip}%
\pgfsetrectcap%
\pgfsetroundjoin%
\pgfsetlinewidth{1.003750pt}%
\definecolor{currentstroke}{rgb}{0.000000,0.000000,0.000000}%
\pgfsetstrokecolor{currentstroke}%
\pgfsetdash{}{0pt}%
\pgfpathmoveto{\pgfqpoint{5.593352in}{1.050908in}}%
\pgfusepath{stroke}%
\end{pgfscope}%
\begin{pgfscope}%
\pgfpathrectangle{\pgfqpoint{3.373521in}{0.804260in}}{\pgfqpoint{2.466479in}{2.466479in}} %
\pgfusepath{clip}%
\pgfsetbuttcap%
\pgfsetroundjoin%
\definecolor{currentfill}{rgb}{0.000000,0.000000,0.000000}%
\pgfsetfillcolor{currentfill}%
\pgfsetlinewidth{2.007500pt}%
\definecolor{currentstroke}{rgb}{0.000000,0.000000,0.000000}%
\pgfsetstrokecolor{currentstroke}%
\pgfsetdash{}{0pt}%
\pgfsys@defobject{currentmarker}{\pgfqpoint{-0.055556in}{-0.055556in}}{\pgfqpoint{0.055556in}{0.055556in}}{%
\pgfpathmoveto{\pgfqpoint{-0.055556in}{-0.055556in}}%
\pgfpathlineto{\pgfqpoint{0.055556in}{0.055556in}}%
\pgfpathmoveto{\pgfqpoint{-0.055556in}{0.055556in}}%
\pgfpathlineto{\pgfqpoint{0.055556in}{-0.055556in}}%
\pgfusepath{stroke,fill}%
}%
\begin{pgfscope}%
\pgfsys@transformshift{5.593352in}{1.050908in}%
\pgfsys@useobject{currentmarker}{}%
\end{pgfscope}%
\end{pgfscope}%
\begin{pgfscope}%
\pgfsetbuttcap%
\pgfsetroundjoin%
\definecolor{currentfill}{rgb}{0.000000,0.000000,0.000000}%
\pgfsetfillcolor{currentfill}%
\pgfsetlinewidth{0.501875pt}%
\definecolor{currentstroke}{rgb}{0.000000,0.000000,0.000000}%
\pgfsetstrokecolor{currentstroke}%
\pgfsetdash{}{0pt}%
\pgfsys@defobject{currentmarker}{\pgfqpoint{0.000000in}{0.000000in}}{\pgfqpoint{0.000000in}{0.055556in}}{%
\pgfpathmoveto{\pgfqpoint{0.000000in}{0.000000in}}%
\pgfpathlineto{\pgfqpoint{0.000000in}{0.055556in}}%
\pgfusepath{stroke,fill}%
}%
\begin{pgfscope}%
\pgfsys@transformshift{3.373521in}{0.804260in}%
\pgfsys@useobject{currentmarker}{}%
\end{pgfscope}%
\end{pgfscope}%
\begin{pgfscope}%
\pgfsetbuttcap%
\pgfsetroundjoin%
\definecolor{currentfill}{rgb}{0.000000,0.000000,0.000000}%
\pgfsetfillcolor{currentfill}%
\pgfsetlinewidth{0.501875pt}%
\definecolor{currentstroke}{rgb}{0.000000,0.000000,0.000000}%
\pgfsetstrokecolor{currentstroke}%
\pgfsetdash{}{0pt}%
\pgfsys@defobject{currentmarker}{\pgfqpoint{0.000000in}{-0.055556in}}{\pgfqpoint{0.000000in}{0.000000in}}{%
\pgfpathmoveto{\pgfqpoint{0.000000in}{0.000000in}}%
\pgfpathlineto{\pgfqpoint{0.000000in}{-0.055556in}}%
\pgfusepath{stroke,fill}%
}%
\begin{pgfscope}%
\pgfsys@transformshift{3.373521in}{3.270740in}%
\pgfsys@useobject{currentmarker}{}%
\end{pgfscope}%
\end{pgfscope}%
\begin{pgfscope}%
\pgftext[x=3.373521in,y=0.748705in,,top]{{\sffamily\fontsize{12.000000}{14.400000}\selectfont \(\displaystyle 0.0\)}}%
\end{pgfscope}%
\begin{pgfscope}%
\pgfsetbuttcap%
\pgfsetroundjoin%
\definecolor{currentfill}{rgb}{0.000000,0.000000,0.000000}%
\pgfsetfillcolor{currentfill}%
\pgfsetlinewidth{0.501875pt}%
\definecolor{currentstroke}{rgb}{0.000000,0.000000,0.000000}%
\pgfsetstrokecolor{currentstroke}%
\pgfsetdash{}{0pt}%
\pgfsys@defobject{currentmarker}{\pgfqpoint{0.000000in}{0.000000in}}{\pgfqpoint{0.000000in}{0.055556in}}{%
\pgfpathmoveto{\pgfqpoint{0.000000in}{0.000000in}}%
\pgfpathlineto{\pgfqpoint{0.000000in}{0.055556in}}%
\pgfusepath{stroke,fill}%
}%
\begin{pgfscope}%
\pgfsys@transformshift{3.866816in}{0.804260in}%
\pgfsys@useobject{currentmarker}{}%
\end{pgfscope}%
\end{pgfscope}%
\begin{pgfscope}%
\pgfsetbuttcap%
\pgfsetroundjoin%
\definecolor{currentfill}{rgb}{0.000000,0.000000,0.000000}%
\pgfsetfillcolor{currentfill}%
\pgfsetlinewidth{0.501875pt}%
\definecolor{currentstroke}{rgb}{0.000000,0.000000,0.000000}%
\pgfsetstrokecolor{currentstroke}%
\pgfsetdash{}{0pt}%
\pgfsys@defobject{currentmarker}{\pgfqpoint{0.000000in}{-0.055556in}}{\pgfqpoint{0.000000in}{0.000000in}}{%
\pgfpathmoveto{\pgfqpoint{0.000000in}{0.000000in}}%
\pgfpathlineto{\pgfqpoint{0.000000in}{-0.055556in}}%
\pgfusepath{stroke,fill}%
}%
\begin{pgfscope}%
\pgfsys@transformshift{3.866816in}{3.270740in}%
\pgfsys@useobject{currentmarker}{}%
\end{pgfscope}%
\end{pgfscope}%
\begin{pgfscope}%
\pgftext[x=3.866816in,y=0.748705in,,top]{{\sffamily\fontsize{12.000000}{14.400000}\selectfont \(\displaystyle 0.2\)}}%
\end{pgfscope}%
\begin{pgfscope}%
\pgfsetbuttcap%
\pgfsetroundjoin%
\definecolor{currentfill}{rgb}{0.000000,0.000000,0.000000}%
\pgfsetfillcolor{currentfill}%
\pgfsetlinewidth{0.501875pt}%
\definecolor{currentstroke}{rgb}{0.000000,0.000000,0.000000}%
\pgfsetstrokecolor{currentstroke}%
\pgfsetdash{}{0pt}%
\pgfsys@defobject{currentmarker}{\pgfqpoint{0.000000in}{0.000000in}}{\pgfqpoint{0.000000in}{0.055556in}}{%
\pgfpathmoveto{\pgfqpoint{0.000000in}{0.000000in}}%
\pgfpathlineto{\pgfqpoint{0.000000in}{0.055556in}}%
\pgfusepath{stroke,fill}%
}%
\begin{pgfscope}%
\pgfsys@transformshift{4.360112in}{0.804260in}%
\pgfsys@useobject{currentmarker}{}%
\end{pgfscope}%
\end{pgfscope}%
\begin{pgfscope}%
\pgfsetbuttcap%
\pgfsetroundjoin%
\definecolor{currentfill}{rgb}{0.000000,0.000000,0.000000}%
\pgfsetfillcolor{currentfill}%
\pgfsetlinewidth{0.501875pt}%
\definecolor{currentstroke}{rgb}{0.000000,0.000000,0.000000}%
\pgfsetstrokecolor{currentstroke}%
\pgfsetdash{}{0pt}%
\pgfsys@defobject{currentmarker}{\pgfqpoint{0.000000in}{-0.055556in}}{\pgfqpoint{0.000000in}{0.000000in}}{%
\pgfpathmoveto{\pgfqpoint{0.000000in}{0.000000in}}%
\pgfpathlineto{\pgfqpoint{0.000000in}{-0.055556in}}%
\pgfusepath{stroke,fill}%
}%
\begin{pgfscope}%
\pgfsys@transformshift{4.360112in}{3.270740in}%
\pgfsys@useobject{currentmarker}{}%
\end{pgfscope}%
\end{pgfscope}%
\begin{pgfscope}%
\pgftext[x=4.360112in,y=0.748705in,,top]{{\sffamily\fontsize{12.000000}{14.400000}\selectfont \(\displaystyle 0.4\)}}%
\end{pgfscope}%
\begin{pgfscope}%
\pgfsetbuttcap%
\pgfsetroundjoin%
\definecolor{currentfill}{rgb}{0.000000,0.000000,0.000000}%
\pgfsetfillcolor{currentfill}%
\pgfsetlinewidth{0.501875pt}%
\definecolor{currentstroke}{rgb}{0.000000,0.000000,0.000000}%
\pgfsetstrokecolor{currentstroke}%
\pgfsetdash{}{0pt}%
\pgfsys@defobject{currentmarker}{\pgfqpoint{0.000000in}{0.000000in}}{\pgfqpoint{0.000000in}{0.055556in}}{%
\pgfpathmoveto{\pgfqpoint{0.000000in}{0.000000in}}%
\pgfpathlineto{\pgfqpoint{0.000000in}{0.055556in}}%
\pgfusepath{stroke,fill}%
}%
\begin{pgfscope}%
\pgfsys@transformshift{4.853408in}{0.804260in}%
\pgfsys@useobject{currentmarker}{}%
\end{pgfscope}%
\end{pgfscope}%
\begin{pgfscope}%
\pgfsetbuttcap%
\pgfsetroundjoin%
\definecolor{currentfill}{rgb}{0.000000,0.000000,0.000000}%
\pgfsetfillcolor{currentfill}%
\pgfsetlinewidth{0.501875pt}%
\definecolor{currentstroke}{rgb}{0.000000,0.000000,0.000000}%
\pgfsetstrokecolor{currentstroke}%
\pgfsetdash{}{0pt}%
\pgfsys@defobject{currentmarker}{\pgfqpoint{0.000000in}{-0.055556in}}{\pgfqpoint{0.000000in}{0.000000in}}{%
\pgfpathmoveto{\pgfqpoint{0.000000in}{0.000000in}}%
\pgfpathlineto{\pgfqpoint{0.000000in}{-0.055556in}}%
\pgfusepath{stroke,fill}%
}%
\begin{pgfscope}%
\pgfsys@transformshift{4.853408in}{3.270740in}%
\pgfsys@useobject{currentmarker}{}%
\end{pgfscope}%
\end{pgfscope}%
\begin{pgfscope}%
\pgftext[x=4.853408in,y=0.748705in,,top]{{\sffamily\fontsize{12.000000}{14.400000}\selectfont \(\displaystyle 0.6\)}}%
\end{pgfscope}%
\begin{pgfscope}%
\pgfsetbuttcap%
\pgfsetroundjoin%
\definecolor{currentfill}{rgb}{0.000000,0.000000,0.000000}%
\pgfsetfillcolor{currentfill}%
\pgfsetlinewidth{0.501875pt}%
\definecolor{currentstroke}{rgb}{0.000000,0.000000,0.000000}%
\pgfsetstrokecolor{currentstroke}%
\pgfsetdash{}{0pt}%
\pgfsys@defobject{currentmarker}{\pgfqpoint{0.000000in}{0.000000in}}{\pgfqpoint{0.000000in}{0.055556in}}{%
\pgfpathmoveto{\pgfqpoint{0.000000in}{0.000000in}}%
\pgfpathlineto{\pgfqpoint{0.000000in}{0.055556in}}%
\pgfusepath{stroke,fill}%
}%
\begin{pgfscope}%
\pgfsys@transformshift{5.346704in}{0.804260in}%
\pgfsys@useobject{currentmarker}{}%
\end{pgfscope}%
\end{pgfscope}%
\begin{pgfscope}%
\pgfsetbuttcap%
\pgfsetroundjoin%
\definecolor{currentfill}{rgb}{0.000000,0.000000,0.000000}%
\pgfsetfillcolor{currentfill}%
\pgfsetlinewidth{0.501875pt}%
\definecolor{currentstroke}{rgb}{0.000000,0.000000,0.000000}%
\pgfsetstrokecolor{currentstroke}%
\pgfsetdash{}{0pt}%
\pgfsys@defobject{currentmarker}{\pgfqpoint{0.000000in}{-0.055556in}}{\pgfqpoint{0.000000in}{0.000000in}}{%
\pgfpathmoveto{\pgfqpoint{0.000000in}{0.000000in}}%
\pgfpathlineto{\pgfqpoint{0.000000in}{-0.055556in}}%
\pgfusepath{stroke,fill}%
}%
\begin{pgfscope}%
\pgfsys@transformshift{5.346704in}{3.270740in}%
\pgfsys@useobject{currentmarker}{}%
\end{pgfscope}%
\end{pgfscope}%
\begin{pgfscope}%
\pgftext[x=5.346704in,y=0.748705in,,top]{{\sffamily\fontsize{12.000000}{14.400000}\selectfont \(\displaystyle 0.8\)}}%
\end{pgfscope}%
\begin{pgfscope}%
\pgfsetbuttcap%
\pgfsetroundjoin%
\definecolor{currentfill}{rgb}{0.000000,0.000000,0.000000}%
\pgfsetfillcolor{currentfill}%
\pgfsetlinewidth{0.501875pt}%
\definecolor{currentstroke}{rgb}{0.000000,0.000000,0.000000}%
\pgfsetstrokecolor{currentstroke}%
\pgfsetdash{}{0pt}%
\pgfsys@defobject{currentmarker}{\pgfqpoint{0.000000in}{0.000000in}}{\pgfqpoint{0.000000in}{0.055556in}}{%
\pgfpathmoveto{\pgfqpoint{0.000000in}{0.000000in}}%
\pgfpathlineto{\pgfqpoint{0.000000in}{0.055556in}}%
\pgfusepath{stroke,fill}%
}%
\begin{pgfscope}%
\pgfsys@transformshift{5.840000in}{0.804260in}%
\pgfsys@useobject{currentmarker}{}%
\end{pgfscope}%
\end{pgfscope}%
\begin{pgfscope}%
\pgfsetbuttcap%
\pgfsetroundjoin%
\definecolor{currentfill}{rgb}{0.000000,0.000000,0.000000}%
\pgfsetfillcolor{currentfill}%
\pgfsetlinewidth{0.501875pt}%
\definecolor{currentstroke}{rgb}{0.000000,0.000000,0.000000}%
\pgfsetstrokecolor{currentstroke}%
\pgfsetdash{}{0pt}%
\pgfsys@defobject{currentmarker}{\pgfqpoint{0.000000in}{-0.055556in}}{\pgfqpoint{0.000000in}{0.000000in}}{%
\pgfpathmoveto{\pgfqpoint{0.000000in}{0.000000in}}%
\pgfpathlineto{\pgfqpoint{0.000000in}{-0.055556in}}%
\pgfusepath{stroke,fill}%
}%
\begin{pgfscope}%
\pgfsys@transformshift{5.840000in}{3.270740in}%
\pgfsys@useobject{currentmarker}{}%
\end{pgfscope}%
\end{pgfscope}%
\begin{pgfscope}%
\pgftext[x=5.840000in,y=0.748705in,,top]{{\sffamily\fontsize{12.000000}{14.400000}\selectfont \(\displaystyle 1.0\)}}%
\end{pgfscope}%
\begin{pgfscope}%
\pgftext[x=4.606760in,y=0.517965in,,top]{{\sffamily\fontsize{12.000000}{14.400000}\selectfont \(\displaystyle P(A|r_2)\)}}%
\end{pgfscope}%
\begin{pgfscope}%
\pgfsetbuttcap%
\pgfsetroundjoin%
\definecolor{currentfill}{rgb}{0.000000,0.000000,0.000000}%
\pgfsetfillcolor{currentfill}%
\pgfsetlinewidth{0.501875pt}%
\definecolor{currentstroke}{rgb}{0.000000,0.000000,0.000000}%
\pgfsetstrokecolor{currentstroke}%
\pgfsetdash{}{0pt}%
\pgfsys@defobject{currentmarker}{\pgfqpoint{0.000000in}{0.000000in}}{\pgfqpoint{0.055556in}{0.000000in}}{%
\pgfpathmoveto{\pgfqpoint{0.000000in}{0.000000in}}%
\pgfpathlineto{\pgfqpoint{0.055556in}{0.000000in}}%
\pgfusepath{stroke,fill}%
}%
\begin{pgfscope}%
\pgfsys@transformshift{3.373521in}{0.804260in}%
\pgfsys@useobject{currentmarker}{}%
\end{pgfscope}%
\end{pgfscope}%
\begin{pgfscope}%
\pgfsetbuttcap%
\pgfsetroundjoin%
\definecolor{currentfill}{rgb}{0.000000,0.000000,0.000000}%
\pgfsetfillcolor{currentfill}%
\pgfsetlinewidth{0.501875pt}%
\definecolor{currentstroke}{rgb}{0.000000,0.000000,0.000000}%
\pgfsetstrokecolor{currentstroke}%
\pgfsetdash{}{0pt}%
\pgfsys@defobject{currentmarker}{\pgfqpoint{-0.055556in}{0.000000in}}{\pgfqpoint{0.000000in}{0.000000in}}{%
\pgfpathmoveto{\pgfqpoint{0.000000in}{0.000000in}}%
\pgfpathlineto{\pgfqpoint{-0.055556in}{0.000000in}}%
\pgfusepath{stroke,fill}%
}%
\begin{pgfscope}%
\pgfsys@transformshift{5.840000in}{0.804260in}%
\pgfsys@useobject{currentmarker}{}%
\end{pgfscope}%
\end{pgfscope}%
\begin{pgfscope}%
\pgfsetbuttcap%
\pgfsetroundjoin%
\definecolor{currentfill}{rgb}{0.000000,0.000000,0.000000}%
\pgfsetfillcolor{currentfill}%
\pgfsetlinewidth{0.501875pt}%
\definecolor{currentstroke}{rgb}{0.000000,0.000000,0.000000}%
\pgfsetstrokecolor{currentstroke}%
\pgfsetdash{}{0pt}%
\pgfsys@defobject{currentmarker}{\pgfqpoint{0.000000in}{0.000000in}}{\pgfqpoint{0.055556in}{0.000000in}}{%
\pgfpathmoveto{\pgfqpoint{0.000000in}{0.000000in}}%
\pgfpathlineto{\pgfqpoint{0.055556in}{0.000000in}}%
\pgfusepath{stroke,fill}%
}%
\begin{pgfscope}%
\pgfsys@transformshift{3.373521in}{1.297556in}%
\pgfsys@useobject{currentmarker}{}%
\end{pgfscope}%
\end{pgfscope}%
\begin{pgfscope}%
\pgfsetbuttcap%
\pgfsetroundjoin%
\definecolor{currentfill}{rgb}{0.000000,0.000000,0.000000}%
\pgfsetfillcolor{currentfill}%
\pgfsetlinewidth{0.501875pt}%
\definecolor{currentstroke}{rgb}{0.000000,0.000000,0.000000}%
\pgfsetstrokecolor{currentstroke}%
\pgfsetdash{}{0pt}%
\pgfsys@defobject{currentmarker}{\pgfqpoint{-0.055556in}{0.000000in}}{\pgfqpoint{0.000000in}{0.000000in}}{%
\pgfpathmoveto{\pgfqpoint{0.000000in}{0.000000in}}%
\pgfpathlineto{\pgfqpoint{-0.055556in}{0.000000in}}%
\pgfusepath{stroke,fill}%
}%
\begin{pgfscope}%
\pgfsys@transformshift{5.840000in}{1.297556in}%
\pgfsys@useobject{currentmarker}{}%
\end{pgfscope}%
\end{pgfscope}%
\begin{pgfscope}%
\pgfsetbuttcap%
\pgfsetroundjoin%
\definecolor{currentfill}{rgb}{0.000000,0.000000,0.000000}%
\pgfsetfillcolor{currentfill}%
\pgfsetlinewidth{0.501875pt}%
\definecolor{currentstroke}{rgb}{0.000000,0.000000,0.000000}%
\pgfsetstrokecolor{currentstroke}%
\pgfsetdash{}{0pt}%
\pgfsys@defobject{currentmarker}{\pgfqpoint{0.000000in}{0.000000in}}{\pgfqpoint{0.055556in}{0.000000in}}{%
\pgfpathmoveto{\pgfqpoint{0.000000in}{0.000000in}}%
\pgfpathlineto{\pgfqpoint{0.055556in}{0.000000in}}%
\pgfusepath{stroke,fill}%
}%
\begin{pgfscope}%
\pgfsys@transformshift{3.373521in}{1.790852in}%
\pgfsys@useobject{currentmarker}{}%
\end{pgfscope}%
\end{pgfscope}%
\begin{pgfscope}%
\pgfsetbuttcap%
\pgfsetroundjoin%
\definecolor{currentfill}{rgb}{0.000000,0.000000,0.000000}%
\pgfsetfillcolor{currentfill}%
\pgfsetlinewidth{0.501875pt}%
\definecolor{currentstroke}{rgb}{0.000000,0.000000,0.000000}%
\pgfsetstrokecolor{currentstroke}%
\pgfsetdash{}{0pt}%
\pgfsys@defobject{currentmarker}{\pgfqpoint{-0.055556in}{0.000000in}}{\pgfqpoint{0.000000in}{0.000000in}}{%
\pgfpathmoveto{\pgfqpoint{0.000000in}{0.000000in}}%
\pgfpathlineto{\pgfqpoint{-0.055556in}{0.000000in}}%
\pgfusepath{stroke,fill}%
}%
\begin{pgfscope}%
\pgfsys@transformshift{5.840000in}{1.790852in}%
\pgfsys@useobject{currentmarker}{}%
\end{pgfscope}%
\end{pgfscope}%
\begin{pgfscope}%
\pgfsetbuttcap%
\pgfsetroundjoin%
\definecolor{currentfill}{rgb}{0.000000,0.000000,0.000000}%
\pgfsetfillcolor{currentfill}%
\pgfsetlinewidth{0.501875pt}%
\definecolor{currentstroke}{rgb}{0.000000,0.000000,0.000000}%
\pgfsetstrokecolor{currentstroke}%
\pgfsetdash{}{0pt}%
\pgfsys@defobject{currentmarker}{\pgfqpoint{0.000000in}{0.000000in}}{\pgfqpoint{0.055556in}{0.000000in}}{%
\pgfpathmoveto{\pgfqpoint{0.000000in}{0.000000in}}%
\pgfpathlineto{\pgfqpoint{0.055556in}{0.000000in}}%
\pgfusepath{stroke,fill}%
}%
\begin{pgfscope}%
\pgfsys@transformshift{3.373521in}{2.284148in}%
\pgfsys@useobject{currentmarker}{}%
\end{pgfscope}%
\end{pgfscope}%
\begin{pgfscope}%
\pgfsetbuttcap%
\pgfsetroundjoin%
\definecolor{currentfill}{rgb}{0.000000,0.000000,0.000000}%
\pgfsetfillcolor{currentfill}%
\pgfsetlinewidth{0.501875pt}%
\definecolor{currentstroke}{rgb}{0.000000,0.000000,0.000000}%
\pgfsetstrokecolor{currentstroke}%
\pgfsetdash{}{0pt}%
\pgfsys@defobject{currentmarker}{\pgfqpoint{-0.055556in}{0.000000in}}{\pgfqpoint{0.000000in}{0.000000in}}{%
\pgfpathmoveto{\pgfqpoint{0.000000in}{0.000000in}}%
\pgfpathlineto{\pgfqpoint{-0.055556in}{0.000000in}}%
\pgfusepath{stroke,fill}%
}%
\begin{pgfscope}%
\pgfsys@transformshift{5.840000in}{2.284148in}%
\pgfsys@useobject{currentmarker}{}%
\end{pgfscope}%
\end{pgfscope}%
\begin{pgfscope}%
\pgfsetbuttcap%
\pgfsetroundjoin%
\definecolor{currentfill}{rgb}{0.000000,0.000000,0.000000}%
\pgfsetfillcolor{currentfill}%
\pgfsetlinewidth{0.501875pt}%
\definecolor{currentstroke}{rgb}{0.000000,0.000000,0.000000}%
\pgfsetstrokecolor{currentstroke}%
\pgfsetdash{}{0pt}%
\pgfsys@defobject{currentmarker}{\pgfqpoint{0.000000in}{0.000000in}}{\pgfqpoint{0.055556in}{0.000000in}}{%
\pgfpathmoveto{\pgfqpoint{0.000000in}{0.000000in}}%
\pgfpathlineto{\pgfqpoint{0.055556in}{0.000000in}}%
\pgfusepath{stroke,fill}%
}%
\begin{pgfscope}%
\pgfsys@transformshift{3.373521in}{2.777444in}%
\pgfsys@useobject{currentmarker}{}%
\end{pgfscope}%
\end{pgfscope}%
\begin{pgfscope}%
\pgfsetbuttcap%
\pgfsetroundjoin%
\definecolor{currentfill}{rgb}{0.000000,0.000000,0.000000}%
\pgfsetfillcolor{currentfill}%
\pgfsetlinewidth{0.501875pt}%
\definecolor{currentstroke}{rgb}{0.000000,0.000000,0.000000}%
\pgfsetstrokecolor{currentstroke}%
\pgfsetdash{}{0pt}%
\pgfsys@defobject{currentmarker}{\pgfqpoint{-0.055556in}{0.000000in}}{\pgfqpoint{0.000000in}{0.000000in}}{%
\pgfpathmoveto{\pgfqpoint{0.000000in}{0.000000in}}%
\pgfpathlineto{\pgfqpoint{-0.055556in}{0.000000in}}%
\pgfusepath{stroke,fill}%
}%
\begin{pgfscope}%
\pgfsys@transformshift{5.840000in}{2.777444in}%
\pgfsys@useobject{currentmarker}{}%
\end{pgfscope}%
\end{pgfscope}%
\begin{pgfscope}%
\pgfsetbuttcap%
\pgfsetroundjoin%
\definecolor{currentfill}{rgb}{0.000000,0.000000,0.000000}%
\pgfsetfillcolor{currentfill}%
\pgfsetlinewidth{0.501875pt}%
\definecolor{currentstroke}{rgb}{0.000000,0.000000,0.000000}%
\pgfsetstrokecolor{currentstroke}%
\pgfsetdash{}{0pt}%
\pgfsys@defobject{currentmarker}{\pgfqpoint{0.000000in}{0.000000in}}{\pgfqpoint{0.055556in}{0.000000in}}{%
\pgfpathmoveto{\pgfqpoint{0.000000in}{0.000000in}}%
\pgfpathlineto{\pgfqpoint{0.055556in}{0.000000in}}%
\pgfusepath{stroke,fill}%
}%
\begin{pgfscope}%
\pgfsys@transformshift{3.373521in}{3.270740in}%
\pgfsys@useobject{currentmarker}{}%
\end{pgfscope}%
\end{pgfscope}%
\begin{pgfscope}%
\pgfsetbuttcap%
\pgfsetroundjoin%
\definecolor{currentfill}{rgb}{0.000000,0.000000,0.000000}%
\pgfsetfillcolor{currentfill}%
\pgfsetlinewidth{0.501875pt}%
\definecolor{currentstroke}{rgb}{0.000000,0.000000,0.000000}%
\pgfsetstrokecolor{currentstroke}%
\pgfsetdash{}{0pt}%
\pgfsys@defobject{currentmarker}{\pgfqpoint{-0.055556in}{0.000000in}}{\pgfqpoint{0.000000in}{0.000000in}}{%
\pgfpathmoveto{\pgfqpoint{0.000000in}{0.000000in}}%
\pgfpathlineto{\pgfqpoint{-0.055556in}{0.000000in}}%
\pgfusepath{stroke,fill}%
}%
\begin{pgfscope}%
\pgfsys@transformshift{5.840000in}{3.270740in}%
\pgfsys@useobject{currentmarker}{}%
\end{pgfscope}%
\end{pgfscope}%
\begin{pgfscope}%
\pgfsetbuttcap%
\pgfsetroundjoin%
\pgfsetlinewidth{1.003750pt}%
\definecolor{currentstroke}{rgb}{0.000000,0.000000,0.000000}%
\pgfsetstrokecolor{currentstroke}%
\pgfsetdash{}{0pt}%
\pgfpathmoveto{\pgfqpoint{3.373521in}{3.270740in}}%
\pgfpathlineto{\pgfqpoint{5.840000in}{3.270740in}}%
\pgfusepath{stroke}%
\end{pgfscope}%
\begin{pgfscope}%
\pgfsetbuttcap%
\pgfsetroundjoin%
\pgfsetlinewidth{1.003750pt}%
\definecolor{currentstroke}{rgb}{0.000000,0.000000,0.000000}%
\pgfsetstrokecolor{currentstroke}%
\pgfsetdash{}{0pt}%
\pgfpathmoveto{\pgfqpoint{5.840000in}{0.804260in}}%
\pgfpathlineto{\pgfqpoint{5.840000in}{3.270740in}}%
\pgfusepath{stroke}%
\end{pgfscope}%
\begin{pgfscope}%
\pgfsetbuttcap%
\pgfsetroundjoin%
\pgfsetlinewidth{1.003750pt}%
\definecolor{currentstroke}{rgb}{0.000000,0.000000,0.000000}%
\pgfsetstrokecolor{currentstroke}%
\pgfsetdash{}{0pt}%
\pgfpathmoveto{\pgfqpoint{3.373521in}{0.804260in}}%
\pgfpathlineto{\pgfqpoint{5.840000in}{0.804260in}}%
\pgfusepath{stroke}%
\end{pgfscope}%
\begin{pgfscope}%
\pgfsetbuttcap%
\pgfsetroundjoin%
\pgfsetlinewidth{1.003750pt}%
\definecolor{currentstroke}{rgb}{0.000000,0.000000,0.000000}%
\pgfsetstrokecolor{currentstroke}%
\pgfsetdash{}{0pt}%
\pgfpathmoveto{\pgfqpoint{3.373521in}{0.804260in}}%
\pgfpathlineto{\pgfqpoint{3.373521in}{3.270740in}}%
\pgfusepath{stroke}%
\end{pgfscope}%
\begin{pgfscope}%
\pgfpathrectangle{\pgfqpoint{6.205000in}{0.611250in}}{\pgfqpoint{0.365000in}{2.852500in}} %
\pgfusepath{clip}%
\pgfsetbuttcap%
\pgfsetroundjoin%
\definecolor{currentfill}{rgb}{1.000000,1.000000,1.000000}%
\pgfsetfillcolor{currentfill}%
\pgfsetlinewidth{0.010037pt}%
\definecolor{currentstroke}{rgb}{1.000000,1.000000,1.000000}%
\pgfsetstrokecolor{currentstroke}%
\pgfsetdash{}{0pt}%
\pgfpathmoveto{\pgfqpoint{6.205000in}{0.611250in}}%
\pgfpathlineto{\pgfqpoint{6.205000in}{0.622393in}}%
\pgfpathlineto{\pgfqpoint{6.205000in}{3.452607in}}%
\pgfpathlineto{\pgfqpoint{6.205000in}{3.463750in}}%
\pgfpathlineto{\pgfqpoint{6.570000in}{3.463750in}}%
\pgfpathlineto{\pgfqpoint{6.570000in}{3.452607in}}%
\pgfpathlineto{\pgfqpoint{6.570000in}{0.622393in}}%
\pgfpathlineto{\pgfqpoint{6.570000in}{0.611250in}}%
\pgfpathclose%
\pgfusepath{stroke,fill}%
\end{pgfscope}%
\begin{pgfscope}%
\pgfpathrectangle{\pgfqpoint{6.205000in}{0.611250in}}{\pgfqpoint{0.365000in}{2.852500in}} %
\pgfusepath{clip}%
\pgfsetbuttcap%
\pgfsetroundjoin%
\definecolor{currentfill}{rgb}{0.229806,0.298718,0.753683}%
\pgfsetfillcolor{currentfill}%
\pgfsetlinewidth{0.000000pt}%
\definecolor{currentstroke}{rgb}{0.000000,0.000000,0.000000}%
\pgfsetstrokecolor{currentstroke}%
\pgfsetdash{}{0pt}%
\pgfpathmoveto{\pgfqpoint{6.205000in}{0.611250in}}%
\pgfpathlineto{\pgfqpoint{6.570000in}{0.611250in}}%
\pgfpathlineto{\pgfqpoint{6.570000in}{0.622393in}}%
\pgfpathlineto{\pgfqpoint{6.205000in}{0.622393in}}%
\pgfpathlineto{\pgfqpoint{6.205000in}{0.611250in}}%
\pgfusepath{fill}%
\end{pgfscope}%
\begin{pgfscope}%
\pgfpathrectangle{\pgfqpoint{6.205000in}{0.611250in}}{\pgfqpoint{0.365000in}{2.852500in}} %
\pgfusepath{clip}%
\pgfsetbuttcap%
\pgfsetroundjoin%
\definecolor{currentfill}{rgb}{0.234377,0.305542,0.759680}%
\pgfsetfillcolor{currentfill}%
\pgfsetlinewidth{0.000000pt}%
\definecolor{currentstroke}{rgb}{0.000000,0.000000,0.000000}%
\pgfsetstrokecolor{currentstroke}%
\pgfsetdash{}{0pt}%
\pgfpathmoveto{\pgfqpoint{6.205000in}{0.622393in}}%
\pgfpathlineto{\pgfqpoint{6.570000in}{0.622393in}}%
\pgfpathlineto{\pgfqpoint{6.570000in}{0.633535in}}%
\pgfpathlineto{\pgfqpoint{6.205000in}{0.633535in}}%
\pgfpathlineto{\pgfqpoint{6.205000in}{0.622393in}}%
\pgfusepath{fill}%
\end{pgfscope}%
\begin{pgfscope}%
\pgfpathrectangle{\pgfqpoint{6.205000in}{0.611250in}}{\pgfqpoint{0.365000in}{2.852500in}} %
\pgfusepath{clip}%
\pgfsetbuttcap%
\pgfsetroundjoin%
\definecolor{currentfill}{rgb}{0.238948,0.312365,0.765676}%
\pgfsetfillcolor{currentfill}%
\pgfsetlinewidth{0.000000pt}%
\definecolor{currentstroke}{rgb}{0.000000,0.000000,0.000000}%
\pgfsetstrokecolor{currentstroke}%
\pgfsetdash{}{0pt}%
\pgfpathmoveto{\pgfqpoint{6.205000in}{0.633535in}}%
\pgfpathlineto{\pgfqpoint{6.570000in}{0.633535in}}%
\pgfpathlineto{\pgfqpoint{6.570000in}{0.644678in}}%
\pgfpathlineto{\pgfqpoint{6.205000in}{0.644678in}}%
\pgfpathlineto{\pgfqpoint{6.205000in}{0.633535in}}%
\pgfusepath{fill}%
\end{pgfscope}%
\begin{pgfscope}%
\pgfpathrectangle{\pgfqpoint{6.205000in}{0.611250in}}{\pgfqpoint{0.365000in}{2.852500in}} %
\pgfusepath{clip}%
\pgfsetbuttcap%
\pgfsetroundjoin%
\definecolor{currentfill}{rgb}{0.243520,0.319189,0.771672}%
\pgfsetfillcolor{currentfill}%
\pgfsetlinewidth{0.000000pt}%
\definecolor{currentstroke}{rgb}{0.000000,0.000000,0.000000}%
\pgfsetstrokecolor{currentstroke}%
\pgfsetdash{}{0pt}%
\pgfpathmoveto{\pgfqpoint{6.205000in}{0.644678in}}%
\pgfpathlineto{\pgfqpoint{6.570000in}{0.644678in}}%
\pgfpathlineto{\pgfqpoint{6.570000in}{0.655820in}}%
\pgfpathlineto{\pgfqpoint{6.205000in}{0.655820in}}%
\pgfpathlineto{\pgfqpoint{6.205000in}{0.644678in}}%
\pgfusepath{fill}%
\end{pgfscope}%
\begin{pgfscope}%
\pgfpathrectangle{\pgfqpoint{6.205000in}{0.611250in}}{\pgfqpoint{0.365000in}{2.852500in}} %
\pgfusepath{clip}%
\pgfsetbuttcap%
\pgfsetroundjoin%
\definecolor{currentfill}{rgb}{0.248091,0.326013,0.777669}%
\pgfsetfillcolor{currentfill}%
\pgfsetlinewidth{0.000000pt}%
\definecolor{currentstroke}{rgb}{0.000000,0.000000,0.000000}%
\pgfsetstrokecolor{currentstroke}%
\pgfsetdash{}{0pt}%
\pgfpathmoveto{\pgfqpoint{6.205000in}{0.655820in}}%
\pgfpathlineto{\pgfqpoint{6.570000in}{0.655820in}}%
\pgfpathlineto{\pgfqpoint{6.570000in}{0.666963in}}%
\pgfpathlineto{\pgfqpoint{6.205000in}{0.666963in}}%
\pgfpathlineto{\pgfqpoint{6.205000in}{0.655820in}}%
\pgfusepath{fill}%
\end{pgfscope}%
\begin{pgfscope}%
\pgfpathrectangle{\pgfqpoint{6.205000in}{0.611250in}}{\pgfqpoint{0.365000in}{2.852500in}} %
\pgfusepath{clip}%
\pgfsetbuttcap%
\pgfsetroundjoin%
\definecolor{currentfill}{rgb}{0.252663,0.332837,0.783665}%
\pgfsetfillcolor{currentfill}%
\pgfsetlinewidth{0.000000pt}%
\definecolor{currentstroke}{rgb}{0.000000,0.000000,0.000000}%
\pgfsetstrokecolor{currentstroke}%
\pgfsetdash{}{0pt}%
\pgfpathmoveto{\pgfqpoint{6.205000in}{0.666963in}}%
\pgfpathlineto{\pgfqpoint{6.570000in}{0.666963in}}%
\pgfpathlineto{\pgfqpoint{6.570000in}{0.678105in}}%
\pgfpathlineto{\pgfqpoint{6.205000in}{0.678105in}}%
\pgfpathlineto{\pgfqpoint{6.205000in}{0.666963in}}%
\pgfusepath{fill}%
\end{pgfscope}%
\begin{pgfscope}%
\pgfpathrectangle{\pgfqpoint{6.205000in}{0.611250in}}{\pgfqpoint{0.365000in}{2.852500in}} %
\pgfusepath{clip}%
\pgfsetbuttcap%
\pgfsetroundjoin%
\definecolor{currentfill}{rgb}{0.257234,0.339661,0.789661}%
\pgfsetfillcolor{currentfill}%
\pgfsetlinewidth{0.000000pt}%
\definecolor{currentstroke}{rgb}{0.000000,0.000000,0.000000}%
\pgfsetstrokecolor{currentstroke}%
\pgfsetdash{}{0pt}%
\pgfpathmoveto{\pgfqpoint{6.205000in}{0.678105in}}%
\pgfpathlineto{\pgfqpoint{6.570000in}{0.678105in}}%
\pgfpathlineto{\pgfqpoint{6.570000in}{0.689248in}}%
\pgfpathlineto{\pgfqpoint{6.205000in}{0.689248in}}%
\pgfpathlineto{\pgfqpoint{6.205000in}{0.678105in}}%
\pgfusepath{fill}%
\end{pgfscope}%
\begin{pgfscope}%
\pgfpathrectangle{\pgfqpoint{6.205000in}{0.611250in}}{\pgfqpoint{0.365000in}{2.852500in}} %
\pgfusepath{clip}%
\pgfsetbuttcap%
\pgfsetroundjoin%
\definecolor{currentfill}{rgb}{0.261805,0.346484,0.795658}%
\pgfsetfillcolor{currentfill}%
\pgfsetlinewidth{0.000000pt}%
\definecolor{currentstroke}{rgb}{0.000000,0.000000,0.000000}%
\pgfsetstrokecolor{currentstroke}%
\pgfsetdash{}{0pt}%
\pgfpathmoveto{\pgfqpoint{6.205000in}{0.689248in}}%
\pgfpathlineto{\pgfqpoint{6.570000in}{0.689248in}}%
\pgfpathlineto{\pgfqpoint{6.570000in}{0.700391in}}%
\pgfpathlineto{\pgfqpoint{6.205000in}{0.700391in}}%
\pgfpathlineto{\pgfqpoint{6.205000in}{0.689248in}}%
\pgfusepath{fill}%
\end{pgfscope}%
\begin{pgfscope}%
\pgfpathrectangle{\pgfqpoint{6.205000in}{0.611250in}}{\pgfqpoint{0.365000in}{2.852500in}} %
\pgfusepath{clip}%
\pgfsetbuttcap%
\pgfsetroundjoin%
\definecolor{currentfill}{rgb}{0.266381,0.353304,0.801637}%
\pgfsetfillcolor{currentfill}%
\pgfsetlinewidth{0.000000pt}%
\definecolor{currentstroke}{rgb}{0.000000,0.000000,0.000000}%
\pgfsetstrokecolor{currentstroke}%
\pgfsetdash{}{0pt}%
\pgfpathmoveto{\pgfqpoint{6.205000in}{0.700391in}}%
\pgfpathlineto{\pgfqpoint{6.570000in}{0.700391in}}%
\pgfpathlineto{\pgfqpoint{6.570000in}{0.711533in}}%
\pgfpathlineto{\pgfqpoint{6.205000in}{0.711533in}}%
\pgfpathlineto{\pgfqpoint{6.205000in}{0.700391in}}%
\pgfusepath{fill}%
\end{pgfscope}%
\begin{pgfscope}%
\pgfpathrectangle{\pgfqpoint{6.205000in}{0.611250in}}{\pgfqpoint{0.365000in}{2.852500in}} %
\pgfusepath{clip}%
\pgfsetbuttcap%
\pgfsetroundjoin%
\definecolor{currentfill}{rgb}{0.271104,0.360011,0.807095}%
\pgfsetfillcolor{currentfill}%
\pgfsetlinewidth{0.000000pt}%
\definecolor{currentstroke}{rgb}{0.000000,0.000000,0.000000}%
\pgfsetstrokecolor{currentstroke}%
\pgfsetdash{}{0pt}%
\pgfpathmoveto{\pgfqpoint{6.205000in}{0.711533in}}%
\pgfpathlineto{\pgfqpoint{6.570000in}{0.711533in}}%
\pgfpathlineto{\pgfqpoint{6.570000in}{0.722676in}}%
\pgfpathlineto{\pgfqpoint{6.205000in}{0.722676in}}%
\pgfpathlineto{\pgfqpoint{6.205000in}{0.711533in}}%
\pgfusepath{fill}%
\end{pgfscope}%
\begin{pgfscope}%
\pgfpathrectangle{\pgfqpoint{6.205000in}{0.611250in}}{\pgfqpoint{0.365000in}{2.852500in}} %
\pgfusepath{clip}%
\pgfsetbuttcap%
\pgfsetroundjoin%
\definecolor{currentfill}{rgb}{0.275827,0.366717,0.812553}%
\pgfsetfillcolor{currentfill}%
\pgfsetlinewidth{0.000000pt}%
\definecolor{currentstroke}{rgb}{0.000000,0.000000,0.000000}%
\pgfsetstrokecolor{currentstroke}%
\pgfsetdash{}{0pt}%
\pgfpathmoveto{\pgfqpoint{6.205000in}{0.722676in}}%
\pgfpathlineto{\pgfqpoint{6.570000in}{0.722676in}}%
\pgfpathlineto{\pgfqpoint{6.570000in}{0.733818in}}%
\pgfpathlineto{\pgfqpoint{6.205000in}{0.733818in}}%
\pgfpathlineto{\pgfqpoint{6.205000in}{0.722676in}}%
\pgfusepath{fill}%
\end{pgfscope}%
\begin{pgfscope}%
\pgfpathrectangle{\pgfqpoint{6.205000in}{0.611250in}}{\pgfqpoint{0.365000in}{2.852500in}} %
\pgfusepath{clip}%
\pgfsetbuttcap%
\pgfsetroundjoin%
\definecolor{currentfill}{rgb}{0.280550,0.373423,0.818011}%
\pgfsetfillcolor{currentfill}%
\pgfsetlinewidth{0.000000pt}%
\definecolor{currentstroke}{rgb}{0.000000,0.000000,0.000000}%
\pgfsetstrokecolor{currentstroke}%
\pgfsetdash{}{0pt}%
\pgfpathmoveto{\pgfqpoint{6.205000in}{0.733818in}}%
\pgfpathlineto{\pgfqpoint{6.570000in}{0.733818in}}%
\pgfpathlineto{\pgfqpoint{6.570000in}{0.744961in}}%
\pgfpathlineto{\pgfqpoint{6.205000in}{0.744961in}}%
\pgfpathlineto{\pgfqpoint{6.205000in}{0.733818in}}%
\pgfusepath{fill}%
\end{pgfscope}%
\begin{pgfscope}%
\pgfpathrectangle{\pgfqpoint{6.205000in}{0.611250in}}{\pgfqpoint{0.365000in}{2.852500in}} %
\pgfusepath{clip}%
\pgfsetbuttcap%
\pgfsetroundjoin%
\definecolor{currentfill}{rgb}{0.285273,0.380129,0.823469}%
\pgfsetfillcolor{currentfill}%
\pgfsetlinewidth{0.000000pt}%
\definecolor{currentstroke}{rgb}{0.000000,0.000000,0.000000}%
\pgfsetstrokecolor{currentstroke}%
\pgfsetdash{}{0pt}%
\pgfpathmoveto{\pgfqpoint{6.205000in}{0.744961in}}%
\pgfpathlineto{\pgfqpoint{6.570000in}{0.744961in}}%
\pgfpathlineto{\pgfqpoint{6.570000in}{0.756104in}}%
\pgfpathlineto{\pgfqpoint{6.205000in}{0.756104in}}%
\pgfpathlineto{\pgfqpoint{6.205000in}{0.744961in}}%
\pgfusepath{fill}%
\end{pgfscope}%
\begin{pgfscope}%
\pgfpathrectangle{\pgfqpoint{6.205000in}{0.611250in}}{\pgfqpoint{0.365000in}{2.852500in}} %
\pgfusepath{clip}%
\pgfsetbuttcap%
\pgfsetroundjoin%
\definecolor{currentfill}{rgb}{0.289996,0.386836,0.828926}%
\pgfsetfillcolor{currentfill}%
\pgfsetlinewidth{0.000000pt}%
\definecolor{currentstroke}{rgb}{0.000000,0.000000,0.000000}%
\pgfsetstrokecolor{currentstroke}%
\pgfsetdash{}{0pt}%
\pgfpathmoveto{\pgfqpoint{6.205000in}{0.756104in}}%
\pgfpathlineto{\pgfqpoint{6.570000in}{0.756104in}}%
\pgfpathlineto{\pgfqpoint{6.570000in}{0.767246in}}%
\pgfpathlineto{\pgfqpoint{6.205000in}{0.767246in}}%
\pgfpathlineto{\pgfqpoint{6.205000in}{0.756104in}}%
\pgfusepath{fill}%
\end{pgfscope}%
\begin{pgfscope}%
\pgfpathrectangle{\pgfqpoint{6.205000in}{0.611250in}}{\pgfqpoint{0.365000in}{2.852500in}} %
\pgfusepath{clip}%
\pgfsetbuttcap%
\pgfsetroundjoin%
\definecolor{currentfill}{rgb}{0.294718,0.393542,0.834384}%
\pgfsetfillcolor{currentfill}%
\pgfsetlinewidth{0.000000pt}%
\definecolor{currentstroke}{rgb}{0.000000,0.000000,0.000000}%
\pgfsetstrokecolor{currentstroke}%
\pgfsetdash{}{0pt}%
\pgfpathmoveto{\pgfqpoint{6.205000in}{0.767246in}}%
\pgfpathlineto{\pgfqpoint{6.570000in}{0.767246in}}%
\pgfpathlineto{\pgfqpoint{6.570000in}{0.778389in}}%
\pgfpathlineto{\pgfqpoint{6.205000in}{0.778389in}}%
\pgfpathlineto{\pgfqpoint{6.205000in}{0.767246in}}%
\pgfusepath{fill}%
\end{pgfscope}%
\begin{pgfscope}%
\pgfpathrectangle{\pgfqpoint{6.205000in}{0.611250in}}{\pgfqpoint{0.365000in}{2.852500in}} %
\pgfusepath{clip}%
\pgfsetbuttcap%
\pgfsetroundjoin%
\definecolor{currentfill}{rgb}{0.299441,0.400248,0.839842}%
\pgfsetfillcolor{currentfill}%
\pgfsetlinewidth{0.000000pt}%
\definecolor{currentstroke}{rgb}{0.000000,0.000000,0.000000}%
\pgfsetstrokecolor{currentstroke}%
\pgfsetdash{}{0pt}%
\pgfpathmoveto{\pgfqpoint{6.205000in}{0.778389in}}%
\pgfpathlineto{\pgfqpoint{6.570000in}{0.778389in}}%
\pgfpathlineto{\pgfqpoint{6.570000in}{0.789531in}}%
\pgfpathlineto{\pgfqpoint{6.205000in}{0.789531in}}%
\pgfpathlineto{\pgfqpoint{6.205000in}{0.778389in}}%
\pgfusepath{fill}%
\end{pgfscope}%
\begin{pgfscope}%
\pgfpathrectangle{\pgfqpoint{6.205000in}{0.611250in}}{\pgfqpoint{0.365000in}{2.852500in}} %
\pgfusepath{clip}%
\pgfsetbuttcap%
\pgfsetroundjoin%
\definecolor{currentfill}{rgb}{0.304174,0.406945,0.845263}%
\pgfsetfillcolor{currentfill}%
\pgfsetlinewidth{0.000000pt}%
\definecolor{currentstroke}{rgb}{0.000000,0.000000,0.000000}%
\pgfsetstrokecolor{currentstroke}%
\pgfsetdash{}{0pt}%
\pgfpathmoveto{\pgfqpoint{6.205000in}{0.789531in}}%
\pgfpathlineto{\pgfqpoint{6.570000in}{0.789531in}}%
\pgfpathlineto{\pgfqpoint{6.570000in}{0.800674in}}%
\pgfpathlineto{\pgfqpoint{6.205000in}{0.800674in}}%
\pgfpathlineto{\pgfqpoint{6.205000in}{0.789531in}}%
\pgfusepath{fill}%
\end{pgfscope}%
\begin{pgfscope}%
\pgfpathrectangle{\pgfqpoint{6.205000in}{0.611250in}}{\pgfqpoint{0.365000in}{2.852500in}} %
\pgfusepath{clip}%
\pgfsetbuttcap%
\pgfsetroundjoin%
\definecolor{currentfill}{rgb}{0.309060,0.413498,0.850128}%
\pgfsetfillcolor{currentfill}%
\pgfsetlinewidth{0.000000pt}%
\definecolor{currentstroke}{rgb}{0.000000,0.000000,0.000000}%
\pgfsetstrokecolor{currentstroke}%
\pgfsetdash{}{0pt}%
\pgfpathmoveto{\pgfqpoint{6.205000in}{0.800674in}}%
\pgfpathlineto{\pgfqpoint{6.570000in}{0.800674in}}%
\pgfpathlineto{\pgfqpoint{6.570000in}{0.811816in}}%
\pgfpathlineto{\pgfqpoint{6.205000in}{0.811816in}}%
\pgfpathlineto{\pgfqpoint{6.205000in}{0.800674in}}%
\pgfusepath{fill}%
\end{pgfscope}%
\begin{pgfscope}%
\pgfpathrectangle{\pgfqpoint{6.205000in}{0.611250in}}{\pgfqpoint{0.365000in}{2.852500in}} %
\pgfusepath{clip}%
\pgfsetbuttcap%
\pgfsetroundjoin%
\definecolor{currentfill}{rgb}{0.313946,0.420052,0.854993}%
\pgfsetfillcolor{currentfill}%
\pgfsetlinewidth{0.000000pt}%
\definecolor{currentstroke}{rgb}{0.000000,0.000000,0.000000}%
\pgfsetstrokecolor{currentstroke}%
\pgfsetdash{}{0pt}%
\pgfpathmoveto{\pgfqpoint{6.205000in}{0.811816in}}%
\pgfpathlineto{\pgfqpoint{6.570000in}{0.811816in}}%
\pgfpathlineto{\pgfqpoint{6.570000in}{0.822959in}}%
\pgfpathlineto{\pgfqpoint{6.205000in}{0.822959in}}%
\pgfpathlineto{\pgfqpoint{6.205000in}{0.811816in}}%
\pgfusepath{fill}%
\end{pgfscope}%
\begin{pgfscope}%
\pgfpathrectangle{\pgfqpoint{6.205000in}{0.611250in}}{\pgfqpoint{0.365000in}{2.852500in}} %
\pgfusepath{clip}%
\pgfsetbuttcap%
\pgfsetroundjoin%
\definecolor{currentfill}{rgb}{0.318832,0.426605,0.859857}%
\pgfsetfillcolor{currentfill}%
\pgfsetlinewidth{0.000000pt}%
\definecolor{currentstroke}{rgb}{0.000000,0.000000,0.000000}%
\pgfsetstrokecolor{currentstroke}%
\pgfsetdash{}{0pt}%
\pgfpathmoveto{\pgfqpoint{6.205000in}{0.822959in}}%
\pgfpathlineto{\pgfqpoint{6.570000in}{0.822959in}}%
\pgfpathlineto{\pgfqpoint{6.570000in}{0.834102in}}%
\pgfpathlineto{\pgfqpoint{6.205000in}{0.834102in}}%
\pgfpathlineto{\pgfqpoint{6.205000in}{0.822959in}}%
\pgfusepath{fill}%
\end{pgfscope}%
\begin{pgfscope}%
\pgfpathrectangle{\pgfqpoint{6.205000in}{0.611250in}}{\pgfqpoint{0.365000in}{2.852500in}} %
\pgfusepath{clip}%
\pgfsetbuttcap%
\pgfsetroundjoin%
\definecolor{currentfill}{rgb}{0.323718,0.433158,0.864722}%
\pgfsetfillcolor{currentfill}%
\pgfsetlinewidth{0.000000pt}%
\definecolor{currentstroke}{rgb}{0.000000,0.000000,0.000000}%
\pgfsetstrokecolor{currentstroke}%
\pgfsetdash{}{0pt}%
\pgfpathmoveto{\pgfqpoint{6.205000in}{0.834102in}}%
\pgfpathlineto{\pgfqpoint{6.570000in}{0.834102in}}%
\pgfpathlineto{\pgfqpoint{6.570000in}{0.845244in}}%
\pgfpathlineto{\pgfqpoint{6.205000in}{0.845244in}}%
\pgfpathlineto{\pgfqpoint{6.205000in}{0.834102in}}%
\pgfusepath{fill}%
\end{pgfscope}%
\begin{pgfscope}%
\pgfpathrectangle{\pgfqpoint{6.205000in}{0.611250in}}{\pgfqpoint{0.365000in}{2.852500in}} %
\pgfusepath{clip}%
\pgfsetbuttcap%
\pgfsetroundjoin%
\definecolor{currentfill}{rgb}{0.328604,0.439712,0.869587}%
\pgfsetfillcolor{currentfill}%
\pgfsetlinewidth{0.000000pt}%
\definecolor{currentstroke}{rgb}{0.000000,0.000000,0.000000}%
\pgfsetstrokecolor{currentstroke}%
\pgfsetdash{}{0pt}%
\pgfpathmoveto{\pgfqpoint{6.205000in}{0.845244in}}%
\pgfpathlineto{\pgfqpoint{6.570000in}{0.845244in}}%
\pgfpathlineto{\pgfqpoint{6.570000in}{0.856387in}}%
\pgfpathlineto{\pgfqpoint{6.205000in}{0.856387in}}%
\pgfpathlineto{\pgfqpoint{6.205000in}{0.845244in}}%
\pgfusepath{fill}%
\end{pgfscope}%
\begin{pgfscope}%
\pgfpathrectangle{\pgfqpoint{6.205000in}{0.611250in}}{\pgfqpoint{0.365000in}{2.852500in}} %
\pgfusepath{clip}%
\pgfsetbuttcap%
\pgfsetroundjoin%
\definecolor{currentfill}{rgb}{0.333490,0.446265,0.874452}%
\pgfsetfillcolor{currentfill}%
\pgfsetlinewidth{0.000000pt}%
\definecolor{currentstroke}{rgb}{0.000000,0.000000,0.000000}%
\pgfsetstrokecolor{currentstroke}%
\pgfsetdash{}{0pt}%
\pgfpathmoveto{\pgfqpoint{6.205000in}{0.856387in}}%
\pgfpathlineto{\pgfqpoint{6.570000in}{0.856387in}}%
\pgfpathlineto{\pgfqpoint{6.570000in}{0.867529in}}%
\pgfpathlineto{\pgfqpoint{6.205000in}{0.867529in}}%
\pgfpathlineto{\pgfqpoint{6.205000in}{0.856387in}}%
\pgfusepath{fill}%
\end{pgfscope}%
\begin{pgfscope}%
\pgfpathrectangle{\pgfqpoint{6.205000in}{0.611250in}}{\pgfqpoint{0.365000in}{2.852500in}} %
\pgfusepath{clip}%
\pgfsetbuttcap%
\pgfsetroundjoin%
\definecolor{currentfill}{rgb}{0.338377,0.452819,0.879317}%
\pgfsetfillcolor{currentfill}%
\pgfsetlinewidth{0.000000pt}%
\definecolor{currentstroke}{rgb}{0.000000,0.000000,0.000000}%
\pgfsetstrokecolor{currentstroke}%
\pgfsetdash{}{0pt}%
\pgfpathmoveto{\pgfqpoint{6.205000in}{0.867529in}}%
\pgfpathlineto{\pgfqpoint{6.570000in}{0.867529in}}%
\pgfpathlineto{\pgfqpoint{6.570000in}{0.878672in}}%
\pgfpathlineto{\pgfqpoint{6.205000in}{0.878672in}}%
\pgfpathlineto{\pgfqpoint{6.205000in}{0.867529in}}%
\pgfusepath{fill}%
\end{pgfscope}%
\begin{pgfscope}%
\pgfpathrectangle{\pgfqpoint{6.205000in}{0.611250in}}{\pgfqpoint{0.365000in}{2.852500in}} %
\pgfusepath{clip}%
\pgfsetbuttcap%
\pgfsetroundjoin%
\definecolor{currentfill}{rgb}{0.343278,0.459354,0.884122}%
\pgfsetfillcolor{currentfill}%
\pgfsetlinewidth{0.000000pt}%
\definecolor{currentstroke}{rgb}{0.000000,0.000000,0.000000}%
\pgfsetstrokecolor{currentstroke}%
\pgfsetdash{}{0pt}%
\pgfpathmoveto{\pgfqpoint{6.205000in}{0.878672in}}%
\pgfpathlineto{\pgfqpoint{6.570000in}{0.878672in}}%
\pgfpathlineto{\pgfqpoint{6.570000in}{0.889814in}}%
\pgfpathlineto{\pgfqpoint{6.205000in}{0.889814in}}%
\pgfpathlineto{\pgfqpoint{6.205000in}{0.878672in}}%
\pgfusepath{fill}%
\end{pgfscope}%
\begin{pgfscope}%
\pgfpathrectangle{\pgfqpoint{6.205000in}{0.611250in}}{\pgfqpoint{0.365000in}{2.852500in}} %
\pgfusepath{clip}%
\pgfsetbuttcap%
\pgfsetroundjoin%
\definecolor{currentfill}{rgb}{0.348323,0.465711,0.888346}%
\pgfsetfillcolor{currentfill}%
\pgfsetlinewidth{0.000000pt}%
\definecolor{currentstroke}{rgb}{0.000000,0.000000,0.000000}%
\pgfsetstrokecolor{currentstroke}%
\pgfsetdash{}{0pt}%
\pgfpathmoveto{\pgfqpoint{6.205000in}{0.889814in}}%
\pgfpathlineto{\pgfqpoint{6.570000in}{0.889814in}}%
\pgfpathlineto{\pgfqpoint{6.570000in}{0.900957in}}%
\pgfpathlineto{\pgfqpoint{6.205000in}{0.900957in}}%
\pgfpathlineto{\pgfqpoint{6.205000in}{0.889814in}}%
\pgfusepath{fill}%
\end{pgfscope}%
\begin{pgfscope}%
\pgfpathrectangle{\pgfqpoint{6.205000in}{0.611250in}}{\pgfqpoint{0.365000in}{2.852500in}} %
\pgfusepath{clip}%
\pgfsetbuttcap%
\pgfsetroundjoin%
\definecolor{currentfill}{rgb}{0.353369,0.472069,0.892570}%
\pgfsetfillcolor{currentfill}%
\pgfsetlinewidth{0.000000pt}%
\definecolor{currentstroke}{rgb}{0.000000,0.000000,0.000000}%
\pgfsetstrokecolor{currentstroke}%
\pgfsetdash{}{0pt}%
\pgfpathmoveto{\pgfqpoint{6.205000in}{0.900957in}}%
\pgfpathlineto{\pgfqpoint{6.570000in}{0.900957in}}%
\pgfpathlineto{\pgfqpoint{6.570000in}{0.912100in}}%
\pgfpathlineto{\pgfqpoint{6.205000in}{0.912100in}}%
\pgfpathlineto{\pgfqpoint{6.205000in}{0.900957in}}%
\pgfusepath{fill}%
\end{pgfscope}%
\begin{pgfscope}%
\pgfpathrectangle{\pgfqpoint{6.205000in}{0.611250in}}{\pgfqpoint{0.365000in}{2.852500in}} %
\pgfusepath{clip}%
\pgfsetbuttcap%
\pgfsetroundjoin%
\definecolor{currentfill}{rgb}{0.358415,0.478426,0.896795}%
\pgfsetfillcolor{currentfill}%
\pgfsetlinewidth{0.000000pt}%
\definecolor{currentstroke}{rgb}{0.000000,0.000000,0.000000}%
\pgfsetstrokecolor{currentstroke}%
\pgfsetdash{}{0pt}%
\pgfpathmoveto{\pgfqpoint{6.205000in}{0.912100in}}%
\pgfpathlineto{\pgfqpoint{6.570000in}{0.912100in}}%
\pgfpathlineto{\pgfqpoint{6.570000in}{0.923242in}}%
\pgfpathlineto{\pgfqpoint{6.205000in}{0.923242in}}%
\pgfpathlineto{\pgfqpoint{6.205000in}{0.912100in}}%
\pgfusepath{fill}%
\end{pgfscope}%
\begin{pgfscope}%
\pgfpathrectangle{\pgfqpoint{6.205000in}{0.611250in}}{\pgfqpoint{0.365000in}{2.852500in}} %
\pgfusepath{clip}%
\pgfsetbuttcap%
\pgfsetroundjoin%
\definecolor{currentfill}{rgb}{0.363461,0.484784,0.901019}%
\pgfsetfillcolor{currentfill}%
\pgfsetlinewidth{0.000000pt}%
\definecolor{currentstroke}{rgb}{0.000000,0.000000,0.000000}%
\pgfsetstrokecolor{currentstroke}%
\pgfsetdash{}{0pt}%
\pgfpathmoveto{\pgfqpoint{6.205000in}{0.923242in}}%
\pgfpathlineto{\pgfqpoint{6.570000in}{0.923242in}}%
\pgfpathlineto{\pgfqpoint{6.570000in}{0.934385in}}%
\pgfpathlineto{\pgfqpoint{6.205000in}{0.934385in}}%
\pgfpathlineto{\pgfqpoint{6.205000in}{0.923242in}}%
\pgfusepath{fill}%
\end{pgfscope}%
\begin{pgfscope}%
\pgfpathrectangle{\pgfqpoint{6.205000in}{0.611250in}}{\pgfqpoint{0.365000in}{2.852500in}} %
\pgfusepath{clip}%
\pgfsetbuttcap%
\pgfsetroundjoin%
\definecolor{currentfill}{rgb}{0.368507,0.491141,0.905243}%
\pgfsetfillcolor{currentfill}%
\pgfsetlinewidth{0.000000pt}%
\definecolor{currentstroke}{rgb}{0.000000,0.000000,0.000000}%
\pgfsetstrokecolor{currentstroke}%
\pgfsetdash{}{0pt}%
\pgfpathmoveto{\pgfqpoint{6.205000in}{0.934385in}}%
\pgfpathlineto{\pgfqpoint{6.570000in}{0.934385in}}%
\pgfpathlineto{\pgfqpoint{6.570000in}{0.945527in}}%
\pgfpathlineto{\pgfqpoint{6.205000in}{0.945527in}}%
\pgfpathlineto{\pgfqpoint{6.205000in}{0.934385in}}%
\pgfusepath{fill}%
\end{pgfscope}%
\begin{pgfscope}%
\pgfpathrectangle{\pgfqpoint{6.205000in}{0.611250in}}{\pgfqpoint{0.365000in}{2.852500in}} %
\pgfusepath{clip}%
\pgfsetbuttcap%
\pgfsetroundjoin%
\definecolor{currentfill}{rgb}{0.373552,0.497499,0.909467}%
\pgfsetfillcolor{currentfill}%
\pgfsetlinewidth{0.000000pt}%
\definecolor{currentstroke}{rgb}{0.000000,0.000000,0.000000}%
\pgfsetstrokecolor{currentstroke}%
\pgfsetdash{}{0pt}%
\pgfpathmoveto{\pgfqpoint{6.205000in}{0.945527in}}%
\pgfpathlineto{\pgfqpoint{6.570000in}{0.945527in}}%
\pgfpathlineto{\pgfqpoint{6.570000in}{0.956670in}}%
\pgfpathlineto{\pgfqpoint{6.205000in}{0.956670in}}%
\pgfpathlineto{\pgfqpoint{6.205000in}{0.945527in}}%
\pgfusepath{fill}%
\end{pgfscope}%
\begin{pgfscope}%
\pgfpathrectangle{\pgfqpoint{6.205000in}{0.611250in}}{\pgfqpoint{0.365000in}{2.852500in}} %
\pgfusepath{clip}%
\pgfsetbuttcap%
\pgfsetroundjoin%
\definecolor{currentfill}{rgb}{0.378598,0.503856,0.913692}%
\pgfsetfillcolor{currentfill}%
\pgfsetlinewidth{0.000000pt}%
\definecolor{currentstroke}{rgb}{0.000000,0.000000,0.000000}%
\pgfsetstrokecolor{currentstroke}%
\pgfsetdash{}{0pt}%
\pgfpathmoveto{\pgfqpoint{6.205000in}{0.956670in}}%
\pgfpathlineto{\pgfqpoint{6.570000in}{0.956670in}}%
\pgfpathlineto{\pgfqpoint{6.570000in}{0.967813in}}%
\pgfpathlineto{\pgfqpoint{6.205000in}{0.967813in}}%
\pgfpathlineto{\pgfqpoint{6.205000in}{0.956670in}}%
\pgfusepath{fill}%
\end{pgfscope}%
\begin{pgfscope}%
\pgfpathrectangle{\pgfqpoint{6.205000in}{0.611250in}}{\pgfqpoint{0.365000in}{2.852500in}} %
\pgfusepath{clip}%
\pgfsetbuttcap%
\pgfsetroundjoin%
\definecolor{currentfill}{rgb}{0.383662,0.510183,0.917831}%
\pgfsetfillcolor{currentfill}%
\pgfsetlinewidth{0.000000pt}%
\definecolor{currentstroke}{rgb}{0.000000,0.000000,0.000000}%
\pgfsetstrokecolor{currentstroke}%
\pgfsetdash{}{0pt}%
\pgfpathmoveto{\pgfqpoint{6.205000in}{0.967813in}}%
\pgfpathlineto{\pgfqpoint{6.570000in}{0.967813in}}%
\pgfpathlineto{\pgfqpoint{6.570000in}{0.978955in}}%
\pgfpathlineto{\pgfqpoint{6.205000in}{0.978955in}}%
\pgfpathlineto{\pgfqpoint{6.205000in}{0.967813in}}%
\pgfusepath{fill}%
\end{pgfscope}%
\begin{pgfscope}%
\pgfpathrectangle{\pgfqpoint{6.205000in}{0.611250in}}{\pgfqpoint{0.365000in}{2.852500in}} %
\pgfusepath{clip}%
\pgfsetbuttcap%
\pgfsetroundjoin%
\definecolor{currentfill}{rgb}{0.388852,0.516298,0.921373}%
\pgfsetfillcolor{currentfill}%
\pgfsetlinewidth{0.000000pt}%
\definecolor{currentstroke}{rgb}{0.000000,0.000000,0.000000}%
\pgfsetstrokecolor{currentstroke}%
\pgfsetdash{}{0pt}%
\pgfpathmoveto{\pgfqpoint{6.205000in}{0.978955in}}%
\pgfpathlineto{\pgfqpoint{6.570000in}{0.978955in}}%
\pgfpathlineto{\pgfqpoint{6.570000in}{0.990098in}}%
\pgfpathlineto{\pgfqpoint{6.205000in}{0.990098in}}%
\pgfpathlineto{\pgfqpoint{6.205000in}{0.978955in}}%
\pgfusepath{fill}%
\end{pgfscope}%
\begin{pgfscope}%
\pgfpathrectangle{\pgfqpoint{6.205000in}{0.611250in}}{\pgfqpoint{0.365000in}{2.852500in}} %
\pgfusepath{clip}%
\pgfsetbuttcap%
\pgfsetroundjoin%
\definecolor{currentfill}{rgb}{0.394042,0.522413,0.924916}%
\pgfsetfillcolor{currentfill}%
\pgfsetlinewidth{0.000000pt}%
\definecolor{currentstroke}{rgb}{0.000000,0.000000,0.000000}%
\pgfsetstrokecolor{currentstroke}%
\pgfsetdash{}{0pt}%
\pgfpathmoveto{\pgfqpoint{6.205000in}{0.990098in}}%
\pgfpathlineto{\pgfqpoint{6.570000in}{0.990098in}}%
\pgfpathlineto{\pgfqpoint{6.570000in}{1.001240in}}%
\pgfpathlineto{\pgfqpoint{6.205000in}{1.001240in}}%
\pgfpathlineto{\pgfqpoint{6.205000in}{0.990098in}}%
\pgfusepath{fill}%
\end{pgfscope}%
\begin{pgfscope}%
\pgfpathrectangle{\pgfqpoint{6.205000in}{0.611250in}}{\pgfqpoint{0.365000in}{2.852500in}} %
\pgfusepath{clip}%
\pgfsetbuttcap%
\pgfsetroundjoin%
\definecolor{currentfill}{rgb}{0.399231,0.528528,0.928459}%
\pgfsetfillcolor{currentfill}%
\pgfsetlinewidth{0.000000pt}%
\definecolor{currentstroke}{rgb}{0.000000,0.000000,0.000000}%
\pgfsetstrokecolor{currentstroke}%
\pgfsetdash{}{0pt}%
\pgfpathmoveto{\pgfqpoint{6.205000in}{1.001240in}}%
\pgfpathlineto{\pgfqpoint{6.570000in}{1.001240in}}%
\pgfpathlineto{\pgfqpoint{6.570000in}{1.012383in}}%
\pgfpathlineto{\pgfqpoint{6.205000in}{1.012383in}}%
\pgfpathlineto{\pgfqpoint{6.205000in}{1.001240in}}%
\pgfusepath{fill}%
\end{pgfscope}%
\begin{pgfscope}%
\pgfpathrectangle{\pgfqpoint{6.205000in}{0.611250in}}{\pgfqpoint{0.365000in}{2.852500in}} %
\pgfusepath{clip}%
\pgfsetbuttcap%
\pgfsetroundjoin%
\definecolor{currentfill}{rgb}{0.404421,0.534643,0.932002}%
\pgfsetfillcolor{currentfill}%
\pgfsetlinewidth{0.000000pt}%
\definecolor{currentstroke}{rgb}{0.000000,0.000000,0.000000}%
\pgfsetstrokecolor{currentstroke}%
\pgfsetdash{}{0pt}%
\pgfpathmoveto{\pgfqpoint{6.205000in}{1.012383in}}%
\pgfpathlineto{\pgfqpoint{6.570000in}{1.012383in}}%
\pgfpathlineto{\pgfqpoint{6.570000in}{1.023525in}}%
\pgfpathlineto{\pgfqpoint{6.205000in}{1.023525in}}%
\pgfpathlineto{\pgfqpoint{6.205000in}{1.012383in}}%
\pgfusepath{fill}%
\end{pgfscope}%
\begin{pgfscope}%
\pgfpathrectangle{\pgfqpoint{6.205000in}{0.611250in}}{\pgfqpoint{0.365000in}{2.852500in}} %
\pgfusepath{clip}%
\pgfsetbuttcap%
\pgfsetroundjoin%
\definecolor{currentfill}{rgb}{0.409611,0.540759,0.935545}%
\pgfsetfillcolor{currentfill}%
\pgfsetlinewidth{0.000000pt}%
\definecolor{currentstroke}{rgb}{0.000000,0.000000,0.000000}%
\pgfsetstrokecolor{currentstroke}%
\pgfsetdash{}{0pt}%
\pgfpathmoveto{\pgfqpoint{6.205000in}{1.023525in}}%
\pgfpathlineto{\pgfqpoint{6.570000in}{1.023525in}}%
\pgfpathlineto{\pgfqpoint{6.570000in}{1.034668in}}%
\pgfpathlineto{\pgfqpoint{6.205000in}{1.034668in}}%
\pgfpathlineto{\pgfqpoint{6.205000in}{1.023525in}}%
\pgfusepath{fill}%
\end{pgfscope}%
\begin{pgfscope}%
\pgfpathrectangle{\pgfqpoint{6.205000in}{0.611250in}}{\pgfqpoint{0.365000in}{2.852500in}} %
\pgfusepath{clip}%
\pgfsetbuttcap%
\pgfsetroundjoin%
\definecolor{currentfill}{rgb}{0.414801,0.546874,0.939088}%
\pgfsetfillcolor{currentfill}%
\pgfsetlinewidth{0.000000pt}%
\definecolor{currentstroke}{rgb}{0.000000,0.000000,0.000000}%
\pgfsetstrokecolor{currentstroke}%
\pgfsetdash{}{0pt}%
\pgfpathmoveto{\pgfqpoint{6.205000in}{1.034668in}}%
\pgfpathlineto{\pgfqpoint{6.570000in}{1.034668in}}%
\pgfpathlineto{\pgfqpoint{6.570000in}{1.045811in}}%
\pgfpathlineto{\pgfqpoint{6.205000in}{1.045811in}}%
\pgfpathlineto{\pgfqpoint{6.205000in}{1.034668in}}%
\pgfusepath{fill}%
\end{pgfscope}%
\begin{pgfscope}%
\pgfpathrectangle{\pgfqpoint{6.205000in}{0.611250in}}{\pgfqpoint{0.365000in}{2.852500in}} %
\pgfusepath{clip}%
\pgfsetbuttcap%
\pgfsetroundjoin%
\definecolor{currentfill}{rgb}{0.419991,0.552989,0.942630}%
\pgfsetfillcolor{currentfill}%
\pgfsetlinewidth{0.000000pt}%
\definecolor{currentstroke}{rgb}{0.000000,0.000000,0.000000}%
\pgfsetstrokecolor{currentstroke}%
\pgfsetdash{}{0pt}%
\pgfpathmoveto{\pgfqpoint{6.205000in}{1.045811in}}%
\pgfpathlineto{\pgfqpoint{6.570000in}{1.045811in}}%
\pgfpathlineto{\pgfqpoint{6.570000in}{1.056953in}}%
\pgfpathlineto{\pgfqpoint{6.205000in}{1.056953in}}%
\pgfpathlineto{\pgfqpoint{6.205000in}{1.045811in}}%
\pgfusepath{fill}%
\end{pgfscope}%
\begin{pgfscope}%
\pgfpathrectangle{\pgfqpoint{6.205000in}{0.611250in}}{\pgfqpoint{0.365000in}{2.852500in}} %
\pgfusepath{clip}%
\pgfsetbuttcap%
\pgfsetroundjoin%
\definecolor{currentfill}{rgb}{0.425199,0.559058,0.946061}%
\pgfsetfillcolor{currentfill}%
\pgfsetlinewidth{0.000000pt}%
\definecolor{currentstroke}{rgb}{0.000000,0.000000,0.000000}%
\pgfsetstrokecolor{currentstroke}%
\pgfsetdash{}{0pt}%
\pgfpathmoveto{\pgfqpoint{6.205000in}{1.056953in}}%
\pgfpathlineto{\pgfqpoint{6.570000in}{1.056953in}}%
\pgfpathlineto{\pgfqpoint{6.570000in}{1.068096in}}%
\pgfpathlineto{\pgfqpoint{6.205000in}{1.068096in}}%
\pgfpathlineto{\pgfqpoint{6.205000in}{1.056953in}}%
\pgfusepath{fill}%
\end{pgfscope}%
\begin{pgfscope}%
\pgfpathrectangle{\pgfqpoint{6.205000in}{0.611250in}}{\pgfqpoint{0.365000in}{2.852500in}} %
\pgfusepath{clip}%
\pgfsetbuttcap%
\pgfsetroundjoin%
\definecolor{currentfill}{rgb}{0.430507,0.564883,0.948889}%
\pgfsetfillcolor{currentfill}%
\pgfsetlinewidth{0.000000pt}%
\definecolor{currentstroke}{rgb}{0.000000,0.000000,0.000000}%
\pgfsetstrokecolor{currentstroke}%
\pgfsetdash{}{0pt}%
\pgfpathmoveto{\pgfqpoint{6.205000in}{1.068096in}}%
\pgfpathlineto{\pgfqpoint{6.570000in}{1.068096in}}%
\pgfpathlineto{\pgfqpoint{6.570000in}{1.079238in}}%
\pgfpathlineto{\pgfqpoint{6.205000in}{1.079238in}}%
\pgfpathlineto{\pgfqpoint{6.205000in}{1.068096in}}%
\pgfusepath{fill}%
\end{pgfscope}%
\begin{pgfscope}%
\pgfpathrectangle{\pgfqpoint{6.205000in}{0.611250in}}{\pgfqpoint{0.365000in}{2.852500in}} %
\pgfusepath{clip}%
\pgfsetbuttcap%
\pgfsetroundjoin%
\definecolor{currentfill}{rgb}{0.435815,0.570707,0.951717}%
\pgfsetfillcolor{currentfill}%
\pgfsetlinewidth{0.000000pt}%
\definecolor{currentstroke}{rgb}{0.000000,0.000000,0.000000}%
\pgfsetstrokecolor{currentstroke}%
\pgfsetdash{}{0pt}%
\pgfpathmoveto{\pgfqpoint{6.205000in}{1.079238in}}%
\pgfpathlineto{\pgfqpoint{6.570000in}{1.079238in}}%
\pgfpathlineto{\pgfqpoint{6.570000in}{1.090381in}}%
\pgfpathlineto{\pgfqpoint{6.205000in}{1.090381in}}%
\pgfpathlineto{\pgfqpoint{6.205000in}{1.079238in}}%
\pgfusepath{fill}%
\end{pgfscope}%
\begin{pgfscope}%
\pgfpathrectangle{\pgfqpoint{6.205000in}{0.611250in}}{\pgfqpoint{0.365000in}{2.852500in}} %
\pgfusepath{clip}%
\pgfsetbuttcap%
\pgfsetroundjoin%
\definecolor{currentfill}{rgb}{0.441123,0.576532,0.954545}%
\pgfsetfillcolor{currentfill}%
\pgfsetlinewidth{0.000000pt}%
\definecolor{currentstroke}{rgb}{0.000000,0.000000,0.000000}%
\pgfsetstrokecolor{currentstroke}%
\pgfsetdash{}{0pt}%
\pgfpathmoveto{\pgfqpoint{6.205000in}{1.090381in}}%
\pgfpathlineto{\pgfqpoint{6.570000in}{1.090381in}}%
\pgfpathlineto{\pgfqpoint{6.570000in}{1.101523in}}%
\pgfpathlineto{\pgfqpoint{6.205000in}{1.101523in}}%
\pgfpathlineto{\pgfqpoint{6.205000in}{1.090381in}}%
\pgfusepath{fill}%
\end{pgfscope}%
\begin{pgfscope}%
\pgfpathrectangle{\pgfqpoint{6.205000in}{0.611250in}}{\pgfqpoint{0.365000in}{2.852500in}} %
\pgfusepath{clip}%
\pgfsetbuttcap%
\pgfsetroundjoin%
\definecolor{currentfill}{rgb}{0.446431,0.582356,0.957373}%
\pgfsetfillcolor{currentfill}%
\pgfsetlinewidth{0.000000pt}%
\definecolor{currentstroke}{rgb}{0.000000,0.000000,0.000000}%
\pgfsetstrokecolor{currentstroke}%
\pgfsetdash{}{0pt}%
\pgfpathmoveto{\pgfqpoint{6.205000in}{1.101523in}}%
\pgfpathlineto{\pgfqpoint{6.570000in}{1.101523in}}%
\pgfpathlineto{\pgfqpoint{6.570000in}{1.112666in}}%
\pgfpathlineto{\pgfqpoint{6.205000in}{1.112666in}}%
\pgfpathlineto{\pgfqpoint{6.205000in}{1.101523in}}%
\pgfusepath{fill}%
\end{pgfscope}%
\begin{pgfscope}%
\pgfpathrectangle{\pgfqpoint{6.205000in}{0.611250in}}{\pgfqpoint{0.365000in}{2.852500in}} %
\pgfusepath{clip}%
\pgfsetbuttcap%
\pgfsetroundjoin%
\definecolor{currentfill}{rgb}{0.451739,0.588181,0.960201}%
\pgfsetfillcolor{currentfill}%
\pgfsetlinewidth{0.000000pt}%
\definecolor{currentstroke}{rgb}{0.000000,0.000000,0.000000}%
\pgfsetstrokecolor{currentstroke}%
\pgfsetdash{}{0pt}%
\pgfpathmoveto{\pgfqpoint{6.205000in}{1.112666in}}%
\pgfpathlineto{\pgfqpoint{6.570000in}{1.112666in}}%
\pgfpathlineto{\pgfqpoint{6.570000in}{1.123809in}}%
\pgfpathlineto{\pgfqpoint{6.205000in}{1.123809in}}%
\pgfpathlineto{\pgfqpoint{6.205000in}{1.112666in}}%
\pgfusepath{fill}%
\end{pgfscope}%
\begin{pgfscope}%
\pgfpathrectangle{\pgfqpoint{6.205000in}{0.611250in}}{\pgfqpoint{0.365000in}{2.852500in}} %
\pgfusepath{clip}%
\pgfsetbuttcap%
\pgfsetroundjoin%
\definecolor{currentfill}{rgb}{0.457046,0.594006,0.963029}%
\pgfsetfillcolor{currentfill}%
\pgfsetlinewidth{0.000000pt}%
\definecolor{currentstroke}{rgb}{0.000000,0.000000,0.000000}%
\pgfsetstrokecolor{currentstroke}%
\pgfsetdash{}{0pt}%
\pgfpathmoveto{\pgfqpoint{6.205000in}{1.123809in}}%
\pgfpathlineto{\pgfqpoint{6.570000in}{1.123809in}}%
\pgfpathlineto{\pgfqpoint{6.570000in}{1.134951in}}%
\pgfpathlineto{\pgfqpoint{6.205000in}{1.134951in}}%
\pgfpathlineto{\pgfqpoint{6.205000in}{1.123809in}}%
\pgfusepath{fill}%
\end{pgfscope}%
\begin{pgfscope}%
\pgfpathrectangle{\pgfqpoint{6.205000in}{0.611250in}}{\pgfqpoint{0.365000in}{2.852500in}} %
\pgfusepath{clip}%
\pgfsetbuttcap%
\pgfsetroundjoin%
\definecolor{currentfill}{rgb}{0.462354,0.599830,0.965857}%
\pgfsetfillcolor{currentfill}%
\pgfsetlinewidth{0.000000pt}%
\definecolor{currentstroke}{rgb}{0.000000,0.000000,0.000000}%
\pgfsetstrokecolor{currentstroke}%
\pgfsetdash{}{0pt}%
\pgfpathmoveto{\pgfqpoint{6.205000in}{1.134951in}}%
\pgfpathlineto{\pgfqpoint{6.570000in}{1.134951in}}%
\pgfpathlineto{\pgfqpoint{6.570000in}{1.146094in}}%
\pgfpathlineto{\pgfqpoint{6.205000in}{1.146094in}}%
\pgfpathlineto{\pgfqpoint{6.205000in}{1.134951in}}%
\pgfusepath{fill}%
\end{pgfscope}%
\begin{pgfscope}%
\pgfpathrectangle{\pgfqpoint{6.205000in}{0.611250in}}{\pgfqpoint{0.365000in}{2.852500in}} %
\pgfusepath{clip}%
\pgfsetbuttcap%
\pgfsetroundjoin%
\definecolor{currentfill}{rgb}{0.467678,0.605591,0.968546}%
\pgfsetfillcolor{currentfill}%
\pgfsetlinewidth{0.000000pt}%
\definecolor{currentstroke}{rgb}{0.000000,0.000000,0.000000}%
\pgfsetstrokecolor{currentstroke}%
\pgfsetdash{}{0pt}%
\pgfpathmoveto{\pgfqpoint{6.205000in}{1.146094in}}%
\pgfpathlineto{\pgfqpoint{6.570000in}{1.146094in}}%
\pgfpathlineto{\pgfqpoint{6.570000in}{1.157236in}}%
\pgfpathlineto{\pgfqpoint{6.205000in}{1.157236in}}%
\pgfpathlineto{\pgfqpoint{6.205000in}{1.146094in}}%
\pgfusepath{fill}%
\end{pgfscope}%
\begin{pgfscope}%
\pgfpathrectangle{\pgfqpoint{6.205000in}{0.611250in}}{\pgfqpoint{0.365000in}{2.852500in}} %
\pgfusepath{clip}%
\pgfsetbuttcap%
\pgfsetroundjoin%
\definecolor{currentfill}{rgb}{0.473070,0.611077,0.970634}%
\pgfsetfillcolor{currentfill}%
\pgfsetlinewidth{0.000000pt}%
\definecolor{currentstroke}{rgb}{0.000000,0.000000,0.000000}%
\pgfsetstrokecolor{currentstroke}%
\pgfsetdash{}{0pt}%
\pgfpathmoveto{\pgfqpoint{6.205000in}{1.157236in}}%
\pgfpathlineto{\pgfqpoint{6.570000in}{1.157236in}}%
\pgfpathlineto{\pgfqpoint{6.570000in}{1.168379in}}%
\pgfpathlineto{\pgfqpoint{6.205000in}{1.168379in}}%
\pgfpathlineto{\pgfqpoint{6.205000in}{1.157236in}}%
\pgfusepath{fill}%
\end{pgfscope}%
\begin{pgfscope}%
\pgfpathrectangle{\pgfqpoint{6.205000in}{0.611250in}}{\pgfqpoint{0.365000in}{2.852500in}} %
\pgfusepath{clip}%
\pgfsetbuttcap%
\pgfsetroundjoin%
\definecolor{currentfill}{rgb}{0.478462,0.616564,0.972721}%
\pgfsetfillcolor{currentfill}%
\pgfsetlinewidth{0.000000pt}%
\definecolor{currentstroke}{rgb}{0.000000,0.000000,0.000000}%
\pgfsetstrokecolor{currentstroke}%
\pgfsetdash{}{0pt}%
\pgfpathmoveto{\pgfqpoint{6.205000in}{1.168379in}}%
\pgfpathlineto{\pgfqpoint{6.570000in}{1.168379in}}%
\pgfpathlineto{\pgfqpoint{6.570000in}{1.179521in}}%
\pgfpathlineto{\pgfqpoint{6.205000in}{1.179521in}}%
\pgfpathlineto{\pgfqpoint{6.205000in}{1.168379in}}%
\pgfusepath{fill}%
\end{pgfscope}%
\begin{pgfscope}%
\pgfpathrectangle{\pgfqpoint{6.205000in}{0.611250in}}{\pgfqpoint{0.365000in}{2.852500in}} %
\pgfusepath{clip}%
\pgfsetbuttcap%
\pgfsetroundjoin%
\definecolor{currentfill}{rgb}{0.483854,0.622050,0.974808}%
\pgfsetfillcolor{currentfill}%
\pgfsetlinewidth{0.000000pt}%
\definecolor{currentstroke}{rgb}{0.000000,0.000000,0.000000}%
\pgfsetstrokecolor{currentstroke}%
\pgfsetdash{}{0pt}%
\pgfpathmoveto{\pgfqpoint{6.205000in}{1.179521in}}%
\pgfpathlineto{\pgfqpoint{6.570000in}{1.179521in}}%
\pgfpathlineto{\pgfqpoint{6.570000in}{1.190664in}}%
\pgfpathlineto{\pgfqpoint{6.205000in}{1.190664in}}%
\pgfpathlineto{\pgfqpoint{6.205000in}{1.179521in}}%
\pgfusepath{fill}%
\end{pgfscope}%
\begin{pgfscope}%
\pgfpathrectangle{\pgfqpoint{6.205000in}{0.611250in}}{\pgfqpoint{0.365000in}{2.852500in}} %
\pgfusepath{clip}%
\pgfsetbuttcap%
\pgfsetroundjoin%
\definecolor{currentfill}{rgb}{0.489246,0.627536,0.976896}%
\pgfsetfillcolor{currentfill}%
\pgfsetlinewidth{0.000000pt}%
\definecolor{currentstroke}{rgb}{0.000000,0.000000,0.000000}%
\pgfsetstrokecolor{currentstroke}%
\pgfsetdash{}{0pt}%
\pgfpathmoveto{\pgfqpoint{6.205000in}{1.190664in}}%
\pgfpathlineto{\pgfqpoint{6.570000in}{1.190664in}}%
\pgfpathlineto{\pgfqpoint{6.570000in}{1.201807in}}%
\pgfpathlineto{\pgfqpoint{6.205000in}{1.201807in}}%
\pgfpathlineto{\pgfqpoint{6.205000in}{1.190664in}}%
\pgfusepath{fill}%
\end{pgfscope}%
\begin{pgfscope}%
\pgfpathrectangle{\pgfqpoint{6.205000in}{0.611250in}}{\pgfqpoint{0.365000in}{2.852500in}} %
\pgfusepath{clip}%
\pgfsetbuttcap%
\pgfsetroundjoin%
\definecolor{currentfill}{rgb}{0.494638,0.633022,0.978983}%
\pgfsetfillcolor{currentfill}%
\pgfsetlinewidth{0.000000pt}%
\definecolor{currentstroke}{rgb}{0.000000,0.000000,0.000000}%
\pgfsetstrokecolor{currentstroke}%
\pgfsetdash{}{0pt}%
\pgfpathmoveto{\pgfqpoint{6.205000in}{1.201807in}}%
\pgfpathlineto{\pgfqpoint{6.570000in}{1.201807in}}%
\pgfpathlineto{\pgfqpoint{6.570000in}{1.212949in}}%
\pgfpathlineto{\pgfqpoint{6.205000in}{1.212949in}}%
\pgfpathlineto{\pgfqpoint{6.205000in}{1.201807in}}%
\pgfusepath{fill}%
\end{pgfscope}%
\begin{pgfscope}%
\pgfpathrectangle{\pgfqpoint{6.205000in}{0.611250in}}{\pgfqpoint{0.365000in}{2.852500in}} %
\pgfusepath{clip}%
\pgfsetbuttcap%
\pgfsetroundjoin%
\definecolor{currentfill}{rgb}{0.500031,0.638508,0.981070}%
\pgfsetfillcolor{currentfill}%
\pgfsetlinewidth{0.000000pt}%
\definecolor{currentstroke}{rgb}{0.000000,0.000000,0.000000}%
\pgfsetstrokecolor{currentstroke}%
\pgfsetdash{}{0pt}%
\pgfpathmoveto{\pgfqpoint{6.205000in}{1.212949in}}%
\pgfpathlineto{\pgfqpoint{6.570000in}{1.212949in}}%
\pgfpathlineto{\pgfqpoint{6.570000in}{1.224092in}}%
\pgfpathlineto{\pgfqpoint{6.205000in}{1.224092in}}%
\pgfpathlineto{\pgfqpoint{6.205000in}{1.212949in}}%
\pgfusepath{fill}%
\end{pgfscope}%
\begin{pgfscope}%
\pgfpathrectangle{\pgfqpoint{6.205000in}{0.611250in}}{\pgfqpoint{0.365000in}{2.852500in}} %
\pgfusepath{clip}%
\pgfsetbuttcap%
\pgfsetroundjoin%
\definecolor{currentfill}{rgb}{0.505423,0.643995,0.983157}%
\pgfsetfillcolor{currentfill}%
\pgfsetlinewidth{0.000000pt}%
\definecolor{currentstroke}{rgb}{0.000000,0.000000,0.000000}%
\pgfsetstrokecolor{currentstroke}%
\pgfsetdash{}{0pt}%
\pgfpathmoveto{\pgfqpoint{6.205000in}{1.224092in}}%
\pgfpathlineto{\pgfqpoint{6.570000in}{1.224092in}}%
\pgfpathlineto{\pgfqpoint{6.570000in}{1.235234in}}%
\pgfpathlineto{\pgfqpoint{6.205000in}{1.235234in}}%
\pgfpathlineto{\pgfqpoint{6.205000in}{1.224092in}}%
\pgfusepath{fill}%
\end{pgfscope}%
\begin{pgfscope}%
\pgfpathrectangle{\pgfqpoint{6.205000in}{0.611250in}}{\pgfqpoint{0.365000in}{2.852500in}} %
\pgfusepath{clip}%
\pgfsetbuttcap%
\pgfsetroundjoin%
\definecolor{currentfill}{rgb}{0.510824,0.649397,0.985079}%
\pgfsetfillcolor{currentfill}%
\pgfsetlinewidth{0.000000pt}%
\definecolor{currentstroke}{rgb}{0.000000,0.000000,0.000000}%
\pgfsetstrokecolor{currentstroke}%
\pgfsetdash{}{0pt}%
\pgfpathmoveto{\pgfqpoint{6.205000in}{1.235234in}}%
\pgfpathlineto{\pgfqpoint{6.570000in}{1.235234in}}%
\pgfpathlineto{\pgfqpoint{6.570000in}{1.246377in}}%
\pgfpathlineto{\pgfqpoint{6.205000in}{1.246377in}}%
\pgfpathlineto{\pgfqpoint{6.205000in}{1.235234in}}%
\pgfusepath{fill}%
\end{pgfscope}%
\begin{pgfscope}%
\pgfpathrectangle{\pgfqpoint{6.205000in}{0.611250in}}{\pgfqpoint{0.365000in}{2.852500in}} %
\pgfusepath{clip}%
\pgfsetbuttcap%
\pgfsetroundjoin%
\definecolor{currentfill}{rgb}{0.516260,0.654498,0.986407}%
\pgfsetfillcolor{currentfill}%
\pgfsetlinewidth{0.000000pt}%
\definecolor{currentstroke}{rgb}{0.000000,0.000000,0.000000}%
\pgfsetstrokecolor{currentstroke}%
\pgfsetdash{}{0pt}%
\pgfpathmoveto{\pgfqpoint{6.205000in}{1.246377in}}%
\pgfpathlineto{\pgfqpoint{6.570000in}{1.246377in}}%
\pgfpathlineto{\pgfqpoint{6.570000in}{1.257520in}}%
\pgfpathlineto{\pgfqpoint{6.205000in}{1.257520in}}%
\pgfpathlineto{\pgfqpoint{6.205000in}{1.246377in}}%
\pgfusepath{fill}%
\end{pgfscope}%
\begin{pgfscope}%
\pgfpathrectangle{\pgfqpoint{6.205000in}{0.611250in}}{\pgfqpoint{0.365000in}{2.852500in}} %
\pgfusepath{clip}%
\pgfsetbuttcap%
\pgfsetroundjoin%
\definecolor{currentfill}{rgb}{0.521696,0.659599,0.987736}%
\pgfsetfillcolor{currentfill}%
\pgfsetlinewidth{0.000000pt}%
\definecolor{currentstroke}{rgb}{0.000000,0.000000,0.000000}%
\pgfsetstrokecolor{currentstroke}%
\pgfsetdash{}{0pt}%
\pgfpathmoveto{\pgfqpoint{6.205000in}{1.257520in}}%
\pgfpathlineto{\pgfqpoint{6.570000in}{1.257520in}}%
\pgfpathlineto{\pgfqpoint{6.570000in}{1.268662in}}%
\pgfpathlineto{\pgfqpoint{6.205000in}{1.268662in}}%
\pgfpathlineto{\pgfqpoint{6.205000in}{1.257520in}}%
\pgfusepath{fill}%
\end{pgfscope}%
\begin{pgfscope}%
\pgfpathrectangle{\pgfqpoint{6.205000in}{0.611250in}}{\pgfqpoint{0.365000in}{2.852500in}} %
\pgfusepath{clip}%
\pgfsetbuttcap%
\pgfsetroundjoin%
\definecolor{currentfill}{rgb}{0.527132,0.664700,0.989065}%
\pgfsetfillcolor{currentfill}%
\pgfsetlinewidth{0.000000pt}%
\definecolor{currentstroke}{rgb}{0.000000,0.000000,0.000000}%
\pgfsetstrokecolor{currentstroke}%
\pgfsetdash{}{0pt}%
\pgfpathmoveto{\pgfqpoint{6.205000in}{1.268662in}}%
\pgfpathlineto{\pgfqpoint{6.570000in}{1.268662in}}%
\pgfpathlineto{\pgfqpoint{6.570000in}{1.279805in}}%
\pgfpathlineto{\pgfqpoint{6.205000in}{1.279805in}}%
\pgfpathlineto{\pgfqpoint{6.205000in}{1.268662in}}%
\pgfusepath{fill}%
\end{pgfscope}%
\begin{pgfscope}%
\pgfpathrectangle{\pgfqpoint{6.205000in}{0.611250in}}{\pgfqpoint{0.365000in}{2.852500in}} %
\pgfusepath{clip}%
\pgfsetbuttcap%
\pgfsetroundjoin%
\definecolor{currentfill}{rgb}{0.532568,0.669801,0.990393}%
\pgfsetfillcolor{currentfill}%
\pgfsetlinewidth{0.000000pt}%
\definecolor{currentstroke}{rgb}{0.000000,0.000000,0.000000}%
\pgfsetstrokecolor{currentstroke}%
\pgfsetdash{}{0pt}%
\pgfpathmoveto{\pgfqpoint{6.205000in}{1.279805in}}%
\pgfpathlineto{\pgfqpoint{6.570000in}{1.279805in}}%
\pgfpathlineto{\pgfqpoint{6.570000in}{1.290947in}}%
\pgfpathlineto{\pgfqpoint{6.205000in}{1.290947in}}%
\pgfpathlineto{\pgfqpoint{6.205000in}{1.279805in}}%
\pgfusepath{fill}%
\end{pgfscope}%
\begin{pgfscope}%
\pgfpathrectangle{\pgfqpoint{6.205000in}{0.611250in}}{\pgfqpoint{0.365000in}{2.852500in}} %
\pgfusepath{clip}%
\pgfsetbuttcap%
\pgfsetroundjoin%
\definecolor{currentfill}{rgb}{0.538004,0.674902,0.991722}%
\pgfsetfillcolor{currentfill}%
\pgfsetlinewidth{0.000000pt}%
\definecolor{currentstroke}{rgb}{0.000000,0.000000,0.000000}%
\pgfsetstrokecolor{currentstroke}%
\pgfsetdash{}{0pt}%
\pgfpathmoveto{\pgfqpoint{6.205000in}{1.290947in}}%
\pgfpathlineto{\pgfqpoint{6.570000in}{1.290947in}}%
\pgfpathlineto{\pgfqpoint{6.570000in}{1.302090in}}%
\pgfpathlineto{\pgfqpoint{6.205000in}{1.302090in}}%
\pgfpathlineto{\pgfqpoint{6.205000in}{1.290947in}}%
\pgfusepath{fill}%
\end{pgfscope}%
\begin{pgfscope}%
\pgfpathrectangle{\pgfqpoint{6.205000in}{0.611250in}}{\pgfqpoint{0.365000in}{2.852500in}} %
\pgfusepath{clip}%
\pgfsetbuttcap%
\pgfsetroundjoin%
\definecolor{currentfill}{rgb}{0.543440,0.680003,0.993051}%
\pgfsetfillcolor{currentfill}%
\pgfsetlinewidth{0.000000pt}%
\definecolor{currentstroke}{rgb}{0.000000,0.000000,0.000000}%
\pgfsetstrokecolor{currentstroke}%
\pgfsetdash{}{0pt}%
\pgfpathmoveto{\pgfqpoint{6.205000in}{1.302090in}}%
\pgfpathlineto{\pgfqpoint{6.570000in}{1.302090in}}%
\pgfpathlineto{\pgfqpoint{6.570000in}{1.313232in}}%
\pgfpathlineto{\pgfqpoint{6.205000in}{1.313232in}}%
\pgfpathlineto{\pgfqpoint{6.205000in}{1.302090in}}%
\pgfusepath{fill}%
\end{pgfscope}%
\begin{pgfscope}%
\pgfpathrectangle{\pgfqpoint{6.205000in}{0.611250in}}{\pgfqpoint{0.365000in}{2.852500in}} %
\pgfusepath{clip}%
\pgfsetbuttcap%
\pgfsetroundjoin%
\definecolor{currentfill}{rgb}{0.548876,0.685104,0.994379}%
\pgfsetfillcolor{currentfill}%
\pgfsetlinewidth{0.000000pt}%
\definecolor{currentstroke}{rgb}{0.000000,0.000000,0.000000}%
\pgfsetstrokecolor{currentstroke}%
\pgfsetdash{}{0pt}%
\pgfpathmoveto{\pgfqpoint{6.205000in}{1.313232in}}%
\pgfpathlineto{\pgfqpoint{6.570000in}{1.313232in}}%
\pgfpathlineto{\pgfqpoint{6.570000in}{1.324375in}}%
\pgfpathlineto{\pgfqpoint{6.205000in}{1.324375in}}%
\pgfpathlineto{\pgfqpoint{6.205000in}{1.313232in}}%
\pgfusepath{fill}%
\end{pgfscope}%
\begin{pgfscope}%
\pgfpathrectangle{\pgfqpoint{6.205000in}{0.611250in}}{\pgfqpoint{0.365000in}{2.852500in}} %
\pgfusepath{clip}%
\pgfsetbuttcap%
\pgfsetroundjoin%
\definecolor{currentfill}{rgb}{0.554312,0.690097,0.995516}%
\pgfsetfillcolor{currentfill}%
\pgfsetlinewidth{0.000000pt}%
\definecolor{currentstroke}{rgb}{0.000000,0.000000,0.000000}%
\pgfsetstrokecolor{currentstroke}%
\pgfsetdash{}{0pt}%
\pgfpathmoveto{\pgfqpoint{6.205000in}{1.324375in}}%
\pgfpathlineto{\pgfqpoint{6.570000in}{1.324375in}}%
\pgfpathlineto{\pgfqpoint{6.570000in}{1.335518in}}%
\pgfpathlineto{\pgfqpoint{6.205000in}{1.335518in}}%
\pgfpathlineto{\pgfqpoint{6.205000in}{1.324375in}}%
\pgfusepath{fill}%
\end{pgfscope}%
\begin{pgfscope}%
\pgfpathrectangle{\pgfqpoint{6.205000in}{0.611250in}}{\pgfqpoint{0.365000in}{2.852500in}} %
\pgfusepath{clip}%
\pgfsetbuttcap%
\pgfsetroundjoin%
\definecolor{currentfill}{rgb}{0.559747,0.694768,0.996075}%
\pgfsetfillcolor{currentfill}%
\pgfsetlinewidth{0.000000pt}%
\definecolor{currentstroke}{rgb}{0.000000,0.000000,0.000000}%
\pgfsetstrokecolor{currentstroke}%
\pgfsetdash{}{0pt}%
\pgfpathmoveto{\pgfqpoint{6.205000in}{1.335518in}}%
\pgfpathlineto{\pgfqpoint{6.570000in}{1.335518in}}%
\pgfpathlineto{\pgfqpoint{6.570000in}{1.346660in}}%
\pgfpathlineto{\pgfqpoint{6.205000in}{1.346660in}}%
\pgfpathlineto{\pgfqpoint{6.205000in}{1.335518in}}%
\pgfusepath{fill}%
\end{pgfscope}%
\begin{pgfscope}%
\pgfpathrectangle{\pgfqpoint{6.205000in}{0.611250in}}{\pgfqpoint{0.365000in}{2.852500in}} %
\pgfusepath{clip}%
\pgfsetbuttcap%
\pgfsetroundjoin%
\definecolor{currentfill}{rgb}{0.565182,0.699438,0.996635}%
\pgfsetfillcolor{currentfill}%
\pgfsetlinewidth{0.000000pt}%
\definecolor{currentstroke}{rgb}{0.000000,0.000000,0.000000}%
\pgfsetstrokecolor{currentstroke}%
\pgfsetdash{}{0pt}%
\pgfpathmoveto{\pgfqpoint{6.205000in}{1.346660in}}%
\pgfpathlineto{\pgfqpoint{6.570000in}{1.346660in}}%
\pgfpathlineto{\pgfqpoint{6.570000in}{1.357803in}}%
\pgfpathlineto{\pgfqpoint{6.205000in}{1.357803in}}%
\pgfpathlineto{\pgfqpoint{6.205000in}{1.346660in}}%
\pgfusepath{fill}%
\end{pgfscope}%
\begin{pgfscope}%
\pgfpathrectangle{\pgfqpoint{6.205000in}{0.611250in}}{\pgfqpoint{0.365000in}{2.852500in}} %
\pgfusepath{clip}%
\pgfsetbuttcap%
\pgfsetroundjoin%
\definecolor{currentfill}{rgb}{0.570616,0.704109,0.997195}%
\pgfsetfillcolor{currentfill}%
\pgfsetlinewidth{0.000000pt}%
\definecolor{currentstroke}{rgb}{0.000000,0.000000,0.000000}%
\pgfsetstrokecolor{currentstroke}%
\pgfsetdash{}{0pt}%
\pgfpathmoveto{\pgfqpoint{6.205000in}{1.357803in}}%
\pgfpathlineto{\pgfqpoint{6.570000in}{1.357803in}}%
\pgfpathlineto{\pgfqpoint{6.570000in}{1.368945in}}%
\pgfpathlineto{\pgfqpoint{6.205000in}{1.368945in}}%
\pgfpathlineto{\pgfqpoint{6.205000in}{1.357803in}}%
\pgfusepath{fill}%
\end{pgfscope}%
\begin{pgfscope}%
\pgfpathrectangle{\pgfqpoint{6.205000in}{0.611250in}}{\pgfqpoint{0.365000in}{2.852500in}} %
\pgfusepath{clip}%
\pgfsetbuttcap%
\pgfsetroundjoin%
\definecolor{currentfill}{rgb}{0.576051,0.708780,0.997755}%
\pgfsetfillcolor{currentfill}%
\pgfsetlinewidth{0.000000pt}%
\definecolor{currentstroke}{rgb}{0.000000,0.000000,0.000000}%
\pgfsetstrokecolor{currentstroke}%
\pgfsetdash{}{0pt}%
\pgfpathmoveto{\pgfqpoint{6.205000in}{1.368945in}}%
\pgfpathlineto{\pgfqpoint{6.570000in}{1.368945in}}%
\pgfpathlineto{\pgfqpoint{6.570000in}{1.380088in}}%
\pgfpathlineto{\pgfqpoint{6.205000in}{1.380088in}}%
\pgfpathlineto{\pgfqpoint{6.205000in}{1.368945in}}%
\pgfusepath{fill}%
\end{pgfscope}%
\begin{pgfscope}%
\pgfpathrectangle{\pgfqpoint{6.205000in}{0.611250in}}{\pgfqpoint{0.365000in}{2.852500in}} %
\pgfusepath{clip}%
\pgfsetbuttcap%
\pgfsetroundjoin%
\definecolor{currentfill}{rgb}{0.581486,0.713451,0.998314}%
\pgfsetfillcolor{currentfill}%
\pgfsetlinewidth{0.000000pt}%
\definecolor{currentstroke}{rgb}{0.000000,0.000000,0.000000}%
\pgfsetstrokecolor{currentstroke}%
\pgfsetdash{}{0pt}%
\pgfpathmoveto{\pgfqpoint{6.205000in}{1.380088in}}%
\pgfpathlineto{\pgfqpoint{6.570000in}{1.380088in}}%
\pgfpathlineto{\pgfqpoint{6.570000in}{1.391230in}}%
\pgfpathlineto{\pgfqpoint{6.205000in}{1.391230in}}%
\pgfpathlineto{\pgfqpoint{6.205000in}{1.380088in}}%
\pgfusepath{fill}%
\end{pgfscope}%
\begin{pgfscope}%
\pgfpathrectangle{\pgfqpoint{6.205000in}{0.611250in}}{\pgfqpoint{0.365000in}{2.852500in}} %
\pgfusepath{clip}%
\pgfsetbuttcap%
\pgfsetroundjoin%
\definecolor{currentfill}{rgb}{0.586921,0.718121,0.998874}%
\pgfsetfillcolor{currentfill}%
\pgfsetlinewidth{0.000000pt}%
\definecolor{currentstroke}{rgb}{0.000000,0.000000,0.000000}%
\pgfsetstrokecolor{currentstroke}%
\pgfsetdash{}{0pt}%
\pgfpathmoveto{\pgfqpoint{6.205000in}{1.391230in}}%
\pgfpathlineto{\pgfqpoint{6.570000in}{1.391230in}}%
\pgfpathlineto{\pgfqpoint{6.570000in}{1.402373in}}%
\pgfpathlineto{\pgfqpoint{6.205000in}{1.402373in}}%
\pgfpathlineto{\pgfqpoint{6.205000in}{1.391230in}}%
\pgfusepath{fill}%
\end{pgfscope}%
\begin{pgfscope}%
\pgfpathrectangle{\pgfqpoint{6.205000in}{0.611250in}}{\pgfqpoint{0.365000in}{2.852500in}} %
\pgfusepath{clip}%
\pgfsetbuttcap%
\pgfsetroundjoin%
\definecolor{currentfill}{rgb}{0.592356,0.722792,0.999434}%
\pgfsetfillcolor{currentfill}%
\pgfsetlinewidth{0.000000pt}%
\definecolor{currentstroke}{rgb}{0.000000,0.000000,0.000000}%
\pgfsetstrokecolor{currentstroke}%
\pgfsetdash{}{0pt}%
\pgfpathmoveto{\pgfqpoint{6.205000in}{1.402373in}}%
\pgfpathlineto{\pgfqpoint{6.570000in}{1.402373in}}%
\pgfpathlineto{\pgfqpoint{6.570000in}{1.413516in}}%
\pgfpathlineto{\pgfqpoint{6.205000in}{1.413516in}}%
\pgfpathlineto{\pgfqpoint{6.205000in}{1.402373in}}%
\pgfusepath{fill}%
\end{pgfscope}%
\begin{pgfscope}%
\pgfpathrectangle{\pgfqpoint{6.205000in}{0.611250in}}{\pgfqpoint{0.365000in}{2.852500in}} %
\pgfusepath{clip}%
\pgfsetbuttcap%
\pgfsetroundjoin%
\definecolor{currentfill}{rgb}{0.597777,0.727330,0.999777}%
\pgfsetfillcolor{currentfill}%
\pgfsetlinewidth{0.000000pt}%
\definecolor{currentstroke}{rgb}{0.000000,0.000000,0.000000}%
\pgfsetstrokecolor{currentstroke}%
\pgfsetdash{}{0pt}%
\pgfpathmoveto{\pgfqpoint{6.205000in}{1.413516in}}%
\pgfpathlineto{\pgfqpoint{6.570000in}{1.413516in}}%
\pgfpathlineto{\pgfqpoint{6.570000in}{1.424658in}}%
\pgfpathlineto{\pgfqpoint{6.205000in}{1.424658in}}%
\pgfpathlineto{\pgfqpoint{6.205000in}{1.413516in}}%
\pgfusepath{fill}%
\end{pgfscope}%
\begin{pgfscope}%
\pgfpathrectangle{\pgfqpoint{6.205000in}{0.611250in}}{\pgfqpoint{0.365000in}{2.852500in}} %
\pgfusepath{clip}%
\pgfsetbuttcap%
\pgfsetroundjoin%
\definecolor{currentfill}{rgb}{0.603162,0.731527,0.999565}%
\pgfsetfillcolor{currentfill}%
\pgfsetlinewidth{0.000000pt}%
\definecolor{currentstroke}{rgb}{0.000000,0.000000,0.000000}%
\pgfsetstrokecolor{currentstroke}%
\pgfsetdash{}{0pt}%
\pgfpathmoveto{\pgfqpoint{6.205000in}{1.424658in}}%
\pgfpathlineto{\pgfqpoint{6.570000in}{1.424658in}}%
\pgfpathlineto{\pgfqpoint{6.570000in}{1.435801in}}%
\pgfpathlineto{\pgfqpoint{6.205000in}{1.435801in}}%
\pgfpathlineto{\pgfqpoint{6.205000in}{1.424658in}}%
\pgfusepath{fill}%
\end{pgfscope}%
\begin{pgfscope}%
\pgfpathrectangle{\pgfqpoint{6.205000in}{0.611250in}}{\pgfqpoint{0.365000in}{2.852500in}} %
\pgfusepath{clip}%
\pgfsetbuttcap%
\pgfsetroundjoin%
\definecolor{currentfill}{rgb}{0.608547,0.735725,0.999354}%
\pgfsetfillcolor{currentfill}%
\pgfsetlinewidth{0.000000pt}%
\definecolor{currentstroke}{rgb}{0.000000,0.000000,0.000000}%
\pgfsetstrokecolor{currentstroke}%
\pgfsetdash{}{0pt}%
\pgfpathmoveto{\pgfqpoint{6.205000in}{1.435801in}}%
\pgfpathlineto{\pgfqpoint{6.570000in}{1.435801in}}%
\pgfpathlineto{\pgfqpoint{6.570000in}{1.446943in}}%
\pgfpathlineto{\pgfqpoint{6.205000in}{1.446943in}}%
\pgfpathlineto{\pgfqpoint{6.205000in}{1.435801in}}%
\pgfusepath{fill}%
\end{pgfscope}%
\begin{pgfscope}%
\pgfpathrectangle{\pgfqpoint{6.205000in}{0.611250in}}{\pgfqpoint{0.365000in}{2.852500in}} %
\pgfusepath{clip}%
\pgfsetbuttcap%
\pgfsetroundjoin%
\definecolor{currentfill}{rgb}{0.613933,0.739923,0.999142}%
\pgfsetfillcolor{currentfill}%
\pgfsetlinewidth{0.000000pt}%
\definecolor{currentstroke}{rgb}{0.000000,0.000000,0.000000}%
\pgfsetstrokecolor{currentstroke}%
\pgfsetdash{}{0pt}%
\pgfpathmoveto{\pgfqpoint{6.205000in}{1.446943in}}%
\pgfpathlineto{\pgfqpoint{6.570000in}{1.446943in}}%
\pgfpathlineto{\pgfqpoint{6.570000in}{1.458086in}}%
\pgfpathlineto{\pgfqpoint{6.205000in}{1.458086in}}%
\pgfpathlineto{\pgfqpoint{6.205000in}{1.446943in}}%
\pgfusepath{fill}%
\end{pgfscope}%
\begin{pgfscope}%
\pgfpathrectangle{\pgfqpoint{6.205000in}{0.611250in}}{\pgfqpoint{0.365000in}{2.852500in}} %
\pgfusepath{clip}%
\pgfsetbuttcap%
\pgfsetroundjoin%
\definecolor{currentfill}{rgb}{0.619318,0.744121,0.998931}%
\pgfsetfillcolor{currentfill}%
\pgfsetlinewidth{0.000000pt}%
\definecolor{currentstroke}{rgb}{0.000000,0.000000,0.000000}%
\pgfsetstrokecolor{currentstroke}%
\pgfsetdash{}{0pt}%
\pgfpathmoveto{\pgfqpoint{6.205000in}{1.458086in}}%
\pgfpathlineto{\pgfqpoint{6.570000in}{1.458086in}}%
\pgfpathlineto{\pgfqpoint{6.570000in}{1.469229in}}%
\pgfpathlineto{\pgfqpoint{6.205000in}{1.469229in}}%
\pgfpathlineto{\pgfqpoint{6.205000in}{1.458086in}}%
\pgfusepath{fill}%
\end{pgfscope}%
\begin{pgfscope}%
\pgfpathrectangle{\pgfqpoint{6.205000in}{0.611250in}}{\pgfqpoint{0.365000in}{2.852500in}} %
\pgfusepath{clip}%
\pgfsetbuttcap%
\pgfsetroundjoin%
\definecolor{currentfill}{rgb}{0.624703,0.748318,0.998719}%
\pgfsetfillcolor{currentfill}%
\pgfsetlinewidth{0.000000pt}%
\definecolor{currentstroke}{rgb}{0.000000,0.000000,0.000000}%
\pgfsetstrokecolor{currentstroke}%
\pgfsetdash{}{0pt}%
\pgfpathmoveto{\pgfqpoint{6.205000in}{1.469229in}}%
\pgfpathlineto{\pgfqpoint{6.570000in}{1.469229in}}%
\pgfpathlineto{\pgfqpoint{6.570000in}{1.480371in}}%
\pgfpathlineto{\pgfqpoint{6.205000in}{1.480371in}}%
\pgfpathlineto{\pgfqpoint{6.205000in}{1.469229in}}%
\pgfusepath{fill}%
\end{pgfscope}%
\begin{pgfscope}%
\pgfpathrectangle{\pgfqpoint{6.205000in}{0.611250in}}{\pgfqpoint{0.365000in}{2.852500in}} %
\pgfusepath{clip}%
\pgfsetbuttcap%
\pgfsetroundjoin%
\definecolor{currentfill}{rgb}{0.630089,0.752516,0.998508}%
\pgfsetfillcolor{currentfill}%
\pgfsetlinewidth{0.000000pt}%
\definecolor{currentstroke}{rgb}{0.000000,0.000000,0.000000}%
\pgfsetstrokecolor{currentstroke}%
\pgfsetdash{}{0pt}%
\pgfpathmoveto{\pgfqpoint{6.205000in}{1.480371in}}%
\pgfpathlineto{\pgfqpoint{6.570000in}{1.480371in}}%
\pgfpathlineto{\pgfqpoint{6.570000in}{1.491514in}}%
\pgfpathlineto{\pgfqpoint{6.205000in}{1.491514in}}%
\pgfpathlineto{\pgfqpoint{6.205000in}{1.480371in}}%
\pgfusepath{fill}%
\end{pgfscope}%
\begin{pgfscope}%
\pgfpathrectangle{\pgfqpoint{6.205000in}{0.611250in}}{\pgfqpoint{0.365000in}{2.852500in}} %
\pgfusepath{clip}%
\pgfsetbuttcap%
\pgfsetroundjoin%
\definecolor{currentfill}{rgb}{0.635474,0.756714,0.998297}%
\pgfsetfillcolor{currentfill}%
\pgfsetlinewidth{0.000000pt}%
\definecolor{currentstroke}{rgb}{0.000000,0.000000,0.000000}%
\pgfsetstrokecolor{currentstroke}%
\pgfsetdash{}{0pt}%
\pgfpathmoveto{\pgfqpoint{6.205000in}{1.491514in}}%
\pgfpathlineto{\pgfqpoint{6.570000in}{1.491514in}}%
\pgfpathlineto{\pgfqpoint{6.570000in}{1.502656in}}%
\pgfpathlineto{\pgfqpoint{6.205000in}{1.502656in}}%
\pgfpathlineto{\pgfqpoint{6.205000in}{1.491514in}}%
\pgfusepath{fill}%
\end{pgfscope}%
\begin{pgfscope}%
\pgfpathrectangle{\pgfqpoint{6.205000in}{0.611250in}}{\pgfqpoint{0.365000in}{2.852500in}} %
\pgfusepath{clip}%
\pgfsetbuttcap%
\pgfsetroundjoin%
\definecolor{currentfill}{rgb}{0.640828,0.760752,0.997846}%
\pgfsetfillcolor{currentfill}%
\pgfsetlinewidth{0.000000pt}%
\definecolor{currentstroke}{rgb}{0.000000,0.000000,0.000000}%
\pgfsetstrokecolor{currentstroke}%
\pgfsetdash{}{0pt}%
\pgfpathmoveto{\pgfqpoint{6.205000in}{1.502656in}}%
\pgfpathlineto{\pgfqpoint{6.570000in}{1.502656in}}%
\pgfpathlineto{\pgfqpoint{6.570000in}{1.513799in}}%
\pgfpathlineto{\pgfqpoint{6.205000in}{1.513799in}}%
\pgfpathlineto{\pgfqpoint{6.205000in}{1.502656in}}%
\pgfusepath{fill}%
\end{pgfscope}%
\begin{pgfscope}%
\pgfpathrectangle{\pgfqpoint{6.205000in}{0.611250in}}{\pgfqpoint{0.365000in}{2.852500in}} %
\pgfusepath{clip}%
\pgfsetbuttcap%
\pgfsetroundjoin%
\definecolor{currentfill}{rgb}{0.646113,0.764436,0.996868}%
\pgfsetfillcolor{currentfill}%
\pgfsetlinewidth{0.000000pt}%
\definecolor{currentstroke}{rgb}{0.000000,0.000000,0.000000}%
\pgfsetstrokecolor{currentstroke}%
\pgfsetdash{}{0pt}%
\pgfpathmoveto{\pgfqpoint{6.205000in}{1.513799in}}%
\pgfpathlineto{\pgfqpoint{6.570000in}{1.513799in}}%
\pgfpathlineto{\pgfqpoint{6.570000in}{1.524941in}}%
\pgfpathlineto{\pgfqpoint{6.205000in}{1.524941in}}%
\pgfpathlineto{\pgfqpoint{6.205000in}{1.513799in}}%
\pgfusepath{fill}%
\end{pgfscope}%
\begin{pgfscope}%
\pgfpathrectangle{\pgfqpoint{6.205000in}{0.611250in}}{\pgfqpoint{0.365000in}{2.852500in}} %
\pgfusepath{clip}%
\pgfsetbuttcap%
\pgfsetroundjoin%
\definecolor{currentfill}{rgb}{0.651398,0.768121,0.995891}%
\pgfsetfillcolor{currentfill}%
\pgfsetlinewidth{0.000000pt}%
\definecolor{currentstroke}{rgb}{0.000000,0.000000,0.000000}%
\pgfsetstrokecolor{currentstroke}%
\pgfsetdash{}{0pt}%
\pgfpathmoveto{\pgfqpoint{6.205000in}{1.524941in}}%
\pgfpathlineto{\pgfqpoint{6.570000in}{1.524941in}}%
\pgfpathlineto{\pgfqpoint{6.570000in}{1.536084in}}%
\pgfpathlineto{\pgfqpoint{6.205000in}{1.536084in}}%
\pgfpathlineto{\pgfqpoint{6.205000in}{1.524941in}}%
\pgfusepath{fill}%
\end{pgfscope}%
\begin{pgfscope}%
\pgfpathrectangle{\pgfqpoint{6.205000in}{0.611250in}}{\pgfqpoint{0.365000in}{2.852500in}} %
\pgfusepath{clip}%
\pgfsetbuttcap%
\pgfsetroundjoin%
\definecolor{currentfill}{rgb}{0.656683,0.771806,0.994914}%
\pgfsetfillcolor{currentfill}%
\pgfsetlinewidth{0.000000pt}%
\definecolor{currentstroke}{rgb}{0.000000,0.000000,0.000000}%
\pgfsetstrokecolor{currentstroke}%
\pgfsetdash{}{0pt}%
\pgfpathmoveto{\pgfqpoint{6.205000in}{1.536084in}}%
\pgfpathlineto{\pgfqpoint{6.570000in}{1.536084in}}%
\pgfpathlineto{\pgfqpoint{6.570000in}{1.547227in}}%
\pgfpathlineto{\pgfqpoint{6.205000in}{1.547227in}}%
\pgfpathlineto{\pgfqpoint{6.205000in}{1.536084in}}%
\pgfusepath{fill}%
\end{pgfscope}%
\begin{pgfscope}%
\pgfpathrectangle{\pgfqpoint{6.205000in}{0.611250in}}{\pgfqpoint{0.365000in}{2.852500in}} %
\pgfusepath{clip}%
\pgfsetbuttcap%
\pgfsetroundjoin%
\definecolor{currentfill}{rgb}{0.661968,0.775491,0.993937}%
\pgfsetfillcolor{currentfill}%
\pgfsetlinewidth{0.000000pt}%
\definecolor{currentstroke}{rgb}{0.000000,0.000000,0.000000}%
\pgfsetstrokecolor{currentstroke}%
\pgfsetdash{}{0pt}%
\pgfpathmoveto{\pgfqpoint{6.205000in}{1.547227in}}%
\pgfpathlineto{\pgfqpoint{6.570000in}{1.547227in}}%
\pgfpathlineto{\pgfqpoint{6.570000in}{1.558369in}}%
\pgfpathlineto{\pgfqpoint{6.205000in}{1.558369in}}%
\pgfpathlineto{\pgfqpoint{6.205000in}{1.547227in}}%
\pgfusepath{fill}%
\end{pgfscope}%
\begin{pgfscope}%
\pgfpathrectangle{\pgfqpoint{6.205000in}{0.611250in}}{\pgfqpoint{0.365000in}{2.852500in}} %
\pgfusepath{clip}%
\pgfsetbuttcap%
\pgfsetroundjoin%
\definecolor{currentfill}{rgb}{0.667253,0.779176,0.992959}%
\pgfsetfillcolor{currentfill}%
\pgfsetlinewidth{0.000000pt}%
\definecolor{currentstroke}{rgb}{0.000000,0.000000,0.000000}%
\pgfsetstrokecolor{currentstroke}%
\pgfsetdash{}{0pt}%
\pgfpathmoveto{\pgfqpoint{6.205000in}{1.558369in}}%
\pgfpathlineto{\pgfqpoint{6.570000in}{1.558369in}}%
\pgfpathlineto{\pgfqpoint{6.570000in}{1.569512in}}%
\pgfpathlineto{\pgfqpoint{6.205000in}{1.569512in}}%
\pgfpathlineto{\pgfqpoint{6.205000in}{1.558369in}}%
\pgfusepath{fill}%
\end{pgfscope}%
\begin{pgfscope}%
\pgfpathrectangle{\pgfqpoint{6.205000in}{0.611250in}}{\pgfqpoint{0.365000in}{2.852500in}} %
\pgfusepath{clip}%
\pgfsetbuttcap%
\pgfsetroundjoin%
\definecolor{currentfill}{rgb}{0.672538,0.782861,0.991982}%
\pgfsetfillcolor{currentfill}%
\pgfsetlinewidth{0.000000pt}%
\definecolor{currentstroke}{rgb}{0.000000,0.000000,0.000000}%
\pgfsetstrokecolor{currentstroke}%
\pgfsetdash{}{0pt}%
\pgfpathmoveto{\pgfqpoint{6.205000in}{1.569512in}}%
\pgfpathlineto{\pgfqpoint{6.570000in}{1.569512in}}%
\pgfpathlineto{\pgfqpoint{6.570000in}{1.580654in}}%
\pgfpathlineto{\pgfqpoint{6.205000in}{1.580654in}}%
\pgfpathlineto{\pgfqpoint{6.205000in}{1.569512in}}%
\pgfusepath{fill}%
\end{pgfscope}%
\begin{pgfscope}%
\pgfpathrectangle{\pgfqpoint{6.205000in}{0.611250in}}{\pgfqpoint{0.365000in}{2.852500in}} %
\pgfusepath{clip}%
\pgfsetbuttcap%
\pgfsetroundjoin%
\definecolor{currentfill}{rgb}{0.677823,0.786546,0.991005}%
\pgfsetfillcolor{currentfill}%
\pgfsetlinewidth{0.000000pt}%
\definecolor{currentstroke}{rgb}{0.000000,0.000000,0.000000}%
\pgfsetstrokecolor{currentstroke}%
\pgfsetdash{}{0pt}%
\pgfpathmoveto{\pgfqpoint{6.205000in}{1.580654in}}%
\pgfpathlineto{\pgfqpoint{6.570000in}{1.580654in}}%
\pgfpathlineto{\pgfqpoint{6.570000in}{1.591797in}}%
\pgfpathlineto{\pgfqpoint{6.205000in}{1.591797in}}%
\pgfpathlineto{\pgfqpoint{6.205000in}{1.580654in}}%
\pgfusepath{fill}%
\end{pgfscope}%
\begin{pgfscope}%
\pgfpathrectangle{\pgfqpoint{6.205000in}{0.611250in}}{\pgfqpoint{0.365000in}{2.852500in}} %
\pgfusepath{clip}%
\pgfsetbuttcap%
\pgfsetroundjoin%
\definecolor{currentfill}{rgb}{0.683056,0.790043,0.989768}%
\pgfsetfillcolor{currentfill}%
\pgfsetlinewidth{0.000000pt}%
\definecolor{currentstroke}{rgb}{0.000000,0.000000,0.000000}%
\pgfsetstrokecolor{currentstroke}%
\pgfsetdash{}{0pt}%
\pgfpathmoveto{\pgfqpoint{6.205000in}{1.591797in}}%
\pgfpathlineto{\pgfqpoint{6.570000in}{1.591797in}}%
\pgfpathlineto{\pgfqpoint{6.570000in}{1.602939in}}%
\pgfpathlineto{\pgfqpoint{6.205000in}{1.602939in}}%
\pgfpathlineto{\pgfqpoint{6.205000in}{1.591797in}}%
\pgfusepath{fill}%
\end{pgfscope}%
\begin{pgfscope}%
\pgfpathrectangle{\pgfqpoint{6.205000in}{0.611250in}}{\pgfqpoint{0.365000in}{2.852500in}} %
\pgfusepath{clip}%
\pgfsetbuttcap%
\pgfsetroundjoin%
\definecolor{currentfill}{rgb}{0.688188,0.793178,0.988038}%
\pgfsetfillcolor{currentfill}%
\pgfsetlinewidth{0.000000pt}%
\definecolor{currentstroke}{rgb}{0.000000,0.000000,0.000000}%
\pgfsetstrokecolor{currentstroke}%
\pgfsetdash{}{0pt}%
\pgfpathmoveto{\pgfqpoint{6.205000in}{1.602939in}}%
\pgfpathlineto{\pgfqpoint{6.570000in}{1.602939in}}%
\pgfpathlineto{\pgfqpoint{6.570000in}{1.614082in}}%
\pgfpathlineto{\pgfqpoint{6.205000in}{1.614082in}}%
\pgfpathlineto{\pgfqpoint{6.205000in}{1.602939in}}%
\pgfusepath{fill}%
\end{pgfscope}%
\begin{pgfscope}%
\pgfpathrectangle{\pgfqpoint{6.205000in}{0.611250in}}{\pgfqpoint{0.365000in}{2.852500in}} %
\pgfusepath{clip}%
\pgfsetbuttcap%
\pgfsetroundjoin%
\definecolor{currentfill}{rgb}{0.693321,0.796314,0.986308}%
\pgfsetfillcolor{currentfill}%
\pgfsetlinewidth{0.000000pt}%
\definecolor{currentstroke}{rgb}{0.000000,0.000000,0.000000}%
\pgfsetstrokecolor{currentstroke}%
\pgfsetdash{}{0pt}%
\pgfpathmoveto{\pgfqpoint{6.205000in}{1.614082in}}%
\pgfpathlineto{\pgfqpoint{6.570000in}{1.614082in}}%
\pgfpathlineto{\pgfqpoint{6.570000in}{1.625225in}}%
\pgfpathlineto{\pgfqpoint{6.205000in}{1.625225in}}%
\pgfpathlineto{\pgfqpoint{6.205000in}{1.614082in}}%
\pgfusepath{fill}%
\end{pgfscope}%
\begin{pgfscope}%
\pgfpathrectangle{\pgfqpoint{6.205000in}{0.611250in}}{\pgfqpoint{0.365000in}{2.852500in}} %
\pgfusepath{clip}%
\pgfsetbuttcap%
\pgfsetroundjoin%
\definecolor{currentfill}{rgb}{0.698454,0.799450,0.984577}%
\pgfsetfillcolor{currentfill}%
\pgfsetlinewidth{0.000000pt}%
\definecolor{currentstroke}{rgb}{0.000000,0.000000,0.000000}%
\pgfsetstrokecolor{currentstroke}%
\pgfsetdash{}{0pt}%
\pgfpathmoveto{\pgfqpoint{6.205000in}{1.625225in}}%
\pgfpathlineto{\pgfqpoint{6.570000in}{1.625225in}}%
\pgfpathlineto{\pgfqpoint{6.570000in}{1.636367in}}%
\pgfpathlineto{\pgfqpoint{6.205000in}{1.636367in}}%
\pgfpathlineto{\pgfqpoint{6.205000in}{1.625225in}}%
\pgfusepath{fill}%
\end{pgfscope}%
\begin{pgfscope}%
\pgfpathrectangle{\pgfqpoint{6.205000in}{0.611250in}}{\pgfqpoint{0.365000in}{2.852500in}} %
\pgfusepath{clip}%
\pgfsetbuttcap%
\pgfsetroundjoin%
\definecolor{currentfill}{rgb}{0.703587,0.802586,0.982847}%
\pgfsetfillcolor{currentfill}%
\pgfsetlinewidth{0.000000pt}%
\definecolor{currentstroke}{rgb}{0.000000,0.000000,0.000000}%
\pgfsetstrokecolor{currentstroke}%
\pgfsetdash{}{0pt}%
\pgfpathmoveto{\pgfqpoint{6.205000in}{1.636367in}}%
\pgfpathlineto{\pgfqpoint{6.570000in}{1.636367in}}%
\pgfpathlineto{\pgfqpoint{6.570000in}{1.647510in}}%
\pgfpathlineto{\pgfqpoint{6.205000in}{1.647510in}}%
\pgfpathlineto{\pgfqpoint{6.205000in}{1.636367in}}%
\pgfusepath{fill}%
\end{pgfscope}%
\begin{pgfscope}%
\pgfpathrectangle{\pgfqpoint{6.205000in}{0.611250in}}{\pgfqpoint{0.365000in}{2.852500in}} %
\pgfusepath{clip}%
\pgfsetbuttcap%
\pgfsetroundjoin%
\definecolor{currentfill}{rgb}{0.708720,0.805721,0.981117}%
\pgfsetfillcolor{currentfill}%
\pgfsetlinewidth{0.000000pt}%
\definecolor{currentstroke}{rgb}{0.000000,0.000000,0.000000}%
\pgfsetstrokecolor{currentstroke}%
\pgfsetdash{}{0pt}%
\pgfpathmoveto{\pgfqpoint{6.205000in}{1.647510in}}%
\pgfpathlineto{\pgfqpoint{6.570000in}{1.647510in}}%
\pgfpathlineto{\pgfqpoint{6.570000in}{1.658652in}}%
\pgfpathlineto{\pgfqpoint{6.205000in}{1.658652in}}%
\pgfpathlineto{\pgfqpoint{6.205000in}{1.647510in}}%
\pgfusepath{fill}%
\end{pgfscope}%
\begin{pgfscope}%
\pgfpathrectangle{\pgfqpoint{6.205000in}{0.611250in}}{\pgfqpoint{0.365000in}{2.852500in}} %
\pgfusepath{clip}%
\pgfsetbuttcap%
\pgfsetroundjoin%
\definecolor{currentfill}{rgb}{0.713852,0.808857,0.979386}%
\pgfsetfillcolor{currentfill}%
\pgfsetlinewidth{0.000000pt}%
\definecolor{currentstroke}{rgb}{0.000000,0.000000,0.000000}%
\pgfsetstrokecolor{currentstroke}%
\pgfsetdash{}{0pt}%
\pgfpathmoveto{\pgfqpoint{6.205000in}{1.658652in}}%
\pgfpathlineto{\pgfqpoint{6.570000in}{1.658652in}}%
\pgfpathlineto{\pgfqpoint{6.570000in}{1.669795in}}%
\pgfpathlineto{\pgfqpoint{6.205000in}{1.669795in}}%
\pgfpathlineto{\pgfqpoint{6.205000in}{1.658652in}}%
\pgfusepath{fill}%
\end{pgfscope}%
\begin{pgfscope}%
\pgfpathrectangle{\pgfqpoint{6.205000in}{0.611250in}}{\pgfqpoint{0.365000in}{2.852500in}} %
\pgfusepath{clip}%
\pgfsetbuttcap%
\pgfsetroundjoin%
\definecolor{currentfill}{rgb}{0.718985,0.811993,0.977656}%
\pgfsetfillcolor{currentfill}%
\pgfsetlinewidth{0.000000pt}%
\definecolor{currentstroke}{rgb}{0.000000,0.000000,0.000000}%
\pgfsetstrokecolor{currentstroke}%
\pgfsetdash{}{0pt}%
\pgfpathmoveto{\pgfqpoint{6.205000in}{1.669795in}}%
\pgfpathlineto{\pgfqpoint{6.570000in}{1.669795in}}%
\pgfpathlineto{\pgfqpoint{6.570000in}{1.680937in}}%
\pgfpathlineto{\pgfqpoint{6.205000in}{1.680937in}}%
\pgfpathlineto{\pgfqpoint{6.205000in}{1.669795in}}%
\pgfusepath{fill}%
\end{pgfscope}%
\begin{pgfscope}%
\pgfpathrectangle{\pgfqpoint{6.205000in}{0.611250in}}{\pgfqpoint{0.365000in}{2.852500in}} %
\pgfusepath{clip}%
\pgfsetbuttcap%
\pgfsetroundjoin%
\definecolor{currentfill}{rgb}{0.724041,0.814910,0.975651}%
\pgfsetfillcolor{currentfill}%
\pgfsetlinewidth{0.000000pt}%
\definecolor{currentstroke}{rgb}{0.000000,0.000000,0.000000}%
\pgfsetstrokecolor{currentstroke}%
\pgfsetdash{}{0pt}%
\pgfpathmoveto{\pgfqpoint{6.205000in}{1.680937in}}%
\pgfpathlineto{\pgfqpoint{6.570000in}{1.680937in}}%
\pgfpathlineto{\pgfqpoint{6.570000in}{1.692080in}}%
\pgfpathlineto{\pgfqpoint{6.205000in}{1.692080in}}%
\pgfpathlineto{\pgfqpoint{6.205000in}{1.680937in}}%
\pgfusepath{fill}%
\end{pgfscope}%
\begin{pgfscope}%
\pgfpathrectangle{\pgfqpoint{6.205000in}{0.611250in}}{\pgfqpoint{0.365000in}{2.852500in}} %
\pgfusepath{clip}%
\pgfsetbuttcap%
\pgfsetroundjoin%
\definecolor{currentfill}{rgb}{0.728970,0.817464,0.973188}%
\pgfsetfillcolor{currentfill}%
\pgfsetlinewidth{0.000000pt}%
\definecolor{currentstroke}{rgb}{0.000000,0.000000,0.000000}%
\pgfsetstrokecolor{currentstroke}%
\pgfsetdash{}{0pt}%
\pgfpathmoveto{\pgfqpoint{6.205000in}{1.692080in}}%
\pgfpathlineto{\pgfqpoint{6.570000in}{1.692080in}}%
\pgfpathlineto{\pgfqpoint{6.570000in}{1.703223in}}%
\pgfpathlineto{\pgfqpoint{6.205000in}{1.703223in}}%
\pgfpathlineto{\pgfqpoint{6.205000in}{1.692080in}}%
\pgfusepath{fill}%
\end{pgfscope}%
\begin{pgfscope}%
\pgfpathrectangle{\pgfqpoint{6.205000in}{0.611250in}}{\pgfqpoint{0.365000in}{2.852500in}} %
\pgfusepath{clip}%
\pgfsetbuttcap%
\pgfsetroundjoin%
\definecolor{currentfill}{rgb}{0.733898,0.820018,0.970724}%
\pgfsetfillcolor{currentfill}%
\pgfsetlinewidth{0.000000pt}%
\definecolor{currentstroke}{rgb}{0.000000,0.000000,0.000000}%
\pgfsetstrokecolor{currentstroke}%
\pgfsetdash{}{0pt}%
\pgfpathmoveto{\pgfqpoint{6.205000in}{1.703223in}}%
\pgfpathlineto{\pgfqpoint{6.570000in}{1.703223in}}%
\pgfpathlineto{\pgfqpoint{6.570000in}{1.714365in}}%
\pgfpathlineto{\pgfqpoint{6.205000in}{1.714365in}}%
\pgfpathlineto{\pgfqpoint{6.205000in}{1.703223in}}%
\pgfusepath{fill}%
\end{pgfscope}%
\begin{pgfscope}%
\pgfpathrectangle{\pgfqpoint{6.205000in}{0.611250in}}{\pgfqpoint{0.365000in}{2.852500in}} %
\pgfusepath{clip}%
\pgfsetbuttcap%
\pgfsetroundjoin%
\definecolor{currentfill}{rgb}{0.738826,0.822572,0.968261}%
\pgfsetfillcolor{currentfill}%
\pgfsetlinewidth{0.000000pt}%
\definecolor{currentstroke}{rgb}{0.000000,0.000000,0.000000}%
\pgfsetstrokecolor{currentstroke}%
\pgfsetdash{}{0pt}%
\pgfpathmoveto{\pgfqpoint{6.205000in}{1.714365in}}%
\pgfpathlineto{\pgfqpoint{6.570000in}{1.714365in}}%
\pgfpathlineto{\pgfqpoint{6.570000in}{1.725508in}}%
\pgfpathlineto{\pgfqpoint{6.205000in}{1.725508in}}%
\pgfpathlineto{\pgfqpoint{6.205000in}{1.714365in}}%
\pgfusepath{fill}%
\end{pgfscope}%
\begin{pgfscope}%
\pgfpathrectangle{\pgfqpoint{6.205000in}{0.611250in}}{\pgfqpoint{0.365000in}{2.852500in}} %
\pgfusepath{clip}%
\pgfsetbuttcap%
\pgfsetroundjoin%
\definecolor{currentfill}{rgb}{0.743754,0.825125,0.965798}%
\pgfsetfillcolor{currentfill}%
\pgfsetlinewidth{0.000000pt}%
\definecolor{currentstroke}{rgb}{0.000000,0.000000,0.000000}%
\pgfsetstrokecolor{currentstroke}%
\pgfsetdash{}{0pt}%
\pgfpathmoveto{\pgfqpoint{6.205000in}{1.725508in}}%
\pgfpathlineto{\pgfqpoint{6.570000in}{1.725508in}}%
\pgfpathlineto{\pgfqpoint{6.570000in}{1.736650in}}%
\pgfpathlineto{\pgfqpoint{6.205000in}{1.736650in}}%
\pgfpathlineto{\pgfqpoint{6.205000in}{1.725508in}}%
\pgfusepath{fill}%
\end{pgfscope}%
\begin{pgfscope}%
\pgfpathrectangle{\pgfqpoint{6.205000in}{0.611250in}}{\pgfqpoint{0.365000in}{2.852500in}} %
\pgfusepath{clip}%
\pgfsetbuttcap%
\pgfsetroundjoin%
\definecolor{currentfill}{rgb}{0.748682,0.827679,0.963334}%
\pgfsetfillcolor{currentfill}%
\pgfsetlinewidth{0.000000pt}%
\definecolor{currentstroke}{rgb}{0.000000,0.000000,0.000000}%
\pgfsetstrokecolor{currentstroke}%
\pgfsetdash{}{0pt}%
\pgfpathmoveto{\pgfqpoint{6.205000in}{1.736650in}}%
\pgfpathlineto{\pgfqpoint{6.570000in}{1.736650in}}%
\pgfpathlineto{\pgfqpoint{6.570000in}{1.747793in}}%
\pgfpathlineto{\pgfqpoint{6.205000in}{1.747793in}}%
\pgfpathlineto{\pgfqpoint{6.205000in}{1.736650in}}%
\pgfusepath{fill}%
\end{pgfscope}%
\begin{pgfscope}%
\pgfpathrectangle{\pgfqpoint{6.205000in}{0.611250in}}{\pgfqpoint{0.365000in}{2.852500in}} %
\pgfusepath{clip}%
\pgfsetbuttcap%
\pgfsetroundjoin%
\definecolor{currentfill}{rgb}{0.753611,0.830233,0.960871}%
\pgfsetfillcolor{currentfill}%
\pgfsetlinewidth{0.000000pt}%
\definecolor{currentstroke}{rgb}{0.000000,0.000000,0.000000}%
\pgfsetstrokecolor{currentstroke}%
\pgfsetdash{}{0pt}%
\pgfpathmoveto{\pgfqpoint{6.205000in}{1.747793in}}%
\pgfpathlineto{\pgfqpoint{6.570000in}{1.747793in}}%
\pgfpathlineto{\pgfqpoint{6.570000in}{1.758936in}}%
\pgfpathlineto{\pgfqpoint{6.205000in}{1.758936in}}%
\pgfpathlineto{\pgfqpoint{6.205000in}{1.747793in}}%
\pgfusepath{fill}%
\end{pgfscope}%
\begin{pgfscope}%
\pgfpathrectangle{\pgfqpoint{6.205000in}{0.611250in}}{\pgfqpoint{0.365000in}{2.852500in}} %
\pgfusepath{clip}%
\pgfsetbuttcap%
\pgfsetroundjoin%
\definecolor{currentfill}{rgb}{0.758539,0.832787,0.958408}%
\pgfsetfillcolor{currentfill}%
\pgfsetlinewidth{0.000000pt}%
\definecolor{currentstroke}{rgb}{0.000000,0.000000,0.000000}%
\pgfsetstrokecolor{currentstroke}%
\pgfsetdash{}{0pt}%
\pgfpathmoveto{\pgfqpoint{6.205000in}{1.758936in}}%
\pgfpathlineto{\pgfqpoint{6.570000in}{1.758936in}}%
\pgfpathlineto{\pgfqpoint{6.570000in}{1.770078in}}%
\pgfpathlineto{\pgfqpoint{6.205000in}{1.770078in}}%
\pgfpathlineto{\pgfqpoint{6.205000in}{1.758936in}}%
\pgfusepath{fill}%
\end{pgfscope}%
\begin{pgfscope}%
\pgfpathrectangle{\pgfqpoint{6.205000in}{0.611250in}}{\pgfqpoint{0.365000in}{2.852500in}} %
\pgfusepath{clip}%
\pgfsetbuttcap%
\pgfsetroundjoin%
\definecolor{currentfill}{rgb}{0.763363,0.835092,0.955658}%
\pgfsetfillcolor{currentfill}%
\pgfsetlinewidth{0.000000pt}%
\definecolor{currentstroke}{rgb}{0.000000,0.000000,0.000000}%
\pgfsetstrokecolor{currentstroke}%
\pgfsetdash{}{0pt}%
\pgfpathmoveto{\pgfqpoint{6.205000in}{1.770078in}}%
\pgfpathlineto{\pgfqpoint{6.570000in}{1.770078in}}%
\pgfpathlineto{\pgfqpoint{6.570000in}{1.781221in}}%
\pgfpathlineto{\pgfqpoint{6.205000in}{1.781221in}}%
\pgfpathlineto{\pgfqpoint{6.205000in}{1.770078in}}%
\pgfusepath{fill}%
\end{pgfscope}%
\begin{pgfscope}%
\pgfpathrectangle{\pgfqpoint{6.205000in}{0.611250in}}{\pgfqpoint{0.365000in}{2.852500in}} %
\pgfusepath{clip}%
\pgfsetbuttcap%
\pgfsetroundjoin%
\definecolor{currentfill}{rgb}{0.768034,0.837035,0.952488}%
\pgfsetfillcolor{currentfill}%
\pgfsetlinewidth{0.000000pt}%
\definecolor{currentstroke}{rgb}{0.000000,0.000000,0.000000}%
\pgfsetstrokecolor{currentstroke}%
\pgfsetdash{}{0pt}%
\pgfpathmoveto{\pgfqpoint{6.205000in}{1.781221in}}%
\pgfpathlineto{\pgfqpoint{6.570000in}{1.781221in}}%
\pgfpathlineto{\pgfqpoint{6.570000in}{1.792363in}}%
\pgfpathlineto{\pgfqpoint{6.205000in}{1.792363in}}%
\pgfpathlineto{\pgfqpoint{6.205000in}{1.781221in}}%
\pgfusepath{fill}%
\end{pgfscope}%
\begin{pgfscope}%
\pgfpathrectangle{\pgfqpoint{6.205000in}{0.611250in}}{\pgfqpoint{0.365000in}{2.852500in}} %
\pgfusepath{clip}%
\pgfsetbuttcap%
\pgfsetroundjoin%
\definecolor{currentfill}{rgb}{0.772706,0.838978,0.949319}%
\pgfsetfillcolor{currentfill}%
\pgfsetlinewidth{0.000000pt}%
\definecolor{currentstroke}{rgb}{0.000000,0.000000,0.000000}%
\pgfsetstrokecolor{currentstroke}%
\pgfsetdash{}{0pt}%
\pgfpathmoveto{\pgfqpoint{6.205000in}{1.792363in}}%
\pgfpathlineto{\pgfqpoint{6.570000in}{1.792363in}}%
\pgfpathlineto{\pgfqpoint{6.570000in}{1.803506in}}%
\pgfpathlineto{\pgfqpoint{6.205000in}{1.803506in}}%
\pgfpathlineto{\pgfqpoint{6.205000in}{1.792363in}}%
\pgfusepath{fill}%
\end{pgfscope}%
\begin{pgfscope}%
\pgfpathrectangle{\pgfqpoint{6.205000in}{0.611250in}}{\pgfqpoint{0.365000in}{2.852500in}} %
\pgfusepath{clip}%
\pgfsetbuttcap%
\pgfsetroundjoin%
\definecolor{currentfill}{rgb}{0.777378,0.840921,0.946149}%
\pgfsetfillcolor{currentfill}%
\pgfsetlinewidth{0.000000pt}%
\definecolor{currentstroke}{rgb}{0.000000,0.000000,0.000000}%
\pgfsetstrokecolor{currentstroke}%
\pgfsetdash{}{0pt}%
\pgfpathmoveto{\pgfqpoint{6.205000in}{1.803506in}}%
\pgfpathlineto{\pgfqpoint{6.570000in}{1.803506in}}%
\pgfpathlineto{\pgfqpoint{6.570000in}{1.814648in}}%
\pgfpathlineto{\pgfqpoint{6.205000in}{1.814648in}}%
\pgfpathlineto{\pgfqpoint{6.205000in}{1.803506in}}%
\pgfusepath{fill}%
\end{pgfscope}%
\begin{pgfscope}%
\pgfpathrectangle{\pgfqpoint{6.205000in}{0.611250in}}{\pgfqpoint{0.365000in}{2.852500in}} %
\pgfusepath{clip}%
\pgfsetbuttcap%
\pgfsetroundjoin%
\definecolor{currentfill}{rgb}{0.782049,0.842864,0.942980}%
\pgfsetfillcolor{currentfill}%
\pgfsetlinewidth{0.000000pt}%
\definecolor{currentstroke}{rgb}{0.000000,0.000000,0.000000}%
\pgfsetstrokecolor{currentstroke}%
\pgfsetdash{}{0pt}%
\pgfpathmoveto{\pgfqpoint{6.205000in}{1.814648in}}%
\pgfpathlineto{\pgfqpoint{6.570000in}{1.814648in}}%
\pgfpathlineto{\pgfqpoint{6.570000in}{1.825791in}}%
\pgfpathlineto{\pgfqpoint{6.205000in}{1.825791in}}%
\pgfpathlineto{\pgfqpoint{6.205000in}{1.814648in}}%
\pgfusepath{fill}%
\end{pgfscope}%
\begin{pgfscope}%
\pgfpathrectangle{\pgfqpoint{6.205000in}{0.611250in}}{\pgfqpoint{0.365000in}{2.852500in}} %
\pgfusepath{clip}%
\pgfsetbuttcap%
\pgfsetroundjoin%
\definecolor{currentfill}{rgb}{0.786721,0.844807,0.939810}%
\pgfsetfillcolor{currentfill}%
\pgfsetlinewidth{0.000000pt}%
\definecolor{currentstroke}{rgb}{0.000000,0.000000,0.000000}%
\pgfsetstrokecolor{currentstroke}%
\pgfsetdash{}{0pt}%
\pgfpathmoveto{\pgfqpoint{6.205000in}{1.825791in}}%
\pgfpathlineto{\pgfqpoint{6.570000in}{1.825791in}}%
\pgfpathlineto{\pgfqpoint{6.570000in}{1.836934in}}%
\pgfpathlineto{\pgfqpoint{6.205000in}{1.836934in}}%
\pgfpathlineto{\pgfqpoint{6.205000in}{1.825791in}}%
\pgfusepath{fill}%
\end{pgfscope}%
\begin{pgfscope}%
\pgfpathrectangle{\pgfqpoint{6.205000in}{0.611250in}}{\pgfqpoint{0.365000in}{2.852500in}} %
\pgfusepath{clip}%
\pgfsetbuttcap%
\pgfsetroundjoin%
\definecolor{currentfill}{rgb}{0.791392,0.846750,0.936641}%
\pgfsetfillcolor{currentfill}%
\pgfsetlinewidth{0.000000pt}%
\definecolor{currentstroke}{rgb}{0.000000,0.000000,0.000000}%
\pgfsetstrokecolor{currentstroke}%
\pgfsetdash{}{0pt}%
\pgfpathmoveto{\pgfqpoint{6.205000in}{1.836934in}}%
\pgfpathlineto{\pgfqpoint{6.570000in}{1.836934in}}%
\pgfpathlineto{\pgfqpoint{6.570000in}{1.848076in}}%
\pgfpathlineto{\pgfqpoint{6.205000in}{1.848076in}}%
\pgfpathlineto{\pgfqpoint{6.205000in}{1.836934in}}%
\pgfusepath{fill}%
\end{pgfscope}%
\begin{pgfscope}%
\pgfpathrectangle{\pgfqpoint{6.205000in}{0.611250in}}{\pgfqpoint{0.365000in}{2.852500in}} %
\pgfusepath{clip}%
\pgfsetbuttcap%
\pgfsetroundjoin%
\definecolor{currentfill}{rgb}{0.796064,0.848693,0.933471}%
\pgfsetfillcolor{currentfill}%
\pgfsetlinewidth{0.000000pt}%
\definecolor{currentstroke}{rgb}{0.000000,0.000000,0.000000}%
\pgfsetstrokecolor{currentstroke}%
\pgfsetdash{}{0pt}%
\pgfpathmoveto{\pgfqpoint{6.205000in}{1.848076in}}%
\pgfpathlineto{\pgfqpoint{6.570000in}{1.848076in}}%
\pgfpathlineto{\pgfqpoint{6.570000in}{1.859219in}}%
\pgfpathlineto{\pgfqpoint{6.205000in}{1.859219in}}%
\pgfpathlineto{\pgfqpoint{6.205000in}{1.848076in}}%
\pgfusepath{fill}%
\end{pgfscope}%
\begin{pgfscope}%
\pgfpathrectangle{\pgfqpoint{6.205000in}{0.611250in}}{\pgfqpoint{0.365000in}{2.852500in}} %
\pgfusepath{clip}%
\pgfsetbuttcap%
\pgfsetroundjoin%
\definecolor{currentfill}{rgb}{0.800601,0.850358,0.930008}%
\pgfsetfillcolor{currentfill}%
\pgfsetlinewidth{0.000000pt}%
\definecolor{currentstroke}{rgb}{0.000000,0.000000,0.000000}%
\pgfsetstrokecolor{currentstroke}%
\pgfsetdash{}{0pt}%
\pgfpathmoveto{\pgfqpoint{6.205000in}{1.859219in}}%
\pgfpathlineto{\pgfqpoint{6.570000in}{1.859219in}}%
\pgfpathlineto{\pgfqpoint{6.570000in}{1.870361in}}%
\pgfpathlineto{\pgfqpoint{6.205000in}{1.870361in}}%
\pgfpathlineto{\pgfqpoint{6.205000in}{1.859219in}}%
\pgfusepath{fill}%
\end{pgfscope}%
\begin{pgfscope}%
\pgfpathrectangle{\pgfqpoint{6.205000in}{0.611250in}}{\pgfqpoint{0.365000in}{2.852500in}} %
\pgfusepath{clip}%
\pgfsetbuttcap%
\pgfsetroundjoin%
\definecolor{currentfill}{rgb}{0.804965,0.851666,0.926165}%
\pgfsetfillcolor{currentfill}%
\pgfsetlinewidth{0.000000pt}%
\definecolor{currentstroke}{rgb}{0.000000,0.000000,0.000000}%
\pgfsetstrokecolor{currentstroke}%
\pgfsetdash{}{0pt}%
\pgfpathmoveto{\pgfqpoint{6.205000in}{1.870361in}}%
\pgfpathlineto{\pgfqpoint{6.570000in}{1.870361in}}%
\pgfpathlineto{\pgfqpoint{6.570000in}{1.881504in}}%
\pgfpathlineto{\pgfqpoint{6.205000in}{1.881504in}}%
\pgfpathlineto{\pgfqpoint{6.205000in}{1.870361in}}%
\pgfusepath{fill}%
\end{pgfscope}%
\begin{pgfscope}%
\pgfpathrectangle{\pgfqpoint{6.205000in}{0.611250in}}{\pgfqpoint{0.365000in}{2.852500in}} %
\pgfusepath{clip}%
\pgfsetbuttcap%
\pgfsetroundjoin%
\definecolor{currentfill}{rgb}{0.809329,0.852974,0.922323}%
\pgfsetfillcolor{currentfill}%
\pgfsetlinewidth{0.000000pt}%
\definecolor{currentstroke}{rgb}{0.000000,0.000000,0.000000}%
\pgfsetstrokecolor{currentstroke}%
\pgfsetdash{}{0pt}%
\pgfpathmoveto{\pgfqpoint{6.205000in}{1.881504in}}%
\pgfpathlineto{\pgfqpoint{6.570000in}{1.881504in}}%
\pgfpathlineto{\pgfqpoint{6.570000in}{1.892646in}}%
\pgfpathlineto{\pgfqpoint{6.205000in}{1.892646in}}%
\pgfpathlineto{\pgfqpoint{6.205000in}{1.881504in}}%
\pgfusepath{fill}%
\end{pgfscope}%
\begin{pgfscope}%
\pgfpathrectangle{\pgfqpoint{6.205000in}{0.611250in}}{\pgfqpoint{0.365000in}{2.852500in}} %
\pgfusepath{clip}%
\pgfsetbuttcap%
\pgfsetroundjoin%
\definecolor{currentfill}{rgb}{0.813693,0.854282,0.918480}%
\pgfsetfillcolor{currentfill}%
\pgfsetlinewidth{0.000000pt}%
\definecolor{currentstroke}{rgb}{0.000000,0.000000,0.000000}%
\pgfsetstrokecolor{currentstroke}%
\pgfsetdash{}{0pt}%
\pgfpathmoveto{\pgfqpoint{6.205000in}{1.892646in}}%
\pgfpathlineto{\pgfqpoint{6.570000in}{1.892646in}}%
\pgfpathlineto{\pgfqpoint{6.570000in}{1.903789in}}%
\pgfpathlineto{\pgfqpoint{6.205000in}{1.903789in}}%
\pgfpathlineto{\pgfqpoint{6.205000in}{1.892646in}}%
\pgfusepath{fill}%
\end{pgfscope}%
\begin{pgfscope}%
\pgfpathrectangle{\pgfqpoint{6.205000in}{0.611250in}}{\pgfqpoint{0.365000in}{2.852500in}} %
\pgfusepath{clip}%
\pgfsetbuttcap%
\pgfsetroundjoin%
\definecolor{currentfill}{rgb}{0.818056,0.855590,0.914638}%
\pgfsetfillcolor{currentfill}%
\pgfsetlinewidth{0.000000pt}%
\definecolor{currentstroke}{rgb}{0.000000,0.000000,0.000000}%
\pgfsetstrokecolor{currentstroke}%
\pgfsetdash{}{0pt}%
\pgfpathmoveto{\pgfqpoint{6.205000in}{1.903789in}}%
\pgfpathlineto{\pgfqpoint{6.570000in}{1.903789in}}%
\pgfpathlineto{\pgfqpoint{6.570000in}{1.914932in}}%
\pgfpathlineto{\pgfqpoint{6.205000in}{1.914932in}}%
\pgfpathlineto{\pgfqpoint{6.205000in}{1.903789in}}%
\pgfusepath{fill}%
\end{pgfscope}%
\begin{pgfscope}%
\pgfpathrectangle{\pgfqpoint{6.205000in}{0.611250in}}{\pgfqpoint{0.365000in}{2.852500in}} %
\pgfusepath{clip}%
\pgfsetbuttcap%
\pgfsetroundjoin%
\definecolor{currentfill}{rgb}{0.822420,0.856898,0.910795}%
\pgfsetfillcolor{currentfill}%
\pgfsetlinewidth{0.000000pt}%
\definecolor{currentstroke}{rgb}{0.000000,0.000000,0.000000}%
\pgfsetstrokecolor{currentstroke}%
\pgfsetdash{}{0pt}%
\pgfpathmoveto{\pgfqpoint{6.205000in}{1.914932in}}%
\pgfpathlineto{\pgfqpoint{6.570000in}{1.914932in}}%
\pgfpathlineto{\pgfqpoint{6.570000in}{1.926074in}}%
\pgfpathlineto{\pgfqpoint{6.205000in}{1.926074in}}%
\pgfpathlineto{\pgfqpoint{6.205000in}{1.914932in}}%
\pgfusepath{fill}%
\end{pgfscope}%
\begin{pgfscope}%
\pgfpathrectangle{\pgfqpoint{6.205000in}{0.611250in}}{\pgfqpoint{0.365000in}{2.852500in}} %
\pgfusepath{clip}%
\pgfsetbuttcap%
\pgfsetroundjoin%
\definecolor{currentfill}{rgb}{0.826784,0.858205,0.906953}%
\pgfsetfillcolor{currentfill}%
\pgfsetlinewidth{0.000000pt}%
\definecolor{currentstroke}{rgb}{0.000000,0.000000,0.000000}%
\pgfsetstrokecolor{currentstroke}%
\pgfsetdash{}{0pt}%
\pgfpathmoveto{\pgfqpoint{6.205000in}{1.926074in}}%
\pgfpathlineto{\pgfqpoint{6.570000in}{1.926074in}}%
\pgfpathlineto{\pgfqpoint{6.570000in}{1.937217in}}%
\pgfpathlineto{\pgfqpoint{6.205000in}{1.937217in}}%
\pgfpathlineto{\pgfqpoint{6.205000in}{1.926074in}}%
\pgfusepath{fill}%
\end{pgfscope}%
\begin{pgfscope}%
\pgfpathrectangle{\pgfqpoint{6.205000in}{0.611250in}}{\pgfqpoint{0.365000in}{2.852500in}} %
\pgfusepath{clip}%
\pgfsetbuttcap%
\pgfsetroundjoin%
\definecolor{currentfill}{rgb}{0.831148,0.859513,0.903110}%
\pgfsetfillcolor{currentfill}%
\pgfsetlinewidth{0.000000pt}%
\definecolor{currentstroke}{rgb}{0.000000,0.000000,0.000000}%
\pgfsetstrokecolor{currentstroke}%
\pgfsetdash{}{0pt}%
\pgfpathmoveto{\pgfqpoint{6.205000in}{1.937217in}}%
\pgfpathlineto{\pgfqpoint{6.570000in}{1.937217in}}%
\pgfpathlineto{\pgfqpoint{6.570000in}{1.948359in}}%
\pgfpathlineto{\pgfqpoint{6.205000in}{1.948359in}}%
\pgfpathlineto{\pgfqpoint{6.205000in}{1.937217in}}%
\pgfusepath{fill}%
\end{pgfscope}%
\begin{pgfscope}%
\pgfpathrectangle{\pgfqpoint{6.205000in}{0.611250in}}{\pgfqpoint{0.365000in}{2.852500in}} %
\pgfusepath{clip}%
\pgfsetbuttcap%
\pgfsetroundjoin%
\definecolor{currentfill}{rgb}{0.835345,0.860514,0.898970}%
\pgfsetfillcolor{currentfill}%
\pgfsetlinewidth{0.000000pt}%
\definecolor{currentstroke}{rgb}{0.000000,0.000000,0.000000}%
\pgfsetstrokecolor{currentstroke}%
\pgfsetdash{}{0pt}%
\pgfpathmoveto{\pgfqpoint{6.205000in}{1.948359in}}%
\pgfpathlineto{\pgfqpoint{6.570000in}{1.948359in}}%
\pgfpathlineto{\pgfqpoint{6.570000in}{1.959502in}}%
\pgfpathlineto{\pgfqpoint{6.205000in}{1.959502in}}%
\pgfpathlineto{\pgfqpoint{6.205000in}{1.948359in}}%
\pgfusepath{fill}%
\end{pgfscope}%
\begin{pgfscope}%
\pgfpathrectangle{\pgfqpoint{6.205000in}{0.611250in}}{\pgfqpoint{0.365000in}{2.852500in}} %
\pgfusepath{clip}%
\pgfsetbuttcap%
\pgfsetroundjoin%
\definecolor{currentfill}{rgb}{0.839351,0.861167,0.894494}%
\pgfsetfillcolor{currentfill}%
\pgfsetlinewidth{0.000000pt}%
\definecolor{currentstroke}{rgb}{0.000000,0.000000,0.000000}%
\pgfsetstrokecolor{currentstroke}%
\pgfsetdash{}{0pt}%
\pgfpathmoveto{\pgfqpoint{6.205000in}{1.959502in}}%
\pgfpathlineto{\pgfqpoint{6.570000in}{1.959502in}}%
\pgfpathlineto{\pgfqpoint{6.570000in}{1.970645in}}%
\pgfpathlineto{\pgfqpoint{6.205000in}{1.970645in}}%
\pgfpathlineto{\pgfqpoint{6.205000in}{1.959502in}}%
\pgfusepath{fill}%
\end{pgfscope}%
\begin{pgfscope}%
\pgfpathrectangle{\pgfqpoint{6.205000in}{0.611250in}}{\pgfqpoint{0.365000in}{2.852500in}} %
\pgfusepath{clip}%
\pgfsetbuttcap%
\pgfsetroundjoin%
\definecolor{currentfill}{rgb}{0.843358,0.861820,0.890017}%
\pgfsetfillcolor{currentfill}%
\pgfsetlinewidth{0.000000pt}%
\definecolor{currentstroke}{rgb}{0.000000,0.000000,0.000000}%
\pgfsetstrokecolor{currentstroke}%
\pgfsetdash{}{0pt}%
\pgfpathmoveto{\pgfqpoint{6.205000in}{1.970645in}}%
\pgfpathlineto{\pgfqpoint{6.570000in}{1.970645in}}%
\pgfpathlineto{\pgfqpoint{6.570000in}{1.981787in}}%
\pgfpathlineto{\pgfqpoint{6.205000in}{1.981787in}}%
\pgfpathlineto{\pgfqpoint{6.205000in}{1.970645in}}%
\pgfusepath{fill}%
\end{pgfscope}%
\begin{pgfscope}%
\pgfpathrectangle{\pgfqpoint{6.205000in}{0.611250in}}{\pgfqpoint{0.365000in}{2.852500in}} %
\pgfusepath{clip}%
\pgfsetbuttcap%
\pgfsetroundjoin%
\definecolor{currentfill}{rgb}{0.847365,0.862472,0.885540}%
\pgfsetfillcolor{currentfill}%
\pgfsetlinewidth{0.000000pt}%
\definecolor{currentstroke}{rgb}{0.000000,0.000000,0.000000}%
\pgfsetstrokecolor{currentstroke}%
\pgfsetdash{}{0pt}%
\pgfpathmoveto{\pgfqpoint{6.205000in}{1.981787in}}%
\pgfpathlineto{\pgfqpoint{6.570000in}{1.981787in}}%
\pgfpathlineto{\pgfqpoint{6.570000in}{1.992930in}}%
\pgfpathlineto{\pgfqpoint{6.205000in}{1.992930in}}%
\pgfpathlineto{\pgfqpoint{6.205000in}{1.981787in}}%
\pgfusepath{fill}%
\end{pgfscope}%
\begin{pgfscope}%
\pgfpathrectangle{\pgfqpoint{6.205000in}{0.611250in}}{\pgfqpoint{0.365000in}{2.852500in}} %
\pgfusepath{clip}%
\pgfsetbuttcap%
\pgfsetroundjoin%
\definecolor{currentfill}{rgb}{0.851372,0.863125,0.881064}%
\pgfsetfillcolor{currentfill}%
\pgfsetlinewidth{0.000000pt}%
\definecolor{currentstroke}{rgb}{0.000000,0.000000,0.000000}%
\pgfsetstrokecolor{currentstroke}%
\pgfsetdash{}{0pt}%
\pgfpathmoveto{\pgfqpoint{6.205000in}{1.992930in}}%
\pgfpathlineto{\pgfqpoint{6.570000in}{1.992930in}}%
\pgfpathlineto{\pgfqpoint{6.570000in}{2.004072in}}%
\pgfpathlineto{\pgfqpoint{6.205000in}{2.004072in}}%
\pgfpathlineto{\pgfqpoint{6.205000in}{1.992930in}}%
\pgfusepath{fill}%
\end{pgfscope}%
\begin{pgfscope}%
\pgfpathrectangle{\pgfqpoint{6.205000in}{0.611250in}}{\pgfqpoint{0.365000in}{2.852500in}} %
\pgfusepath{clip}%
\pgfsetbuttcap%
\pgfsetroundjoin%
\definecolor{currentfill}{rgb}{0.855378,0.863778,0.876587}%
\pgfsetfillcolor{currentfill}%
\pgfsetlinewidth{0.000000pt}%
\definecolor{currentstroke}{rgb}{0.000000,0.000000,0.000000}%
\pgfsetstrokecolor{currentstroke}%
\pgfsetdash{}{0pt}%
\pgfpathmoveto{\pgfqpoint{6.205000in}{2.004072in}}%
\pgfpathlineto{\pgfqpoint{6.570000in}{2.004072in}}%
\pgfpathlineto{\pgfqpoint{6.570000in}{2.015215in}}%
\pgfpathlineto{\pgfqpoint{6.205000in}{2.015215in}}%
\pgfpathlineto{\pgfqpoint{6.205000in}{2.004072in}}%
\pgfusepath{fill}%
\end{pgfscope}%
\begin{pgfscope}%
\pgfpathrectangle{\pgfqpoint{6.205000in}{0.611250in}}{\pgfqpoint{0.365000in}{2.852500in}} %
\pgfusepath{clip}%
\pgfsetbuttcap%
\pgfsetroundjoin%
\definecolor{currentfill}{rgb}{0.859385,0.864431,0.872111}%
\pgfsetfillcolor{currentfill}%
\pgfsetlinewidth{0.000000pt}%
\definecolor{currentstroke}{rgb}{0.000000,0.000000,0.000000}%
\pgfsetstrokecolor{currentstroke}%
\pgfsetdash{}{0pt}%
\pgfpathmoveto{\pgfqpoint{6.205000in}{2.015215in}}%
\pgfpathlineto{\pgfqpoint{6.570000in}{2.015215in}}%
\pgfpathlineto{\pgfqpoint{6.570000in}{2.026357in}}%
\pgfpathlineto{\pgfqpoint{6.205000in}{2.026357in}}%
\pgfpathlineto{\pgfqpoint{6.205000in}{2.015215in}}%
\pgfusepath{fill}%
\end{pgfscope}%
\begin{pgfscope}%
\pgfpathrectangle{\pgfqpoint{6.205000in}{0.611250in}}{\pgfqpoint{0.365000in}{2.852500in}} %
\pgfusepath{clip}%
\pgfsetbuttcap%
\pgfsetroundjoin%
\definecolor{currentfill}{rgb}{0.863392,0.865084,0.867634}%
\pgfsetfillcolor{currentfill}%
\pgfsetlinewidth{0.000000pt}%
\definecolor{currentstroke}{rgb}{0.000000,0.000000,0.000000}%
\pgfsetstrokecolor{currentstroke}%
\pgfsetdash{}{0pt}%
\pgfpathmoveto{\pgfqpoint{6.205000in}{2.026357in}}%
\pgfpathlineto{\pgfqpoint{6.570000in}{2.026357in}}%
\pgfpathlineto{\pgfqpoint{6.570000in}{2.037500in}}%
\pgfpathlineto{\pgfqpoint{6.205000in}{2.037500in}}%
\pgfpathlineto{\pgfqpoint{6.205000in}{2.026357in}}%
\pgfusepath{fill}%
\end{pgfscope}%
\begin{pgfscope}%
\pgfpathrectangle{\pgfqpoint{6.205000in}{0.611250in}}{\pgfqpoint{0.365000in}{2.852500in}} %
\pgfusepath{clip}%
\pgfsetbuttcap%
\pgfsetroundjoin%
\definecolor{currentfill}{rgb}{0.867428,0.864377,0.862602}%
\pgfsetfillcolor{currentfill}%
\pgfsetlinewidth{0.000000pt}%
\definecolor{currentstroke}{rgb}{0.000000,0.000000,0.000000}%
\pgfsetstrokecolor{currentstroke}%
\pgfsetdash{}{0pt}%
\pgfpathmoveto{\pgfqpoint{6.205000in}{2.037500in}}%
\pgfpathlineto{\pgfqpoint{6.570000in}{2.037500in}}%
\pgfpathlineto{\pgfqpoint{6.570000in}{2.048643in}}%
\pgfpathlineto{\pgfqpoint{6.205000in}{2.048643in}}%
\pgfpathlineto{\pgfqpoint{6.205000in}{2.037500in}}%
\pgfusepath{fill}%
\end{pgfscope}%
\begin{pgfscope}%
\pgfpathrectangle{\pgfqpoint{6.205000in}{0.611250in}}{\pgfqpoint{0.365000in}{2.852500in}} %
\pgfusepath{clip}%
\pgfsetbuttcap%
\pgfsetroundjoin%
\definecolor{currentfill}{rgb}{0.871493,0.862309,0.857016}%
\pgfsetfillcolor{currentfill}%
\pgfsetlinewidth{0.000000pt}%
\definecolor{currentstroke}{rgb}{0.000000,0.000000,0.000000}%
\pgfsetstrokecolor{currentstroke}%
\pgfsetdash{}{0pt}%
\pgfpathmoveto{\pgfqpoint{6.205000in}{2.048643in}}%
\pgfpathlineto{\pgfqpoint{6.570000in}{2.048643in}}%
\pgfpathlineto{\pgfqpoint{6.570000in}{2.059785in}}%
\pgfpathlineto{\pgfqpoint{6.205000in}{2.059785in}}%
\pgfpathlineto{\pgfqpoint{6.205000in}{2.048643in}}%
\pgfusepath{fill}%
\end{pgfscope}%
\begin{pgfscope}%
\pgfpathrectangle{\pgfqpoint{6.205000in}{0.611250in}}{\pgfqpoint{0.365000in}{2.852500in}} %
\pgfusepath{clip}%
\pgfsetbuttcap%
\pgfsetroundjoin%
\definecolor{currentfill}{rgb}{0.875557,0.860242,0.851430}%
\pgfsetfillcolor{currentfill}%
\pgfsetlinewidth{0.000000pt}%
\definecolor{currentstroke}{rgb}{0.000000,0.000000,0.000000}%
\pgfsetstrokecolor{currentstroke}%
\pgfsetdash{}{0pt}%
\pgfpathmoveto{\pgfqpoint{6.205000in}{2.059785in}}%
\pgfpathlineto{\pgfqpoint{6.570000in}{2.059785in}}%
\pgfpathlineto{\pgfqpoint{6.570000in}{2.070928in}}%
\pgfpathlineto{\pgfqpoint{6.205000in}{2.070928in}}%
\pgfpathlineto{\pgfqpoint{6.205000in}{2.059785in}}%
\pgfusepath{fill}%
\end{pgfscope}%
\begin{pgfscope}%
\pgfpathrectangle{\pgfqpoint{6.205000in}{0.611250in}}{\pgfqpoint{0.365000in}{2.852500in}} %
\pgfusepath{clip}%
\pgfsetbuttcap%
\pgfsetroundjoin%
\definecolor{currentfill}{rgb}{0.879622,0.858175,0.845844}%
\pgfsetfillcolor{currentfill}%
\pgfsetlinewidth{0.000000pt}%
\definecolor{currentstroke}{rgb}{0.000000,0.000000,0.000000}%
\pgfsetstrokecolor{currentstroke}%
\pgfsetdash{}{0pt}%
\pgfpathmoveto{\pgfqpoint{6.205000in}{2.070928in}}%
\pgfpathlineto{\pgfqpoint{6.570000in}{2.070928in}}%
\pgfpathlineto{\pgfqpoint{6.570000in}{2.082070in}}%
\pgfpathlineto{\pgfqpoint{6.205000in}{2.082070in}}%
\pgfpathlineto{\pgfqpoint{6.205000in}{2.070928in}}%
\pgfusepath{fill}%
\end{pgfscope}%
\begin{pgfscope}%
\pgfpathrectangle{\pgfqpoint{6.205000in}{0.611250in}}{\pgfqpoint{0.365000in}{2.852500in}} %
\pgfusepath{clip}%
\pgfsetbuttcap%
\pgfsetroundjoin%
\definecolor{currentfill}{rgb}{0.883687,0.856108,0.840258}%
\pgfsetfillcolor{currentfill}%
\pgfsetlinewidth{0.000000pt}%
\definecolor{currentstroke}{rgb}{0.000000,0.000000,0.000000}%
\pgfsetstrokecolor{currentstroke}%
\pgfsetdash{}{0pt}%
\pgfpathmoveto{\pgfqpoint{6.205000in}{2.082070in}}%
\pgfpathlineto{\pgfqpoint{6.570000in}{2.082070in}}%
\pgfpathlineto{\pgfqpoint{6.570000in}{2.093213in}}%
\pgfpathlineto{\pgfqpoint{6.205000in}{2.093213in}}%
\pgfpathlineto{\pgfqpoint{6.205000in}{2.082070in}}%
\pgfusepath{fill}%
\end{pgfscope}%
\begin{pgfscope}%
\pgfpathrectangle{\pgfqpoint{6.205000in}{0.611250in}}{\pgfqpoint{0.365000in}{2.852500in}} %
\pgfusepath{clip}%
\pgfsetbuttcap%
\pgfsetroundjoin%
\definecolor{currentfill}{rgb}{0.887752,0.854040,0.834671}%
\pgfsetfillcolor{currentfill}%
\pgfsetlinewidth{0.000000pt}%
\definecolor{currentstroke}{rgb}{0.000000,0.000000,0.000000}%
\pgfsetstrokecolor{currentstroke}%
\pgfsetdash{}{0pt}%
\pgfpathmoveto{\pgfqpoint{6.205000in}{2.093213in}}%
\pgfpathlineto{\pgfqpoint{6.570000in}{2.093213in}}%
\pgfpathlineto{\pgfqpoint{6.570000in}{2.104355in}}%
\pgfpathlineto{\pgfqpoint{6.205000in}{2.104355in}}%
\pgfpathlineto{\pgfqpoint{6.205000in}{2.093213in}}%
\pgfusepath{fill}%
\end{pgfscope}%
\begin{pgfscope}%
\pgfpathrectangle{\pgfqpoint{6.205000in}{0.611250in}}{\pgfqpoint{0.365000in}{2.852500in}} %
\pgfusepath{clip}%
\pgfsetbuttcap%
\pgfsetroundjoin%
\definecolor{currentfill}{rgb}{0.891817,0.851973,0.829085}%
\pgfsetfillcolor{currentfill}%
\pgfsetlinewidth{0.000000pt}%
\definecolor{currentstroke}{rgb}{0.000000,0.000000,0.000000}%
\pgfsetstrokecolor{currentstroke}%
\pgfsetdash{}{0pt}%
\pgfpathmoveto{\pgfqpoint{6.205000in}{2.104355in}}%
\pgfpathlineto{\pgfqpoint{6.570000in}{2.104355in}}%
\pgfpathlineto{\pgfqpoint{6.570000in}{2.115498in}}%
\pgfpathlineto{\pgfqpoint{6.205000in}{2.115498in}}%
\pgfpathlineto{\pgfqpoint{6.205000in}{2.104355in}}%
\pgfusepath{fill}%
\end{pgfscope}%
\begin{pgfscope}%
\pgfpathrectangle{\pgfqpoint{6.205000in}{0.611250in}}{\pgfqpoint{0.365000in}{2.852500in}} %
\pgfusepath{clip}%
\pgfsetbuttcap%
\pgfsetroundjoin%
\definecolor{currentfill}{rgb}{0.895882,0.849906,0.823499}%
\pgfsetfillcolor{currentfill}%
\pgfsetlinewidth{0.000000pt}%
\definecolor{currentstroke}{rgb}{0.000000,0.000000,0.000000}%
\pgfsetstrokecolor{currentstroke}%
\pgfsetdash{}{0pt}%
\pgfpathmoveto{\pgfqpoint{6.205000in}{2.115498in}}%
\pgfpathlineto{\pgfqpoint{6.570000in}{2.115498in}}%
\pgfpathlineto{\pgfqpoint{6.570000in}{2.126641in}}%
\pgfpathlineto{\pgfqpoint{6.205000in}{2.126641in}}%
\pgfpathlineto{\pgfqpoint{6.205000in}{2.115498in}}%
\pgfusepath{fill}%
\end{pgfscope}%
\begin{pgfscope}%
\pgfpathrectangle{\pgfqpoint{6.205000in}{0.611250in}}{\pgfqpoint{0.365000in}{2.852500in}} %
\pgfusepath{clip}%
\pgfsetbuttcap%
\pgfsetroundjoin%
\definecolor{currentfill}{rgb}{0.899543,0.847500,0.817789}%
\pgfsetfillcolor{currentfill}%
\pgfsetlinewidth{0.000000pt}%
\definecolor{currentstroke}{rgb}{0.000000,0.000000,0.000000}%
\pgfsetstrokecolor{currentstroke}%
\pgfsetdash{}{0pt}%
\pgfpathmoveto{\pgfqpoint{6.205000in}{2.126641in}}%
\pgfpathlineto{\pgfqpoint{6.570000in}{2.126641in}}%
\pgfpathlineto{\pgfqpoint{6.570000in}{2.137783in}}%
\pgfpathlineto{\pgfqpoint{6.205000in}{2.137783in}}%
\pgfpathlineto{\pgfqpoint{6.205000in}{2.126641in}}%
\pgfusepath{fill}%
\end{pgfscope}%
\begin{pgfscope}%
\pgfpathrectangle{\pgfqpoint{6.205000in}{0.611250in}}{\pgfqpoint{0.365000in}{2.852500in}} %
\pgfusepath{clip}%
\pgfsetbuttcap%
\pgfsetroundjoin%
\definecolor{currentfill}{rgb}{0.902849,0.844796,0.811970}%
\pgfsetfillcolor{currentfill}%
\pgfsetlinewidth{0.000000pt}%
\definecolor{currentstroke}{rgb}{0.000000,0.000000,0.000000}%
\pgfsetstrokecolor{currentstroke}%
\pgfsetdash{}{0pt}%
\pgfpathmoveto{\pgfqpoint{6.205000in}{2.137783in}}%
\pgfpathlineto{\pgfqpoint{6.570000in}{2.137783in}}%
\pgfpathlineto{\pgfqpoint{6.570000in}{2.148926in}}%
\pgfpathlineto{\pgfqpoint{6.205000in}{2.148926in}}%
\pgfpathlineto{\pgfqpoint{6.205000in}{2.137783in}}%
\pgfusepath{fill}%
\end{pgfscope}%
\begin{pgfscope}%
\pgfpathrectangle{\pgfqpoint{6.205000in}{0.611250in}}{\pgfqpoint{0.365000in}{2.852500in}} %
\pgfusepath{clip}%
\pgfsetbuttcap%
\pgfsetroundjoin%
\definecolor{currentfill}{rgb}{0.906154,0.842091,0.806151}%
\pgfsetfillcolor{currentfill}%
\pgfsetlinewidth{0.000000pt}%
\definecolor{currentstroke}{rgb}{0.000000,0.000000,0.000000}%
\pgfsetstrokecolor{currentstroke}%
\pgfsetdash{}{0pt}%
\pgfpathmoveto{\pgfqpoint{6.205000in}{2.148926in}}%
\pgfpathlineto{\pgfqpoint{6.570000in}{2.148926in}}%
\pgfpathlineto{\pgfqpoint{6.570000in}{2.160068in}}%
\pgfpathlineto{\pgfqpoint{6.205000in}{2.160068in}}%
\pgfpathlineto{\pgfqpoint{6.205000in}{2.148926in}}%
\pgfusepath{fill}%
\end{pgfscope}%
\begin{pgfscope}%
\pgfpathrectangle{\pgfqpoint{6.205000in}{0.611250in}}{\pgfqpoint{0.365000in}{2.852500in}} %
\pgfusepath{clip}%
\pgfsetbuttcap%
\pgfsetroundjoin%
\definecolor{currentfill}{rgb}{0.909460,0.839386,0.800331}%
\pgfsetfillcolor{currentfill}%
\pgfsetlinewidth{0.000000pt}%
\definecolor{currentstroke}{rgb}{0.000000,0.000000,0.000000}%
\pgfsetstrokecolor{currentstroke}%
\pgfsetdash{}{0pt}%
\pgfpathmoveto{\pgfqpoint{6.205000in}{2.160068in}}%
\pgfpathlineto{\pgfqpoint{6.570000in}{2.160068in}}%
\pgfpathlineto{\pgfqpoint{6.570000in}{2.171211in}}%
\pgfpathlineto{\pgfqpoint{6.205000in}{2.171211in}}%
\pgfpathlineto{\pgfqpoint{6.205000in}{2.160068in}}%
\pgfusepath{fill}%
\end{pgfscope}%
\begin{pgfscope}%
\pgfpathrectangle{\pgfqpoint{6.205000in}{0.611250in}}{\pgfqpoint{0.365000in}{2.852500in}} %
\pgfusepath{clip}%
\pgfsetbuttcap%
\pgfsetroundjoin%
\definecolor{currentfill}{rgb}{0.912765,0.836682,0.794512}%
\pgfsetfillcolor{currentfill}%
\pgfsetlinewidth{0.000000pt}%
\definecolor{currentstroke}{rgb}{0.000000,0.000000,0.000000}%
\pgfsetstrokecolor{currentstroke}%
\pgfsetdash{}{0pt}%
\pgfpathmoveto{\pgfqpoint{6.205000in}{2.171211in}}%
\pgfpathlineto{\pgfqpoint{6.570000in}{2.171211in}}%
\pgfpathlineto{\pgfqpoint{6.570000in}{2.182354in}}%
\pgfpathlineto{\pgfqpoint{6.205000in}{2.182354in}}%
\pgfpathlineto{\pgfqpoint{6.205000in}{2.171211in}}%
\pgfusepath{fill}%
\end{pgfscope}%
\begin{pgfscope}%
\pgfpathrectangle{\pgfqpoint{6.205000in}{0.611250in}}{\pgfqpoint{0.365000in}{2.852500in}} %
\pgfusepath{clip}%
\pgfsetbuttcap%
\pgfsetroundjoin%
\definecolor{currentfill}{rgb}{0.916071,0.833977,0.788693}%
\pgfsetfillcolor{currentfill}%
\pgfsetlinewidth{0.000000pt}%
\definecolor{currentstroke}{rgb}{0.000000,0.000000,0.000000}%
\pgfsetstrokecolor{currentstroke}%
\pgfsetdash{}{0pt}%
\pgfpathmoveto{\pgfqpoint{6.205000in}{2.182354in}}%
\pgfpathlineto{\pgfqpoint{6.570000in}{2.182354in}}%
\pgfpathlineto{\pgfqpoint{6.570000in}{2.193496in}}%
\pgfpathlineto{\pgfqpoint{6.205000in}{2.193496in}}%
\pgfpathlineto{\pgfqpoint{6.205000in}{2.182354in}}%
\pgfusepath{fill}%
\end{pgfscope}%
\begin{pgfscope}%
\pgfpathrectangle{\pgfqpoint{6.205000in}{0.611250in}}{\pgfqpoint{0.365000in}{2.852500in}} %
\pgfusepath{clip}%
\pgfsetbuttcap%
\pgfsetroundjoin%
\definecolor{currentfill}{rgb}{0.919376,0.831273,0.782874}%
\pgfsetfillcolor{currentfill}%
\pgfsetlinewidth{0.000000pt}%
\definecolor{currentstroke}{rgb}{0.000000,0.000000,0.000000}%
\pgfsetstrokecolor{currentstroke}%
\pgfsetdash{}{0pt}%
\pgfpathmoveto{\pgfqpoint{6.205000in}{2.193496in}}%
\pgfpathlineto{\pgfqpoint{6.570000in}{2.193496in}}%
\pgfpathlineto{\pgfqpoint{6.570000in}{2.204639in}}%
\pgfpathlineto{\pgfqpoint{6.205000in}{2.204639in}}%
\pgfpathlineto{\pgfqpoint{6.205000in}{2.193496in}}%
\pgfusepath{fill}%
\end{pgfscope}%
\begin{pgfscope}%
\pgfpathrectangle{\pgfqpoint{6.205000in}{0.611250in}}{\pgfqpoint{0.365000in}{2.852500in}} %
\pgfusepath{clip}%
\pgfsetbuttcap%
\pgfsetroundjoin%
\definecolor{currentfill}{rgb}{0.922681,0.828568,0.777054}%
\pgfsetfillcolor{currentfill}%
\pgfsetlinewidth{0.000000pt}%
\definecolor{currentstroke}{rgb}{0.000000,0.000000,0.000000}%
\pgfsetstrokecolor{currentstroke}%
\pgfsetdash{}{0pt}%
\pgfpathmoveto{\pgfqpoint{6.205000in}{2.204639in}}%
\pgfpathlineto{\pgfqpoint{6.570000in}{2.204639in}}%
\pgfpathlineto{\pgfqpoint{6.570000in}{2.215781in}}%
\pgfpathlineto{\pgfqpoint{6.205000in}{2.215781in}}%
\pgfpathlineto{\pgfqpoint{6.205000in}{2.204639in}}%
\pgfusepath{fill}%
\end{pgfscope}%
\begin{pgfscope}%
\pgfpathrectangle{\pgfqpoint{6.205000in}{0.611250in}}{\pgfqpoint{0.365000in}{2.852500in}} %
\pgfusepath{clip}%
\pgfsetbuttcap%
\pgfsetroundjoin%
\definecolor{currentfill}{rgb}{0.925563,0.825517,0.771136}%
\pgfsetfillcolor{currentfill}%
\pgfsetlinewidth{0.000000pt}%
\definecolor{currentstroke}{rgb}{0.000000,0.000000,0.000000}%
\pgfsetstrokecolor{currentstroke}%
\pgfsetdash{}{0pt}%
\pgfpathmoveto{\pgfqpoint{6.205000in}{2.215781in}}%
\pgfpathlineto{\pgfqpoint{6.570000in}{2.215781in}}%
\pgfpathlineto{\pgfqpoint{6.570000in}{2.226924in}}%
\pgfpathlineto{\pgfqpoint{6.205000in}{2.226924in}}%
\pgfpathlineto{\pgfqpoint{6.205000in}{2.215781in}}%
\pgfusepath{fill}%
\end{pgfscope}%
\begin{pgfscope}%
\pgfpathrectangle{\pgfqpoint{6.205000in}{0.611250in}}{\pgfqpoint{0.365000in}{2.852500in}} %
\pgfusepath{clip}%
\pgfsetbuttcap%
\pgfsetroundjoin%
\definecolor{currentfill}{rgb}{0.928116,0.822197,0.765141}%
\pgfsetfillcolor{currentfill}%
\pgfsetlinewidth{0.000000pt}%
\definecolor{currentstroke}{rgb}{0.000000,0.000000,0.000000}%
\pgfsetstrokecolor{currentstroke}%
\pgfsetdash{}{0pt}%
\pgfpathmoveto{\pgfqpoint{6.205000in}{2.226924in}}%
\pgfpathlineto{\pgfqpoint{6.570000in}{2.226924in}}%
\pgfpathlineto{\pgfqpoint{6.570000in}{2.238066in}}%
\pgfpathlineto{\pgfqpoint{6.205000in}{2.238066in}}%
\pgfpathlineto{\pgfqpoint{6.205000in}{2.226924in}}%
\pgfusepath{fill}%
\end{pgfscope}%
\begin{pgfscope}%
\pgfpathrectangle{\pgfqpoint{6.205000in}{0.611250in}}{\pgfqpoint{0.365000in}{2.852500in}} %
\pgfusepath{clip}%
\pgfsetbuttcap%
\pgfsetroundjoin%
\definecolor{currentfill}{rgb}{0.930669,0.818877,0.759146}%
\pgfsetfillcolor{currentfill}%
\pgfsetlinewidth{0.000000pt}%
\definecolor{currentstroke}{rgb}{0.000000,0.000000,0.000000}%
\pgfsetstrokecolor{currentstroke}%
\pgfsetdash{}{0pt}%
\pgfpathmoveto{\pgfqpoint{6.205000in}{2.238066in}}%
\pgfpathlineto{\pgfqpoint{6.570000in}{2.238066in}}%
\pgfpathlineto{\pgfqpoint{6.570000in}{2.249209in}}%
\pgfpathlineto{\pgfqpoint{6.205000in}{2.249209in}}%
\pgfpathlineto{\pgfqpoint{6.205000in}{2.238066in}}%
\pgfusepath{fill}%
\end{pgfscope}%
\begin{pgfscope}%
\pgfpathrectangle{\pgfqpoint{6.205000in}{0.611250in}}{\pgfqpoint{0.365000in}{2.852500in}} %
\pgfusepath{clip}%
\pgfsetbuttcap%
\pgfsetroundjoin%
\definecolor{currentfill}{rgb}{0.933221,0.815557,0.753151}%
\pgfsetfillcolor{currentfill}%
\pgfsetlinewidth{0.000000pt}%
\definecolor{currentstroke}{rgb}{0.000000,0.000000,0.000000}%
\pgfsetstrokecolor{currentstroke}%
\pgfsetdash{}{0pt}%
\pgfpathmoveto{\pgfqpoint{6.205000in}{2.249209in}}%
\pgfpathlineto{\pgfqpoint{6.570000in}{2.249209in}}%
\pgfpathlineto{\pgfqpoint{6.570000in}{2.260352in}}%
\pgfpathlineto{\pgfqpoint{6.205000in}{2.260352in}}%
\pgfpathlineto{\pgfqpoint{6.205000in}{2.249209in}}%
\pgfusepath{fill}%
\end{pgfscope}%
\begin{pgfscope}%
\pgfpathrectangle{\pgfqpoint{6.205000in}{0.611250in}}{\pgfqpoint{0.365000in}{2.852500in}} %
\pgfusepath{clip}%
\pgfsetbuttcap%
\pgfsetroundjoin%
\definecolor{currentfill}{rgb}{0.935774,0.812237,0.747156}%
\pgfsetfillcolor{currentfill}%
\pgfsetlinewidth{0.000000pt}%
\definecolor{currentstroke}{rgb}{0.000000,0.000000,0.000000}%
\pgfsetstrokecolor{currentstroke}%
\pgfsetdash{}{0pt}%
\pgfpathmoveto{\pgfqpoint{6.205000in}{2.260352in}}%
\pgfpathlineto{\pgfqpoint{6.570000in}{2.260352in}}%
\pgfpathlineto{\pgfqpoint{6.570000in}{2.271494in}}%
\pgfpathlineto{\pgfqpoint{6.205000in}{2.271494in}}%
\pgfpathlineto{\pgfqpoint{6.205000in}{2.260352in}}%
\pgfusepath{fill}%
\end{pgfscope}%
\begin{pgfscope}%
\pgfpathrectangle{\pgfqpoint{6.205000in}{0.611250in}}{\pgfqpoint{0.365000in}{2.852500in}} %
\pgfusepath{clip}%
\pgfsetbuttcap%
\pgfsetroundjoin%
\definecolor{currentfill}{rgb}{0.938326,0.808917,0.741162}%
\pgfsetfillcolor{currentfill}%
\pgfsetlinewidth{0.000000pt}%
\definecolor{currentstroke}{rgb}{0.000000,0.000000,0.000000}%
\pgfsetstrokecolor{currentstroke}%
\pgfsetdash{}{0pt}%
\pgfpathmoveto{\pgfqpoint{6.205000in}{2.271494in}}%
\pgfpathlineto{\pgfqpoint{6.570000in}{2.271494in}}%
\pgfpathlineto{\pgfqpoint{6.570000in}{2.282637in}}%
\pgfpathlineto{\pgfqpoint{6.205000in}{2.282637in}}%
\pgfpathlineto{\pgfqpoint{6.205000in}{2.271494in}}%
\pgfusepath{fill}%
\end{pgfscope}%
\begin{pgfscope}%
\pgfpathrectangle{\pgfqpoint{6.205000in}{0.611250in}}{\pgfqpoint{0.365000in}{2.852500in}} %
\pgfusepath{clip}%
\pgfsetbuttcap%
\pgfsetroundjoin%
\definecolor{currentfill}{rgb}{0.940879,0.805596,0.735167}%
\pgfsetfillcolor{currentfill}%
\pgfsetlinewidth{0.000000pt}%
\definecolor{currentstroke}{rgb}{0.000000,0.000000,0.000000}%
\pgfsetstrokecolor{currentstroke}%
\pgfsetdash{}{0pt}%
\pgfpathmoveto{\pgfqpoint{6.205000in}{2.282637in}}%
\pgfpathlineto{\pgfqpoint{6.570000in}{2.282637in}}%
\pgfpathlineto{\pgfqpoint{6.570000in}{2.293779in}}%
\pgfpathlineto{\pgfqpoint{6.205000in}{2.293779in}}%
\pgfpathlineto{\pgfqpoint{6.205000in}{2.282637in}}%
\pgfusepath{fill}%
\end{pgfscope}%
\begin{pgfscope}%
\pgfpathrectangle{\pgfqpoint{6.205000in}{0.611250in}}{\pgfqpoint{0.365000in}{2.852500in}} %
\pgfusepath{clip}%
\pgfsetbuttcap%
\pgfsetroundjoin%
\definecolor{currentfill}{rgb}{0.943432,0.802276,0.729172}%
\pgfsetfillcolor{currentfill}%
\pgfsetlinewidth{0.000000pt}%
\definecolor{currentstroke}{rgb}{0.000000,0.000000,0.000000}%
\pgfsetstrokecolor{currentstroke}%
\pgfsetdash{}{0pt}%
\pgfpathmoveto{\pgfqpoint{6.205000in}{2.293779in}}%
\pgfpathlineto{\pgfqpoint{6.570000in}{2.293779in}}%
\pgfpathlineto{\pgfqpoint{6.570000in}{2.304922in}}%
\pgfpathlineto{\pgfqpoint{6.205000in}{2.304922in}}%
\pgfpathlineto{\pgfqpoint{6.205000in}{2.293779in}}%
\pgfusepath{fill}%
\end{pgfscope}%
\begin{pgfscope}%
\pgfpathrectangle{\pgfqpoint{6.205000in}{0.611250in}}{\pgfqpoint{0.365000in}{2.852500in}} %
\pgfusepath{clip}%
\pgfsetbuttcap%
\pgfsetroundjoin%
\definecolor{currentfill}{rgb}{0.945540,0.798606,0.723105}%
\pgfsetfillcolor{currentfill}%
\pgfsetlinewidth{0.000000pt}%
\definecolor{currentstroke}{rgb}{0.000000,0.000000,0.000000}%
\pgfsetstrokecolor{currentstroke}%
\pgfsetdash{}{0pt}%
\pgfpathmoveto{\pgfqpoint{6.205000in}{2.304922in}}%
\pgfpathlineto{\pgfqpoint{6.570000in}{2.304922in}}%
\pgfpathlineto{\pgfqpoint{6.570000in}{2.316064in}}%
\pgfpathlineto{\pgfqpoint{6.205000in}{2.316064in}}%
\pgfpathlineto{\pgfqpoint{6.205000in}{2.304922in}}%
\pgfusepath{fill}%
\end{pgfscope}%
\begin{pgfscope}%
\pgfpathrectangle{\pgfqpoint{6.205000in}{0.611250in}}{\pgfqpoint{0.365000in}{2.852500in}} %
\pgfusepath{clip}%
\pgfsetbuttcap%
\pgfsetroundjoin%
\definecolor{currentfill}{rgb}{0.947345,0.794696,0.716991}%
\pgfsetfillcolor{currentfill}%
\pgfsetlinewidth{0.000000pt}%
\definecolor{currentstroke}{rgb}{0.000000,0.000000,0.000000}%
\pgfsetstrokecolor{currentstroke}%
\pgfsetdash{}{0pt}%
\pgfpathmoveto{\pgfqpoint{6.205000in}{2.316064in}}%
\pgfpathlineto{\pgfqpoint{6.570000in}{2.316064in}}%
\pgfpathlineto{\pgfqpoint{6.570000in}{2.327207in}}%
\pgfpathlineto{\pgfqpoint{6.205000in}{2.327207in}}%
\pgfpathlineto{\pgfqpoint{6.205000in}{2.316064in}}%
\pgfusepath{fill}%
\end{pgfscope}%
\begin{pgfscope}%
\pgfpathrectangle{\pgfqpoint{6.205000in}{0.611250in}}{\pgfqpoint{0.365000in}{2.852500in}} %
\pgfusepath{clip}%
\pgfsetbuttcap%
\pgfsetroundjoin%
\definecolor{currentfill}{rgb}{0.949151,0.790785,0.710876}%
\pgfsetfillcolor{currentfill}%
\pgfsetlinewidth{0.000000pt}%
\definecolor{currentstroke}{rgb}{0.000000,0.000000,0.000000}%
\pgfsetstrokecolor{currentstroke}%
\pgfsetdash{}{0pt}%
\pgfpathmoveto{\pgfqpoint{6.205000in}{2.327207in}}%
\pgfpathlineto{\pgfqpoint{6.570000in}{2.327207in}}%
\pgfpathlineto{\pgfqpoint{6.570000in}{2.338350in}}%
\pgfpathlineto{\pgfqpoint{6.205000in}{2.338350in}}%
\pgfpathlineto{\pgfqpoint{6.205000in}{2.327207in}}%
\pgfusepath{fill}%
\end{pgfscope}%
\begin{pgfscope}%
\pgfpathrectangle{\pgfqpoint{6.205000in}{0.611250in}}{\pgfqpoint{0.365000in}{2.852500in}} %
\pgfusepath{clip}%
\pgfsetbuttcap%
\pgfsetroundjoin%
\definecolor{currentfill}{rgb}{0.950956,0.786875,0.704761}%
\pgfsetfillcolor{currentfill}%
\pgfsetlinewidth{0.000000pt}%
\definecolor{currentstroke}{rgb}{0.000000,0.000000,0.000000}%
\pgfsetstrokecolor{currentstroke}%
\pgfsetdash{}{0pt}%
\pgfpathmoveto{\pgfqpoint{6.205000in}{2.338350in}}%
\pgfpathlineto{\pgfqpoint{6.570000in}{2.338350in}}%
\pgfpathlineto{\pgfqpoint{6.570000in}{2.349492in}}%
\pgfpathlineto{\pgfqpoint{6.205000in}{2.349492in}}%
\pgfpathlineto{\pgfqpoint{6.205000in}{2.338350in}}%
\pgfusepath{fill}%
\end{pgfscope}%
\begin{pgfscope}%
\pgfpathrectangle{\pgfqpoint{6.205000in}{0.611250in}}{\pgfqpoint{0.365000in}{2.852500in}} %
\pgfusepath{clip}%
\pgfsetbuttcap%
\pgfsetroundjoin%
\definecolor{currentfill}{rgb}{0.952761,0.782965,0.698646}%
\pgfsetfillcolor{currentfill}%
\pgfsetlinewidth{0.000000pt}%
\definecolor{currentstroke}{rgb}{0.000000,0.000000,0.000000}%
\pgfsetstrokecolor{currentstroke}%
\pgfsetdash{}{0pt}%
\pgfpathmoveto{\pgfqpoint{6.205000in}{2.349492in}}%
\pgfpathlineto{\pgfqpoint{6.570000in}{2.349492in}}%
\pgfpathlineto{\pgfqpoint{6.570000in}{2.360635in}}%
\pgfpathlineto{\pgfqpoint{6.205000in}{2.360635in}}%
\pgfpathlineto{\pgfqpoint{6.205000in}{2.349492in}}%
\pgfusepath{fill}%
\end{pgfscope}%
\begin{pgfscope}%
\pgfpathrectangle{\pgfqpoint{6.205000in}{0.611250in}}{\pgfqpoint{0.365000in}{2.852500in}} %
\pgfusepath{clip}%
\pgfsetbuttcap%
\pgfsetroundjoin%
\definecolor{currentfill}{rgb}{0.954566,0.779055,0.692531}%
\pgfsetfillcolor{currentfill}%
\pgfsetlinewidth{0.000000pt}%
\definecolor{currentstroke}{rgb}{0.000000,0.000000,0.000000}%
\pgfsetstrokecolor{currentstroke}%
\pgfsetdash{}{0pt}%
\pgfpathmoveto{\pgfqpoint{6.205000in}{2.360635in}}%
\pgfpathlineto{\pgfqpoint{6.570000in}{2.360635in}}%
\pgfpathlineto{\pgfqpoint{6.570000in}{2.371777in}}%
\pgfpathlineto{\pgfqpoint{6.205000in}{2.371777in}}%
\pgfpathlineto{\pgfqpoint{6.205000in}{2.360635in}}%
\pgfusepath{fill}%
\end{pgfscope}%
\begin{pgfscope}%
\pgfpathrectangle{\pgfqpoint{6.205000in}{0.611250in}}{\pgfqpoint{0.365000in}{2.852500in}} %
\pgfusepath{clip}%
\pgfsetbuttcap%
\pgfsetroundjoin%
\definecolor{currentfill}{rgb}{0.956371,0.775144,0.686416}%
\pgfsetfillcolor{currentfill}%
\pgfsetlinewidth{0.000000pt}%
\definecolor{currentstroke}{rgb}{0.000000,0.000000,0.000000}%
\pgfsetstrokecolor{currentstroke}%
\pgfsetdash{}{0pt}%
\pgfpathmoveto{\pgfqpoint{6.205000in}{2.371777in}}%
\pgfpathlineto{\pgfqpoint{6.570000in}{2.371777in}}%
\pgfpathlineto{\pgfqpoint{6.570000in}{2.382920in}}%
\pgfpathlineto{\pgfqpoint{6.205000in}{2.382920in}}%
\pgfpathlineto{\pgfqpoint{6.205000in}{2.371777in}}%
\pgfusepath{fill}%
\end{pgfscope}%
\begin{pgfscope}%
\pgfpathrectangle{\pgfqpoint{6.205000in}{0.611250in}}{\pgfqpoint{0.365000in}{2.852500in}} %
\pgfusepath{clip}%
\pgfsetbuttcap%
\pgfsetroundjoin%
\definecolor{currentfill}{rgb}{0.958176,0.771234,0.680301}%
\pgfsetfillcolor{currentfill}%
\pgfsetlinewidth{0.000000pt}%
\definecolor{currentstroke}{rgb}{0.000000,0.000000,0.000000}%
\pgfsetstrokecolor{currentstroke}%
\pgfsetdash{}{0pt}%
\pgfpathmoveto{\pgfqpoint{6.205000in}{2.382920in}}%
\pgfpathlineto{\pgfqpoint{6.570000in}{2.382920in}}%
\pgfpathlineto{\pgfqpoint{6.570000in}{2.394062in}}%
\pgfpathlineto{\pgfqpoint{6.205000in}{2.394062in}}%
\pgfpathlineto{\pgfqpoint{6.205000in}{2.382920in}}%
\pgfusepath{fill}%
\end{pgfscope}%
\begin{pgfscope}%
\pgfpathrectangle{\pgfqpoint{6.205000in}{0.611250in}}{\pgfqpoint{0.365000in}{2.852500in}} %
\pgfusepath{clip}%
\pgfsetbuttcap%
\pgfsetroundjoin%
\definecolor{currentfill}{rgb}{0.959518,0.766973,0.674145}%
\pgfsetfillcolor{currentfill}%
\pgfsetlinewidth{0.000000pt}%
\definecolor{currentstroke}{rgb}{0.000000,0.000000,0.000000}%
\pgfsetstrokecolor{currentstroke}%
\pgfsetdash{}{0pt}%
\pgfpathmoveto{\pgfqpoint{6.205000in}{2.394062in}}%
\pgfpathlineto{\pgfqpoint{6.570000in}{2.394062in}}%
\pgfpathlineto{\pgfqpoint{6.570000in}{2.405205in}}%
\pgfpathlineto{\pgfqpoint{6.205000in}{2.405205in}}%
\pgfpathlineto{\pgfqpoint{6.205000in}{2.394062in}}%
\pgfusepath{fill}%
\end{pgfscope}%
\begin{pgfscope}%
\pgfpathrectangle{\pgfqpoint{6.205000in}{0.611250in}}{\pgfqpoint{0.365000in}{2.852500in}} %
\pgfusepath{clip}%
\pgfsetbuttcap%
\pgfsetroundjoin%
\definecolor{currentfill}{rgb}{0.960581,0.762501,0.667964}%
\pgfsetfillcolor{currentfill}%
\pgfsetlinewidth{0.000000pt}%
\definecolor{currentstroke}{rgb}{0.000000,0.000000,0.000000}%
\pgfsetstrokecolor{currentstroke}%
\pgfsetdash{}{0pt}%
\pgfpathmoveto{\pgfqpoint{6.205000in}{2.405205in}}%
\pgfpathlineto{\pgfqpoint{6.570000in}{2.405205in}}%
\pgfpathlineto{\pgfqpoint{6.570000in}{2.416348in}}%
\pgfpathlineto{\pgfqpoint{6.205000in}{2.416348in}}%
\pgfpathlineto{\pgfqpoint{6.205000in}{2.405205in}}%
\pgfusepath{fill}%
\end{pgfscope}%
\begin{pgfscope}%
\pgfpathrectangle{\pgfqpoint{6.205000in}{0.611250in}}{\pgfqpoint{0.365000in}{2.852500in}} %
\pgfusepath{clip}%
\pgfsetbuttcap%
\pgfsetroundjoin%
\definecolor{currentfill}{rgb}{0.961645,0.758029,0.661782}%
\pgfsetfillcolor{currentfill}%
\pgfsetlinewidth{0.000000pt}%
\definecolor{currentstroke}{rgb}{0.000000,0.000000,0.000000}%
\pgfsetstrokecolor{currentstroke}%
\pgfsetdash{}{0pt}%
\pgfpathmoveto{\pgfqpoint{6.205000in}{2.416348in}}%
\pgfpathlineto{\pgfqpoint{6.570000in}{2.416348in}}%
\pgfpathlineto{\pgfqpoint{6.570000in}{2.427490in}}%
\pgfpathlineto{\pgfqpoint{6.205000in}{2.427490in}}%
\pgfpathlineto{\pgfqpoint{6.205000in}{2.416348in}}%
\pgfusepath{fill}%
\end{pgfscope}%
\begin{pgfscope}%
\pgfpathrectangle{\pgfqpoint{6.205000in}{0.611250in}}{\pgfqpoint{0.365000in}{2.852500in}} %
\pgfusepath{clip}%
\pgfsetbuttcap%
\pgfsetroundjoin%
\definecolor{currentfill}{rgb}{0.962708,0.753557,0.655601}%
\pgfsetfillcolor{currentfill}%
\pgfsetlinewidth{0.000000pt}%
\definecolor{currentstroke}{rgb}{0.000000,0.000000,0.000000}%
\pgfsetstrokecolor{currentstroke}%
\pgfsetdash{}{0pt}%
\pgfpathmoveto{\pgfqpoint{6.205000in}{2.427490in}}%
\pgfpathlineto{\pgfqpoint{6.570000in}{2.427490in}}%
\pgfpathlineto{\pgfqpoint{6.570000in}{2.438633in}}%
\pgfpathlineto{\pgfqpoint{6.205000in}{2.438633in}}%
\pgfpathlineto{\pgfqpoint{6.205000in}{2.427490in}}%
\pgfusepath{fill}%
\end{pgfscope}%
\begin{pgfscope}%
\pgfpathrectangle{\pgfqpoint{6.205000in}{0.611250in}}{\pgfqpoint{0.365000in}{2.852500in}} %
\pgfusepath{clip}%
\pgfsetbuttcap%
\pgfsetroundjoin%
\definecolor{currentfill}{rgb}{0.963772,0.749086,0.649420}%
\pgfsetfillcolor{currentfill}%
\pgfsetlinewidth{0.000000pt}%
\definecolor{currentstroke}{rgb}{0.000000,0.000000,0.000000}%
\pgfsetstrokecolor{currentstroke}%
\pgfsetdash{}{0pt}%
\pgfpathmoveto{\pgfqpoint{6.205000in}{2.438633in}}%
\pgfpathlineto{\pgfqpoint{6.570000in}{2.438633in}}%
\pgfpathlineto{\pgfqpoint{6.570000in}{2.449775in}}%
\pgfpathlineto{\pgfqpoint{6.205000in}{2.449775in}}%
\pgfpathlineto{\pgfqpoint{6.205000in}{2.438633in}}%
\pgfusepath{fill}%
\end{pgfscope}%
\begin{pgfscope}%
\pgfpathrectangle{\pgfqpoint{6.205000in}{0.611250in}}{\pgfqpoint{0.365000in}{2.852500in}} %
\pgfusepath{clip}%
\pgfsetbuttcap%
\pgfsetroundjoin%
\definecolor{currentfill}{rgb}{0.964835,0.744614,0.643239}%
\pgfsetfillcolor{currentfill}%
\pgfsetlinewidth{0.000000pt}%
\definecolor{currentstroke}{rgb}{0.000000,0.000000,0.000000}%
\pgfsetstrokecolor{currentstroke}%
\pgfsetdash{}{0pt}%
\pgfpathmoveto{\pgfqpoint{6.205000in}{2.449775in}}%
\pgfpathlineto{\pgfqpoint{6.570000in}{2.449775in}}%
\pgfpathlineto{\pgfqpoint{6.570000in}{2.460918in}}%
\pgfpathlineto{\pgfqpoint{6.205000in}{2.460918in}}%
\pgfpathlineto{\pgfqpoint{6.205000in}{2.449775in}}%
\pgfusepath{fill}%
\end{pgfscope}%
\begin{pgfscope}%
\pgfpathrectangle{\pgfqpoint{6.205000in}{0.611250in}}{\pgfqpoint{0.365000in}{2.852500in}} %
\pgfusepath{clip}%
\pgfsetbuttcap%
\pgfsetroundjoin%
\definecolor{currentfill}{rgb}{0.965899,0.740142,0.637058}%
\pgfsetfillcolor{currentfill}%
\pgfsetlinewidth{0.000000pt}%
\definecolor{currentstroke}{rgb}{0.000000,0.000000,0.000000}%
\pgfsetstrokecolor{currentstroke}%
\pgfsetdash{}{0pt}%
\pgfpathmoveto{\pgfqpoint{6.205000in}{2.460918in}}%
\pgfpathlineto{\pgfqpoint{6.570000in}{2.460918in}}%
\pgfpathlineto{\pgfqpoint{6.570000in}{2.472061in}}%
\pgfpathlineto{\pgfqpoint{6.205000in}{2.472061in}}%
\pgfpathlineto{\pgfqpoint{6.205000in}{2.460918in}}%
\pgfusepath{fill}%
\end{pgfscope}%
\begin{pgfscope}%
\pgfpathrectangle{\pgfqpoint{6.205000in}{0.611250in}}{\pgfqpoint{0.365000in}{2.852500in}} %
\pgfusepath{clip}%
\pgfsetbuttcap%
\pgfsetroundjoin%
\definecolor{currentfill}{rgb}{0.966962,0.735670,0.630877}%
\pgfsetfillcolor{currentfill}%
\pgfsetlinewidth{0.000000pt}%
\definecolor{currentstroke}{rgb}{0.000000,0.000000,0.000000}%
\pgfsetstrokecolor{currentstroke}%
\pgfsetdash{}{0pt}%
\pgfpathmoveto{\pgfqpoint{6.205000in}{2.472061in}}%
\pgfpathlineto{\pgfqpoint{6.570000in}{2.472061in}}%
\pgfpathlineto{\pgfqpoint{6.570000in}{2.483203in}}%
\pgfpathlineto{\pgfqpoint{6.205000in}{2.483203in}}%
\pgfpathlineto{\pgfqpoint{6.205000in}{2.472061in}}%
\pgfusepath{fill}%
\end{pgfscope}%
\begin{pgfscope}%
\pgfpathrectangle{\pgfqpoint{6.205000in}{0.611250in}}{\pgfqpoint{0.365000in}{2.852500in}} %
\pgfusepath{clip}%
\pgfsetbuttcap%
\pgfsetroundjoin%
\definecolor{currentfill}{rgb}{0.967544,0.730850,0.624685}%
\pgfsetfillcolor{currentfill}%
\pgfsetlinewidth{0.000000pt}%
\definecolor{currentstroke}{rgb}{0.000000,0.000000,0.000000}%
\pgfsetstrokecolor{currentstroke}%
\pgfsetdash{}{0pt}%
\pgfpathmoveto{\pgfqpoint{6.205000in}{2.483203in}}%
\pgfpathlineto{\pgfqpoint{6.570000in}{2.483203in}}%
\pgfpathlineto{\pgfqpoint{6.570000in}{2.494346in}}%
\pgfpathlineto{\pgfqpoint{6.205000in}{2.494346in}}%
\pgfpathlineto{\pgfqpoint{6.205000in}{2.483203in}}%
\pgfusepath{fill}%
\end{pgfscope}%
\begin{pgfscope}%
\pgfpathrectangle{\pgfqpoint{6.205000in}{0.611250in}}{\pgfqpoint{0.365000in}{2.852500in}} %
\pgfusepath{clip}%
\pgfsetbuttcap%
\pgfsetroundjoin%
\definecolor{currentfill}{rgb}{0.967874,0.725847,0.618489}%
\pgfsetfillcolor{currentfill}%
\pgfsetlinewidth{0.000000pt}%
\definecolor{currentstroke}{rgb}{0.000000,0.000000,0.000000}%
\pgfsetstrokecolor{currentstroke}%
\pgfsetdash{}{0pt}%
\pgfpathmoveto{\pgfqpoint{6.205000in}{2.494346in}}%
\pgfpathlineto{\pgfqpoint{6.570000in}{2.494346in}}%
\pgfpathlineto{\pgfqpoint{6.570000in}{2.505488in}}%
\pgfpathlineto{\pgfqpoint{6.205000in}{2.505488in}}%
\pgfpathlineto{\pgfqpoint{6.205000in}{2.494346in}}%
\pgfusepath{fill}%
\end{pgfscope}%
\begin{pgfscope}%
\pgfpathrectangle{\pgfqpoint{6.205000in}{0.611250in}}{\pgfqpoint{0.365000in}{2.852500in}} %
\pgfusepath{clip}%
\pgfsetbuttcap%
\pgfsetroundjoin%
\definecolor{currentfill}{rgb}{0.968203,0.720844,0.612293}%
\pgfsetfillcolor{currentfill}%
\pgfsetlinewidth{0.000000pt}%
\definecolor{currentstroke}{rgb}{0.000000,0.000000,0.000000}%
\pgfsetstrokecolor{currentstroke}%
\pgfsetdash{}{0pt}%
\pgfpathmoveto{\pgfqpoint{6.205000in}{2.505488in}}%
\pgfpathlineto{\pgfqpoint{6.570000in}{2.505488in}}%
\pgfpathlineto{\pgfqpoint{6.570000in}{2.516631in}}%
\pgfpathlineto{\pgfqpoint{6.205000in}{2.516631in}}%
\pgfpathlineto{\pgfqpoint{6.205000in}{2.505488in}}%
\pgfusepath{fill}%
\end{pgfscope}%
\begin{pgfscope}%
\pgfpathrectangle{\pgfqpoint{6.205000in}{0.611250in}}{\pgfqpoint{0.365000in}{2.852500in}} %
\pgfusepath{clip}%
\pgfsetbuttcap%
\pgfsetroundjoin%
\definecolor{currentfill}{rgb}{0.968533,0.715841,0.606097}%
\pgfsetfillcolor{currentfill}%
\pgfsetlinewidth{0.000000pt}%
\definecolor{currentstroke}{rgb}{0.000000,0.000000,0.000000}%
\pgfsetstrokecolor{currentstroke}%
\pgfsetdash{}{0pt}%
\pgfpathmoveto{\pgfqpoint{6.205000in}{2.516631in}}%
\pgfpathlineto{\pgfqpoint{6.570000in}{2.516631in}}%
\pgfpathlineto{\pgfqpoint{6.570000in}{2.527773in}}%
\pgfpathlineto{\pgfqpoint{6.205000in}{2.527773in}}%
\pgfpathlineto{\pgfqpoint{6.205000in}{2.516631in}}%
\pgfusepath{fill}%
\end{pgfscope}%
\begin{pgfscope}%
\pgfpathrectangle{\pgfqpoint{6.205000in}{0.611250in}}{\pgfqpoint{0.365000in}{2.852500in}} %
\pgfusepath{clip}%
\pgfsetbuttcap%
\pgfsetroundjoin%
\definecolor{currentfill}{rgb}{0.968863,0.710838,0.599901}%
\pgfsetfillcolor{currentfill}%
\pgfsetlinewidth{0.000000pt}%
\definecolor{currentstroke}{rgb}{0.000000,0.000000,0.000000}%
\pgfsetstrokecolor{currentstroke}%
\pgfsetdash{}{0pt}%
\pgfpathmoveto{\pgfqpoint{6.205000in}{2.527773in}}%
\pgfpathlineto{\pgfqpoint{6.570000in}{2.527773in}}%
\pgfpathlineto{\pgfqpoint{6.570000in}{2.538916in}}%
\pgfpathlineto{\pgfqpoint{6.205000in}{2.538916in}}%
\pgfpathlineto{\pgfqpoint{6.205000in}{2.527773in}}%
\pgfusepath{fill}%
\end{pgfscope}%
\begin{pgfscope}%
\pgfpathrectangle{\pgfqpoint{6.205000in}{0.611250in}}{\pgfqpoint{0.365000in}{2.852500in}} %
\pgfusepath{clip}%
\pgfsetbuttcap%
\pgfsetroundjoin%
\definecolor{currentfill}{rgb}{0.969192,0.705836,0.593704}%
\pgfsetfillcolor{currentfill}%
\pgfsetlinewidth{0.000000pt}%
\definecolor{currentstroke}{rgb}{0.000000,0.000000,0.000000}%
\pgfsetstrokecolor{currentstroke}%
\pgfsetdash{}{0pt}%
\pgfpathmoveto{\pgfqpoint{6.205000in}{2.538916in}}%
\pgfpathlineto{\pgfqpoint{6.570000in}{2.538916in}}%
\pgfpathlineto{\pgfqpoint{6.570000in}{2.550059in}}%
\pgfpathlineto{\pgfqpoint{6.205000in}{2.550059in}}%
\pgfpathlineto{\pgfqpoint{6.205000in}{2.538916in}}%
\pgfusepath{fill}%
\end{pgfscope}%
\begin{pgfscope}%
\pgfpathrectangle{\pgfqpoint{6.205000in}{0.611250in}}{\pgfqpoint{0.365000in}{2.852500in}} %
\pgfusepath{clip}%
\pgfsetbuttcap%
\pgfsetroundjoin%
\definecolor{currentfill}{rgb}{0.969522,0.700833,0.587508}%
\pgfsetfillcolor{currentfill}%
\pgfsetlinewidth{0.000000pt}%
\definecolor{currentstroke}{rgb}{0.000000,0.000000,0.000000}%
\pgfsetstrokecolor{currentstroke}%
\pgfsetdash{}{0pt}%
\pgfpathmoveto{\pgfqpoint{6.205000in}{2.550059in}}%
\pgfpathlineto{\pgfqpoint{6.570000in}{2.550059in}}%
\pgfpathlineto{\pgfqpoint{6.570000in}{2.561201in}}%
\pgfpathlineto{\pgfqpoint{6.205000in}{2.561201in}}%
\pgfpathlineto{\pgfqpoint{6.205000in}{2.550059in}}%
\pgfusepath{fill}%
\end{pgfscope}%
\begin{pgfscope}%
\pgfpathrectangle{\pgfqpoint{6.205000in}{0.611250in}}{\pgfqpoint{0.365000in}{2.852500in}} %
\pgfusepath{clip}%
\pgfsetbuttcap%
\pgfsetroundjoin%
\definecolor{currentfill}{rgb}{0.969851,0.695830,0.581312}%
\pgfsetfillcolor{currentfill}%
\pgfsetlinewidth{0.000000pt}%
\definecolor{currentstroke}{rgb}{0.000000,0.000000,0.000000}%
\pgfsetstrokecolor{currentstroke}%
\pgfsetdash{}{0pt}%
\pgfpathmoveto{\pgfqpoint{6.205000in}{2.561201in}}%
\pgfpathlineto{\pgfqpoint{6.570000in}{2.561201in}}%
\pgfpathlineto{\pgfqpoint{6.570000in}{2.572344in}}%
\pgfpathlineto{\pgfqpoint{6.205000in}{2.572344in}}%
\pgfpathlineto{\pgfqpoint{6.205000in}{2.561201in}}%
\pgfusepath{fill}%
\end{pgfscope}%
\begin{pgfscope}%
\pgfpathrectangle{\pgfqpoint{6.205000in}{0.611250in}}{\pgfqpoint{0.365000in}{2.852500in}} %
\pgfusepath{clip}%
\pgfsetbuttcap%
\pgfsetroundjoin%
\definecolor{currentfill}{rgb}{0.969683,0.690484,0.575138}%
\pgfsetfillcolor{currentfill}%
\pgfsetlinewidth{0.000000pt}%
\definecolor{currentstroke}{rgb}{0.000000,0.000000,0.000000}%
\pgfsetstrokecolor{currentstroke}%
\pgfsetdash{}{0pt}%
\pgfpathmoveto{\pgfqpoint{6.205000in}{2.572344in}}%
\pgfpathlineto{\pgfqpoint{6.570000in}{2.572344in}}%
\pgfpathlineto{\pgfqpoint{6.570000in}{2.583486in}}%
\pgfpathlineto{\pgfqpoint{6.205000in}{2.583486in}}%
\pgfpathlineto{\pgfqpoint{6.205000in}{2.572344in}}%
\pgfusepath{fill}%
\end{pgfscope}%
\begin{pgfscope}%
\pgfpathrectangle{\pgfqpoint{6.205000in}{0.611250in}}{\pgfqpoint{0.365000in}{2.852500in}} %
\pgfusepath{clip}%
\pgfsetbuttcap%
\pgfsetroundjoin%
\definecolor{currentfill}{rgb}{0.969289,0.684982,0.568975}%
\pgfsetfillcolor{currentfill}%
\pgfsetlinewidth{0.000000pt}%
\definecolor{currentstroke}{rgb}{0.000000,0.000000,0.000000}%
\pgfsetstrokecolor{currentstroke}%
\pgfsetdash{}{0pt}%
\pgfpathmoveto{\pgfqpoint{6.205000in}{2.583486in}}%
\pgfpathlineto{\pgfqpoint{6.570000in}{2.583486in}}%
\pgfpathlineto{\pgfqpoint{6.570000in}{2.594629in}}%
\pgfpathlineto{\pgfqpoint{6.205000in}{2.594629in}}%
\pgfpathlineto{\pgfqpoint{6.205000in}{2.583486in}}%
\pgfusepath{fill}%
\end{pgfscope}%
\begin{pgfscope}%
\pgfpathrectangle{\pgfqpoint{6.205000in}{0.611250in}}{\pgfqpoint{0.365000in}{2.852500in}} %
\pgfusepath{clip}%
\pgfsetbuttcap%
\pgfsetroundjoin%
\definecolor{currentfill}{rgb}{0.968894,0.679480,0.562812}%
\pgfsetfillcolor{currentfill}%
\pgfsetlinewidth{0.000000pt}%
\definecolor{currentstroke}{rgb}{0.000000,0.000000,0.000000}%
\pgfsetstrokecolor{currentstroke}%
\pgfsetdash{}{0pt}%
\pgfpathmoveto{\pgfqpoint{6.205000in}{2.594629in}}%
\pgfpathlineto{\pgfqpoint{6.570000in}{2.594629in}}%
\pgfpathlineto{\pgfqpoint{6.570000in}{2.605771in}}%
\pgfpathlineto{\pgfqpoint{6.205000in}{2.605771in}}%
\pgfpathlineto{\pgfqpoint{6.205000in}{2.594629in}}%
\pgfusepath{fill}%
\end{pgfscope}%
\begin{pgfscope}%
\pgfpathrectangle{\pgfqpoint{6.205000in}{0.611250in}}{\pgfqpoint{0.365000in}{2.852500in}} %
\pgfusepath{clip}%
\pgfsetbuttcap%
\pgfsetroundjoin%
\definecolor{currentfill}{rgb}{0.968500,0.673977,0.556649}%
\pgfsetfillcolor{currentfill}%
\pgfsetlinewidth{0.000000pt}%
\definecolor{currentstroke}{rgb}{0.000000,0.000000,0.000000}%
\pgfsetstrokecolor{currentstroke}%
\pgfsetdash{}{0pt}%
\pgfpathmoveto{\pgfqpoint{6.205000in}{2.605771in}}%
\pgfpathlineto{\pgfqpoint{6.570000in}{2.605771in}}%
\pgfpathlineto{\pgfqpoint{6.570000in}{2.616914in}}%
\pgfpathlineto{\pgfqpoint{6.205000in}{2.616914in}}%
\pgfpathlineto{\pgfqpoint{6.205000in}{2.605771in}}%
\pgfusepath{fill}%
\end{pgfscope}%
\begin{pgfscope}%
\pgfpathrectangle{\pgfqpoint{6.205000in}{0.611250in}}{\pgfqpoint{0.365000in}{2.852500in}} %
\pgfusepath{clip}%
\pgfsetbuttcap%
\pgfsetroundjoin%
\definecolor{currentfill}{rgb}{0.968105,0.668475,0.550486}%
\pgfsetfillcolor{currentfill}%
\pgfsetlinewidth{0.000000pt}%
\definecolor{currentstroke}{rgb}{0.000000,0.000000,0.000000}%
\pgfsetstrokecolor{currentstroke}%
\pgfsetdash{}{0pt}%
\pgfpathmoveto{\pgfqpoint{6.205000in}{2.616914in}}%
\pgfpathlineto{\pgfqpoint{6.570000in}{2.616914in}}%
\pgfpathlineto{\pgfqpoint{6.570000in}{2.628057in}}%
\pgfpathlineto{\pgfqpoint{6.205000in}{2.628057in}}%
\pgfpathlineto{\pgfqpoint{6.205000in}{2.616914in}}%
\pgfusepath{fill}%
\end{pgfscope}%
\begin{pgfscope}%
\pgfpathrectangle{\pgfqpoint{6.205000in}{0.611250in}}{\pgfqpoint{0.365000in}{2.852500in}} %
\pgfusepath{clip}%
\pgfsetbuttcap%
\pgfsetroundjoin%
\definecolor{currentfill}{rgb}{0.967711,0.662973,0.544323}%
\pgfsetfillcolor{currentfill}%
\pgfsetlinewidth{0.000000pt}%
\definecolor{currentstroke}{rgb}{0.000000,0.000000,0.000000}%
\pgfsetstrokecolor{currentstroke}%
\pgfsetdash{}{0pt}%
\pgfpathmoveto{\pgfqpoint{6.205000in}{2.628057in}}%
\pgfpathlineto{\pgfqpoint{6.570000in}{2.628057in}}%
\pgfpathlineto{\pgfqpoint{6.570000in}{2.639199in}}%
\pgfpathlineto{\pgfqpoint{6.205000in}{2.639199in}}%
\pgfpathlineto{\pgfqpoint{6.205000in}{2.628057in}}%
\pgfusepath{fill}%
\end{pgfscope}%
\begin{pgfscope}%
\pgfpathrectangle{\pgfqpoint{6.205000in}{0.611250in}}{\pgfqpoint{0.365000in}{2.852500in}} %
\pgfusepath{clip}%
\pgfsetbuttcap%
\pgfsetroundjoin%
\definecolor{currentfill}{rgb}{0.967317,0.657471,0.538160}%
\pgfsetfillcolor{currentfill}%
\pgfsetlinewidth{0.000000pt}%
\definecolor{currentstroke}{rgb}{0.000000,0.000000,0.000000}%
\pgfsetstrokecolor{currentstroke}%
\pgfsetdash{}{0pt}%
\pgfpathmoveto{\pgfqpoint{6.205000in}{2.639199in}}%
\pgfpathlineto{\pgfqpoint{6.570000in}{2.639199in}}%
\pgfpathlineto{\pgfqpoint{6.570000in}{2.650342in}}%
\pgfpathlineto{\pgfqpoint{6.205000in}{2.650342in}}%
\pgfpathlineto{\pgfqpoint{6.205000in}{2.639199in}}%
\pgfusepath{fill}%
\end{pgfscope}%
\begin{pgfscope}%
\pgfpathrectangle{\pgfqpoint{6.205000in}{0.611250in}}{\pgfqpoint{0.365000in}{2.852500in}} %
\pgfusepath{clip}%
\pgfsetbuttcap%
\pgfsetroundjoin%
\definecolor{currentfill}{rgb}{0.966922,0.651969,0.531997}%
\pgfsetfillcolor{currentfill}%
\pgfsetlinewidth{0.000000pt}%
\definecolor{currentstroke}{rgb}{0.000000,0.000000,0.000000}%
\pgfsetstrokecolor{currentstroke}%
\pgfsetdash{}{0pt}%
\pgfpathmoveto{\pgfqpoint{6.205000in}{2.650342in}}%
\pgfpathlineto{\pgfqpoint{6.570000in}{2.650342in}}%
\pgfpathlineto{\pgfqpoint{6.570000in}{2.661484in}}%
\pgfpathlineto{\pgfqpoint{6.205000in}{2.661484in}}%
\pgfpathlineto{\pgfqpoint{6.205000in}{2.650342in}}%
\pgfusepath{fill}%
\end{pgfscope}%
\begin{pgfscope}%
\pgfpathrectangle{\pgfqpoint{6.205000in}{0.611250in}}{\pgfqpoint{0.365000in}{2.852500in}} %
\pgfusepath{clip}%
\pgfsetbuttcap%
\pgfsetroundjoin%
\definecolor{currentfill}{rgb}{0.966017,0.646130,0.525890}%
\pgfsetfillcolor{currentfill}%
\pgfsetlinewidth{0.000000pt}%
\definecolor{currentstroke}{rgb}{0.000000,0.000000,0.000000}%
\pgfsetstrokecolor{currentstroke}%
\pgfsetdash{}{0pt}%
\pgfpathmoveto{\pgfqpoint{6.205000in}{2.661484in}}%
\pgfpathlineto{\pgfqpoint{6.570000in}{2.661484in}}%
\pgfpathlineto{\pgfqpoint{6.570000in}{2.672627in}}%
\pgfpathlineto{\pgfqpoint{6.205000in}{2.672627in}}%
\pgfpathlineto{\pgfqpoint{6.205000in}{2.661484in}}%
\pgfusepath{fill}%
\end{pgfscope}%
\begin{pgfscope}%
\pgfpathrectangle{\pgfqpoint{6.205000in}{0.611250in}}{\pgfqpoint{0.365000in}{2.852500in}} %
\pgfusepath{clip}%
\pgfsetbuttcap%
\pgfsetroundjoin%
\definecolor{currentfill}{rgb}{0.964911,0.640159,0.519806}%
\pgfsetfillcolor{currentfill}%
\pgfsetlinewidth{0.000000pt}%
\definecolor{currentstroke}{rgb}{0.000000,0.000000,0.000000}%
\pgfsetstrokecolor{currentstroke}%
\pgfsetdash{}{0pt}%
\pgfpathmoveto{\pgfqpoint{6.205000in}{2.672627in}}%
\pgfpathlineto{\pgfqpoint{6.570000in}{2.672627in}}%
\pgfpathlineto{\pgfqpoint{6.570000in}{2.683770in}}%
\pgfpathlineto{\pgfqpoint{6.205000in}{2.683770in}}%
\pgfpathlineto{\pgfqpoint{6.205000in}{2.672627in}}%
\pgfusepath{fill}%
\end{pgfscope}%
\begin{pgfscope}%
\pgfpathrectangle{\pgfqpoint{6.205000in}{0.611250in}}{\pgfqpoint{0.365000in}{2.852500in}} %
\pgfusepath{clip}%
\pgfsetbuttcap%
\pgfsetroundjoin%
\definecolor{currentfill}{rgb}{0.963806,0.634188,0.513721}%
\pgfsetfillcolor{currentfill}%
\pgfsetlinewidth{0.000000pt}%
\definecolor{currentstroke}{rgb}{0.000000,0.000000,0.000000}%
\pgfsetstrokecolor{currentstroke}%
\pgfsetdash{}{0pt}%
\pgfpathmoveto{\pgfqpoint{6.205000in}{2.683770in}}%
\pgfpathlineto{\pgfqpoint{6.570000in}{2.683770in}}%
\pgfpathlineto{\pgfqpoint{6.570000in}{2.694912in}}%
\pgfpathlineto{\pgfqpoint{6.205000in}{2.694912in}}%
\pgfpathlineto{\pgfqpoint{6.205000in}{2.683770in}}%
\pgfusepath{fill}%
\end{pgfscope}%
\begin{pgfscope}%
\pgfpathrectangle{\pgfqpoint{6.205000in}{0.611250in}}{\pgfqpoint{0.365000in}{2.852500in}} %
\pgfusepath{clip}%
\pgfsetbuttcap%
\pgfsetroundjoin%
\definecolor{currentfill}{rgb}{0.962701,0.628218,0.507636}%
\pgfsetfillcolor{currentfill}%
\pgfsetlinewidth{0.000000pt}%
\definecolor{currentstroke}{rgb}{0.000000,0.000000,0.000000}%
\pgfsetstrokecolor{currentstroke}%
\pgfsetdash{}{0pt}%
\pgfpathmoveto{\pgfqpoint{6.205000in}{2.694912in}}%
\pgfpathlineto{\pgfqpoint{6.570000in}{2.694912in}}%
\pgfpathlineto{\pgfqpoint{6.570000in}{2.706055in}}%
\pgfpathlineto{\pgfqpoint{6.205000in}{2.706055in}}%
\pgfpathlineto{\pgfqpoint{6.205000in}{2.694912in}}%
\pgfusepath{fill}%
\end{pgfscope}%
\begin{pgfscope}%
\pgfpathrectangle{\pgfqpoint{6.205000in}{0.611250in}}{\pgfqpoint{0.365000in}{2.852500in}} %
\pgfusepath{clip}%
\pgfsetbuttcap%
\pgfsetroundjoin%
\definecolor{currentfill}{rgb}{0.961595,0.622247,0.501551}%
\pgfsetfillcolor{currentfill}%
\pgfsetlinewidth{0.000000pt}%
\definecolor{currentstroke}{rgb}{0.000000,0.000000,0.000000}%
\pgfsetstrokecolor{currentstroke}%
\pgfsetdash{}{0pt}%
\pgfpathmoveto{\pgfqpoint{6.205000in}{2.706055in}}%
\pgfpathlineto{\pgfqpoint{6.570000in}{2.706055in}}%
\pgfpathlineto{\pgfqpoint{6.570000in}{2.717197in}}%
\pgfpathlineto{\pgfqpoint{6.205000in}{2.717197in}}%
\pgfpathlineto{\pgfqpoint{6.205000in}{2.706055in}}%
\pgfusepath{fill}%
\end{pgfscope}%
\begin{pgfscope}%
\pgfpathrectangle{\pgfqpoint{6.205000in}{0.611250in}}{\pgfqpoint{0.365000in}{2.852500in}} %
\pgfusepath{clip}%
\pgfsetbuttcap%
\pgfsetroundjoin%
\definecolor{currentfill}{rgb}{0.960490,0.616276,0.495467}%
\pgfsetfillcolor{currentfill}%
\pgfsetlinewidth{0.000000pt}%
\definecolor{currentstroke}{rgb}{0.000000,0.000000,0.000000}%
\pgfsetstrokecolor{currentstroke}%
\pgfsetdash{}{0pt}%
\pgfpathmoveto{\pgfqpoint{6.205000in}{2.717197in}}%
\pgfpathlineto{\pgfqpoint{6.570000in}{2.717197in}}%
\pgfpathlineto{\pgfqpoint{6.570000in}{2.728340in}}%
\pgfpathlineto{\pgfqpoint{6.205000in}{2.728340in}}%
\pgfpathlineto{\pgfqpoint{6.205000in}{2.717197in}}%
\pgfusepath{fill}%
\end{pgfscope}%
\begin{pgfscope}%
\pgfpathrectangle{\pgfqpoint{6.205000in}{0.611250in}}{\pgfqpoint{0.365000in}{2.852500in}} %
\pgfusepath{clip}%
\pgfsetbuttcap%
\pgfsetroundjoin%
\definecolor{currentfill}{rgb}{0.959385,0.610306,0.489382}%
\pgfsetfillcolor{currentfill}%
\pgfsetlinewidth{0.000000pt}%
\definecolor{currentstroke}{rgb}{0.000000,0.000000,0.000000}%
\pgfsetstrokecolor{currentstroke}%
\pgfsetdash{}{0pt}%
\pgfpathmoveto{\pgfqpoint{6.205000in}{2.728340in}}%
\pgfpathlineto{\pgfqpoint{6.570000in}{2.728340in}}%
\pgfpathlineto{\pgfqpoint{6.570000in}{2.739482in}}%
\pgfpathlineto{\pgfqpoint{6.205000in}{2.739482in}}%
\pgfpathlineto{\pgfqpoint{6.205000in}{2.728340in}}%
\pgfusepath{fill}%
\end{pgfscope}%
\begin{pgfscope}%
\pgfpathrectangle{\pgfqpoint{6.205000in}{0.611250in}}{\pgfqpoint{0.365000in}{2.852500in}} %
\pgfusepath{clip}%
\pgfsetbuttcap%
\pgfsetroundjoin%
\definecolor{currentfill}{rgb}{0.958279,0.604335,0.483297}%
\pgfsetfillcolor{currentfill}%
\pgfsetlinewidth{0.000000pt}%
\definecolor{currentstroke}{rgb}{0.000000,0.000000,0.000000}%
\pgfsetstrokecolor{currentstroke}%
\pgfsetdash{}{0pt}%
\pgfpathmoveto{\pgfqpoint{6.205000in}{2.739482in}}%
\pgfpathlineto{\pgfqpoint{6.570000in}{2.739482in}}%
\pgfpathlineto{\pgfqpoint{6.570000in}{2.750625in}}%
\pgfpathlineto{\pgfqpoint{6.205000in}{2.750625in}}%
\pgfpathlineto{\pgfqpoint{6.205000in}{2.739482in}}%
\pgfusepath{fill}%
\end{pgfscope}%
\begin{pgfscope}%
\pgfpathrectangle{\pgfqpoint{6.205000in}{0.611250in}}{\pgfqpoint{0.365000in}{2.852500in}} %
\pgfusepath{clip}%
\pgfsetbuttcap%
\pgfsetroundjoin%
\definecolor{currentfill}{rgb}{0.956653,0.598034,0.477302}%
\pgfsetfillcolor{currentfill}%
\pgfsetlinewidth{0.000000pt}%
\definecolor{currentstroke}{rgb}{0.000000,0.000000,0.000000}%
\pgfsetstrokecolor{currentstroke}%
\pgfsetdash{}{0pt}%
\pgfpathmoveto{\pgfqpoint{6.205000in}{2.750625in}}%
\pgfpathlineto{\pgfqpoint{6.570000in}{2.750625in}}%
\pgfpathlineto{\pgfqpoint{6.570000in}{2.761768in}}%
\pgfpathlineto{\pgfqpoint{6.205000in}{2.761768in}}%
\pgfpathlineto{\pgfqpoint{6.205000in}{2.750625in}}%
\pgfusepath{fill}%
\end{pgfscope}%
\begin{pgfscope}%
\pgfpathrectangle{\pgfqpoint{6.205000in}{0.611250in}}{\pgfqpoint{0.365000in}{2.852500in}} %
\pgfusepath{clip}%
\pgfsetbuttcap%
\pgfsetroundjoin%
\definecolor{currentfill}{rgb}{0.954853,0.591622,0.471337}%
\pgfsetfillcolor{currentfill}%
\pgfsetlinewidth{0.000000pt}%
\definecolor{currentstroke}{rgb}{0.000000,0.000000,0.000000}%
\pgfsetstrokecolor{currentstroke}%
\pgfsetdash{}{0pt}%
\pgfpathmoveto{\pgfqpoint{6.205000in}{2.761768in}}%
\pgfpathlineto{\pgfqpoint{6.570000in}{2.761768in}}%
\pgfpathlineto{\pgfqpoint{6.570000in}{2.772910in}}%
\pgfpathlineto{\pgfqpoint{6.205000in}{2.772910in}}%
\pgfpathlineto{\pgfqpoint{6.205000in}{2.761768in}}%
\pgfusepath{fill}%
\end{pgfscope}%
\begin{pgfscope}%
\pgfpathrectangle{\pgfqpoint{6.205000in}{0.611250in}}{\pgfqpoint{0.365000in}{2.852500in}} %
\pgfusepath{clip}%
\pgfsetbuttcap%
\pgfsetroundjoin%
\definecolor{currentfill}{rgb}{0.953054,0.585211,0.465373}%
\pgfsetfillcolor{currentfill}%
\pgfsetlinewidth{0.000000pt}%
\definecolor{currentstroke}{rgb}{0.000000,0.000000,0.000000}%
\pgfsetstrokecolor{currentstroke}%
\pgfsetdash{}{0pt}%
\pgfpathmoveto{\pgfqpoint{6.205000in}{2.772910in}}%
\pgfpathlineto{\pgfqpoint{6.570000in}{2.772910in}}%
\pgfpathlineto{\pgfqpoint{6.570000in}{2.784053in}}%
\pgfpathlineto{\pgfqpoint{6.205000in}{2.784053in}}%
\pgfpathlineto{\pgfqpoint{6.205000in}{2.772910in}}%
\pgfusepath{fill}%
\end{pgfscope}%
\begin{pgfscope}%
\pgfpathrectangle{\pgfqpoint{6.205000in}{0.611250in}}{\pgfqpoint{0.365000in}{2.852500in}} %
\pgfusepath{clip}%
\pgfsetbuttcap%
\pgfsetroundjoin%
\definecolor{currentfill}{rgb}{0.951254,0.578799,0.459408}%
\pgfsetfillcolor{currentfill}%
\pgfsetlinewidth{0.000000pt}%
\definecolor{currentstroke}{rgb}{0.000000,0.000000,0.000000}%
\pgfsetstrokecolor{currentstroke}%
\pgfsetdash{}{0pt}%
\pgfpathmoveto{\pgfqpoint{6.205000in}{2.784053in}}%
\pgfpathlineto{\pgfqpoint{6.570000in}{2.784053in}}%
\pgfpathlineto{\pgfqpoint{6.570000in}{2.795195in}}%
\pgfpathlineto{\pgfqpoint{6.205000in}{2.795195in}}%
\pgfpathlineto{\pgfqpoint{6.205000in}{2.784053in}}%
\pgfusepath{fill}%
\end{pgfscope}%
\begin{pgfscope}%
\pgfpathrectangle{\pgfqpoint{6.205000in}{0.611250in}}{\pgfqpoint{0.365000in}{2.852500in}} %
\pgfusepath{clip}%
\pgfsetbuttcap%
\pgfsetroundjoin%
\definecolor{currentfill}{rgb}{0.949454,0.572388,0.453443}%
\pgfsetfillcolor{currentfill}%
\pgfsetlinewidth{0.000000pt}%
\definecolor{currentstroke}{rgb}{0.000000,0.000000,0.000000}%
\pgfsetstrokecolor{currentstroke}%
\pgfsetdash{}{0pt}%
\pgfpathmoveto{\pgfqpoint{6.205000in}{2.795195in}}%
\pgfpathlineto{\pgfqpoint{6.570000in}{2.795195in}}%
\pgfpathlineto{\pgfqpoint{6.570000in}{2.806338in}}%
\pgfpathlineto{\pgfqpoint{6.205000in}{2.806338in}}%
\pgfpathlineto{\pgfqpoint{6.205000in}{2.795195in}}%
\pgfusepath{fill}%
\end{pgfscope}%
\begin{pgfscope}%
\pgfpathrectangle{\pgfqpoint{6.205000in}{0.611250in}}{\pgfqpoint{0.365000in}{2.852500in}} %
\pgfusepath{clip}%
\pgfsetbuttcap%
\pgfsetroundjoin%
\definecolor{currentfill}{rgb}{0.947654,0.565976,0.447478}%
\pgfsetfillcolor{currentfill}%
\pgfsetlinewidth{0.000000pt}%
\definecolor{currentstroke}{rgb}{0.000000,0.000000,0.000000}%
\pgfsetstrokecolor{currentstroke}%
\pgfsetdash{}{0pt}%
\pgfpathmoveto{\pgfqpoint{6.205000in}{2.806338in}}%
\pgfpathlineto{\pgfqpoint{6.570000in}{2.806338in}}%
\pgfpathlineto{\pgfqpoint{6.570000in}{2.817480in}}%
\pgfpathlineto{\pgfqpoint{6.205000in}{2.817480in}}%
\pgfpathlineto{\pgfqpoint{6.205000in}{2.806338in}}%
\pgfusepath{fill}%
\end{pgfscope}%
\begin{pgfscope}%
\pgfpathrectangle{\pgfqpoint{6.205000in}{0.611250in}}{\pgfqpoint{0.365000in}{2.852500in}} %
\pgfusepath{clip}%
\pgfsetbuttcap%
\pgfsetroundjoin%
\definecolor{currentfill}{rgb}{0.945854,0.559565,0.441513}%
\pgfsetfillcolor{currentfill}%
\pgfsetlinewidth{0.000000pt}%
\definecolor{currentstroke}{rgb}{0.000000,0.000000,0.000000}%
\pgfsetstrokecolor{currentstroke}%
\pgfsetdash{}{0pt}%
\pgfpathmoveto{\pgfqpoint{6.205000in}{2.817480in}}%
\pgfpathlineto{\pgfqpoint{6.570000in}{2.817480in}}%
\pgfpathlineto{\pgfqpoint{6.570000in}{2.828623in}}%
\pgfpathlineto{\pgfqpoint{6.205000in}{2.828623in}}%
\pgfpathlineto{\pgfqpoint{6.205000in}{2.817480in}}%
\pgfusepath{fill}%
\end{pgfscope}%
\begin{pgfscope}%
\pgfpathrectangle{\pgfqpoint{6.205000in}{0.611250in}}{\pgfqpoint{0.365000in}{2.852500in}} %
\pgfusepath{clip}%
\pgfsetbuttcap%
\pgfsetroundjoin%
\definecolor{currentfill}{rgb}{0.944055,0.553153,0.435548}%
\pgfsetfillcolor{currentfill}%
\pgfsetlinewidth{0.000000pt}%
\definecolor{currentstroke}{rgb}{0.000000,0.000000,0.000000}%
\pgfsetstrokecolor{currentstroke}%
\pgfsetdash{}{0pt}%
\pgfpathmoveto{\pgfqpoint{6.205000in}{2.828623in}}%
\pgfpathlineto{\pgfqpoint{6.570000in}{2.828623in}}%
\pgfpathlineto{\pgfqpoint{6.570000in}{2.839766in}}%
\pgfpathlineto{\pgfqpoint{6.205000in}{2.839766in}}%
\pgfpathlineto{\pgfqpoint{6.205000in}{2.828623in}}%
\pgfusepath{fill}%
\end{pgfscope}%
\begin{pgfscope}%
\pgfpathrectangle{\pgfqpoint{6.205000in}{0.611250in}}{\pgfqpoint{0.365000in}{2.852500in}} %
\pgfusepath{clip}%
\pgfsetbuttcap%
\pgfsetroundjoin%
\definecolor{currentfill}{rgb}{0.941728,0.546413,0.429707}%
\pgfsetfillcolor{currentfill}%
\pgfsetlinewidth{0.000000pt}%
\definecolor{currentstroke}{rgb}{0.000000,0.000000,0.000000}%
\pgfsetstrokecolor{currentstroke}%
\pgfsetdash{}{0pt}%
\pgfpathmoveto{\pgfqpoint{6.205000in}{2.839766in}}%
\pgfpathlineto{\pgfqpoint{6.570000in}{2.839766in}}%
\pgfpathlineto{\pgfqpoint{6.570000in}{2.850908in}}%
\pgfpathlineto{\pgfqpoint{6.205000in}{2.850908in}}%
\pgfpathlineto{\pgfqpoint{6.205000in}{2.839766in}}%
\pgfusepath{fill}%
\end{pgfscope}%
\begin{pgfscope}%
\pgfpathrectangle{\pgfqpoint{6.205000in}{0.611250in}}{\pgfqpoint{0.365000in}{2.852500in}} %
\pgfusepath{clip}%
\pgfsetbuttcap%
\pgfsetroundjoin%
\definecolor{currentfill}{rgb}{0.939254,0.539581,0.423900}%
\pgfsetfillcolor{currentfill}%
\pgfsetlinewidth{0.000000pt}%
\definecolor{currentstroke}{rgb}{0.000000,0.000000,0.000000}%
\pgfsetstrokecolor{currentstroke}%
\pgfsetdash{}{0pt}%
\pgfpathmoveto{\pgfqpoint{6.205000in}{2.850908in}}%
\pgfpathlineto{\pgfqpoint{6.570000in}{2.850908in}}%
\pgfpathlineto{\pgfqpoint{6.570000in}{2.862051in}}%
\pgfpathlineto{\pgfqpoint{6.205000in}{2.862051in}}%
\pgfpathlineto{\pgfqpoint{6.205000in}{2.850908in}}%
\pgfusepath{fill}%
\end{pgfscope}%
\begin{pgfscope}%
\pgfpathrectangle{\pgfqpoint{6.205000in}{0.611250in}}{\pgfqpoint{0.365000in}{2.852500in}} %
\pgfusepath{clip}%
\pgfsetbuttcap%
\pgfsetroundjoin%
\definecolor{currentfill}{rgb}{0.936780,0.532750,0.418093}%
\pgfsetfillcolor{currentfill}%
\pgfsetlinewidth{0.000000pt}%
\definecolor{currentstroke}{rgb}{0.000000,0.000000,0.000000}%
\pgfsetstrokecolor{currentstroke}%
\pgfsetdash{}{0pt}%
\pgfpathmoveto{\pgfqpoint{6.205000in}{2.862051in}}%
\pgfpathlineto{\pgfqpoint{6.570000in}{2.862051in}}%
\pgfpathlineto{\pgfqpoint{6.570000in}{2.873193in}}%
\pgfpathlineto{\pgfqpoint{6.205000in}{2.873193in}}%
\pgfpathlineto{\pgfqpoint{6.205000in}{2.862051in}}%
\pgfusepath{fill}%
\end{pgfscope}%
\begin{pgfscope}%
\pgfpathrectangle{\pgfqpoint{6.205000in}{0.611250in}}{\pgfqpoint{0.365000in}{2.852500in}} %
\pgfusepath{clip}%
\pgfsetbuttcap%
\pgfsetroundjoin%
\definecolor{currentfill}{rgb}{0.934305,0.525918,0.412286}%
\pgfsetfillcolor{currentfill}%
\pgfsetlinewidth{0.000000pt}%
\definecolor{currentstroke}{rgb}{0.000000,0.000000,0.000000}%
\pgfsetstrokecolor{currentstroke}%
\pgfsetdash{}{0pt}%
\pgfpathmoveto{\pgfqpoint{6.205000in}{2.873193in}}%
\pgfpathlineto{\pgfqpoint{6.570000in}{2.873193in}}%
\pgfpathlineto{\pgfqpoint{6.570000in}{2.884336in}}%
\pgfpathlineto{\pgfqpoint{6.205000in}{2.884336in}}%
\pgfpathlineto{\pgfqpoint{6.205000in}{2.873193in}}%
\pgfusepath{fill}%
\end{pgfscope}%
\begin{pgfscope}%
\pgfpathrectangle{\pgfqpoint{6.205000in}{0.611250in}}{\pgfqpoint{0.365000in}{2.852500in}} %
\pgfusepath{clip}%
\pgfsetbuttcap%
\pgfsetroundjoin%
\definecolor{currentfill}{rgb}{0.931831,0.519086,0.406480}%
\pgfsetfillcolor{currentfill}%
\pgfsetlinewidth{0.000000pt}%
\definecolor{currentstroke}{rgb}{0.000000,0.000000,0.000000}%
\pgfsetstrokecolor{currentstroke}%
\pgfsetdash{}{0pt}%
\pgfpathmoveto{\pgfqpoint{6.205000in}{2.884336in}}%
\pgfpathlineto{\pgfqpoint{6.570000in}{2.884336in}}%
\pgfpathlineto{\pgfqpoint{6.570000in}{2.895479in}}%
\pgfpathlineto{\pgfqpoint{6.205000in}{2.895479in}}%
\pgfpathlineto{\pgfqpoint{6.205000in}{2.884336in}}%
\pgfusepath{fill}%
\end{pgfscope}%
\begin{pgfscope}%
\pgfpathrectangle{\pgfqpoint{6.205000in}{0.611250in}}{\pgfqpoint{0.365000in}{2.852500in}} %
\pgfusepath{clip}%
\pgfsetbuttcap%
\pgfsetroundjoin%
\definecolor{currentfill}{rgb}{0.929357,0.512254,0.400673}%
\pgfsetfillcolor{currentfill}%
\pgfsetlinewidth{0.000000pt}%
\definecolor{currentstroke}{rgb}{0.000000,0.000000,0.000000}%
\pgfsetstrokecolor{currentstroke}%
\pgfsetdash{}{0pt}%
\pgfpathmoveto{\pgfqpoint{6.205000in}{2.895479in}}%
\pgfpathlineto{\pgfqpoint{6.570000in}{2.895479in}}%
\pgfpathlineto{\pgfqpoint{6.570000in}{2.906621in}}%
\pgfpathlineto{\pgfqpoint{6.205000in}{2.906621in}}%
\pgfpathlineto{\pgfqpoint{6.205000in}{2.895479in}}%
\pgfusepath{fill}%
\end{pgfscope}%
\begin{pgfscope}%
\pgfpathrectangle{\pgfqpoint{6.205000in}{0.611250in}}{\pgfqpoint{0.365000in}{2.852500in}} %
\pgfusepath{clip}%
\pgfsetbuttcap%
\pgfsetroundjoin%
\definecolor{currentfill}{rgb}{0.926883,0.505422,0.394866}%
\pgfsetfillcolor{currentfill}%
\pgfsetlinewidth{0.000000pt}%
\definecolor{currentstroke}{rgb}{0.000000,0.000000,0.000000}%
\pgfsetstrokecolor{currentstroke}%
\pgfsetdash{}{0pt}%
\pgfpathmoveto{\pgfqpoint{6.205000in}{2.906621in}}%
\pgfpathlineto{\pgfqpoint{6.570000in}{2.906621in}}%
\pgfpathlineto{\pgfqpoint{6.570000in}{2.917764in}}%
\pgfpathlineto{\pgfqpoint{6.205000in}{2.917764in}}%
\pgfpathlineto{\pgfqpoint{6.205000in}{2.906621in}}%
\pgfusepath{fill}%
\end{pgfscope}%
\begin{pgfscope}%
\pgfpathrectangle{\pgfqpoint{6.205000in}{0.611250in}}{\pgfqpoint{0.365000in}{2.852500in}} %
\pgfusepath{clip}%
\pgfsetbuttcap%
\pgfsetroundjoin%
\definecolor{currentfill}{rgb}{0.924409,0.498590,0.389059}%
\pgfsetfillcolor{currentfill}%
\pgfsetlinewidth{0.000000pt}%
\definecolor{currentstroke}{rgb}{0.000000,0.000000,0.000000}%
\pgfsetstrokecolor{currentstroke}%
\pgfsetdash{}{0pt}%
\pgfpathmoveto{\pgfqpoint{6.205000in}{2.917764in}}%
\pgfpathlineto{\pgfqpoint{6.570000in}{2.917764in}}%
\pgfpathlineto{\pgfqpoint{6.570000in}{2.928906in}}%
\pgfpathlineto{\pgfqpoint{6.205000in}{2.928906in}}%
\pgfpathlineto{\pgfqpoint{6.205000in}{2.917764in}}%
\pgfusepath{fill}%
\end{pgfscope}%
\begin{pgfscope}%
\pgfpathrectangle{\pgfqpoint{6.205000in}{0.611250in}}{\pgfqpoint{0.365000in}{2.852500in}} %
\pgfusepath{clip}%
\pgfsetbuttcap%
\pgfsetroundjoin%
\definecolor{currentfill}{rgb}{0.921406,0.491420,0.383408}%
\pgfsetfillcolor{currentfill}%
\pgfsetlinewidth{0.000000pt}%
\definecolor{currentstroke}{rgb}{0.000000,0.000000,0.000000}%
\pgfsetstrokecolor{currentstroke}%
\pgfsetdash{}{0pt}%
\pgfpathmoveto{\pgfqpoint{6.205000in}{2.928906in}}%
\pgfpathlineto{\pgfqpoint{6.570000in}{2.928906in}}%
\pgfpathlineto{\pgfqpoint{6.570000in}{2.940049in}}%
\pgfpathlineto{\pgfqpoint{6.205000in}{2.940049in}}%
\pgfpathlineto{\pgfqpoint{6.205000in}{2.928906in}}%
\pgfusepath{fill}%
\end{pgfscope}%
\begin{pgfscope}%
\pgfpathrectangle{\pgfqpoint{6.205000in}{0.611250in}}{\pgfqpoint{0.365000in}{2.852500in}} %
\pgfusepath{clip}%
\pgfsetbuttcap%
\pgfsetroundjoin%
\definecolor{currentfill}{rgb}{0.918282,0.484173,0.377794}%
\pgfsetfillcolor{currentfill}%
\pgfsetlinewidth{0.000000pt}%
\definecolor{currentstroke}{rgb}{0.000000,0.000000,0.000000}%
\pgfsetstrokecolor{currentstroke}%
\pgfsetdash{}{0pt}%
\pgfpathmoveto{\pgfqpoint{6.205000in}{2.940049in}}%
\pgfpathlineto{\pgfqpoint{6.570000in}{2.940049in}}%
\pgfpathlineto{\pgfqpoint{6.570000in}{2.951191in}}%
\pgfpathlineto{\pgfqpoint{6.205000in}{2.951191in}}%
\pgfpathlineto{\pgfqpoint{6.205000in}{2.940049in}}%
\pgfusepath{fill}%
\end{pgfscope}%
\begin{pgfscope}%
\pgfpathrectangle{\pgfqpoint{6.205000in}{0.611250in}}{\pgfqpoint{0.365000in}{2.852500in}} %
\pgfusepath{clip}%
\pgfsetbuttcap%
\pgfsetroundjoin%
\definecolor{currentfill}{rgb}{0.915157,0.476927,0.372179}%
\pgfsetfillcolor{currentfill}%
\pgfsetlinewidth{0.000000pt}%
\definecolor{currentstroke}{rgb}{0.000000,0.000000,0.000000}%
\pgfsetstrokecolor{currentstroke}%
\pgfsetdash{}{0pt}%
\pgfpathmoveto{\pgfqpoint{6.205000in}{2.951191in}}%
\pgfpathlineto{\pgfqpoint{6.570000in}{2.951191in}}%
\pgfpathlineto{\pgfqpoint{6.570000in}{2.962334in}}%
\pgfpathlineto{\pgfqpoint{6.205000in}{2.962334in}}%
\pgfpathlineto{\pgfqpoint{6.205000in}{2.951191in}}%
\pgfusepath{fill}%
\end{pgfscope}%
\begin{pgfscope}%
\pgfpathrectangle{\pgfqpoint{6.205000in}{0.611250in}}{\pgfqpoint{0.365000in}{2.852500in}} %
\pgfusepath{clip}%
\pgfsetbuttcap%
\pgfsetroundjoin%
\definecolor{currentfill}{rgb}{0.912033,0.469680,0.366565}%
\pgfsetfillcolor{currentfill}%
\pgfsetlinewidth{0.000000pt}%
\definecolor{currentstroke}{rgb}{0.000000,0.000000,0.000000}%
\pgfsetstrokecolor{currentstroke}%
\pgfsetdash{}{0pt}%
\pgfpathmoveto{\pgfqpoint{6.205000in}{2.962334in}}%
\pgfpathlineto{\pgfqpoint{6.570000in}{2.962334in}}%
\pgfpathlineto{\pgfqpoint{6.570000in}{2.973477in}}%
\pgfpathlineto{\pgfqpoint{6.205000in}{2.973477in}}%
\pgfpathlineto{\pgfqpoint{6.205000in}{2.962334in}}%
\pgfusepath{fill}%
\end{pgfscope}%
\begin{pgfscope}%
\pgfpathrectangle{\pgfqpoint{6.205000in}{0.611250in}}{\pgfqpoint{0.365000in}{2.852500in}} %
\pgfusepath{clip}%
\pgfsetbuttcap%
\pgfsetroundjoin%
\definecolor{currentfill}{rgb}{0.908908,0.462433,0.360950}%
\pgfsetfillcolor{currentfill}%
\pgfsetlinewidth{0.000000pt}%
\definecolor{currentstroke}{rgb}{0.000000,0.000000,0.000000}%
\pgfsetstrokecolor{currentstroke}%
\pgfsetdash{}{0pt}%
\pgfpathmoveto{\pgfqpoint{6.205000in}{2.973477in}}%
\pgfpathlineto{\pgfqpoint{6.570000in}{2.973477in}}%
\pgfpathlineto{\pgfqpoint{6.570000in}{2.984619in}}%
\pgfpathlineto{\pgfqpoint{6.205000in}{2.984619in}}%
\pgfpathlineto{\pgfqpoint{6.205000in}{2.973477in}}%
\pgfusepath{fill}%
\end{pgfscope}%
\begin{pgfscope}%
\pgfpathrectangle{\pgfqpoint{6.205000in}{0.611250in}}{\pgfqpoint{0.365000in}{2.852500in}} %
\pgfusepath{clip}%
\pgfsetbuttcap%
\pgfsetroundjoin%
\definecolor{currentfill}{rgb}{0.905783,0.455186,0.355336}%
\pgfsetfillcolor{currentfill}%
\pgfsetlinewidth{0.000000pt}%
\definecolor{currentstroke}{rgb}{0.000000,0.000000,0.000000}%
\pgfsetstrokecolor{currentstroke}%
\pgfsetdash{}{0pt}%
\pgfpathmoveto{\pgfqpoint{6.205000in}{2.984619in}}%
\pgfpathlineto{\pgfqpoint{6.570000in}{2.984619in}}%
\pgfpathlineto{\pgfqpoint{6.570000in}{2.995762in}}%
\pgfpathlineto{\pgfqpoint{6.205000in}{2.995762in}}%
\pgfpathlineto{\pgfqpoint{6.205000in}{2.984619in}}%
\pgfusepath{fill}%
\end{pgfscope}%
\begin{pgfscope}%
\pgfpathrectangle{\pgfqpoint{6.205000in}{0.611250in}}{\pgfqpoint{0.365000in}{2.852500in}} %
\pgfusepath{clip}%
\pgfsetbuttcap%
\pgfsetroundjoin%
\definecolor{currentfill}{rgb}{0.902659,0.447939,0.349721}%
\pgfsetfillcolor{currentfill}%
\pgfsetlinewidth{0.000000pt}%
\definecolor{currentstroke}{rgb}{0.000000,0.000000,0.000000}%
\pgfsetstrokecolor{currentstroke}%
\pgfsetdash{}{0pt}%
\pgfpathmoveto{\pgfqpoint{6.205000in}{2.995762in}}%
\pgfpathlineto{\pgfqpoint{6.570000in}{2.995762in}}%
\pgfpathlineto{\pgfqpoint{6.570000in}{3.006904in}}%
\pgfpathlineto{\pgfqpoint{6.205000in}{3.006904in}}%
\pgfpathlineto{\pgfqpoint{6.205000in}{2.995762in}}%
\pgfusepath{fill}%
\end{pgfscope}%
\begin{pgfscope}%
\pgfpathrectangle{\pgfqpoint{6.205000in}{0.611250in}}{\pgfqpoint{0.365000in}{2.852500in}} %
\pgfusepath{clip}%
\pgfsetbuttcap%
\pgfsetroundjoin%
\definecolor{currentfill}{rgb}{0.899534,0.440692,0.344107}%
\pgfsetfillcolor{currentfill}%
\pgfsetlinewidth{0.000000pt}%
\definecolor{currentstroke}{rgb}{0.000000,0.000000,0.000000}%
\pgfsetstrokecolor{currentstroke}%
\pgfsetdash{}{0pt}%
\pgfpathmoveto{\pgfqpoint{6.205000in}{3.006904in}}%
\pgfpathlineto{\pgfqpoint{6.570000in}{3.006904in}}%
\pgfpathlineto{\pgfqpoint{6.570000in}{3.018047in}}%
\pgfpathlineto{\pgfqpoint{6.205000in}{3.018047in}}%
\pgfpathlineto{\pgfqpoint{6.205000in}{3.006904in}}%
\pgfusepath{fill}%
\end{pgfscope}%
\begin{pgfscope}%
\pgfpathrectangle{\pgfqpoint{6.205000in}{0.611250in}}{\pgfqpoint{0.365000in}{2.852500in}} %
\pgfusepath{clip}%
\pgfsetbuttcap%
\pgfsetroundjoin%
\definecolor{currentfill}{rgb}{0.895885,0.433075,0.338681}%
\pgfsetfillcolor{currentfill}%
\pgfsetlinewidth{0.000000pt}%
\definecolor{currentstroke}{rgb}{0.000000,0.000000,0.000000}%
\pgfsetstrokecolor{currentstroke}%
\pgfsetdash{}{0pt}%
\pgfpathmoveto{\pgfqpoint{6.205000in}{3.018047in}}%
\pgfpathlineto{\pgfqpoint{6.570000in}{3.018047in}}%
\pgfpathlineto{\pgfqpoint{6.570000in}{3.029189in}}%
\pgfpathlineto{\pgfqpoint{6.205000in}{3.029189in}}%
\pgfpathlineto{\pgfqpoint{6.205000in}{3.018047in}}%
\pgfusepath{fill}%
\end{pgfscope}%
\begin{pgfscope}%
\pgfpathrectangle{\pgfqpoint{6.205000in}{0.611250in}}{\pgfqpoint{0.365000in}{2.852500in}} %
\pgfusepath{clip}%
\pgfsetbuttcap%
\pgfsetroundjoin%
\definecolor{currentfill}{rgb}{0.892138,0.425389,0.333289}%
\pgfsetfillcolor{currentfill}%
\pgfsetlinewidth{0.000000pt}%
\definecolor{currentstroke}{rgb}{0.000000,0.000000,0.000000}%
\pgfsetstrokecolor{currentstroke}%
\pgfsetdash{}{0pt}%
\pgfpathmoveto{\pgfqpoint{6.205000in}{3.029189in}}%
\pgfpathlineto{\pgfqpoint{6.570000in}{3.029189in}}%
\pgfpathlineto{\pgfqpoint{6.570000in}{3.040332in}}%
\pgfpathlineto{\pgfqpoint{6.205000in}{3.040332in}}%
\pgfpathlineto{\pgfqpoint{6.205000in}{3.029189in}}%
\pgfusepath{fill}%
\end{pgfscope}%
\begin{pgfscope}%
\pgfpathrectangle{\pgfqpoint{6.205000in}{0.611250in}}{\pgfqpoint{0.365000in}{2.852500in}} %
\pgfusepath{clip}%
\pgfsetbuttcap%
\pgfsetroundjoin%
\definecolor{currentfill}{rgb}{0.888390,0.417703,0.327898}%
\pgfsetfillcolor{currentfill}%
\pgfsetlinewidth{0.000000pt}%
\definecolor{currentstroke}{rgb}{0.000000,0.000000,0.000000}%
\pgfsetstrokecolor{currentstroke}%
\pgfsetdash{}{0pt}%
\pgfpathmoveto{\pgfqpoint{6.205000in}{3.040332in}}%
\pgfpathlineto{\pgfqpoint{6.570000in}{3.040332in}}%
\pgfpathlineto{\pgfqpoint{6.570000in}{3.051475in}}%
\pgfpathlineto{\pgfqpoint{6.205000in}{3.051475in}}%
\pgfpathlineto{\pgfqpoint{6.205000in}{3.040332in}}%
\pgfusepath{fill}%
\end{pgfscope}%
\begin{pgfscope}%
\pgfpathrectangle{\pgfqpoint{6.205000in}{0.611250in}}{\pgfqpoint{0.365000in}{2.852500in}} %
\pgfusepath{clip}%
\pgfsetbuttcap%
\pgfsetroundjoin%
\definecolor{currentfill}{rgb}{0.884643,0.410017,0.322507}%
\pgfsetfillcolor{currentfill}%
\pgfsetlinewidth{0.000000pt}%
\definecolor{currentstroke}{rgb}{0.000000,0.000000,0.000000}%
\pgfsetstrokecolor{currentstroke}%
\pgfsetdash{}{0pt}%
\pgfpathmoveto{\pgfqpoint{6.205000in}{3.051475in}}%
\pgfpathlineto{\pgfqpoint{6.570000in}{3.051475in}}%
\pgfpathlineto{\pgfqpoint{6.570000in}{3.062617in}}%
\pgfpathlineto{\pgfqpoint{6.205000in}{3.062617in}}%
\pgfpathlineto{\pgfqpoint{6.205000in}{3.051475in}}%
\pgfusepath{fill}%
\end{pgfscope}%
\begin{pgfscope}%
\pgfpathrectangle{\pgfqpoint{6.205000in}{0.611250in}}{\pgfqpoint{0.365000in}{2.852500in}} %
\pgfusepath{clip}%
\pgfsetbuttcap%
\pgfsetroundjoin%
\definecolor{currentfill}{rgb}{0.880896,0.402331,0.317115}%
\pgfsetfillcolor{currentfill}%
\pgfsetlinewidth{0.000000pt}%
\definecolor{currentstroke}{rgb}{0.000000,0.000000,0.000000}%
\pgfsetstrokecolor{currentstroke}%
\pgfsetdash{}{0pt}%
\pgfpathmoveto{\pgfqpoint{6.205000in}{3.062617in}}%
\pgfpathlineto{\pgfqpoint{6.570000in}{3.062617in}}%
\pgfpathlineto{\pgfqpoint{6.570000in}{3.073760in}}%
\pgfpathlineto{\pgfqpoint{6.205000in}{3.073760in}}%
\pgfpathlineto{\pgfqpoint{6.205000in}{3.062617in}}%
\pgfusepath{fill}%
\end{pgfscope}%
\begin{pgfscope}%
\pgfpathrectangle{\pgfqpoint{6.205000in}{0.611250in}}{\pgfqpoint{0.365000in}{2.852500in}} %
\pgfusepath{clip}%
\pgfsetbuttcap%
\pgfsetroundjoin%
\definecolor{currentfill}{rgb}{0.877149,0.394645,0.311724}%
\pgfsetfillcolor{currentfill}%
\pgfsetlinewidth{0.000000pt}%
\definecolor{currentstroke}{rgb}{0.000000,0.000000,0.000000}%
\pgfsetstrokecolor{currentstroke}%
\pgfsetdash{}{0pt}%
\pgfpathmoveto{\pgfqpoint{6.205000in}{3.073760in}}%
\pgfpathlineto{\pgfqpoint{6.570000in}{3.073760in}}%
\pgfpathlineto{\pgfqpoint{6.570000in}{3.084902in}}%
\pgfpathlineto{\pgfqpoint{6.205000in}{3.084902in}}%
\pgfpathlineto{\pgfqpoint{6.205000in}{3.073760in}}%
\pgfusepath{fill}%
\end{pgfscope}%
\begin{pgfscope}%
\pgfpathrectangle{\pgfqpoint{6.205000in}{0.611250in}}{\pgfqpoint{0.365000in}{2.852500in}} %
\pgfusepath{clip}%
\pgfsetbuttcap%
\pgfsetroundjoin%
\definecolor{currentfill}{rgb}{0.873402,0.386960,0.306332}%
\pgfsetfillcolor{currentfill}%
\pgfsetlinewidth{0.000000pt}%
\definecolor{currentstroke}{rgb}{0.000000,0.000000,0.000000}%
\pgfsetstrokecolor{currentstroke}%
\pgfsetdash{}{0pt}%
\pgfpathmoveto{\pgfqpoint{6.205000in}{3.084902in}}%
\pgfpathlineto{\pgfqpoint{6.570000in}{3.084902in}}%
\pgfpathlineto{\pgfqpoint{6.570000in}{3.096045in}}%
\pgfpathlineto{\pgfqpoint{6.205000in}{3.096045in}}%
\pgfpathlineto{\pgfqpoint{6.205000in}{3.084902in}}%
\pgfusepath{fill}%
\end{pgfscope}%
\begin{pgfscope}%
\pgfpathrectangle{\pgfqpoint{6.205000in}{0.611250in}}{\pgfqpoint{0.365000in}{2.852500in}} %
\pgfusepath{clip}%
\pgfsetbuttcap%
\pgfsetroundjoin%
\definecolor{currentfill}{rgb}{0.869655,0.379274,0.300941}%
\pgfsetfillcolor{currentfill}%
\pgfsetlinewidth{0.000000pt}%
\definecolor{currentstroke}{rgb}{0.000000,0.000000,0.000000}%
\pgfsetstrokecolor{currentstroke}%
\pgfsetdash{}{0pt}%
\pgfpathmoveto{\pgfqpoint{6.205000in}{3.096045in}}%
\pgfpathlineto{\pgfqpoint{6.570000in}{3.096045in}}%
\pgfpathlineto{\pgfqpoint{6.570000in}{3.107188in}}%
\pgfpathlineto{\pgfqpoint{6.205000in}{3.107188in}}%
\pgfpathlineto{\pgfqpoint{6.205000in}{3.096045in}}%
\pgfusepath{fill}%
\end{pgfscope}%
\begin{pgfscope}%
\pgfpathrectangle{\pgfqpoint{6.205000in}{0.611250in}}{\pgfqpoint{0.365000in}{2.852500in}} %
\pgfusepath{clip}%
\pgfsetbuttcap%
\pgfsetroundjoin%
\definecolor{currentfill}{rgb}{0.865391,0.371128,0.295769}%
\pgfsetfillcolor{currentfill}%
\pgfsetlinewidth{0.000000pt}%
\definecolor{currentstroke}{rgb}{0.000000,0.000000,0.000000}%
\pgfsetstrokecolor{currentstroke}%
\pgfsetdash{}{0pt}%
\pgfpathmoveto{\pgfqpoint{6.205000in}{3.107188in}}%
\pgfpathlineto{\pgfqpoint{6.570000in}{3.107188in}}%
\pgfpathlineto{\pgfqpoint{6.570000in}{3.118330in}}%
\pgfpathlineto{\pgfqpoint{6.205000in}{3.118330in}}%
\pgfpathlineto{\pgfqpoint{6.205000in}{3.107188in}}%
\pgfusepath{fill}%
\end{pgfscope}%
\begin{pgfscope}%
\pgfpathrectangle{\pgfqpoint{6.205000in}{0.611250in}}{\pgfqpoint{0.365000in}{2.852500in}} %
\pgfusepath{clip}%
\pgfsetbuttcap%
\pgfsetroundjoin%
\definecolor{currentfill}{rgb}{0.861054,0.362916,0.290628}%
\pgfsetfillcolor{currentfill}%
\pgfsetlinewidth{0.000000pt}%
\definecolor{currentstroke}{rgb}{0.000000,0.000000,0.000000}%
\pgfsetstrokecolor{currentstroke}%
\pgfsetdash{}{0pt}%
\pgfpathmoveto{\pgfqpoint{6.205000in}{3.118330in}}%
\pgfpathlineto{\pgfqpoint{6.570000in}{3.118330in}}%
\pgfpathlineto{\pgfqpoint{6.570000in}{3.129473in}}%
\pgfpathlineto{\pgfqpoint{6.205000in}{3.129473in}}%
\pgfpathlineto{\pgfqpoint{6.205000in}{3.118330in}}%
\pgfusepath{fill}%
\end{pgfscope}%
\begin{pgfscope}%
\pgfpathrectangle{\pgfqpoint{6.205000in}{0.611250in}}{\pgfqpoint{0.365000in}{2.852500in}} %
\pgfusepath{clip}%
\pgfsetbuttcap%
\pgfsetroundjoin%
\definecolor{currentfill}{rgb}{0.856716,0.354704,0.285487}%
\pgfsetfillcolor{currentfill}%
\pgfsetlinewidth{0.000000pt}%
\definecolor{currentstroke}{rgb}{0.000000,0.000000,0.000000}%
\pgfsetstrokecolor{currentstroke}%
\pgfsetdash{}{0pt}%
\pgfpathmoveto{\pgfqpoint{6.205000in}{3.129473in}}%
\pgfpathlineto{\pgfqpoint{6.570000in}{3.129473in}}%
\pgfpathlineto{\pgfqpoint{6.570000in}{3.140615in}}%
\pgfpathlineto{\pgfqpoint{6.205000in}{3.140615in}}%
\pgfpathlineto{\pgfqpoint{6.205000in}{3.129473in}}%
\pgfusepath{fill}%
\end{pgfscope}%
\begin{pgfscope}%
\pgfpathrectangle{\pgfqpoint{6.205000in}{0.611250in}}{\pgfqpoint{0.365000in}{2.852500in}} %
\pgfusepath{clip}%
\pgfsetbuttcap%
\pgfsetroundjoin%
\definecolor{currentfill}{rgb}{0.852378,0.346492,0.280346}%
\pgfsetfillcolor{currentfill}%
\pgfsetlinewidth{0.000000pt}%
\definecolor{currentstroke}{rgb}{0.000000,0.000000,0.000000}%
\pgfsetstrokecolor{currentstroke}%
\pgfsetdash{}{0pt}%
\pgfpathmoveto{\pgfqpoint{6.205000in}{3.140615in}}%
\pgfpathlineto{\pgfqpoint{6.570000in}{3.140615in}}%
\pgfpathlineto{\pgfqpoint{6.570000in}{3.151758in}}%
\pgfpathlineto{\pgfqpoint{6.205000in}{3.151758in}}%
\pgfpathlineto{\pgfqpoint{6.205000in}{3.140615in}}%
\pgfusepath{fill}%
\end{pgfscope}%
\begin{pgfscope}%
\pgfpathrectangle{\pgfqpoint{6.205000in}{0.611250in}}{\pgfqpoint{0.365000in}{2.852500in}} %
\pgfusepath{clip}%
\pgfsetbuttcap%
\pgfsetroundjoin%
\definecolor{currentfill}{rgb}{0.848040,0.338280,0.275206}%
\pgfsetfillcolor{currentfill}%
\pgfsetlinewidth{0.000000pt}%
\definecolor{currentstroke}{rgb}{0.000000,0.000000,0.000000}%
\pgfsetstrokecolor{currentstroke}%
\pgfsetdash{}{0pt}%
\pgfpathmoveto{\pgfqpoint{6.205000in}{3.151758in}}%
\pgfpathlineto{\pgfqpoint{6.570000in}{3.151758in}}%
\pgfpathlineto{\pgfqpoint{6.570000in}{3.162900in}}%
\pgfpathlineto{\pgfqpoint{6.205000in}{3.162900in}}%
\pgfpathlineto{\pgfqpoint{6.205000in}{3.151758in}}%
\pgfusepath{fill}%
\end{pgfscope}%
\begin{pgfscope}%
\pgfpathrectangle{\pgfqpoint{6.205000in}{0.611250in}}{\pgfqpoint{0.365000in}{2.852500in}} %
\pgfusepath{clip}%
\pgfsetbuttcap%
\pgfsetroundjoin%
\definecolor{currentfill}{rgb}{0.843703,0.330068,0.270065}%
\pgfsetfillcolor{currentfill}%
\pgfsetlinewidth{0.000000pt}%
\definecolor{currentstroke}{rgb}{0.000000,0.000000,0.000000}%
\pgfsetstrokecolor{currentstroke}%
\pgfsetdash{}{0pt}%
\pgfpathmoveto{\pgfqpoint{6.205000in}{3.162900in}}%
\pgfpathlineto{\pgfqpoint{6.570000in}{3.162900in}}%
\pgfpathlineto{\pgfqpoint{6.570000in}{3.174043in}}%
\pgfpathlineto{\pgfqpoint{6.205000in}{3.174043in}}%
\pgfpathlineto{\pgfqpoint{6.205000in}{3.162900in}}%
\pgfusepath{fill}%
\end{pgfscope}%
\begin{pgfscope}%
\pgfpathrectangle{\pgfqpoint{6.205000in}{0.611250in}}{\pgfqpoint{0.365000in}{2.852500in}} %
\pgfusepath{clip}%
\pgfsetbuttcap%
\pgfsetroundjoin%
\definecolor{currentfill}{rgb}{0.839365,0.321856,0.264924}%
\pgfsetfillcolor{currentfill}%
\pgfsetlinewidth{0.000000pt}%
\definecolor{currentstroke}{rgb}{0.000000,0.000000,0.000000}%
\pgfsetstrokecolor{currentstroke}%
\pgfsetdash{}{0pt}%
\pgfpathmoveto{\pgfqpoint{6.205000in}{3.174043in}}%
\pgfpathlineto{\pgfqpoint{6.570000in}{3.174043in}}%
\pgfpathlineto{\pgfqpoint{6.570000in}{3.185186in}}%
\pgfpathlineto{\pgfqpoint{6.205000in}{3.185186in}}%
\pgfpathlineto{\pgfqpoint{6.205000in}{3.174043in}}%
\pgfusepath{fill}%
\end{pgfscope}%
\begin{pgfscope}%
\pgfpathrectangle{\pgfqpoint{6.205000in}{0.611250in}}{\pgfqpoint{0.365000in}{2.852500in}} %
\pgfusepath{clip}%
\pgfsetbuttcap%
\pgfsetroundjoin%
\definecolor{currentfill}{rgb}{0.835027,0.313644,0.259783}%
\pgfsetfillcolor{currentfill}%
\pgfsetlinewidth{0.000000pt}%
\definecolor{currentstroke}{rgb}{0.000000,0.000000,0.000000}%
\pgfsetstrokecolor{currentstroke}%
\pgfsetdash{}{0pt}%
\pgfpathmoveto{\pgfqpoint{6.205000in}{3.185186in}}%
\pgfpathlineto{\pgfqpoint{6.570000in}{3.185186in}}%
\pgfpathlineto{\pgfqpoint{6.570000in}{3.196328in}}%
\pgfpathlineto{\pgfqpoint{6.205000in}{3.196328in}}%
\pgfpathlineto{\pgfqpoint{6.205000in}{3.185186in}}%
\pgfusepath{fill}%
\end{pgfscope}%
\begin{pgfscope}%
\pgfpathrectangle{\pgfqpoint{6.205000in}{0.611250in}}{\pgfqpoint{0.365000in}{2.852500in}} %
\pgfusepath{clip}%
\pgfsetbuttcap%
\pgfsetroundjoin%
\definecolor{currentfill}{rgb}{0.830187,0.304733,0.254891}%
\pgfsetfillcolor{currentfill}%
\pgfsetlinewidth{0.000000pt}%
\definecolor{currentstroke}{rgb}{0.000000,0.000000,0.000000}%
\pgfsetstrokecolor{currentstroke}%
\pgfsetdash{}{0pt}%
\pgfpathmoveto{\pgfqpoint{6.205000in}{3.196328in}}%
\pgfpathlineto{\pgfqpoint{6.570000in}{3.196328in}}%
\pgfpathlineto{\pgfqpoint{6.570000in}{3.207471in}}%
\pgfpathlineto{\pgfqpoint{6.205000in}{3.207471in}}%
\pgfpathlineto{\pgfqpoint{6.205000in}{3.196328in}}%
\pgfusepath{fill}%
\end{pgfscope}%
\begin{pgfscope}%
\pgfpathrectangle{\pgfqpoint{6.205000in}{0.611250in}}{\pgfqpoint{0.365000in}{2.852500in}} %
\pgfusepath{clip}%
\pgfsetbuttcap%
\pgfsetroundjoin%
\definecolor{currentfill}{rgb}{0.825294,0.295749,0.250025}%
\pgfsetfillcolor{currentfill}%
\pgfsetlinewidth{0.000000pt}%
\definecolor{currentstroke}{rgb}{0.000000,0.000000,0.000000}%
\pgfsetstrokecolor{currentstroke}%
\pgfsetdash{}{0pt}%
\pgfpathmoveto{\pgfqpoint{6.205000in}{3.207471in}}%
\pgfpathlineto{\pgfqpoint{6.570000in}{3.207471in}}%
\pgfpathlineto{\pgfqpoint{6.570000in}{3.218613in}}%
\pgfpathlineto{\pgfqpoint{6.205000in}{3.218613in}}%
\pgfpathlineto{\pgfqpoint{6.205000in}{3.207471in}}%
\pgfusepath{fill}%
\end{pgfscope}%
\begin{pgfscope}%
\pgfpathrectangle{\pgfqpoint{6.205000in}{0.611250in}}{\pgfqpoint{0.365000in}{2.852500in}} %
\pgfusepath{clip}%
\pgfsetbuttcap%
\pgfsetroundjoin%
\definecolor{currentfill}{rgb}{0.820401,0.286765,0.245160}%
\pgfsetfillcolor{currentfill}%
\pgfsetlinewidth{0.000000pt}%
\definecolor{currentstroke}{rgb}{0.000000,0.000000,0.000000}%
\pgfsetstrokecolor{currentstroke}%
\pgfsetdash{}{0pt}%
\pgfpathmoveto{\pgfqpoint{6.205000in}{3.218613in}}%
\pgfpathlineto{\pgfqpoint{6.570000in}{3.218613in}}%
\pgfpathlineto{\pgfqpoint{6.570000in}{3.229756in}}%
\pgfpathlineto{\pgfqpoint{6.205000in}{3.229756in}}%
\pgfpathlineto{\pgfqpoint{6.205000in}{3.218613in}}%
\pgfusepath{fill}%
\end{pgfscope}%
\begin{pgfscope}%
\pgfpathrectangle{\pgfqpoint{6.205000in}{0.611250in}}{\pgfqpoint{0.365000in}{2.852500in}} %
\pgfusepath{clip}%
\pgfsetbuttcap%
\pgfsetroundjoin%
\definecolor{currentfill}{rgb}{0.815508,0.277781,0.240294}%
\pgfsetfillcolor{currentfill}%
\pgfsetlinewidth{0.000000pt}%
\definecolor{currentstroke}{rgb}{0.000000,0.000000,0.000000}%
\pgfsetstrokecolor{currentstroke}%
\pgfsetdash{}{0pt}%
\pgfpathmoveto{\pgfqpoint{6.205000in}{3.229756in}}%
\pgfpathlineto{\pgfqpoint{6.570000in}{3.229756in}}%
\pgfpathlineto{\pgfqpoint{6.570000in}{3.240898in}}%
\pgfpathlineto{\pgfqpoint{6.205000in}{3.240898in}}%
\pgfpathlineto{\pgfqpoint{6.205000in}{3.229756in}}%
\pgfusepath{fill}%
\end{pgfscope}%
\begin{pgfscope}%
\pgfpathrectangle{\pgfqpoint{6.205000in}{0.611250in}}{\pgfqpoint{0.365000in}{2.852500in}} %
\pgfusepath{clip}%
\pgfsetbuttcap%
\pgfsetroundjoin%
\definecolor{currentfill}{rgb}{0.810616,0.268797,0.235428}%
\pgfsetfillcolor{currentfill}%
\pgfsetlinewidth{0.000000pt}%
\definecolor{currentstroke}{rgb}{0.000000,0.000000,0.000000}%
\pgfsetstrokecolor{currentstroke}%
\pgfsetdash{}{0pt}%
\pgfpathmoveto{\pgfqpoint{6.205000in}{3.240898in}}%
\pgfpathlineto{\pgfqpoint{6.570000in}{3.240898in}}%
\pgfpathlineto{\pgfqpoint{6.570000in}{3.252041in}}%
\pgfpathlineto{\pgfqpoint{6.205000in}{3.252041in}}%
\pgfpathlineto{\pgfqpoint{6.205000in}{3.240898in}}%
\pgfusepath{fill}%
\end{pgfscope}%
\begin{pgfscope}%
\pgfpathrectangle{\pgfqpoint{6.205000in}{0.611250in}}{\pgfqpoint{0.365000in}{2.852500in}} %
\pgfusepath{clip}%
\pgfsetbuttcap%
\pgfsetroundjoin%
\definecolor{currentfill}{rgb}{0.805723,0.259813,0.230562}%
\pgfsetfillcolor{currentfill}%
\pgfsetlinewidth{0.000000pt}%
\definecolor{currentstroke}{rgb}{0.000000,0.000000,0.000000}%
\pgfsetstrokecolor{currentstroke}%
\pgfsetdash{}{0pt}%
\pgfpathmoveto{\pgfqpoint{6.205000in}{3.252041in}}%
\pgfpathlineto{\pgfqpoint{6.570000in}{3.252041in}}%
\pgfpathlineto{\pgfqpoint{6.570000in}{3.263184in}}%
\pgfpathlineto{\pgfqpoint{6.205000in}{3.263184in}}%
\pgfpathlineto{\pgfqpoint{6.205000in}{3.252041in}}%
\pgfusepath{fill}%
\end{pgfscope}%
\begin{pgfscope}%
\pgfpathrectangle{\pgfqpoint{6.205000in}{0.611250in}}{\pgfqpoint{0.365000in}{2.852500in}} %
\pgfusepath{clip}%
\pgfsetbuttcap%
\pgfsetroundjoin%
\definecolor{currentfill}{rgb}{0.800830,0.250829,0.225696}%
\pgfsetfillcolor{currentfill}%
\pgfsetlinewidth{0.000000pt}%
\definecolor{currentstroke}{rgb}{0.000000,0.000000,0.000000}%
\pgfsetstrokecolor{currentstroke}%
\pgfsetdash{}{0pt}%
\pgfpathmoveto{\pgfqpoint{6.205000in}{3.263184in}}%
\pgfpathlineto{\pgfqpoint{6.570000in}{3.263184in}}%
\pgfpathlineto{\pgfqpoint{6.570000in}{3.274326in}}%
\pgfpathlineto{\pgfqpoint{6.205000in}{3.274326in}}%
\pgfpathlineto{\pgfqpoint{6.205000in}{3.263184in}}%
\pgfusepath{fill}%
\end{pgfscope}%
\begin{pgfscope}%
\pgfpathrectangle{\pgfqpoint{6.205000in}{0.611250in}}{\pgfqpoint{0.365000in}{2.852500in}} %
\pgfusepath{clip}%
\pgfsetbuttcap%
\pgfsetroundjoin%
\definecolor{currentfill}{rgb}{0.795938,0.241845,0.220830}%
\pgfsetfillcolor{currentfill}%
\pgfsetlinewidth{0.000000pt}%
\definecolor{currentstroke}{rgb}{0.000000,0.000000,0.000000}%
\pgfsetstrokecolor{currentstroke}%
\pgfsetdash{}{0pt}%
\pgfpathmoveto{\pgfqpoint{6.205000in}{3.274326in}}%
\pgfpathlineto{\pgfqpoint{6.570000in}{3.274326in}}%
\pgfpathlineto{\pgfqpoint{6.570000in}{3.285469in}}%
\pgfpathlineto{\pgfqpoint{6.205000in}{3.285469in}}%
\pgfpathlineto{\pgfqpoint{6.205000in}{3.274326in}}%
\pgfusepath{fill}%
\end{pgfscope}%
\begin{pgfscope}%
\pgfpathrectangle{\pgfqpoint{6.205000in}{0.611250in}}{\pgfqpoint{0.365000in}{2.852500in}} %
\pgfusepath{clip}%
\pgfsetbuttcap%
\pgfsetroundjoin%
\definecolor{currentfill}{rgb}{0.790562,0.231397,0.216242}%
\pgfsetfillcolor{currentfill}%
\pgfsetlinewidth{0.000000pt}%
\definecolor{currentstroke}{rgb}{0.000000,0.000000,0.000000}%
\pgfsetstrokecolor{currentstroke}%
\pgfsetdash{}{0pt}%
\pgfpathmoveto{\pgfqpoint{6.205000in}{3.285469in}}%
\pgfpathlineto{\pgfqpoint{6.570000in}{3.285469in}}%
\pgfpathlineto{\pgfqpoint{6.570000in}{3.296611in}}%
\pgfpathlineto{\pgfqpoint{6.205000in}{3.296611in}}%
\pgfpathlineto{\pgfqpoint{6.205000in}{3.285469in}}%
\pgfusepath{fill}%
\end{pgfscope}%
\begin{pgfscope}%
\pgfpathrectangle{\pgfqpoint{6.205000in}{0.611250in}}{\pgfqpoint{0.365000in}{2.852500in}} %
\pgfusepath{clip}%
\pgfsetbuttcap%
\pgfsetroundjoin%
\definecolor{currentfill}{rgb}{0.785153,0.220851,0.211673}%
\pgfsetfillcolor{currentfill}%
\pgfsetlinewidth{0.000000pt}%
\definecolor{currentstroke}{rgb}{0.000000,0.000000,0.000000}%
\pgfsetstrokecolor{currentstroke}%
\pgfsetdash{}{0pt}%
\pgfpathmoveto{\pgfqpoint{6.205000in}{3.296611in}}%
\pgfpathlineto{\pgfqpoint{6.570000in}{3.296611in}}%
\pgfpathlineto{\pgfqpoint{6.570000in}{3.307754in}}%
\pgfpathlineto{\pgfqpoint{6.205000in}{3.307754in}}%
\pgfpathlineto{\pgfqpoint{6.205000in}{3.296611in}}%
\pgfusepath{fill}%
\end{pgfscope}%
\begin{pgfscope}%
\pgfpathrectangle{\pgfqpoint{6.205000in}{0.611250in}}{\pgfqpoint{0.365000in}{2.852500in}} %
\pgfusepath{clip}%
\pgfsetbuttcap%
\pgfsetroundjoin%
\definecolor{currentfill}{rgb}{0.779745,0.210305,0.207104}%
\pgfsetfillcolor{currentfill}%
\pgfsetlinewidth{0.000000pt}%
\definecolor{currentstroke}{rgb}{0.000000,0.000000,0.000000}%
\pgfsetstrokecolor{currentstroke}%
\pgfsetdash{}{0pt}%
\pgfpathmoveto{\pgfqpoint{6.205000in}{3.307754in}}%
\pgfpathlineto{\pgfqpoint{6.570000in}{3.307754in}}%
\pgfpathlineto{\pgfqpoint{6.570000in}{3.318896in}}%
\pgfpathlineto{\pgfqpoint{6.205000in}{3.318896in}}%
\pgfpathlineto{\pgfqpoint{6.205000in}{3.307754in}}%
\pgfusepath{fill}%
\end{pgfscope}%
\begin{pgfscope}%
\pgfpathrectangle{\pgfqpoint{6.205000in}{0.611250in}}{\pgfqpoint{0.365000in}{2.852500in}} %
\pgfusepath{clip}%
\pgfsetbuttcap%
\pgfsetroundjoin%
\definecolor{currentfill}{rgb}{0.774337,0.199759,0.202535}%
\pgfsetfillcolor{currentfill}%
\pgfsetlinewidth{0.000000pt}%
\definecolor{currentstroke}{rgb}{0.000000,0.000000,0.000000}%
\pgfsetstrokecolor{currentstroke}%
\pgfsetdash{}{0pt}%
\pgfpathmoveto{\pgfqpoint{6.205000in}{3.318896in}}%
\pgfpathlineto{\pgfqpoint{6.570000in}{3.318896in}}%
\pgfpathlineto{\pgfqpoint{6.570000in}{3.330039in}}%
\pgfpathlineto{\pgfqpoint{6.205000in}{3.330039in}}%
\pgfpathlineto{\pgfqpoint{6.205000in}{3.318896in}}%
\pgfusepath{fill}%
\end{pgfscope}%
\begin{pgfscope}%
\pgfpathrectangle{\pgfqpoint{6.205000in}{0.611250in}}{\pgfqpoint{0.365000in}{2.852500in}} %
\pgfusepath{clip}%
\pgfsetbuttcap%
\pgfsetroundjoin%
\definecolor{currentfill}{rgb}{0.768929,0.189213,0.197965}%
\pgfsetfillcolor{currentfill}%
\pgfsetlinewidth{0.000000pt}%
\definecolor{currentstroke}{rgb}{0.000000,0.000000,0.000000}%
\pgfsetstrokecolor{currentstroke}%
\pgfsetdash{}{0pt}%
\pgfpathmoveto{\pgfqpoint{6.205000in}{3.330039in}}%
\pgfpathlineto{\pgfqpoint{6.570000in}{3.330039in}}%
\pgfpathlineto{\pgfqpoint{6.570000in}{3.341182in}}%
\pgfpathlineto{\pgfqpoint{6.205000in}{3.341182in}}%
\pgfpathlineto{\pgfqpoint{6.205000in}{3.330039in}}%
\pgfusepath{fill}%
\end{pgfscope}%
\begin{pgfscope}%
\pgfpathrectangle{\pgfqpoint{6.205000in}{0.611250in}}{\pgfqpoint{0.365000in}{2.852500in}} %
\pgfusepath{clip}%
\pgfsetbuttcap%
\pgfsetroundjoin%
\definecolor{currentfill}{rgb}{0.763520,0.178667,0.193396}%
\pgfsetfillcolor{currentfill}%
\pgfsetlinewidth{0.000000pt}%
\definecolor{currentstroke}{rgb}{0.000000,0.000000,0.000000}%
\pgfsetstrokecolor{currentstroke}%
\pgfsetdash{}{0pt}%
\pgfpathmoveto{\pgfqpoint{6.205000in}{3.341182in}}%
\pgfpathlineto{\pgfqpoint{6.570000in}{3.341182in}}%
\pgfpathlineto{\pgfqpoint{6.570000in}{3.352324in}}%
\pgfpathlineto{\pgfqpoint{6.205000in}{3.352324in}}%
\pgfpathlineto{\pgfqpoint{6.205000in}{3.341182in}}%
\pgfusepath{fill}%
\end{pgfscope}%
\begin{pgfscope}%
\pgfpathrectangle{\pgfqpoint{6.205000in}{0.611250in}}{\pgfqpoint{0.365000in}{2.852500in}} %
\pgfusepath{clip}%
\pgfsetbuttcap%
\pgfsetroundjoin%
\definecolor{currentfill}{rgb}{0.758112,0.168122,0.188827}%
\pgfsetfillcolor{currentfill}%
\pgfsetlinewidth{0.000000pt}%
\definecolor{currentstroke}{rgb}{0.000000,0.000000,0.000000}%
\pgfsetstrokecolor{currentstroke}%
\pgfsetdash{}{0pt}%
\pgfpathmoveto{\pgfqpoint{6.205000in}{3.352324in}}%
\pgfpathlineto{\pgfqpoint{6.570000in}{3.352324in}}%
\pgfpathlineto{\pgfqpoint{6.570000in}{3.363467in}}%
\pgfpathlineto{\pgfqpoint{6.205000in}{3.363467in}}%
\pgfpathlineto{\pgfqpoint{6.205000in}{3.352324in}}%
\pgfusepath{fill}%
\end{pgfscope}%
\begin{pgfscope}%
\pgfpathrectangle{\pgfqpoint{6.205000in}{0.611250in}}{\pgfqpoint{0.365000in}{2.852500in}} %
\pgfusepath{clip}%
\pgfsetbuttcap%
\pgfsetroundjoin%
\definecolor{currentfill}{rgb}{0.752704,0.157576,0.184258}%
\pgfsetfillcolor{currentfill}%
\pgfsetlinewidth{0.000000pt}%
\definecolor{currentstroke}{rgb}{0.000000,0.000000,0.000000}%
\pgfsetstrokecolor{currentstroke}%
\pgfsetdash{}{0pt}%
\pgfpathmoveto{\pgfqpoint{6.205000in}{3.363467in}}%
\pgfpathlineto{\pgfqpoint{6.570000in}{3.363467in}}%
\pgfpathlineto{\pgfqpoint{6.570000in}{3.374609in}}%
\pgfpathlineto{\pgfqpoint{6.205000in}{3.374609in}}%
\pgfpathlineto{\pgfqpoint{6.205000in}{3.363467in}}%
\pgfusepath{fill}%
\end{pgfscope}%
\begin{pgfscope}%
\pgfpathrectangle{\pgfqpoint{6.205000in}{0.611250in}}{\pgfqpoint{0.365000in}{2.852500in}} %
\pgfusepath{clip}%
\pgfsetbuttcap%
\pgfsetroundjoin%
\definecolor{currentfill}{rgb}{0.746838,0.140021,0.179996}%
\pgfsetfillcolor{currentfill}%
\pgfsetlinewidth{0.000000pt}%
\definecolor{currentstroke}{rgb}{0.000000,0.000000,0.000000}%
\pgfsetstrokecolor{currentstroke}%
\pgfsetdash{}{0pt}%
\pgfpathmoveto{\pgfqpoint{6.205000in}{3.374609in}}%
\pgfpathlineto{\pgfqpoint{6.570000in}{3.374609in}}%
\pgfpathlineto{\pgfqpoint{6.570000in}{3.385752in}}%
\pgfpathlineto{\pgfqpoint{6.205000in}{3.385752in}}%
\pgfpathlineto{\pgfqpoint{6.205000in}{3.374609in}}%
\pgfusepath{fill}%
\end{pgfscope}%
\begin{pgfscope}%
\pgfpathrectangle{\pgfqpoint{6.205000in}{0.611250in}}{\pgfqpoint{0.365000in}{2.852500in}} %
\pgfusepath{clip}%
\pgfsetbuttcap%
\pgfsetroundjoin%
\definecolor{currentfill}{rgb}{0.740957,0.122240,0.175744}%
\pgfsetfillcolor{currentfill}%
\pgfsetlinewidth{0.000000pt}%
\definecolor{currentstroke}{rgb}{0.000000,0.000000,0.000000}%
\pgfsetstrokecolor{currentstroke}%
\pgfsetdash{}{0pt}%
\pgfpathmoveto{\pgfqpoint{6.205000in}{3.385752in}}%
\pgfpathlineto{\pgfqpoint{6.570000in}{3.385752in}}%
\pgfpathlineto{\pgfqpoint{6.570000in}{3.396895in}}%
\pgfpathlineto{\pgfqpoint{6.205000in}{3.396895in}}%
\pgfpathlineto{\pgfqpoint{6.205000in}{3.385752in}}%
\pgfusepath{fill}%
\end{pgfscope}%
\begin{pgfscope}%
\pgfpathrectangle{\pgfqpoint{6.205000in}{0.611250in}}{\pgfqpoint{0.365000in}{2.852500in}} %
\pgfusepath{clip}%
\pgfsetbuttcap%
\pgfsetroundjoin%
\definecolor{currentfill}{rgb}{0.735077,0.104460,0.171492}%
\pgfsetfillcolor{currentfill}%
\pgfsetlinewidth{0.000000pt}%
\definecolor{currentstroke}{rgb}{0.000000,0.000000,0.000000}%
\pgfsetstrokecolor{currentstroke}%
\pgfsetdash{}{0pt}%
\pgfpathmoveto{\pgfqpoint{6.205000in}{3.396895in}}%
\pgfpathlineto{\pgfqpoint{6.570000in}{3.396895in}}%
\pgfpathlineto{\pgfqpoint{6.570000in}{3.408037in}}%
\pgfpathlineto{\pgfqpoint{6.205000in}{3.408037in}}%
\pgfpathlineto{\pgfqpoint{6.205000in}{3.396895in}}%
\pgfusepath{fill}%
\end{pgfscope}%
\begin{pgfscope}%
\pgfpathrectangle{\pgfqpoint{6.205000in}{0.611250in}}{\pgfqpoint{0.365000in}{2.852500in}} %
\pgfusepath{clip}%
\pgfsetbuttcap%
\pgfsetroundjoin%
\definecolor{currentfill}{rgb}{0.729196,0.086679,0.167240}%
\pgfsetfillcolor{currentfill}%
\pgfsetlinewidth{0.000000pt}%
\definecolor{currentstroke}{rgb}{0.000000,0.000000,0.000000}%
\pgfsetstrokecolor{currentstroke}%
\pgfsetdash{}{0pt}%
\pgfpathmoveto{\pgfqpoint{6.205000in}{3.408037in}}%
\pgfpathlineto{\pgfqpoint{6.570000in}{3.408037in}}%
\pgfpathlineto{\pgfqpoint{6.570000in}{3.419180in}}%
\pgfpathlineto{\pgfqpoint{6.205000in}{3.419180in}}%
\pgfpathlineto{\pgfqpoint{6.205000in}{3.408037in}}%
\pgfusepath{fill}%
\end{pgfscope}%
\begin{pgfscope}%
\pgfpathrectangle{\pgfqpoint{6.205000in}{0.611250in}}{\pgfqpoint{0.365000in}{2.852500in}} %
\pgfusepath{clip}%
\pgfsetbuttcap%
\pgfsetroundjoin%
\definecolor{currentfill}{rgb}{0.723315,0.068898,0.162989}%
\pgfsetfillcolor{currentfill}%
\pgfsetlinewidth{0.000000pt}%
\definecolor{currentstroke}{rgb}{0.000000,0.000000,0.000000}%
\pgfsetstrokecolor{currentstroke}%
\pgfsetdash{}{0pt}%
\pgfpathmoveto{\pgfqpoint{6.205000in}{3.419180in}}%
\pgfpathlineto{\pgfqpoint{6.570000in}{3.419180in}}%
\pgfpathlineto{\pgfqpoint{6.570000in}{3.430322in}}%
\pgfpathlineto{\pgfqpoint{6.205000in}{3.430322in}}%
\pgfpathlineto{\pgfqpoint{6.205000in}{3.419180in}}%
\pgfusepath{fill}%
\end{pgfscope}%
\begin{pgfscope}%
\pgfpathrectangle{\pgfqpoint{6.205000in}{0.611250in}}{\pgfqpoint{0.365000in}{2.852500in}} %
\pgfusepath{clip}%
\pgfsetbuttcap%
\pgfsetroundjoin%
\definecolor{currentfill}{rgb}{0.717435,0.051118,0.158737}%
\pgfsetfillcolor{currentfill}%
\pgfsetlinewidth{0.000000pt}%
\definecolor{currentstroke}{rgb}{0.000000,0.000000,0.000000}%
\pgfsetstrokecolor{currentstroke}%
\pgfsetdash{}{0pt}%
\pgfpathmoveto{\pgfqpoint{6.205000in}{3.430322in}}%
\pgfpathlineto{\pgfqpoint{6.570000in}{3.430322in}}%
\pgfpathlineto{\pgfqpoint{6.570000in}{3.441465in}}%
\pgfpathlineto{\pgfqpoint{6.205000in}{3.441465in}}%
\pgfpathlineto{\pgfqpoint{6.205000in}{3.430322in}}%
\pgfusepath{fill}%
\end{pgfscope}%
\begin{pgfscope}%
\pgfpathrectangle{\pgfqpoint{6.205000in}{0.611250in}}{\pgfqpoint{0.365000in}{2.852500in}} %
\pgfusepath{clip}%
\pgfsetbuttcap%
\pgfsetroundjoin%
\definecolor{currentfill}{rgb}{0.711554,0.033337,0.154485}%
\pgfsetfillcolor{currentfill}%
\pgfsetlinewidth{0.000000pt}%
\definecolor{currentstroke}{rgb}{0.000000,0.000000,0.000000}%
\pgfsetstrokecolor{currentstroke}%
\pgfsetdash{}{0pt}%
\pgfpathmoveto{\pgfqpoint{6.205000in}{3.441465in}}%
\pgfpathlineto{\pgfqpoint{6.570000in}{3.441465in}}%
\pgfpathlineto{\pgfqpoint{6.570000in}{3.452607in}}%
\pgfpathlineto{\pgfqpoint{6.205000in}{3.452607in}}%
\pgfpathlineto{\pgfqpoint{6.205000in}{3.441465in}}%
\pgfusepath{fill}%
\end{pgfscope}%
\begin{pgfscope}%
\pgfpathrectangle{\pgfqpoint{6.205000in}{0.611250in}}{\pgfqpoint{0.365000in}{2.852500in}} %
\pgfusepath{clip}%
\pgfsetbuttcap%
\pgfsetroundjoin%
\definecolor{currentfill}{rgb}{0.705673,0.015556,0.150233}%
\pgfsetfillcolor{currentfill}%
\pgfsetlinewidth{0.000000pt}%
\definecolor{currentstroke}{rgb}{0.000000,0.000000,0.000000}%
\pgfsetstrokecolor{currentstroke}%
\pgfsetdash{}{0pt}%
\pgfpathmoveto{\pgfqpoint{6.205000in}{3.452607in}}%
\pgfpathlineto{\pgfqpoint{6.570000in}{3.452607in}}%
\pgfpathlineto{\pgfqpoint{6.570000in}{3.463750in}}%
\pgfpathlineto{\pgfqpoint{6.205000in}{3.463750in}}%
\pgfpathlineto{\pgfqpoint{6.205000in}{3.452607in}}%
\pgfusepath{fill}%
\end{pgfscope}%
\begin{pgfscope}%
\pgfsetrectcap%
\pgfsetroundjoin%
\pgfsetlinewidth{1.003750pt}%
\definecolor{currentstroke}{rgb}{0.000000,0.000000,0.000000}%
\pgfsetstrokecolor{currentstroke}%
\pgfsetdash{}{0pt}%
\pgfpathmoveto{\pgfqpoint{6.205000in}{0.611250in}}%
\pgfpathlineto{\pgfqpoint{6.205000in}{0.622393in}}%
\pgfpathlineto{\pgfqpoint{6.205000in}{3.452607in}}%
\pgfpathlineto{\pgfqpoint{6.205000in}{3.463750in}}%
\pgfpathlineto{\pgfqpoint{6.570000in}{3.463750in}}%
\pgfpathlineto{\pgfqpoint{6.570000in}{3.452607in}}%
\pgfpathlineto{\pgfqpoint{6.570000in}{0.622393in}}%
\pgfpathlineto{\pgfqpoint{6.570000in}{0.611250in}}%
\pgfpathlineto{\pgfqpoint{6.205000in}{0.611250in}}%
\pgfusepath{stroke}%
\end{pgfscope}%
\begin{pgfscope}%
\pgfsetbuttcap%
\pgfsetroundjoin%
\definecolor{currentfill}{rgb}{0.000000,0.000000,0.000000}%
\pgfsetfillcolor{currentfill}%
\pgfsetlinewidth{0.501875pt}%
\definecolor{currentstroke}{rgb}{0.000000,0.000000,0.000000}%
\pgfsetstrokecolor{currentstroke}%
\pgfsetdash{}{0pt}%
\pgfsys@defobject{currentmarker}{\pgfqpoint{-0.055556in}{0.000000in}}{\pgfqpoint{0.000000in}{0.000000in}}{%
\pgfpathmoveto{\pgfqpoint{0.000000in}{0.000000in}}%
\pgfpathlineto{\pgfqpoint{-0.055556in}{0.000000in}}%
\pgfusepath{stroke,fill}%
}%
\begin{pgfscope}%
\pgfsys@transformshift{6.570000in}{0.887768in}%
\pgfsys@useobject{currentmarker}{}%
\end{pgfscope}%
\end{pgfscope}%
\begin{pgfscope}%
\pgftext[x=6.625556in,y=0.887768in,left,]{{\sffamily\fontsize{12.000000}{14.400000}\bfseries\selectfont \(\displaystyle -140\)}}%
\end{pgfscope}%
\begin{pgfscope}%
\pgfsetbuttcap%
\pgfsetroundjoin%
\definecolor{currentfill}{rgb}{0.000000,0.000000,0.000000}%
\pgfsetfillcolor{currentfill}%
\pgfsetlinewidth{0.501875pt}%
\definecolor{currentstroke}{rgb}{0.000000,0.000000,0.000000}%
\pgfsetstrokecolor{currentstroke}%
\pgfsetdash{}{0pt}%
\pgfsys@defobject{currentmarker}{\pgfqpoint{-0.055556in}{0.000000in}}{\pgfqpoint{0.000000in}{0.000000in}}{%
\pgfpathmoveto{\pgfqpoint{0.000000in}{0.000000in}}%
\pgfpathlineto{\pgfqpoint{-0.055556in}{0.000000in}}%
\pgfusepath{stroke,fill}%
}%
\begin{pgfscope}%
\pgfsys@transformshift{6.570000in}{1.178839in}%
\pgfsys@useobject{currentmarker}{}%
\end{pgfscope}%
\end{pgfscope}%
\begin{pgfscope}%
\pgftext[x=6.625556in,y=1.178839in,left,]{{\sffamily\fontsize{12.000000}{14.400000}\bfseries\selectfont \(\displaystyle -120\)}}%
\end{pgfscope}%
\begin{pgfscope}%
\pgfsetbuttcap%
\pgfsetroundjoin%
\definecolor{currentfill}{rgb}{0.000000,0.000000,0.000000}%
\pgfsetfillcolor{currentfill}%
\pgfsetlinewidth{0.501875pt}%
\definecolor{currentstroke}{rgb}{0.000000,0.000000,0.000000}%
\pgfsetstrokecolor{currentstroke}%
\pgfsetdash{}{0pt}%
\pgfsys@defobject{currentmarker}{\pgfqpoint{-0.055556in}{0.000000in}}{\pgfqpoint{0.000000in}{0.000000in}}{%
\pgfpathmoveto{\pgfqpoint{0.000000in}{0.000000in}}%
\pgfpathlineto{\pgfqpoint{-0.055556in}{0.000000in}}%
\pgfusepath{stroke,fill}%
}%
\begin{pgfscope}%
\pgfsys@transformshift{6.570000in}{1.469911in}%
\pgfsys@useobject{currentmarker}{}%
\end{pgfscope}%
\end{pgfscope}%
\begin{pgfscope}%
\pgftext[x=6.625556in,y=1.469911in,left,]{{\sffamily\fontsize{12.000000}{14.400000}\bfseries\selectfont \(\displaystyle -100\)}}%
\end{pgfscope}%
\begin{pgfscope}%
\pgfsetbuttcap%
\pgfsetroundjoin%
\definecolor{currentfill}{rgb}{0.000000,0.000000,0.000000}%
\pgfsetfillcolor{currentfill}%
\pgfsetlinewidth{0.501875pt}%
\definecolor{currentstroke}{rgb}{0.000000,0.000000,0.000000}%
\pgfsetstrokecolor{currentstroke}%
\pgfsetdash{}{0pt}%
\pgfsys@defobject{currentmarker}{\pgfqpoint{-0.055556in}{0.000000in}}{\pgfqpoint{0.000000in}{0.000000in}}{%
\pgfpathmoveto{\pgfqpoint{0.000000in}{0.000000in}}%
\pgfpathlineto{\pgfqpoint{-0.055556in}{0.000000in}}%
\pgfusepath{stroke,fill}%
}%
\begin{pgfscope}%
\pgfsys@transformshift{6.570000in}{1.760982in}%
\pgfsys@useobject{currentmarker}{}%
\end{pgfscope}%
\end{pgfscope}%
\begin{pgfscope}%
\pgftext[x=6.625556in,y=1.760982in,left,]{{\sffamily\fontsize{12.000000}{14.400000}\bfseries\selectfont \(\displaystyle -80\)}}%
\end{pgfscope}%
\begin{pgfscope}%
\pgfsetbuttcap%
\pgfsetroundjoin%
\definecolor{currentfill}{rgb}{0.000000,0.000000,0.000000}%
\pgfsetfillcolor{currentfill}%
\pgfsetlinewidth{0.501875pt}%
\definecolor{currentstroke}{rgb}{0.000000,0.000000,0.000000}%
\pgfsetstrokecolor{currentstroke}%
\pgfsetdash{}{0pt}%
\pgfsys@defobject{currentmarker}{\pgfqpoint{-0.055556in}{0.000000in}}{\pgfqpoint{0.000000in}{0.000000in}}{%
\pgfpathmoveto{\pgfqpoint{0.000000in}{0.000000in}}%
\pgfpathlineto{\pgfqpoint{-0.055556in}{0.000000in}}%
\pgfusepath{stroke,fill}%
}%
\begin{pgfscope}%
\pgfsys@transformshift{6.570000in}{2.052054in}%
\pgfsys@useobject{currentmarker}{}%
\end{pgfscope}%
\end{pgfscope}%
\begin{pgfscope}%
\pgftext[x=6.625556in,y=2.052054in,left,]{{\sffamily\fontsize{12.000000}{14.400000}\bfseries\selectfont \(\displaystyle -60\)}}%
\end{pgfscope}%
\begin{pgfscope}%
\pgfsetbuttcap%
\pgfsetroundjoin%
\definecolor{currentfill}{rgb}{0.000000,0.000000,0.000000}%
\pgfsetfillcolor{currentfill}%
\pgfsetlinewidth{0.501875pt}%
\definecolor{currentstroke}{rgb}{0.000000,0.000000,0.000000}%
\pgfsetstrokecolor{currentstroke}%
\pgfsetdash{}{0pt}%
\pgfsys@defobject{currentmarker}{\pgfqpoint{-0.055556in}{0.000000in}}{\pgfqpoint{0.000000in}{0.000000in}}{%
\pgfpathmoveto{\pgfqpoint{0.000000in}{0.000000in}}%
\pgfpathlineto{\pgfqpoint{-0.055556in}{0.000000in}}%
\pgfusepath{stroke,fill}%
}%
\begin{pgfscope}%
\pgfsys@transformshift{6.570000in}{2.343125in}%
\pgfsys@useobject{currentmarker}{}%
\end{pgfscope}%
\end{pgfscope}%
\begin{pgfscope}%
\pgftext[x=6.625556in,y=2.343125in,left,]{{\sffamily\fontsize{12.000000}{14.400000}\bfseries\selectfont \(\displaystyle -40\)}}%
\end{pgfscope}%
\begin{pgfscope}%
\pgfsetbuttcap%
\pgfsetroundjoin%
\definecolor{currentfill}{rgb}{0.000000,0.000000,0.000000}%
\pgfsetfillcolor{currentfill}%
\pgfsetlinewidth{0.501875pt}%
\definecolor{currentstroke}{rgb}{0.000000,0.000000,0.000000}%
\pgfsetstrokecolor{currentstroke}%
\pgfsetdash{}{0pt}%
\pgfsys@defobject{currentmarker}{\pgfqpoint{-0.055556in}{0.000000in}}{\pgfqpoint{0.000000in}{0.000000in}}{%
\pgfpathmoveto{\pgfqpoint{0.000000in}{0.000000in}}%
\pgfpathlineto{\pgfqpoint{-0.055556in}{0.000000in}}%
\pgfusepath{stroke,fill}%
}%
\begin{pgfscope}%
\pgfsys@transformshift{6.570000in}{2.634196in}%
\pgfsys@useobject{currentmarker}{}%
\end{pgfscope}%
\end{pgfscope}%
\begin{pgfscope}%
\pgftext[x=6.625556in,y=2.634196in,left,]{{\sffamily\fontsize{12.000000}{14.400000}\bfseries\selectfont \(\displaystyle -20\)}}%
\end{pgfscope}%
\begin{pgfscope}%
\pgfsetbuttcap%
\pgfsetroundjoin%
\definecolor{currentfill}{rgb}{0.000000,0.000000,0.000000}%
\pgfsetfillcolor{currentfill}%
\pgfsetlinewidth{0.501875pt}%
\definecolor{currentstroke}{rgb}{0.000000,0.000000,0.000000}%
\pgfsetstrokecolor{currentstroke}%
\pgfsetdash{}{0pt}%
\pgfsys@defobject{currentmarker}{\pgfqpoint{-0.055556in}{0.000000in}}{\pgfqpoint{0.000000in}{0.000000in}}{%
\pgfpathmoveto{\pgfqpoint{0.000000in}{0.000000in}}%
\pgfpathlineto{\pgfqpoint{-0.055556in}{0.000000in}}%
\pgfusepath{stroke,fill}%
}%
\begin{pgfscope}%
\pgfsys@transformshift{6.570000in}{2.925268in}%
\pgfsys@useobject{currentmarker}{}%
\end{pgfscope}%
\end{pgfscope}%
\begin{pgfscope}%
\pgftext[x=6.625556in,y=2.925268in,left,]{{\sffamily\fontsize{12.000000}{14.400000}\bfseries\selectfont \(\displaystyle 0\)}}%
\end{pgfscope}%
\begin{pgfscope}%
\pgfsetbuttcap%
\pgfsetroundjoin%
\definecolor{currentfill}{rgb}{0.000000,0.000000,0.000000}%
\pgfsetfillcolor{currentfill}%
\pgfsetlinewidth{0.501875pt}%
\definecolor{currentstroke}{rgb}{0.000000,0.000000,0.000000}%
\pgfsetstrokecolor{currentstroke}%
\pgfsetdash{}{0pt}%
\pgfsys@defobject{currentmarker}{\pgfqpoint{-0.055556in}{0.000000in}}{\pgfqpoint{0.000000in}{0.000000in}}{%
\pgfpathmoveto{\pgfqpoint{0.000000in}{0.000000in}}%
\pgfpathlineto{\pgfqpoint{-0.055556in}{0.000000in}}%
\pgfusepath{stroke,fill}%
}%
\begin{pgfscope}%
\pgfsys@transformshift{6.570000in}{3.216339in}%
\pgfsys@useobject{currentmarker}{}%
\end{pgfscope}%
\end{pgfscope}%
\begin{pgfscope}%
\pgftext[x=6.625556in,y=3.216339in,left,]{{\sffamily\fontsize{12.000000}{14.400000}\bfseries\selectfont \(\displaystyle 20\)}}%
\end{pgfscope}%
\begin{pgfscope}%
\pgftext[x=7.069419in,y=2.037500in,,top,rotate=90.000000]{{\sffamily\fontsize{12.000000}{14.400000}\selectfont \(\displaystyle f(i)\)}}%
\end{pgfscope}%
\end{pgfpicture}%
\makeatother%
\endgroup%
}
\caption{$f(i)$ with respect to partner $j$ (marked as ``X") in a two-referent language game, with $cost_A=80$, $cost_{r_1}=60$, $cost_{r_2}=120$, and $S=85$. Note that in the scenario presented on the left, with $x_j = (0.1, 0.9)$, $f$'s global optimum is located in the bottom-left region of the search space, whereas in the scenario presented on the right, where $x_j = (0.9, 0.1)$, the global optimum is located at the bottom-right.}
\label{fig:1}
\end{figure}

There are a few issues to note with this formulation of $f$. Because the value of $f$ is dependent not only on the speaker's strategy, but also on the strategy their partner uses, because these strategies are a direct result of the positions of particles within the particle swarm, and because the positions of particles may change on every iteration of the PSO task, $f$ is \textit{extremely} dynamic. This dynamism is illustrated in \autoref{fig:1}. PSO is well-known for its resilience to dynamic functions, when accommodations are made by changing the methods by which particle velocities and best positions are updated \citep{engelbrecht2005}. However, the established accommodation techniques as presented in \cite{engelbrecht2005} are unsuitable for use here because they universally assume an objective function which updates periodically, not constantly. Further, many of the techniques proposed make assumptions that would be implausible within the context of the \citeauthor{rohde2012} language game. For example, one technique is to reinitialize all or part of the particle swarm when the objective function changes, however, this would not only result in participants randomly changing strategies every iteration, but would be difficult to justify from a real-world perspective. In another technique, the associated scores for personal best and global best positions are recalculated when the objective function changes. However, in the context of the language game, this would be equivalent to allowing participants to continuously re-evaluate previously held strategies against their partner's current strategy. Because no appropriate adaptations for dynamic fitness functions were found, the implementation of PSO used was not altered to accommodate $f$.

\subsection{Model parameter optimization}
To complete the models described above, the model parameters were optimized to best fit the experimental data. These parameters included the standard PSO parameters, namely, the cognitive and social components, which were restricted to the interval $[0,4]$, an inertial dampening factor, which was restricted to the interval $[1,1.1]$, and an initial inertia, which was restricted to $[0,4]$. Additionally, a velocity dampening constant, which was used in the calculation of particle positions and was restricted to $[0,2]$, and the number of iterations over which the model was to be run, as restricted to $[100,1000]$, were also optimized.

Optimization of these parameters was itself performed via PSO over a $6$-dimensional search space. The parameter values used for this meta-optimization task were those recommended in \cite{shi1998} and \cite{solnon2010}. Parameters were optimized over $1255$ iterations of the particle swarm algorithm for the ``repair" model; these parameters were then also applied to the ``rejection" model. The optimization task itself utilized the unbounded ``rejection" technique as recommended in \cite{engelbrecht2005}. Initial particle positions were assigned using the randomized nonuniform method presented in \cite{mitchell1991}.

The parameter optimization task sought to minimize the discrepancy in rates of ambiguous form coordination, unambiguous form coordination, and failure to coordinate between the model and the experimental data, across all language game variants. In order to evaluate this, pairs were considered to have coordinated if, when the PSO task had completed, referents could be successfully communicated between the pair $\geq 95\%$ of the time.


\section{Results}
\subsection{Meta-optimization task (PSO parameters)}
\begin{center}
    \begin{tabular}{ l l l r }
    Parameter & Optimized & Baseline\footnotemark[1] & $\% \Delta$ \\ \hline
    Cognitive component       & $0.689$ & $2.0$   & $-65.55\%$ \\ \hline
    Social component          & $2.897$ & $2.0$   & $+44.86\%$ \\ \hline
    Inertial dampening factor & $1.027$ & $1.001$ & $+2.58\%$\\ \hline
    Initial inertia           & $0.658$ & $1.2$   & $-45.17\%$ \\ \hline
    Velocity dampening factor & $1.202$ & N/A     & N/A\\ \hline
    Iterations                & $305$   & N/A     & N/A\\ 
    \end{tabular}
\end{center}
\footnotetext[1]{These constitute standard PSO parameters optimized to work well for a number of general problems, as recommended in \cite{shi1998} and \cite{solnon2010}.}

\subsection{Comparison of models to experimental data}
Results for both the rejection and repair models are compared against a baseline model, which makes use of the standard parameters used in the meta-optimization task. For each model, 250 simulations of 10 pairs were performed. Costs from each experiment as taken from the \citeauthor{rohde2012} studies are presented below.
\begin{center}
    \begin{tabular}{ l r r r }
     & $cost_{r_1}$ & $cost_{r_2}$ & $cost_{r_3}$ \\ \hline
    Experiment 1 & $60$ & $120$ & $280$ \\ \hline
    Experiment 2 & $60$ & $120$ & $250$ \\ \hline
    Experiment 3 & $80$ & $140$ & $165$ \\ \hline
    Experiment 4 & $80$ & $135$ & $170$ \\ 
    \end{tabular}
\end{center}

\textit{The results ended up seeming much harder to interpret than when I originally presented them to you. I'll bring my data for our next meeting, in the meanwhile, I've included graphs below, along with my questions/concerns where appropriate.}

\begin{center}
\includegraphics[width=\textwidth]{ambi_coord_exp1.png}
\textit{Is there a more cohesive way of presenting these? Should these box/whisker plots in particular perhaps be moved to an appendix?}
\includegraphics[width=\textwidth]{ambi_coord_exp2.png}
\includegraphics[width=\textwidth]{ambi_coord_exp3.png}
\includegraphics[width=\textwidth]{ambi_coord_exp4.png}
\end{center}

\begin{center}
\includegraphics[width=\textwidth]{ambiguous_coordination_rejection.png}
\textit{Given that these are all at different scales, is there a way to present them together for comparison? How to choose/justify the scaling?}
\includegraphics[width=\textwidth]{ambiguous_coordination_repair.png}
\includegraphics[width=\textwidth]{ambiguous_coordination_baseline.png}
\includegraphics[width=\textwidth]{ambiguous_vs_unambiguous_smaller.png}
\textit{What is the most appropriate way to measure whether either of the two optimized models tracks the experimental data more effectively than the baseline model?}
\end{center}

\subsection{Predictions from successful model}
\textit{No longer clear which should be considered ``successful".}

\section{Discussion}
\begin{itemize}
\item Perhaps mention coevolutionary aspect here?

\item Model has response in keeping with experimental data without agents performing complex modeling (i.e. general optimization is sufficient to produce effects)

\item Repercussions of predictions

\item Possibility of applying PSO to other language games / experiments

\end{itemize}
\section{Conclusion}



\bibliographystyle{apacite}
\bibliography{dissertation.bib}


\end{document}
